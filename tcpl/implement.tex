\startcomponent implement
\product prd-srd

\chapter{IMPLEMENTATION}
\placecontent
The implementation must deal with the two problem mentioned in last chapter.

\section{deal with "endian mode"}
Without gcc's native support, we still has another weapon, \code{MACRO}. Think
that, we defined a bit-field list in order of msb->lsb:
\startccode
SRD_REG_DEF(
	bf1 : 2,
	bf2 : 4,
	bf3 : 3,
	bf4 : 7);
\stopccode
in LE mode, we need lsb->msb, so
the MACRO should reverse it to lsb->msb:
\startccode
bf4 : 7;
bf3 : 3;
bf2 : 4;
bf1 : 2;
\stopccode
but for BE mode, just hold it:
\startccode
bf1 : 2;
bf2 : 4;
bf3 : 3;
bf4 : 7;
\stopccode

for the example, we can implement \code{SRD_REG_DEF} as that:
\startccode
#if defined(__LITTLE_ENDIAN)
#define SRD_REG_DEF(BF1, BF2, BF3, BF4) BF4;BF3;BF2;BF1
#elif defined(__BIG_ENDIAN)
#define SRD_REG_DEF(BF1, BF2, BF3, BF4) BF1;BF2;BF3;BF4
#endif
\stopccode

\subsection{the number of bit-fields}
Because for every register, the number of bit-fields is not fixed, we should
use \code{Variadic Macro}, e.g. for BE mode:
\startccode
#define SRD_REG_DEF(BFX, ...) BFX, ##__VA_ARGS__
\stopccode

But what about LE mode? To reverse the macro arguments, we must know the number
of arguments first. The maxmium register width supported is 64bit.
\startccode
#define PP_NARG(...) PP_NARG_(__VA_ARGS__)
#define PP_NARG_(...) \
	PP_NARG__(x, ##__VA_ARGS__, PP_RSEQ_N())
#define PP_NARG__(...) \
	PP_ARG_N(__VA_ARGS__)
#define PP_ARG_N( \
	 _0, _1, _2, _3, _4, _5, _6, _7, _8, _9, \
	_10,_11,_12,_13,_14,_15,_16,_17,_18,_19, \
	_20,_21,_22,_23,_24,_25,_26,_27,_28,_29, \
	_30,_31,_32,_33,_34,_35,_36,_37,_38,_39, \
	_40,_41,_42,_43,_44,_45,_46,_47,_48,_49, \
	_50,_51,_52,_53,_54,_55,_56,_57,_58,_59, \
	_60,_61,_62,_63,_64,N,...) N
#define PP_RSEQ_N() \
	64,63,62,61,60,                   \
	59,58,57,56,55,54,53,52,51,50, \
	49,48,47,46,45,44,43,42,41,40, \
	39,38,37,36,35,34,33,32,31,30, \
	29,28,27,26,25,24,23,22,21,20, \
	19,18,17,16,15,14,13,12,11,10, \
	9,8,7,6,5,4,3,2,1,0
\stopccode
The first macro \code{PP_NARG} is used to get the argument number. E.g. \code{
PP_NARG()} is \code{0} after extended. The second macro \code{PP_NARG_} will add
one argument \code{x} into the arguments, and followed by a number list(64~0).
\code{PP_ARG_N} will return the 66th argument. If original arguments is null, then
the \code{PP_NARG__} will be extended to:
\startccode
x, 64, 63, ..., 2, 1, 0
\stopccode
The 66th argument is \code{0}, which is the final result. what about only one
argument?
\startccode
x, arg1, 64, 63, ..., 2, 1, 0
\stopccode
The 66th argument - \code{1} is the final result.

And so forth...

\subsection{reverse}
Let's reverse the arguments next. We can get it in mind easily:
\startccode
reverse(x, ...) PP_NARG(##__VA_ARGS__) == 0 ? : reverse(##__VA_ARGS__, x)
\stopccode
Unfortunately, it can't work, because the C's macro don't allow recurrence, even
use two macro, reference each other.

But - without C++'s meta program - we can do that like:
\startccode
#define REVERSE_16(x, y, ...) REVERSE_15(y, ##__VA_ARGS__), x
#define REVERSE_15(x, y, ...) REVERSE_14(y, ##__VA_ARGS__), x
#define REVERSE_14(x, y, ...) REVERSE_13(y, ##__VA_ARGS__), x
...
#define REVERSE_3(x, y, ...) REVERSE_2(y, ##__VA_ARGS__), x
#define REVERSE_2(x, y) y, x

#define REVERSE(x, ...) REVERSE_(PP_NARG(x, ##_VA_ARGS__), x, ##_VA_ARGS__)
#define REVERSE_(n, x, ...) REVERSE_##n(x, ##_VA_ARGS__)
\stopccode
Of course, the maximum number of arguments it supports is 16, but you can extends
it easily.

\section{deal with "access width"}
To avoid the compiler optimization, if using gcc<4.6, we can't access the bit-
field directly, but we can:
\startccode
(typeof(reg)) tmp = reg;
tmp.bfx = 123;
reg = tmp;
\stopccode

\section{result}
By now, we have conquered all the technique problems. We got that finally:
\startccode
/*
 * you need define your own macro to distinguish the kernel space and user space.
 * and in kernel space, we add __BYTE_ORDER to adapt to user space, maybe it is
 * not the best manner, you can go with your own.
 */
#if defined(__KERNEL_SPACE__)
#include <asm/byteorder.h>

#if defined(__LITTLE_ENDIAN)
#if defined(__BIG_ENDIAN)
#error "both big and little endian are defined."
#endif
#endif

#if defined(__LITTLE_ENDIAN)

#define __BYTE_ORDER __LITTLE_ENDIAN
#define __BIG_ENDIAN 4321

#elif defined(__BIG_ENDIAN)

#define __BYTE_ORDER __BIG_ENDIAN
#define __LITTLE_ENDIAN 1234

#else
#error "I don't know big or little"
#endif

#else
#include <endian.h>
#endif

// count the arguments
#define PP_NARG(...) PP_NARG_(__VA_ARGS__)
#define PP_NARG_(...) \
	PP_NARG__(x, ##__VA_ARGS__, PP_RSEQ_N())
#define PP_NARG__(...) \
	PP_ARG_N(__VA_ARGS__)
#define PP_ARG_N( \
	 _0, _1, _2, _3, _4, _5, _6, _7, _8, _9, \
	_10,_11,_12,_13,_14,_15,_16,_17,_18,_19, \
	_20,_21,_22,_23,_24,_25,_26,_27,_28,_29, \
	_30,_31,_32,_33,_34,_35,_36,_37,_38,_39, \
	_40,_41,_42,_43,_44,_45,_46,_47,_48,_49, \
	_50,_51,_52,_53,_54,_55,_56,_57,_58,_59, \
	_60,_61,_62,_63,_64,N,...) N
#define PP_RSEQ_N() \
	64,63,62,61,60,                   \
	59,58,57,56,55,54,53,52,51,50, \
	49,48,47,46,45,44,43,42,41,40, \
	39,38,37,36,35,34,33,32,31,30, \
	29,28,27,26,25,24,23,22,21,20, \
	19,18,17,16,15,14,13,12,11,10, \
	9,8,7,6,5,4,3,2,1,0

// concatenate two str
#define PP_CONCATENATE(arg1, arg2) PP_CONCATENATE_(arg1, arg2)
#define PP_CONCATENATE_(arg1, arg2) PP_CONCATENATE__(arg1, arg2)
#define PP_CONCATENATE__(arg1, arg2) arg1##arg2

// forward / backward traverse
#define SRD_FW(x, y, ...) x; y, ##__VA_ARGS__
#define SRD_BW(x, y, ...) y, ##__VA_ARGS__; x

// recurrence of bit-fields
#define SRD_FOREACH_16(dir, type, x, ...) SRD_##dir( \
		SRD_FOREACH_1(dir, type, x), \
		SRD_FOREACH_15(dir, type, __VA_ARGS__))
#define SRD_FOREACH_15(dir, type, x, ...) SRD_##dir( \
		SRD_FOREACH_1(dir, type, x), \
		SRD_FOREACH_14(dir, type, __VA_ARGS__))
#define SRD_FOREACH_14(dir, type, x, ...) SRD_##dir( \
		SRD_FOREACH_1(dir, type, x), \
		SRD_FOREACH_13(dir, type, __VA_ARGS__))
#define SRD_FOREACH_13(dir, type, x, ...) SRD_##dir( \
		SRD_FOREACH_1(dir, type, x), \
		SRD_FOREACH_12(dir, type, __VA_ARGS__))
#define SRD_FOREACH_12(dir, type, x, ...) SRD_##dir( \
		SRD_FOREACH_1(dir, type, x), \
		SRD_FOREACH_11(dir, type, __VA_ARGS__))
#define SRD_FOREACH_11(dir, type, x, ...) SRD_##dir( \
		SRD_FOREACH_1(dir, type, x), \
		SRD_FOREACH_10(dir, type, __VA_ARGS__))
#define SRD_FOREACH_10(dir, type, x, ...) SRD_##dir( \
		SRD_FOREACH_1(dir, type, x), \
		SRD_FOREACH_9(dir, type, __VA_ARGS__))
#define SRD_FOREACH_9(dir, type, x, ...) SRD_##dir( \
		SRD_FOREACH_1(dir, type, x), \
		SRD_FOREACH_8(dir, type, __VA_ARGS__))
#define SRD_FOREACH_8(dir, type, x, ...) SRD_##dir( \
		SRD_FOREACH_1(dir, type, x), \
		SRD_FOREACH_7(dir, type, __VA_ARGS__))
#define SRD_FOREACH_7(dir, type, x, ...) SRD_##dir( \
		SRD_FOREACH_1(dir, type, x), \
		SRD_FOREACH_6(dir, type, __VA_ARGS__))
#define SRD_FOREACH_6(dir, type, x, ...) SRD_##dir( \
		SRD_FOREACH_1(dir, type, x), \
		SRD_FOREACH_5(dir, type, __VA_ARGS__))
#define SRD_FOREACH_5(dir, type, x, ...) SRD_##dir( \
		SRD_FOREACH_1(dir, type, x), \
		SRD_FOREACH_4(dir, type, __VA_ARGS__))
#define SRD_FOREACH_4(dir, type, x, ...) SRD_##dir( \
		SRD_FOREACH_1(dir, type, x), \
		SRD_FOREACH_3(dir, type, __VA_ARGS__))
#define SRD_FOREACH_3(dir, type, x, ...) SRD_##dir( \
		SRD_FOREACH_1(dir, type, x), \
		SRD_FOREACH_2(dir, type, __VA_ARGS__))
#define SRD_FOREACH_2(dir, type, x, ...) SRD_##dir( \
		SRD_FOREACH_1(dir, type, x), \
		SRD_FOREACH_1(dir, type, __VA_ARGS__))
#define SRD_FOREACH_1(dir, type, x) type x

// traverse bit-fields by specific direction (backward or forward)
#define SRD_FOREACH(dir, type, x, ...) \
	PP_CONCATENATE(SRD_FOREACH_, PP_NARG(x, ##__VA_ARGS__)) \
		(dir, type, x, ##__VA_ARGS__)

// used to define bit-fields, assume listed from msb to lsb
#if __BYTE_ORDER == __LITTLE_ENDIAN
# define SRD_REG_BF(type, x, ...) SRD_FOREACH(BW, type, x, ##__VA_ARGS__)
#elif __BYTE_ORDER == __BIG_ENDIAN
# define SRD_REG_BF(type, x, ...) SRD_FOREACH(FW, type, x, ##__VA_ARGS__)
#else
#error "I don't know the endian mode"
#endif
\stopccode

\subsection{usage}
You can extends it to common use. Here is the usage:
\startccode
#define SRD_TEST_REG(name, bf, ...) \
	union { \
		volatile u_int16_t v; \
		struct { \
			SRD_REG_BF(volatile u_int16_t, bf, ##__VA_ARGS__); \
		} _bf; \
	} name

/*
 * the *SRD_RBF* and *SRD_WBF* is used to read/write bit-field.
 * @param:
 *     @reg	- the register
 *     @bf	- the bit-field
 *     @p	- the pointer pointed to the data store
 */
#define SRD_RBF(reg, bf, p) do { \
		(typeof(reg)) tmp = { .v = (reg).v, };
		*(p) = tmp._bf.bf;
	} while(0)
#define SRD_WBF(reg, bf, p) do { \
		(typeof(reg)) tmp = { .v = (reg).v, };
		tmp._bf.bf = *(p);
		(reg).v = tmp.v;
	} while(0)

struct test_chip_t {
	SRD_TEST_REG(test_reg1,
		a : 3,
		b : 4,
		c : 5,
		d : 4);
	SRD_TEST_REG(test_reg2,
		a : 3,
		b : 4,
		c : 5,
		d : 4);
} test_chip;

u_int16_t data = 0;

SRD_RBF(test_chip.test_reg1, b, &data);
SRD_WBF(test_chip.test_reg2, d, &data);
\stopccode

\section{NOTE}
In the example, we access the register directly, at least on powerpc's localbus.
If you are using other CPUs or other bus, you need read the manual for further
information. And, maybe the barrier is needed.

And there are many further issues you need to consider, if you want to use it in
your product. E.g. \code{RO/WO/RW}, of course, for \code{RO}, you can add
\code{const} preceding the bit-field name, but what about \code{WO}? Maybe there
are some address unused in the address space of one chip, and I'm sure you don't
want code like:
\startccode
struct xx {
	...
	u_int16_t unused1[32];
	...
	u_int16_t unused2[16];
	...
};
\stopccode
For this problem, I can share my idea with you:
\startccode
/*
 * for unused registers
 * @param:
 *     type	- register type, u_int{8|16|32}_t
 *     start	- the first register offset in the gap region, closed
 *     end	- the last register offset in the gap region, open
 * the gap region is [start, end), the start and end can't be expression, only
 * value be supported.
 */
#define SRD_GAP(type, start, end) type gap_##start_##end[(end) - (start)]
\stopccode

Finally, how to ensure we write the right bit-field definition? One aspect of it
is the size. My trial:
\startccode
struct xxx {
	...
};

typedef char xxx_size[sizeof(struct xxx) == N ? 1 : -1];
\stopccode
In which, the \code{N} is your expect.

And ..., it's the show time of you. Good luck!!!

\useURL[author-email][mailto:niqingliang@insigma.com.cn][][niqingliang@insigma.com.cn]
\useURL[author-blog][http://niqingliang2003.wordpress.com][][http://niqingliang2003.wordpress.com]
If any problem, please contact author \from[author-email], or visit \from[author-blog].

\stopcomponent

