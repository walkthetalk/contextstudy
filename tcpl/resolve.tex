\startcomponent resolve
\product prd-srd

\chapter{RESOLVE PROBLEM}
\placecontent
So we got the direction: use \code{structure} in 'C' to represent the registers.

\section{endian mode}
PAUSE! For compiler, e.g. gcc, when compiling bit-field code, the order is depends
on CPU's endian mode. For little endian, e.g. x86, it is from lsb to msb, and for
big endian, it will be from msb to lsb. (Generally speaking, there is no need to
consider the bit-order in one field, because it is masked by hardware.)

In other words, if we write code like this:
\startccode
struct xxx {
	u_int8_t bf1 : 4;
	u_int8_t bf2 : 4;
} var;

var = 0x12;
var.bf1 = ?;
\stopccode
we will get different value about \code{var.bf1}, 0x1 on BE CPU, and 0x2 on LE
CPU.

Now, we want the code section above result same result on BE and LE CPUs, how to
do that?

\subsection{native support by gcc}
\useURL
  [gcc-bf-order-ctrl]
  [http://gcc.gnu.org/ml/gcc/2011-07/msg00142.html]
  []
  [gcc.gnu.org]
At this moment, the gcc doesn't provide what we want, but it is prospective. We
can find some information on \from[gcc-bf-order-ctrl]. And then we can assign
an lsb/msb attribute to one structure type, after gcc implemented that feature.

\section{access width}
Another problem, access-width comes. Some compiler, e.g. gcc (still gcc, don't
ask me why), will generate optimized machine code when accessing bit-field. Consider that:
\startccode
struct xxx {
	u_int16_t bf1 : 8;
	u_int16_t bf2 : 8;
} var;

var.bf1 = 0x1;
\stopccode

If you deassemle the binary generated from it, you may find that it equals to:
\startccode
struct xxx {
	u_int16_t bf1 : 8;
	u_int16_t bf2 : 8;
} var;

*(u_int8_t *)(&var) = 0x1;
\stopccode
In other words, gcc found \code{var.bf1} is one full byte, and direct assigns value
to that byte without regard to the container type (\code{u_int16_t}). For memory,
there is no problem, but think that you are accessing register! Generally speaking,
it is always need to read/write the whole register, but not part of it. So the
assignment in the example may not work (if the var is an register).

\subsection{native support by gcc}
\useURL
  [gcc-strict-volatile-bf]
  [http://gcc.gnu.org/gcc-4.6/changes.html]
  []
  [gcc.gnu.org]
Indeed, when representing registers, we need to add \code{volatile} to the type.
In gcc 4.6's new feature list(\from[gcc-strict-volatile-bf]), we can find that:
\startframedtextsec
A new switch \code{-fstrict-volatile-bitfields} has been added. Using it indicates
that accesses to volatile bitfields should use a single access of the width of the
field's type. This option can be useful for precisely defining and accessing
memory-mapped peripheral registers from C or C++.
\stopframedtextsec
That's it! For different CPU architecture, it's default state is different. So
maybe we need add it into \code{CFLAGS} manually. After turn on this switch, all
bit-fields with qualifier \code{volatile} will be accessed according it's container
type.

\stopcomponent

