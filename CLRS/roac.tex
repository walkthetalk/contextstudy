\startcomponent roac
\product prd-clrs

\chapter{算法在计算中的角色}
\placecontent
\section{算法}

input -> algorithm -> output

解决computational problem。最基本的问题:排序。

一个具体问题就是一个实例(instance)。只有对任一输入实例,都能产生正确的输出,这个算法才是正确(correct)的。

算法能解决哪些问题:人体基因工程、管理操控大量因特网数据、电子商务中的公钥加密和数字签名、企业分配稀有资源来获取最大利润……

也可以解决一些特定的问题:最短路径、最长公共子串、拓扑排序、求n个点的最小凸包、……

算法的两点共性:需要在众多算法中找到一个能解决当前问题或者最好的那个,有切实的用途。

数据结构(data structure)是存储和组织数据的一种方式。

技术:掌握如何设计、分析算法。

难题:NP完全问题,没有人能证明或否证对其有效算法的存在性。如果任一NP完全问题存在相应的有效算法,则所有NP完全问题都存在这样的算法。一些NP完全问题都有一个与其相似但不同并且存在有效算法的问题。

并行。

\section{如技术之算法}
效率:插入排序和归并排序。

算法和其它技术。

\stopcomponent

