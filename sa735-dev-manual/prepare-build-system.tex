\startcomponent prepare-build-system
\product prd-sa735-dev-manual

\useURL
  [yocto-official-address]
  [http://www.yoctoproject.org/]
  []
  [yocto 官网]
\useURL
  [optimus-address]
  [http://10.10.20.149/gitweb/?p=optimus.git;a=summary]
  []
  [optimus 网址]

\chapter{构建编译系统}
我们的编译系统由 optimus 构建而来,而 optimus 是基于 LFS 的 yocto 项目。
关于 yocto 项目,请参考\from[yocto-official-address]。
optimus 为了支持我们自己的产品,在 yocto 基础之上进行了二次开发,
添加了我们自己的一些软件包和脚本。
optimus 目前由 git 来进行版本管理。请使用如下命令将 optimus 克隆到本地:
\startccode
git clone git://10.10.20.149/optimus <optimus-dir>
\stopccode
其中\code{<optimus-dir>}即本地路径。
你也可以通过\from[optimus-address]来查看其更新历史。

后面的步骤都以 sa735 产品为例,对于其它产品仅供参考。

我们所说的编译系统,即进行产品软件开发所需的开发环境(部分)。
其实 optimus 本身只是一个 build framework。
将 optimus 克隆到本地后,继续准备编译目录,此目录下的文件就是我们的开发环境。
\startccode
<optimus-dir>/meta-sa735/scripts/setupbuildir -e <dev-env-dir>
\stopccode
其中\code{<dev-env-dir>}就是编译目录。
选项\code{-e}指明我们需要的模拟版本的开发环境,这样编译出的开发环境可以编译 x86 版本的软件
(一般在 pc 上用模拟器 qemu 运行)。
如果不加此参数,则编译出的开发环境可以编译 powerpc 版本的软件(一般在真实设备上运行)。

然后开始编译:
\startccode
<dev-env-dir>/optimus-build core-image-sa735 for eclipse
\stopccode
此步骤会生成编译工具链、 Linux 内核、 rootfs。
而\code{for eclipse}则是为了生成 eclipse 相关的支持工具。

\stopcomponent

