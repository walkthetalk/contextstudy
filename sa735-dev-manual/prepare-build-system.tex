\startcomponent prepare-build-system
\product prd-sa735-dev-manual

\setupexternalfigures[directory=fig]
\useURL
  [yocto-official-address]
  [http://www.yoctoproject.org/]
  []
  [yocto 官网]
\useURL
  [optimus-address]
  [http://10.10.20.149/gitweb/?p=optimus.git;a=summary]
  []
  [optimus 网址]

\chapter{构建编译系统}

注意:命令中形如 \code{<src_dir>} 以尖括号括起来的内容为目录,用户可以自行指定,
比如第一条命令中有 \code{<optimus-dir>},用户指定了目录 \code{/media/word/optimus},
则在后续命令中再次出现 \code{<optimus-dir>} 时,请用 \code{/media/word/optimus} 自行替换。

我们的编译系统由 optimus 构建而来,而 optimus 是基于 LFS 的 yocto 项目。
关于 yocto 项目,请参考\from[yocto-official-address]。
optimus 为了支持我们自己的产品,在 yocto 基础之上进行了二次开发,
添加了我们自己的一些软件包和脚本。
optimus 目前由 git 来进行版本管理。请使用如下命令将 optimus 克隆到本地:
\startccode
git clone git://10.10.20.149/optimus.git <optimus-dir>
\stopccode
其中\code{<optimus-dir>}即本地路径。
你也可以通过\from[optimus-address]来查看其更新历史。

后面的步骤都以 sa735 产品为例,对于其它产品仅供参考。

我们所说的编译系统,即进行产品软件开发所需的开发环境(部分)。
其实 optimus 本身只是一个 build framework。
将 optimus 克隆到本地后,继续准备编译目录,此目录下的文件就是我们的开发环境。
\startccode
<optimus-dir>/meta-sa735/scripts/setupbuildir -d <local_src_dir> -e <dev-env-dir>
\stopccode

其中\code{<dev-env-dir>}就是编译目录。
而\code{-d <local_src_dir>}则是指定本地代码的目录,
如果不指定此选项,则会使用代码服务器上的代码库,对于 sa735 而言,
\code{<local_src_dir>}目录下应该有两个文件夹,
\code{omt}是平台代码,\code{sa735}是产品代码。
选项\code{-e}指明我们需要的模拟版本的开发环境,
这样编译出的开发环境可以编译 x86 版本的软件
(一般在 pc 上用模拟器 qemu 运行)。
如果不加此参数,则编译出的开发环境可以编译 powerpc 版本的软件(一般在真实设备上运行)。

\code{<dev-env-dir>}下会有两个可执行文件,一个是\code{optimus-build},
用来编译单个软件包或生成开发环境:
\startccode
<dev-env-dir>/optimus-build core-image-sa735 for eclipse
\stopccode
此步骤会生成编译工具链、 Linux 内核、 rootfs。
而\code{for eclipse}则是为了生成 eclipse 相关的支持工具;
另一个就是\code{taijitu},此为 GUI 编译工具,目前仅支持我们自己的软件包,界面如\in{图 }{}[fig: taijitu]所示,其使用方式请参考相关文档。
\placefigure[here,force][fig: taijitu]
   {GUI编译工具——太极图}
   {\externalfigure[taijitu.png][scale=1500]}

\stopcomponent

