\startcomponent configure-eclipse
\product prd-sa735-dev-manual

\setupexternalfigures[directory=fig]

\useURL
  [eclipse-optimus-address]
  [http://10.10.20.149/optimustools/eclipse-optimus.tar.xz]
  []
  [内部服务器]
\useURL
  [yocto-official-doc-address]
  [http://www.yoctoproject.org/docs/current/adt-manual/adt-manual.html\#adt-eclipse]
  []
  [yocto 官方文档]

\chapter{配置 Eclipse}
首先需要为 Eclipse 安装一些插件,简单起见,我们直接从\from[eclipse-optimus-address]下载。
关于 Eclipse 相关内容,你也可以参考\from[yocto-official-doc-address]。

为了更方便的使用 Eclipse, yocto 为其开发了一个插件: Yocto ADT。
下面主要说一下此插件的配置。

首先打开菜单\code{Window -> Preferences},在左侧栏中选中\code{Yocto ADT},
我们可以看到如\in{图 }{}[fig: adt-plugin]所示的配置界面。
\placefigure[here,force][fig: adt-plugin]
   {Yocto ADT 配置界面}
   {\externalfigure[adt-plugin.png][scale=1500]}
然后在\code{Cross Compiler Options:}中选中\code{Build system derived toolchain};

点击 \code{Toolchain Root Location:} 右边的 \code{Browse} 按钮
选择目录 \code{<dev-env-dir>}。这时 \code{Target Architecture:}
右侧下拉菜单中会自动检测,本例中显示的是 \code{i586-insigma-linux}。

点击 \code{Sysroot Location:} 右边的 \code{Browse} 按钮
选择目录:
\startccode
<dev-env-dir>/sysroot_for_eclipse
\stopccode

选中 \code{Target Options:} 中的 \code{QEMU},然后点击 \code{Kernel:} 右侧的
\code{Browse} 按钮选择内核,本例中为
\startccode
<dev-env-dir>/tmp/deploy/images/bzImage-sa735emu.bin
\stopccode
\placefigure[here,force]
   {Yocto ADT 配置完成}
   {\externalfigure[adt-plugin-cfg.png][scale=1500]}
点击按钮 \code{OK} 关闭配置界面。


\stopcomponent

