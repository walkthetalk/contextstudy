\startcomponent sa735-project
\product prd-sa735-dev-manual

\setupexternalfigures[directory=fig]

\chapter{在 eclipse 中创建 sa735 项目}
首先要把sa735的源码下载到本地:
\startccode
cd <src_dir>
hg clone hg://iot-hg.insigma.com.cn/liye/omt
hg clone hg://iot-hg.insigma.com.cn/niqingliang/sa735
\stopccode
其中 \code{<src_dir>} 就是存储源码的目录。
至于后面两步克隆,前者是下载平台代码,后者是下载产品代码。

然后对我们的软件进行配置:
\startccode
source <dev-env-dir>/tmp/environment-setup-i586-insigma-linux
cd <src_dir>/sa735/build-optimus/emu
<src_dir>/bintools-optimus/configure
\stopccode
第一步的作用是设置环境变量,以便使用我们的开发环境。
本例我们用的模拟版本的软件,所以第二步用的目录是 \code{emu},否则请使用目录 \code{dev}。

下面我们在 Eclipse 下创建新工程。

打开 Eclipse,点击菜单 \code{File -> New -> C++ Project},
可以看到\in{图 }{}[fig: new prj]。

首先给工程起个名字,如 sa735,将其输入到 \code{Project name:} 右侧的文本框中。
然后在左侧栏 \code{Project type:} 中展开 \code{Yocto ADT Project},
选中其中的 \code{Empty Project}。
然后取消复选框 \code{Use default location},这时 \code{Location:} 一行会变成可用状态。
点击右侧的 \code{Browse...} 按钮选择我们的 Makefile 所在目录,本例中是
\startccode
<src_dir>/sa735/build-optimus/emu
\stopccode
这时如\in{图 }{}[fig: new prj],然后点击按钮 \code{Finish},如果有提示,直接点击 \code{OK},至此,我们成功创建了工程。
\placefigure[here,force][fig: new prj]{新建工程}
   {\startcombination
      {\externalfigure[new-cpp-prj.png][width=.5\textwidth]} {\placefloatcaption{新建工程}}
      {\externalfigure[sa735prj.png][width=.5\textwidth]} {\placefloatcaption{配置工程}}
    \stopcombination}

然后在 Eclipse 主界面的 \code{Project Explorer} 中选中我们所创建的项目。

% 此部分暂时不做
%下面我们修改一下工程的配置,点击 \code{Project -> Properties},会出现工程的配置窗口。
%在左侧栏中选择 \code{Builders},然后移除复选框 \code{Autotools Makefile Generator}
%(我们的工程不是用的 Autotools),并单击 \code{OK}。

\useexternalfigure[hammer][build_exec.png][height=2ex]
现在我们就可以编译程序了。
点击工具栏中的\externalfigure[hammer],来生成我们的程序。
你也可以通过菜单 \code{Project -> Build Configurations -> manage...} 为工程生成多个配置(比如,debug、release),并通过\externalfigure[hammer]右侧的箭头进行选择。

由于我们使用的是模拟版本,所以需要启动虚拟机,此虚拟机用 qemu 运行。
(注意,Eclipse 默认使用 xterm 运行 qemu,请安装此软件包。)
请点击菜单 \code{Run -> External Tools -> qemu_i586-insigma-linux} 来运行虚拟机。
此虚拟机用户名为 root,密码为空。启动后,你可能需要执行 \code{killall tinbox} 来杀死自动启动的 tinbox (因为我们要调试此程序,此处杀死的是 rootfs 中自带的)。

下面我们开始调试。首先请单击菜单 \code{Run -> Debug Configurations...},
出现调试的配置窗口,在左侧栏中展开 \code{C/C++ Remote Application},
选中下面的 \code{sa735_gdb_i586-insigma-linux},可以看到\in{图 }{}[fig: dbg-cfg]。
\placefigure[here,force][fig: dbg-cfg]
   {调试配置界面}
   {\externalfigure[dbg-cfg.png][scale=1500]}

在右侧 \code{Main} 选项卡中点击 \code{Connection:} 右边的按钮 \code{New...},
然后选择 \code{TCF},如\in{图 }{}[fig: tcf]。
\placefigure[here,force][fig: tcf]
   {选择 TCF}
   {\externalfigure[tcf.png][scale=1500]}

继续点击 \code{Next},在\in{图 }{}[fig: connection]中的 \code{Host name:}
右侧文本框中填写虚拟机的 IP 地址(你可以在虚拟机中用命令 \code{ifconfig} 查看),
然后点击 \code{Finish}。
\placefigure[here,force][fig: connection]
   {创建到虚拟机的远程连接}
   {\externalfigure[connection.png][scale=1500]}

下面又回到\in{图 }{}[fig: dbg-cfg],
在 \code{Connection:} 右侧的下拉选单中选择刚创建的连接。
然后点击按钮 \code{Search Project...} 选择应用程序,
本例中我们选择 tinbox (参见\in{图 }{}[fig: program-sel])。
\placefigure[here,force][fig: program-sel]
   {选择要调试的程序}
   {\externalfigure[program-sel.png][scale=1500]}

然后选择运行此程序时的路径名,如\code{Remote Absolute File Path for C/C++ Application:},本例中指定为 \code{/usr/sbin/tinbox.debug}。

如有需要,在选项卡 \code{Arguments} 中的 \code{Program arguments:} 中填入参数,
本例中填的是 \code{DE 1 1},如\in{图 }{}[fig: arg]所示。
\placefigure[here,force][fig: arg]
   {程序参数}
   {\externalfigure[arg.png][scale=1500]}

配置好后如\in{图 }{}[fig: dbg-cfg-done]所示。
\placefigure[here,force][fig: dbg-cfg-done]
   {配置完毕}
   {\externalfigure[dbg-cfg-done.png][scale=1500]}

然后点击右下角的 \code{Debug} 按钮,出现远程登录的界面\in{图 }{}[fig: remote-login]。
\placefigure[here,force][fig: remote-login]
   {远程登录}
   {\externalfigure[remote-login.png][scale=1500]}

在 \code{User ID:} 右侧文本框中输入用户名 \code{root},点击 \code{OK} 开始调试。

\stopcomponent

