\usemodule[zhfonts][style=rm, size=10.5pt]
%\setupzhfonts[feature][onum=yes, pnum=yes]

%% use `mtxrun --script font --list --all` to list all fonts
% set base fonts
\setupzhfonts
[serif]
[regular=adobesongstd,
bold=adobeheitistd,
italic=adobekaitistd,
bolditalic=adobeheitistd]

\setupzhfonts
[sans]
[regular=adobefangsongstd,
bold=adobeheitistd,
italic=adobefangsongstd,
bolditalic=adobeheitistd]

\setupzhfonts
[mono]
[regular=adobekaitistd,
bold=adobeheitistd,
italic=adobekaitistd,
bolditalic=adobeheitistd]

% set latin fonts
\setupzhfonts
[latin, serif]
[regular=texgyrepagellaregular,
bold=texgyrepagellabold,
italic=texgyrepagellaitalic,
bolditalic=texgyrepagellabolditalic]

\setupzhfonts
[latin, mono]
[regular=texgyrecursorregular,
bold=texgyrecursorbold,
italic=texgyrecursoritalic,
bolditalic=texgyrecursorbolditalic]

\setupzhfonts
[latin, sans]
[regular=texgyreherosregular,
bold=texgyreherosbold,
italic=texgyreherositalic,
bolditalic=texgyreherosbolditalic]

% 启用中文断行
\setscript[hanzi]
\mainlanguage[cn]

\setuppapersize[A4]

\input moderncv

\def\entryspace{0.5em}
\def\cvname{倪庆亮}
\def\cvalias{Y\'i Q\`ing Li\`ang}
\def\cvaddress{浙江省杭州市文三西路}
\def\cvpersonalinfo{男,汉族,群众}
\def\cvbirthday{1982 年 10 月}
\def\email{niqingliang2003@tom.com}
\def\tel{+8613588371863}

\define[6]\cvworkexp{
\cventry{#1}{#2 \hskip 1em #3 \hskip 1em #4 \hskip 1em #5\par\parindent2em #6}
}
% ==================== Main text ====================>
\starttext
\placehead

% self profile
\subject{自我评价}
7 年工作经验(主要是嵌入式软件),具有扎实的技术功底、刻苦的钻研精神;处事稳重,工作踏实、认真,责任心强,思维缜密,逻辑严谨,善于独立思考,具有较强的分析解决问题的能力。

% work experience
\subject{工作经历}

\cvworkexp{2010 ~ 今}{浙大网新}{物联网事业部}{杭州}{}{
负责软件部基础设施的预研及建设,参与软件平台的整体设计,负责底层软件平台的设计开发,开源软件/项目的研究及使用,相关配套工具的设计开发,部分招聘和技术培训工作,另外作为 SA735 产品的软件开发代表负责其软件开发和版本发布管理等工作。

其中开发工作包括但不限于:
嵌入式 OS 构建系统 optimus(基于 yocto)。
GUI 编译工具:太极图(与 optimus 配套使用,用 wxpython 實現)。
嵌入式设备双系统冗馀备份方案的設計開發(包括配套的升级工具)。
软件平台中的进程执行模型、
驱动框架(主要用于实现驱动在用户态和内核态的无缝切换)、
进程及设备管理模块、
用于加载 FPGA 和 CPLD 的软件模块等。
以及 Linux下使用的 mm/md(对 uboot 中相应命令功能进行了扩展,以bash脚本实现)等。
用于生成 UBIFS image 的工具(bash脚本,自动构建时使用)。
p2020 和 mpc8315 相关 bsp 的调试开发。
}

\cvworkexp{2009 ~ 2010}{诺基亚西门子}{NGMGW}{杭州}{}{
带领 TDM 团队进行 TDM linecard 相关的需求分析、相关文档的撰写以及软件设计开发,其中包括 SDH、
PDH 的配置管理(包括告警、性能、APS 等),时钟管理,及相应 redundancy 的处理。
}

\cvworkexp{2007 ~ 2009}{UT 斯达康}{ONS}{杭州}{}{
主要是低阶交叉芯片的研究及驱动开发,低阶交叉板、混合光板的开发,热备份 / SSF / MAPS 等新功能的设计
开发,另外还有配置管理、业务、算法等模块的开发和维护等。
}

\cvworkexp{2006 ~ 2007}{霓虹灯设计与演示系统}{西安}{}{}{
负责网格编辑模块的设计实现。

包括动作管理器、存档序列化/反序列化、undo / redo 、各种移动/花样效果、调色盘、UI 设计等。
}

\cvworkexp{2005 ~ 2006}{港湾网络}{研发部交换产品线接入服务组}{北京}{}{
负责 Qos、Acl、Dhcp 模块的开发和维护、新芯片的功能测试及相应的软件设计开发工作。
}

% education
\subject{教育背景}

\cventry{2004 ~ 2007}{
{\bf 硕士} \hskip 1em {西安电子科技大学} \hskip 1em {西安市} \hskip 1em {计算机科学与技术}
}

\cventry{2000 ~ 2004}{
{\bf 学士} \hskip 1em {西安电子科技大学} \hskip 1em {西安市} \hskip 1em {计算机科学与技术}
}

% language
\subject{语言}

\cventry{英语}{六级}

\stoptext

