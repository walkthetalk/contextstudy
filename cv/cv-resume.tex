\usemodule[zhfonts][style=rm, size=10.5pt]
%\setupzhfonts[feature][onum=yes, pnum=yes]

%% use `mtxrun --script font --list --all` to list all fonts
% set base fonts
\setupzhfonts
[serif]
[regular=adobesongstd,
bold=adobeheitistd,
italic=adobekaitistd,
bolditalic=adobeheitistd]

\setupzhfonts
[sans]
[regular=adobefangsongstd,
bold=adobeheitistd,
italic=adobefangsongstd,
bolditalic=adobeheitistd]

\setupzhfonts
[mono]
[regular=adobekaitistd,
bold=adobeheitistd,
italic=adobekaitistd,
bolditalic=adobeheitistd]

% set latin fonts
\setupzhfonts
[latin, serif]
[regular=texgyrepagellaregular,
bold=texgyrepagellabold,
italic=texgyrepagellaitalic,
bolditalic=texgyrepagellabolditalic]

\setupzhfonts
[latin, mono]
[regular=texgyrecursorregular,
bold=texgyrecursorbold,
italic=texgyrecursoritalic,
bolditalic=texgyrecursorbolditalic]

\setupzhfonts
[latin, sans]
[regular=texgyreherosregular,
bold=texgyreherosbold,
italic=texgyreherositalic,
bolditalic=texgyreherosbolditalic]

% 启用中文断行
\setscript[hanzi]
\mainlanguage[cn]

\setuppapersize[A4]
\defineitemgroup[igBase][levels=1]
\setupitemgroup[igBase]
[1]
[packed,joinedup,]
[
  leftmargin=2em,	%no standard dimension
  before=,	%command
  inbetween=,	%command
  after=,	%command
]

\input moderncv

\def\entryspace{0.5em}
\def\cvname{倪庆亮}
\def\cvalias{Y\'i Q\`ing Li\`ang}
\def\cvaddress{浙江省杭州市文三西路}
\def\cvpersonalinfo{男,汉族,群众}
\def\cvbirthday{1982 年 10 月}
\def\cvgithub{https://github.com/walkthetalk}
\def\cvhomepage{https://niqingliang2003.wordpress.com/}
\def\email{niqingliang2003@tom.com}
\def\tel{+8613588371863}

\define[6]\cvworkexp{
\cventry{#1}{#2 \hskip 1em #3 \hskip 1em #4 \hskip 1em #5\par\parindent2em #6}
}
% ==================== Main text ====================>
\starttext
\placehead

% self profile
\subject{自我评价}
7 年工作经验(主要是嵌入式软件),具有扎实的技术功底、刻苦的钻研精神;
处事稳重,工作踏实、认真,责任心强,思维缜密,逻辑严谨,善于独立思考,
具有较强的分析解决问题的能力。

\subject{技能}

\cventry{编程语言}{
从开始工作至今使用的编程语言主要是 C / C++,对 Java / C\sharp 曾有过接触,
底层汇编语言相關调试工作也比較熟悉。
对于 bash / python / lua 等动态语言或脚本语言也有一定的使用经验。
另外对于面向对象、设计模式等相关技术也有较深的理解。
}

\cventry{操作系统}{日常使用的是 ArchLinux。
最近三年的开发工作主要集中在 Linux 上面,从 bootloader、驱动到上层应用都有所涉及。
另外 ftp / tftp / http / git 等类型的服务器也有一定的配置经验。}

\cventry{开发工具}{主要使用的开发工具为 gcc / gdb / makefile。
软件版本管理工具主要使用 git 和 hg。}

\cventry{应用软件}{虚拟机软件主要使用的是 virtualbox。
文档编辑目前主要使用 Con\TeX t,偶尔使用 LibreOffice。}

% work experience
\subject{工作经历}

\cvworkexp{2010 ~ 今}{浙大网新}{物联网事业部}{杭州}{高级软件工程师}{
负责软件部基础设施的预研及建设,参与软件平台的整体设计,负责底层软件平台的设计开发,
开源软件/项目的研究及使用,相关配套工具的设计开发,
部分招聘和技术培训工作,另外作为 SA735 产品的开发代表负责其软件开发和版本发布管理等工作。
还参与了 SN6500 产品的软件设计开发和数据中心的前期调研工作。

开发环境主要是 Linux 和 C/C++,其中开发工作包括但不限于:
\startigBase
\item 嵌入式 OS 构建系统 optimus(基于 yocto),
主要是为了解决产品分化的问题,将多个产品统一到同一个平台上,
最开始曾经试图用 LTIB 搭建,但由于 LTIB 自身固有问题效果不大理想,后来改用 yocto。
\item GUI 编译工具:太极图(与 optimus 配套使用,用 wxpython 实现),
主要是以图形化的方式显式软件包之间的依赖关系,并完成自动推导递归编译。
\item 嵌入式设备双系统冗余备份方案的设计开发(包括配套的升级工具)。
\item 软件平台中的进程执行模型、
驱动框架(主要用于实现驱动在用户态和内核态的无缝切换)、
进程及设备管理模块、
用于加载 FPGA 和 CPLD 的软件模块等。
\item 以及 Linux下使用的 mm/md(对 uboot 中相应命令功能进行了扩展,以bash脚本实现)等。
\item 用于生成 UBIFS image 的工具(bash脚本,自动构建时使用)。
\item p2020 和 mpc8315 (freescale / powerpc)相关 bsp 的调试开发。
\item \dots
\stopigBase
}

\cvworkexp{2009 ~ 2010}{诺基亚西门子}{NGMGW}{杭州}{高级软件工程师 & TDM 组长}{
带领 TDM 团队进行 TDM 相关业务的需求分析、相关文档的撰写以及软件设计开发,其中包括 SDH、
PDH 的配置管理(包括告警、性能、APS 等),时钟管理,及相应 redundancy 的处理。

开发环境主要是 Linux,编程语言为 C/C++ 与 TNSDL。
}

\cvworkexp{2007 ~ 2009}{UT 斯达康}{ONS}{杭州}{}{
主要是低阶交叉芯片的研究及驱动开发,低阶交叉板、混合光板的开发,
热备份 / SSF / MAPS 等新功能的设计开发,
另外还有配置管理、业务、算法等模块的开发和维护等。

开发环境主要是 windows/vxworks + C/C++。
}

\cvworkexp{2006 ~ 2007}{霓虹灯设计与演示系统}{西安}{}{}{
负责网格编辑模块的设计实现。

包括动作管理器、存档序列化/反序列化、undo / redo 、各种移动/花样效果、调色盘、UI 设计等。

开发环境为 windows + VC。
}

\cvworkexp{2005 ~ 2006}{港湾网络}{研发部交换产品线接入服务组}{北京}{}{
负责BH68系列交换机的软件研发工作。

负责QoS、Acl模块的相关工作。
包括:代码维护,包括上层软件和芯片Qos驱动;
新需求的开发,如NP板对MassACL的支持,QoS、Acl对IPv6的支持等;
新芯片(broadcom / firebolt easyridar)功能测试,
并为使QoS、Acl模块支持新芯片进行了详细设计。
另外还负责DHCP模块的维护优化。

开发环境主要是 windows/vxworks + C/C++。
}

% education
\subject{教育背景}

\cventry{2004 ~ 2007}{
{\bf 硕士} \hskip 1em {西安电子科技大学} \hskip 1em {西安市} \hskip 1em {计算机科学与技术}
}

\cventry{2000 ~ 2004}{
{\bf 学士} \hskip 1em {西安电子科技大学} \hskip 1em {西安市} \hskip 1em {计算机科学与技术}
}

% language
\subject{语言}

\cventry{英语}{六级}

\stoptext

