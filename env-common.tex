% 拷贝自glib-notes,将默认中文字体改为了adobe的字体。

\startenvironment env-common
% C 代码高亮抄录模块
% \usemodule[pretty-c]

% 中文字体模块,支持中西文字体设置与标点微排版
% 下载地址: http://github.com/liyanrui/t-zhfonts
\usemodule[zhfonts][style=rm, size=11pt]
\setupzhfonts[feature][onum=yes, pnum=yes]

\setupzhfonts
[serif]
[regular=adobesongstd,
bold=adobeheitistd,
italic=adobesongstd,
bolditalic=adobeheitistd]

\setupzhfonts
[mono]
[regular=adobekaitistd,
bold=adobeheitistd,
italic=adobekaitistd,
bolditalic=adobeheitistd]

\setupzhfonts
[sans]
[regular=wenquanyizenhei,%adobefangsongstd,
bold=adobeheitistd,
italic=wenquanyizenhei,%adobefangsongstd,
bolditalic=adobeheitistd]

\setupzhfonts[latin, mono][regular=Monaco]

% ascii 数学公式模块
\usemodule[asciimath]

\setupinteraction[state=start,
  color=darkblue,
  contrastcolor=darkblue,
  focus=standard]

% 适合 A4 纸张打印的版式
\setuplayout[backspace=25mm,
             width=135mm,
             rightmargin=30mm,
             topspace=20mm,
             header=15mm,
             footer=10mm,
             height=260mm,]
\setuppagenumbering[alternative=doublesided]

% 封面 layout
\newdimen\AppendicesTextWidth
\AppendicesTextWidth=\textwidth
\advance\AppendicesTextWidth by\rightmarginwidth
\definelayout[normal][width=\AppendicesTextWidth]

% 页眉与页脚
\setuppagenumbering[location=]

\newdimen\headerwidth
\headerwidth=\the\textwidth
\advance\headerwidth by \rightmarginwidth
\newdimen\LineWidth
\LineWidth=2pt

\def\HeaderFrame#1{%
  \framed[width={\headerwidth}, 
          frame=off,
          offset=none,
          bottomframe=on, 
          framecolor=darkgray,
          rulethickness=\LineWidth]{\ss#1}}
\def\PageFrame{%
  \inframed[width=14mm, 
            height=6mm,
            frame=off,
            offset=0pt,
            framecolor=darkgray,
            background=color, 
            backgroundcolor=darkgray]{\color[white]{\ssx\pagenumber}}}
\def\HeadStr#1{\headnumber[#1]\hskip1em\getmarking[#1]}
\def\RightHeader{\HeaderFrame{\HeadStr{section}\hfill\PageFrame\hbox to -\LineWidth{}}}
\def\LeftHeader{\HeaderFrame{\hbox to -\LineWidth{}\PageFrame\hfill\HeadStr{chapter}}}

\def\FooterFrame#1{%
  \framed[width={\headerwidth}, 
          frame=off,
          offset=0pt]{#1}}
\def\BookNameFrame[#1]{%
  \framed[width=fit, 
            height=fit,
            frame=off,
            offset=2pt,
            background=color,
            backgroundcolor=darkgray]{%
    \color[white]{\ss \rotate[rotation=#1]{\docname}}}}
\def\RightFooter{\FooterFrame{\hfill\BookNameFrame[-90]}}
\def\LeftFooter{\FooterFrame{\BookNameFrame[90]\hfill}}

\startsetups Text
\setupheadertexts[text][\RightHeader][][][\LeftHeader]
\setupfootertexts[text][\RightFooter][][][\LeftFooter]
\stopsetups

\startsetups Appendices
\def\RightAppendicesHeader{\HeaderFrame{\ssx\getmarking[title]\hfill%
                                        \PageFrame\hbox to -\LineWidth{}}}
\def\LeftAppendicesHeader{\HeaderFrame{\hbox to -\LineWidth{}\PageFrame\hfill\ssx 附录}}
\setupheadertexts[text][\RightAppendicesHeader][][][\LeftAppendicesHeader]
\stopsetups

\startsetups Empty
\setupheadertexts[text][][][][]
\setupfootertexts[text][][][][]
\stopsetups

% 标题
\setupheads[indentnext=yes]
\definepagebreak[headpagebreak][yes, header, footer, odd]
\setuphead[chapter,title][header=empty, style=\ssc, page=headpagebreak,]
\def\HeadOffset{\hbox to -3cm{}}
\def\chaptercmd#1#2{\hbox to \hsize{#2\hfill\switchtobodyfont[48pt]{%
                                    \color[darkgray]{#1}}\HeadOffset}}
\setuphead[chapter][command=\chaptercmd,after={\blank[2cm]}]

\def\ContentTitle#1{%
  \inframed[width=fit, 
            height=fit,
            frame=off,
            offset=4pt,
            loffset=10mm,
            roffset=10mm,
            framecolor=darkgray,
            background=color, 
            backgroundcolor=darkgray]{\color[white]{#1}}}
\def\titlecmd#1#2{\hbox to \hsize{\hfill%
                        \ContentTitle{#2}\HeadOffset}}
\setuphead[title][command=\titlecmd,after={\blank[2cm]}]
\setuphead[section, subject][style=\ssa]

% 段落
\setupindenting[first,always,2em]
\setupinterlinespace[line=1.5em]

\setupfootnotes[bodyfont=9pt]

% 抄录
\setuplinenumbering[location=inmargin,color=darkgreen]
\setuptype[style=\ttx]
\setuptyping[bodyfont=9pt, numbering=line, before=\blank, after=\blank]
\setupitemize[paragraph, packed, broad]

% 目录
\setupcombinedlist[content][alternative=c,interaction=all]
\setuplist[section][margin=2em,]

% 封面
\defineoverlay[CruxOrnament][\useMPgraphic{crux}]
\def\CruxFramed#1{\framed[frame=off,width=12cm,height=fit]{#1}}

\startuseMPgraphic{crux}
color ccc ;
pair p, h[], v[] ;
u := \overlaywidth ; v := \overlayheight ;
hdelta := .15u ; vdelta := .03v ;
drawoptions (withpen pensquare scaled 2pt) ;
randomseed := day + time*epsilon ;
show time * epsilon ;
for i :=1 upto 128 :
    ccc := (uniformdeviate (1), uniformdeviate (1), uniformdeviate (1)) ;
    p := (uniformdeviate (u), uniformdeviate (v)) ;
    h0 := (xpart (p) - uniformdeviate (hdelta) , ypart (p)) ;
    h1 := (xpart (p) + uniformdeviate (hdelta) , ypart (p)) ;
    v0 := (xpart (p), ypart (p) - uniformdeviate (vdelta)) ;
    v1 := (xpart (p), ypart (p) + uniformdeviate (vdelta)) ;
    draw h0 -- h1 withcolor transparent(1,.4,ccc) ;
    draw v0 -- v1 withcolor transparent(1,.4,ccc) ;
endfor ;
\stopuseMPgraphic

\startsetups BG
\defineoverlay[bg][\useMPgraphic{crux}]
\setupbackgrounds[page][background=bg]
\stopsetups
\definestartstop[BG][commands=\setups{BG}]

% misc
\setupheadtext[en][pubs=参考文献]
\setupheadtext[en][content=目录]
\setupheadtext[en][index=索引]
\setuplabeltext[en][figure=图\;]
\setuplabeltext[en][table=表\;]
\setupcaptions[style=\tfx,headstyle=\normal]

%%%%%%%%%%%%%%%   字体
\define\ftref{\rm\it}
\define\ftemp{\rm\bf}
\definetype[code][
  style=\tt\tf,
]
%%%%%%%%%%%%%%%   枚举
% igBase
\defineitemgroup[igBase][levels=1]
\setupitemgroup[igBase][each]
[packed,joinedup,]
[
  leftmargin=2em,
  before=,
  inbetween=,
  after=,
]
\setupitemgroup[igBase][1][1]
%%%%%%%%%%%%%%  例子
\defineenumeration[example][
  text=例,
  location=inleft,
  width=fit,
  headstyle=bold,
  headcolor=darkgreen,
  prefix=yes,
  prefixsegments=chapter,
  right={.~},
]
\setupnumber[example][way=bychapter]
%%%%%%%%%%%%%%   代码
\definetextbackground
  [verb]
  [frame=on,location=paragraph,
   before=,after=,
   leftoffset=0.1em,rightoffset=0.1em,
   topoffset=0.1em,bottomoffset=0.1em,
   rulethickness=0.75pt]
\defineframedtext[framecode][
  width=fit,offset=0pt,background=color,backgroundcolor=lightgray,
]
\definetyping[bb][%option=c,
  bodyfont=8pt,
%  before={\startframecode\startlinenumbering},
  before={\blank[0.5em]\starttextbackground[verb]},
%  after={\stoplinenumbering\stopframecode},
  after={\stoptextbackground\blank[0.5em]},
  margin=no,
  tab=8,
  numbering=line,
]

\definetyping[ccode][%option=c,
  bodyfont=8pt,
%  before={\startframecode\startlinenumbering},
  before={\blank[0.5em]\starttextbackground[verb]},
%  after={\stoplinenumbering\stopframecode},
  after={\stoptextbackground\blank[0.5em]},
  margin=no,
  tab=8,
  numbering=line,
]

% 数学公式自动设置标点间距
\setupmathematics[autopunctuation=no]

% 开启 ascii 模式 % 如果asciimode后面有%开始的注释,则会导致多出两页封面
\asciimode

\stopenvironment
