%
% author:	Ni Qingliang
% date:		2012-10-11
%
\startcomponent cpn-en
\product examination
%\startEnglish
\setscript[no]
\mainlanguage[en]
\setupindenting[no]%
%\setupwhitespace[line]%
%\setupinterlinespace[medium]

\startcolumns[n=2]
{\ftEmp
\startlines
姓名:\AI{填写姓名处}
\vskip.5em
学校:\AI{填写毕业学校处}
联系方式:\AI{填写联系方式处}
\vskip.5em
专业:\AI{填写所学专业处}
\stoplines
}
\stopcolumns
{\ftEmp{注意:可以用中文作答!!!}}

\startQ
According to the requirement of software design,
the feature that continuing to run without failure
in the prescribed time and under specific conditions is called (\AI{B}). (5\%)
\startcolumns[n=4,blank=small]
\startCASES
\item availability
\item reliability
\item correctness
\item integrity
\stopCASES
\stopcolumns
\stopQ

\startQ
Which of the following mechanism CANNOT be used as inter-process communication way
in the GNU/Linux operation system? (5\%) (\AI{C})
\startcolumns[blank=small]
\startCASES
\item Pipe/Message Queue
\item Share Memory
\item Global Variable
\item Berkeley Sockets
\stopCASES
\stopcolumns
\stopQ

\startQ
{\ftEmp{FIFO}} algorithm and {\ftEmp{LRU}} algorithm are two typical page replacement algorithm
when a page fault occurs.
A process has ownership of three memory pages,
think about the page access sequence of this processis 1321215123,
if {\ftEmp{a}} is the fault number of use FIFO as page replacement algorithm,
and {\ftEmp{b}} is the fault number of use LRU as page replacement algorithm,
what is the correct value of a:b? (5\%) (\AI{B})
\startcolumns[blank=small]
\startCASES
\item 4:3
\item 3:2
\item 1:1
\item 2:1
\stopCASES
\stopcolumns
\stopQ

\startQ
When storing data 7,34,55,25,64,46,20,10,
if use $H(K)=K \% 9$ as hash function,
then how many elements are there whose hash address is 1? (5\%) (\AI{D})
\startcolumns[n=4,blank=small]
\startCASES
\item 1
\item 2
\item 3
\item 4
\stopCASES
\stopcolumns
\stopQ

\startQ
The {\ftEmp{POSSIBLE}} result of following code is (\AI{C}). (10\%)
\startC[numbering=line]
char c;
int v0=1, v1=0, v2=0;

do {
	switch (c = getchar()) {
		case 'a': case 'A':
		case 'e': case 'E':
		case 'i': case 'I':
		case 'o': case 'O':
		case 'u': case 'U': v1 += 1;
		default: v0 += 1; v2 += 1;
	}
} while (c != '\n');

printf("v0=%d, v1=%d, v2=%d\n",v0,v1,v2);
\stopC
\startcolumns[n=2,blank=small,distance=0.5em]
\startCASES
\item \ccmm{v0=8, v1=7, v2=7}
\item \ccmm{v0=7, v1=7, v2=8}
\item \ccmm{v0=8, v1=6, v2=7}
\item \ccmm{v0=7, v1=8, v2=6}
\stopCASES
\stopcolumns
\stopQ

\startQ
Please fill in the missed statements of the following binary-search function.
({\ftEmp{hint: recurrence}})
\startC[numbering=line]
struct node_t {
	int8_t	data;
	node_t	*left;
	node_t	*right;
};

bool bin_search(node_t *pNode, int8_t item)
{
	if (pNode == nullptr) {
		return false;
	} else {
		if (item == pNode->data) {
			return /BTEX\AI{true}/ETEX;		/BTEX(1\%)/ETEX
		} else if( item < pNode->data) {
			return  bin_search(/BTEX\AI{pNode->left}/ETEX, item);	/BTEX(2\%)/ETEX
		} else {
			return bin_search(/BTEX\AI{pNode->right}/ETEX, item);	/BTEX(2\%)/ETEX
		}
	}
}
\stopC
\stopQ

\startQ
The function of following function is: \startAI计算 \ccmm{x} 中比特 1 的个数\stopAI。(5\%)
\startC
uint8_t anonymous_func(uint64_t x)
{
	uint8_t ret = 0;

	while (x) {
		++ret;
		x &= x - 1;
	};

	return ret;
}
\stopC
\stopQ

\startQ
When following code section running on 32 bit CPU,
the value of
\ccmm{sizeof(name)}, \ccmm{sizeof(str)},
\ccmm{strlen(str)} and \ccmm{sizeof(p)} are: (\AI{B}). (5\%)
\startC
void func(char name[5]);
char str[] = "hello";
char *p = str;
\stopC
\startcolumns[n=2,blank=small]
\startCASES
\item \ccmm{4}, \ccmm{5}, \ccmm{5}, \ccmm{4}
\item \ccmm{4}, \ccmm{6}, \ccmm{5}, \ccmm{4}
\item \ccmm{5}, \ccmm{5}, \ccmm{5}, \ccmm{4}
\item \ccmm{5}, \ccmm{6}, \ccmm{5}, \ccmm{4}
\stopCASES
\stopcolumns
\stopQ

\startQ
As shown in the following figure,
there are two sites located on both sides of the river,
and the two banks(河岸) of the river parallel to each other.
Now we want to build a road between the two sites,
and the bridge crossing the river must be perpendicular(垂直) to the river bank.
How to build the road to make it shortest? ({\ftEmp{hint: shifting}}) (10\%)

\startreusableMPgraphic{shortestRoad}
%pen penCmm;
%color memColor;

u := 4mm;
v := 4mm;
%ahangle := 30;
%ahlength := .5v;
%penCmm := pencircle scaled 2;
%memColor := (1,0.93,0.98);

% ori pic data
pair pA, pB;
numeric hB;%half of bridge
pA := (5u,8v);
pB := (15u,-7v);
hB := 2v;
path pathT; % top boundary
pathT := (0u, hB)--(20u, hB);
path pathB; % bottom boundary
pathB := (0u,-hB)--(20u,-hB);

% 原图
def picOri :=
image(
dotlabel.lft(btex \mplabel{A} etex, pA);
draw pathT;
draw pathB;
dotlabel.rt(btex \mplabel{B} etex, pB);
)
enddef;

%%%%%%%%%%%%%%%%%%%%%%%%%%%%%% answer
def colorAnswer :=
transparent(1, 1, \MPcolor{COLORA})
enddef;
def styAux :=
dashed evenly withcolor colorAnswer
enddef;
def styAnswer :=
withcolor colorAnswer
enddef;
def styLbl :=
withcolor colorAnswer
enddef;

% step 1: 过 A 向河岸作垂线
draw pA--(xpart pA, -hB) styAux;

pair pointC;
pointC := (xpart pA, ypart pA - 2hB);
dotlabel.lft(btex \mplabel{C} etex, pointC) styLbl;
path pathAux;
pathAux := pointC--pB;
draw pathAux styAux;

% bridge
pair pointD;
pointD := pathAux intersectionpoint pathB;
dotlabel.llft(btex \mplabel{D} etex, pointD) styLbl;
pair pointE;
pointE := (xpart pointD, hB);
dotlabel.llft(btex \mplabel{E} etex, pointE) styLbl;

path pathBridge;
pathBridge := pointD--pointE;
path pathAR;
pathAR := pA--pointE;
path pathBR;
pathBR := pointD--pB;

draw pathAR--pathBridge--pathBR styAnswer;

%%%%%%%%%%%%%%%%%%%%%%%%%%%%%%%%%%%%%%
draw picOri;

\stopreusableMPgraphic

\reuseMPgraphic{shortestRoad}
\stopQ

\startQ
Please write out the access specifiers in C++,
and explain the purpose of everyone respectively. (10\%)
\stopQ
\startA
常用的 \ckey{public}、 \ckey{protected}、 \ckey{private}。
还有一个不常用的 \ckey{friend}。
\startLIST
\item \ckey{public} 即公共的,任 何人都能访问;
\item \ckey{protected} 即只有自己和子类可以访问;
\item \ckey{private} 即只有自己才能访问。
\stopLIST
\stopA

\startQ
Please define a macro to return the smaller one of two variables. (5\%)
\stopQ
\startA
\startC
#define MIN(x, y) ((x) < (y) ? (x) : (y))
\stopC
\stopA

\startQ
The output of following program is ({\AI{B}}). (10\%)
\startcolumns[rule=on,blank=small]
\typefile[C]{src/cpp.cpp}
\stopcolumns
\startcolumns[blank=small]
\startCASES
\item 1468
\item 1468752
\item 14638752
\item 4168752
\stopCASES
\stopcolumns
\stopQ

\startQ
what is the least number of pound weights (砝码) that can be used
on a scale pan (秤盘) to weigh {\ftEmp{ANY}} integral number of pounds
from 1 to 40 inclusive,
if the weights can be placed in either of the scale pans?
({\ftEmp{Using Mathematical Induction to get the recurrence formula}}) (10\%)
\stopQ
\startA
\startformula \startalign
\NC a_{n + 1} \NC = 2S_n + 1 \NR
\NC a_{n + 1} \NC = 3a_n \NR
\stopalign \stopformula

四个砝码就够了: 1、 3、 9、 27。

本题关键在于递推式。
\stopA

\startQ
Write a function to calculate the following formula: (10\%)
\startformula
1 + 2 - 3 + 4 - 5 + 6 - ... n
\stopformula
\stopQ
\startA
\startC
int sum(int n)
{
	if (n <= 0) {
		return -1;
	}

	if (n % 2) {
		return ((3 - n)/2);
	} else {
		return ((n + 4)/2);
	}
}
\stopC
\stopA

%\stopEnglish
\stopcomponent
