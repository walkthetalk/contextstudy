%
% author:	Ni Qingliang
% date:		2011-02-11
%
\startcomponent cpn-main
\product examination

%\chapter{引言}

\startQ
以下是 ARM 平台上的 32 位 C 程序,请计算出 \ccmm{sizeof} 的数值:

\startcolumns[n=3,distance=0.5em]
\startCASES[paragraph,intext]
\startitem
\startC
char str[] = "Hello";
char*p = str;
int n = 100;
sizeof(str) = /BTEX{\BLANK{\setA6}}/ETEX;
sizeof(p) = /BTEX{\BLANK{\setA4}}/ETEX;
sizeof(n) = /BTEX{\BLANK{\setA4}}/ETEX;
\stopC
\stopitem

\column

\startitem
\startC
void Func(char str[100])
{
	sizeof(str) = /BTEX{\BLANK{\setA4}}/ETEX;
}
\stopC
\stopitem

\column

\startitem
\startC
void*p = malloc(100);
sizeof(p) = /BTEX{\BLANK{\setA4}}/ETEX;
\stopC
\stopitem
\stopCASES
\stopcolumns
\stopQ

\startQ
\startCASES
\startitem
对于 32 位整形变量 \ccmm{A=0X12345678},
请画出在 little endian 及 big endian 的方式下在内存中是如何存储的。
\stopitem
\startitem
写一个简单检测处理器大小端的函数。
\stopitem
\stopCASES
\stopQ
% answer
\startA
\startcolumns[n=2,distance=0.5em,blank=small]
\startCASES[paragraph,intext]
\startitem
存储方式如下:

\midaligned{
\bTABLE[frame=off,split=off,style=\tt\tfx]
\bTR
 \bTD[nr=2,style=\tfb,align={lohi}] L: \eTD
 \bTD 偏移字節數 \math{\rightarrow} \eTD
 \bTD 0 \eTD \bTD 1 \eTD \bTD 2 \eTD \bTD 3 \eTD
\eTR
\bTR[align={middle,lohi}]
\bTD 内存中的值 \math{\rightarrow} \eTD
\bTD[frame=on,width=.1\textwidth] 0x78 \eTD
\bTD[frame=on,width=.1\textwidth] 0x56 \eTD
\bTD[frame=on,width=.1\textwidth] 0x34 \eTD
\bTD[frame=on,width=.1\textwidth] 0x12 \eTD
\eTR
\eTABLE
}

\midaligned{
\bTABLE[frame=off,split=off,style=\tt\tfx]
\bTR
 \bTD[nr=2,style=\tfb,align={lohi}] B: \eTD
 \bTD 偏移字節數 \math{\rightarrow} \eTD
 \bTD 0 \eTD \bTD 1 \eTD \bTD 2 \eTD \bTD 3 \eTD
\eTR
\bTR[align={middle,lohi}]
\bTD 内存中的值 \math{\rightarrow} \eTD
\bTD[frame=on,width=.1\textwidth] 0x12 \eTD
\bTD[frame=on,width=.1\textwidth] 0x34 \eTD
\bTD[frame=on,width=.1\textwidth] 0x56 \eTD
\bTD[frame=on,width=.1\textwidth] 0x78 \eTD
\eTR
\eTABLE
}
\stopitem

\column

\startitem
示例程序如下所示,
返回 \ccmm{false} 则为 big endian;
否则为 little endian:

\startC
bool is_little_endian(void)
{
	union {
		int a;
		char b;
	} c;
	c.a = 1;
	return (c.b == 1);
}
\stopC
\stopitem
\stopCASES
\stopcolumns
\stopA

\startQ
头文件中的 \cdrt{#ifndef} / \cdrt{#define} / \cdrt{#endif} 的作用是什么?
Linux 设备驱动按功能分为哪几类?
\stopQ
\startA
头文件中的 \cdrt{#ifndef} / \cdrt{#define} / \cdrt{#endif} 的作用是为防止重复包含同一头文件。

Linux 设备驱动主要分为字符设备驱动、块设备驱动、网络设备驱动。
\stopA

\startQ
什么情况下需要 \ckey{static} 修饰符?
\stopQ
\startA
除明确标识出变量的生命周期(英语:Object lifetime)外,将变量声明为static存储类还会根据变量属性不同而有一些特殊的作用:
\startCASES[margin=2em]
\item 对于静态全局变量来说,
针对某一源文件的以 static 声明的文件级变量与函数的作用域只限于文件内(只在文件内可见),
也即“内部连接”,因而可以用来限定变量的作用域;
\item 对于静态局部变量来说,
在函数内以static声明的变量虽然与自动局部变量的作用域相同(即作用域都只限于函数内),
但存储空间是以静态分配而非默认的自动分配方式获取的,
因而存储空间所在区域不同(一般来说,静态分配时存储空间于编译时在程序数据段分配,一次分配全程有效;
而自动分配时存储空间则是于调用栈上分配,只在调用时分配与释放),且两次调用间变量值始终保持一致;
\item 对于静态成员变量(英语:Member variable)来说,
在 C++ 中,在类的定义中以 static 声明的成员变量属于类变量(英语:Class variable),
也即在所有类实例中共享,与之相对的就是过程变量(英语:Instance variable)。
\stopCASES
\stopA

\startQ
请阅读下面一段代码,若有错误,指出是何错误并更正:
\startC[numbering=line]
class A {
public:
	void Func(void)		/BTEX\setA//不修改成员变量的成员函数应该加上 \ccmm{const},不会访问非 \ccmm{static} 成员的函数应该加上 \ccmm{static}/ETEX
	{
		cout << "Func of class A" << endl;	/BTEX\setA//没有包含头文件,另外还需要 \ccmm{using namespace std;}/ETEX
	}
};

void main()		/BTEX\setA//\ccmm{main} 的返回值类型应该是 \ccmm{int}/ETEX
{
	A * pa;					/BTEX\setA//没有初始化/ETEX
	char * p = (char *)malloc(100);		/BTEX\setA//申请内存后一般应该检查成功与否,但是由于 OS 的原因检查可能无效/ETEX
	strcpy(p, "Hello");			/BTEX\setA//最好使用带 \ccmm{n} 的字串函数族,与 \ccmm{strcpy} 对应的是 \ccmm{strncpy}/ETEX
	free(p);				/BTEX\setA//如果释放内存后还需要使用此指针,则应将其置为 \ccmm{NULL}/ETEX
	if (p != NULL)				/BTEX\setA//在 \ccmm{if}、 \ccmm{for}、 \ccmm{while} 等语句块中,即使只有一条语句,也最好加上大括号/ETEX
		strcpy(p, "World");
	{
		A a;
		pa = &a;
	}
	pa->Func();	/BTEX\setA//此处 \ccmm{pa} 所指向的对象 \ccmm{a} 已经析构,不存在了/ETEX
}
\stopC
\stopQ

\startQ
请说明什么叫做指针函数,什么叫做函数指针,并写出各自的函数原型。
\stopQ
\startA
\startcolumns[n=2,blank=small]
{\ftEmp{指针函数}}是一种函数,其返回值的类型是一种指针。如:
\startC
int * f(void);
\stopC
\column
{\ftEmp{函数指针}}是一种指针,指向具有特定特征的函数。如:
\startC
int (* p)(void);
\stopC
\stopcolumns
\stopA

\startQ
为方便嵌入式系统中对于变量或者寄存器进行位操作,
现给定一整型变量 a,请写两个函数,
第一个设置 a 的第 5 位为 1,
第二个清除 a 的第 5 位为 0,
上述两个操作要求保持其他位不变。
\stopQ
\startA
\startC
#define BIT(n) (1 << (n))

void set_bf(int * p)
{
	*p |= BIT(5);
}

void clr_bf(int * p)
{
	*p &= ~BIT(5);
}
\stopC
\stopA

\startQ
填空题:
\startCASES
\item 在 Linux 系统下,
第二个 IDE 通道的硬盘(从盘)被标识为\BLANK{\setA\ccmm{/dev/sdb}}。
\item 在 Linux 操作系统中,设备都是通过\BLANK{\setA设备文件}来访问。
\item Linux 内核引导时,从文件\BLANK{\setA\ccmm{/etc/fstab}}中读取要加载的文件系统。
\item 在 Linux 系统中,用来存放系统所需要的配置文件和子目录的目录是\BLANK{\setA\ccmm{/etc}},
目录\BLANK{\setA\ccmm{/sbin}}用来存放系统管理员使用的管理程序。
\item 某文件的权限为 \ccmm{drw-r--r--},用数值形式表示该权限,
则该八进制数为\BLANK{\setA\ccmm{1644}},该文件属性是\BLANK{\setA目录}。
\item Linux 内核分为\BLANK{\setA进程管理、内存管理、虚拟文件系统、网络接口}等四个子系统。
\item 进程与程序的区别在于其动态性,动态的产生和终止,
从产生到终止进程可以具有的基本形态为\BLANK{\setA运行}和\BLANK{\setA睡眠}。
\item 为脚本程序指定执行权的命令及参数是\BLANK{\setA\ccmm{chmod +x}}。
\item 将 \ccmm{/opt/QT4.5/build/arm} 目录做归档压缩,
压缩后生成 \ccmm{arm.tar.gz} 文件,
并将此文件保存到 \ccmm{/home/yhm} 目录下,
实现此任务的 tar 命令为\BLANK{\setA\ccmm{tar -czf /home/yhm/arm.tar.gz /opt/QT4.5/build/arm}}。
\stopCASES
\stopQ

\startQ
有一个 HelloWorld 项目包括如下 3 个文件:

\startcolumns[n=2,blank=small]
\cemp{fun.h}:
\typefile{./src/main/fun.h}
\cemp{main.c}:
\typefile{./src/main/main.c}
\cemp{fun.c}:
\typefile{./src/main/fun.c}
\stopcolumns
请编写一个编译该项目的 Makefile 文件,
生成的目标程序 \ccmm{main} 运行在嵌入式开发板上,
不需使用自动变量、变量替换、隐形规则等,书写最原始 Makefile 文件即可。
\stopQ
\startA
\typefile{./src/main/Makefile.mf}
\stopA

\startQ
设置 PLL 的作用是什么,
请简述 armLinux 系统中 u-boot 启动的两个阶段,
各阶段所对应的主要文件及其启动流程。
\stopQ
\startA
{\ftEmp{PLL}} 就是锁相环路,简称锁相环,是一种反馈控制电路。
许多电子设备要正常工作,通常需要外部的输入信号与内部的振荡信号同步,利用锁相环路就可以实现这个目的。

u-boot 启动的两个阶段:
\startLIST
\item 第一阶段是依赖于 CPU 体系结构的代码(如设备初始化代码等),一般用汇编语言来实现。
主要进行以下方面的设置:设置 ARM 进入 SVC 模式、禁止 IRQ 和 FIQ、关闭看门狗、屏蔽所有中断。
设置时钟(FCLK、 HCLK、 PCLK)、清空 I/D cache、清空 TLB、禁止 MMU 和 cache、
配置内存控制器、为搬运代码做准备、搬移 uboot 映像到 RAM 中(使用 copy_loop 实现)、
分配堆栈、清空 bss 段(使用 clbss_l 实现)。
\item 第二阶段通常用 C 语言来实现。
一系列初始化(cpu、板卡、中断、串口、控制台等),开启 I/D cache。
初始化 FLASH,根据系统配置执行其他初始化操作。
打印 LOG,使能中断,获取环境变量,初始化网卡。
最后进入 main_loop 函数。在 main_loop 函数中会检查 bootdelay 和 bootcmd 环境变量,
如果 bootcmd 已经设置过,则在等待 bootdelay 个毫秒后会自动执行 bootcmd。
如果等待过程中被用户中断或者没有设置 bootcmd,则会等待用户输入命令。
\stopLIST
\stopA

\stopcomponent
