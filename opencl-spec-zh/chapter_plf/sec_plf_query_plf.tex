\section{查询平台信息}
可以使用下面函数获取可用平台的清单:
\startclc
cl_int clGetPlatformIDs(cl_uint num_entries,
			cl_platform_id *platforms,
			cl_uint *num_platforms)
\stopclc
\carg{num_entries}是\carg{platforms}中可以容纳\ctype{cl_platform_id}表项的数目。如果\carg{platforms}不是\cenum{NULL},\carg{num_entries}必须大于0。

\carg{platforms}会返回所找到的\scopencl平台的清单。\carg{platforms}中的\ctype{cl_platform_id}可以用来标识一个特定的\scopencl平台。如果\carg{platforms}是\cenum{NULL},则被忽略。所返回的\scopencl平台的数目是\carg{num_entries}和实际可用数目中较小的那个。

\carg{num_platforms}返回实际可用的\scopencl平台的数目。如果\carg{num_platforms}是\cenum{NULL},则被忽略。

如果执行成功,\capi{clGetPlatformIDs}会返回\cenum{CL_SUCCESS}。否则,会返回下列错误之一:
\startigBase
\item \cenum{CL_INVALID_VALUE},如果\carg{num_entries}等于零,且\carg{platforms}不是\cenum{NULL},或者\carg{num_platforms}和\carg{platforms}都是\cenum{NULL}。
\item \cenum{CL_OUT_OF_HOST_MEMORY},如果\schostfailres。
\stopigBase

可以使用如下函数获取\scopencl平台的特定信息。这些信息如表4.1所示。
\startclc
cl_int clGetPlatformInfo(
		cl_platform_id platform,
		cl_platform_info param_name,
		size_t param_value_size,
		void *param_value,
		size_t *param_value_size_ret)
\stopclc

\carg{platform}即平台ID,指明要查询哪个平台,或者也可能是\cenum{NULL}。如果是\cenum{NULL},其行为\cnglo{impdef}。

\carg{param_name}是一个枚举常量,指明要查询什么信息。其值如\reftab{plfquery}所示。

\carg{param_value}是一个指针,指向一块内存,用来存储所返回的信息。如果是\cenum{NULL},则忽略。

\carg{param_value_size}指明\carg{param_value}所指内存块的大小(单位:字节)。其值必须>=\reftab{plfquery}中返回类型的大小。

\carg{param_value_size_ret}会返回所查询信息的实际大小。如果是\cenum{NULL},则忽略。

\cltable
{\placetable[here,force][tab:plfquery]{\scopencl平台查询}}
{\startETD[cl_platform_info][
返回型別
\footnote{對於 OpenCL 查詢函式,如果所查詢資訊的型別為 \ctype{char[]},則會返回一個以 null 結尾的字串。}
]

\clETD{CL_PLATFORM_PROFILE}{char[]}{
  OpenCL 規格字串。
  返回實作所支持的規格的名字。
  可以是下列之一:
  \startigBase
  \item FULL\_PROFILE——表示實作支持 OpenCL 規範(核心規範所定義的功能,無須支持任何擴展)。
  \item EMBEDDED\_PROFILE——表示實作支持 OpenCL 嵌入式規格。他是對應版本 OpenCL 的一個子集。
    OpenCL \scver 的嵌入式規格請參考\todo{secion 10}。
  \stopigBase
}

\clETD{CL_PLATFORM_VERSION}{char[]}{
  OpenCL 版本字串。返回所支持的 OpenCL 版本。其格式如下:

  {\ftRef OpenCL<space><major\_version.minor\_version><space><platform-specific information>}

  所返回的 {\ftRef major\_version.minor\_version} 將是 \scver。
}

\clETD{CL_PLATFORM_NAME}{char[]}{
  平台名字。
}

\clETD{CL_PLATFORM_VENDOR}{char[]}{
  平台供應商的名字。
}

\clETD{CL_PLATFORM_EXTENSIONS}{char[]}{
  返回一個清單,列出平台所支持的擴展名,以空格分隔(擴展名本身不包含空格)。
  此平台相關的所有\cnglo{device}都要支持此處定義的擴展。
}

\stopETD

}

如果执行成功,\capi{clGetPlatformInfo}会返回\cenum{CL_SUCCESS}。否则,返回下列错误之一\footnote{\scopencl规范没有规定API调用返回错误码时的优先顺序。}:
\startigBase
\item \cenum{CL_INVALID_PLATFORM},如果\carg{platform}无效。
\item \cenum{CL_INVALID_VALUE},如果\carg{param_name}不被支持,或者\carg{param_value_size}<\reftab{plfquery}中返回值的大小且\carg{param_value}不是\cenum{NULL}。
\item \cenum{CL_OUT_OF_HOST_MEMORY},如果\schostfailres。
\stopigBase
