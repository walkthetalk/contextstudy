\startETD[cl_platform_info][
對於 OpenCL 查詢函數,如果所查詢的信息類型是 \ctype{char[]},則會返回一個以 null 結尾的字符串。
]

\clETD{CL_PLATFORM_PROFILE}{char[]}{
  OpenCL 規格字符串。
  返回實現所支持的規格的名字。
  可以是下列之一:
  \startigBase
  \item FULL\_PROFILE——表示實現支持 OpenCL 規範(核心規範所定義的功能,無須支持任何擴展)。
  \item EMBEDDED\_PROFILE——表示實現支持 OpenCL 嵌入式規格。它是對應版本 OpenCL 的一個子集。
    OpenCL \scver 的嵌入式規格請參考\todo{secion 10}。
  \stopigBase
}

\clETD{CL_PLATFORM_VERSION}{char[]}{
  OpenCL 版本字符串。返回所支持的 OpenCL 版本。其格式如下:

  {\ftRef OpenCL<space><major\_version.minor\_version><space><platform-specific information>}

  所返回的 {\ftRef major\_version.minor\_version} 將是 \scver。
}

\clETD{CL_PLATFORM_NAME}{char[]}{
  平台名字。
}

\clETD{CL_PLATFORM_VENDOR}{char[]}{
  平台供應商的名字。
}

\clETD{CL_PLATFORM_EXTENSIONS}{char[]}{
  返回一個清單,列出平台所支持的擴展名,以空格分隔(擴展名本身不包含空格)。
  此平台相關的所有\cnglo{device}都要支持此處定義的擴展。
}

\stopETD

