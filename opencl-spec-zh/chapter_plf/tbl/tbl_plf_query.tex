\startETD[cl_platform_info][%
返回型別\footnote{%
對於 OpenCL 查詢函式,如果所查詢資訊的型別為 \ctype{char[]},
則會返回一個以 null 終止的字串。}%
]

\clETD{CL_PLATFORM_PROFILE}{char[]}{
OpenCL 規格字串。
返回實作所支持的規格名。
可以是下列之一:
\startigBase
\item \cenum{FULL_PROFILE}——表示實作支持 OpenCL 規範(核心規範所定義的功能,無須支持任何擴展)。
\item \cenum{EMBEDDED_PROFILE}——表示實作支持 OpenCL 嵌入式規格。
他是對應版本 OpenCL 的一個子集。
OpenCL \scver 的嵌入式規格請參考\refchapter{openclEmbProfile}。
\stopigBase
}

\clETD{CL_PLATFORM_VERSION}{char[]}{
OpenCL 版本字串。返回所支持的 OpenCL 版本。其格式如下:

\cfmt{OpenCL<space><major_version.minor_version><space><platform-specific information>}

所返回的 \cfmt{major_version.minor_version} 將是 \scver。
}

\clETD{CL_PLATFORM_NAME}{char[]}{
平台名字。
}

\clETD{CL_PLATFORM_VENDOR}{char[]}{
平台供應商的名字。
}

\clETD{CL_PLATFORM_EXTENSIONS}{char[]}{
返回平台所支持的擴展名,以空格分隔(擴展名本身不包含空格)。
此平台關聯的所有\cnglo{device}都要支持此處定義的擴展。
}

\stopETD

