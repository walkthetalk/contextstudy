\startETD[cl_device_partition_property][partition value]

\clETD{CL_DEVICE_PARTITION_EQUALLY}{unsigned int}{
  將一個大的\cnglo{device}集合分割成許多小的\cnglo{device}集合,每個都含有$n$個\cnglo{computeunit}。
  $n$伴隨此屬性一起傳遞。
  如果不能均勻平分 \cenum{CL_DEVICE_PARTITION_MAX_COMPUTE_UNITS},則不使用剩下的\cnglo{computeunit}。
}

\clETD{CL_DEVICE_PARTITION_BY_COUNTS}{unsigned int}{
  此屬性後面跟着的是\cnglo{computeunit}數目清單,以 \cenum{CL_DEVICE_PARTITION_BY_COUNTS_LIST_END} 終止。
  對於清單中每個非零的$m$,都會創建一個具有$m$個\cnglo{computeunit}的\cnglo{subdev}。
  \cenum{CL_DEVICE_PARTITION_BY_COUNTS_LIST_END}的值是 0。
  清單中的非零數目不能超過 \cenum{CL_DEVICE_PARTITION_MAX_SUB_DEVICES}。
  \cnglo{computeunit}的總數不能超過 \cenum{CL_DEVICE_PARTITION_MAX_COMPUTE_UNITS}。
}

\clETD{CL_DEVICE_PARTITION_BY_AFFINITY_DOMAIN}{cl_device_affinity_domain}{
  將\cnglo{device}分割成許多小的\cnglo{device}集合,每個集合中包含一個或多個\cnglo{computeunit}。
  他們共享部分 cache 層級系統。
  跟隨此屬性的值從下列值中選取:
  \startigBase
  \item \cenum{CL_DEVICE_AFFINITY_DOMAIN_NUMA}——\cnglo{subdev}中的\cnglo{computeunit}之間共享一個 NUMA 節點。
  \item \cenum{CL_DEVICE_AFFINITY_DOMAIN_L4_CACHE}——\cnglo{subdev}中的\cnglo{computeunit}之間共享一個 4 級數據 cache。
  \item \cenum{CL_DEVICE_AFFINITY_DOMAIN_L3_CACHE}——\cnglo{subdev}中的\cnglo{computeunit}之間共享一個 3 級數據 cache。
  \item \cenum{CL_DEVICE_AFFINITY_DOMAIN_L2_CACHE}——\cnglo{subdev}中的\cnglo{computeunit}之間共享一個 2 級數據 cache。
  \item \cenum{CL_DEVICE_AFFINITY_DOMAIN_L1_CACHE}——\cnglo{subdev}中的\cnglo{computeunit}之間共享一個 1 級數據 cache。
  \item \cenum{CL_DEVICE_AFFINITY_DOMAIN_NEXT_PARTITIONABLE}——將\cnglo{device}按下一個可再分的相似域進行分割。
    實作會判定此\cnglo{device}或者\cnglo{subdev}可以怎樣進一步細分,按 NUMA、L4、L3、L2、L1 的順序,取最前面的那個。
    對於所劃分成的\cnglo{subdev},其中的\cnglo{computeunit}間共享這一級的內存子系統。
  \stopigBase

  用戶可以通過對\cnglo{subdev}調用 \capi{clGetDeviceInfo}(\cenum{CL_DEVICE_PARTITION_TYPE}) 來確定發生了什麼。
}

\stopETD
