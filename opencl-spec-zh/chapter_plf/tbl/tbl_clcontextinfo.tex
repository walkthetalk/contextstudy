\startETD[cl_context_info][返回型別]

\clETD{CL_CONTEXT_REFERENCE_COUNT}{cl_unit}{
  返回 \carg{context} 的引用計數
  \footnote{在返回的那一刻,此引用計數就已過時。應用中一般不太適用。提供此特性主要是為了檢測內存泄漏。}。
}

\clETD{CL_CONTEXT_NUM_DEVICES}{cl_unit}{
  返回 \carg{context} 中\cnglo{device}的數目。
}

\clETD{CL_CONTEXT_DEVICES}{cl_device_id[]}{
  返回 \carg{context} 中\cnglo{device}的清單。
}

\clETD{CL_CONTEXT_PROPERTIES}{cl_context_properties[]}{
  返回調用 \capi{clCreateContext} 或 \capi{clCreateContextFromType} 時所指定的參數 \carg{properties}。

  對於調用 \capi{clCreateContext} 或 \capi{clCreateContextFromType} 創建 \carg{context} 時所指定的參數 \carg{properties} 而言,
  如果此參數不是 \cmacro{NULL},實作必須返回此參數的值。
  而如果此參數是 \cmacro{NULL},實作可以選擇將 \carg{param_value_size_ret} 置為 0,表示沒有返回屬性值,即没有返回属性值,
  或者將 \carg{param_value} 的內容置為 0 ( 0 用\cnglo{context}屬性清單的終止標記)。
}

\stopETD

