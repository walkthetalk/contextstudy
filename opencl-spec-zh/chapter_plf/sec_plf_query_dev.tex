\startbuffer[sectitlequerydevice]
查询\cnglo{device}
\stopbuffer
\section{\getbuffer[sectitlequerydevice]}
%%%%%%%%%%%%%%%%%%%%%%%%%%%%%%%%%%%%%%%%%%%%%clGetDeviceIDs
可以用如下函数获得一个\cnglo{platform}上可用\cnglo{device}的清单\footnote{\capi{clGetDeviceIDs}可能返回\carg{platform}中所有与\carg{device_type}匹配的真正的物理设备,也可能只是其中一个子集。}。
\startclc
cl_int clGetDeviceIDs(
		cl_platform_id platform,
		cl_device_type device_type,
		cl_uint num_entries,
		cl_device_id *devices,
		cl_uint *num_devices)
\stopclc

\carg{platform}用来指明要查询哪个平台的\cnglo{device},它可能是\capi{clGetPlatformIDs}所返回的,也可能是\cenum{NULL}。如果是\cenum{NULL},则行为\cnglo{impdef}。

\carg{device_type}是位域,用来指明要查询哪些类型的\scopencl\cnglo{device},可以仅查询某一种,也可以查询所有的。其有效值如\reftab{cldevctgr}。

\startbuffer[tblcapdevctgrlist]
\scopencl\cnglo{device}种类清单
\stopbuffer
\cltable
{\placetable[here,force][tab:cldevctgr]{\getbuffer[tblcapdevctgrlist]}}
{\startED[cl_device_type][返回類型]

\clED{CL_DEVICE_TYPE_CPU}{
  \cnglo{host}處理器。OpenCL 實現運行其上,是單核或多核 CPU。
}

\clED{CL_DEVICE_TYPE_GPU}{
  GPU。這意味着此\cnglo{device}也可以用來加速 3D API (如 OpenGL 或 DirectX)。
}

\clED{CL_DEVICE_TYPE_ACCELERATOR}{
  OpenCL 專用加速器(如 IBM CELL Blade)。
  這些設備通過外圍設備互聯總線(如 PCIe)與\cnglo{host}處理器通信。
}

\clED{CL_DEVICE_TYPE_CUSTOM}{
  一種專用加速器,但是不支持用 OpenCL C 編寫的\cnglo{program}。
}

\clED{CL_DEVICE_TYPE_DEFAULT}{
  系統中默認的 OpenCL \cnglo{device}。
  不能是 \cenum{CL_DEVICE_TYPE_CUSTOM} 類型的設備。
}

\clED{CL_DEVICE_TYPE_ALL}{
  系統中所有可用的 OpenCL \cnglo{device}。
  不包括 \cenum{CL_DEVICE_TYPE_CUSTOM} 類型的設備。
}

\stopED

}

\carg{num_entries}是可以加入\carg{devices}中的\ctype{cl_device}表项的数目。如果\carg{devices}不是\cenum{NULL},则\carg{num_entries}必须大于0。

\carg{devices}用来返回所找到的\scopencl\cnglo{device}的清单。\carg{devices}中的\ctype{cl_device_id}的值可以用来标识一个特定的\scopencl\cnglo{device}。如果参数\carg{devices}是NULL,则忽略此参数。所返回的\scopencl\cnglo{device}的数目是如下两个数目中较小的一个:\carg{num_entries},类型为\ctype{device_type}的\scopencl\cnglo{device}的数目。

\carg{num_devices}返回与\ctype{device_type}相匹配的可用\scopencl\cnglo{device}的数目。如果\carg{num_devices}是NULL,则忽略此参数。

如果执行成功,则\capi{clGetDeviceIDs}会返回\cenum{CL_SUCCESS}。否则,返回下列错误码之一:
\startigBase
\item \cenum{CL_INVALID_PLATFORM},如果\carg{platform}无效。
\item \cenum{CL_INVALID_DEVICE_TYPE},如果\carg{device_type}无效。
\item \cenum{CL_INVALID_VALUE},如果\carg{num_entries}等于零且\carg{devices}不是\cenum{NULL},或者\carg{num_devices}和\carg{devices}都是\cenum{NULL}。
\item \cenum{CL_DEVICE_NOT_FOUND},如果没有找到任何与\carg{device_type}匹配的\scopencl\cnglo{device}。
\item \cenum{CL_OUT_OF_RESOURCES},如果\scdevfailres。
\item \cenum{CL_OUT_OF_HOST_MEMORY},如果\schostfailres。
\stopigBase

对于\capi{clGetDeviceIDs}返回的\scopencl\cnglo{device},\cnglo{app}可以查询其\sccapability。\cnglo{app}可以据其来决定使用哪些\cnglo{device}。

%%%%%%%%%%%%%%%%%%%%%%%%%%%%%%%%%%%%%%%%%clGetDeviceInfo
可以用函数\capi{clGetDeviceInfo}获取一个\scopencl\cnglo{device}的特定信息,这些信息如表 4.3所示。
\startclc
cl_int clGetDeviceInfo(
		cl_device_id device,
		cl_device_info param_name,
		size_t param_value_size,
		void *param_value,
		size_t *param_value_size_ret)
\stopclc

\carg{device}是\capi{clGetDeviceIDs}所返回的一个\cnglo{device}。

\carg{param_name}是一个枚举常量,用来指名要查询那种信息,其值可以在表4.3所列数值中选取。

\carg{param_value}是一个指针,所指内存中存储有\carg{param_name}所对应的值。如果\carg{param_value}是\cenum{NULL},则忽略。

\carg{param_value_size}的值就是\carg{param_value}所指内存的字节数,其值必须>=表 4.3所列的返回类型的大小。
 
\carg{param_value_size_ret}返回\carg{param_value}所对应数据的实际大小。如果\carg{param_value_size_ret}是\cenum{NULL},则忽略。

\startbuffer[tblcapdevquery]
\scopencl\cnglo{device}查询
\stopbuffer

\startbuffer[footnoteprofile]
平台\scprofile返回\scopencl\cnglo{framework}所实现的\scprofile。如果返回的是\cenum{FULL_PROFILE},则\scopencl\cnglo{framework}支持\cenum{FULL_PROFILE}的\cnglo{device},可能也支持\cenum{EMBEDDED_PROFILE}的\cnglo{device}。编译器必须对所有\cnglo{device}可用,即\cenum{CL_DEVICE_COMPILER_AVAILABLE}必须是\cenum{CL_TRUE}。如果\cnglo{platform}\scprofile是\cenum{EMBEDDED_PROFILE},则只支持\cenum{EMBEDDED_PROFILE}的\cnglo{device}。
\stopbuffer

\cltable
{\placetable[here,force][tab:cldevquery]{\getbuffer[tblcapdevquery]}}
{\startETD[cl_device_info][返回類型]

\clETD{CL_DEVICE_TYPE}{cl_device_type}{
  OpenCL \cnglo{device}類型,當前支持:
  \startigBase
  \item \cenum{CL_DEVICE_TYPE_CPU},
  \item \cenum{CL_DEVICE_TYPE_GPU},
  \item \cenum{CL_DEVICE_TYPE_ACCELERATOR},
  \item \cenum{CL_DEVICE_TYPE_DEFAULT},或者
  \item 以上值的组合,或者
  \item \cenum{CL_DEVICE_TYPE_CUSTOM}。
  \stopigBase
}

\clETD{CL_DEVICE_VENDOR_ID}{cl_uint}{
  唯一的\cnglo{device}供應商標識。例如可以是 PCIe ID。
}

\clETD{CL_DEVICE_MAX_COMPUTE_UNITS}{cl_uint}{
  OpenCL \cnglo{device}上的並行\cnglo{computeunit}的數目。最小值是1。
}

\clETD{CL_DEVICE_MAX_WORK_ITEM_DIMENSIONS}{cl_uint}{
  \cnglo{dppm}中所用全局和局部\cnglo{workitem} ID 的最大維數(參見 \capi{clEnqueueNDRangeKernel})。
  最小值是3。
}

\clETD{CL_DEVICE_MAX_WORK_ITEM_SIZES}{size_t[]}{
  對\capi{clEnqueueNDRangeKernel}而言,\cnglo{workgrp}中每個維度上\cnglo{workitem}的最大數目。

  返回 \carg{n} 個 \ctype{size_t} 類型的表項。
  其中 \carg{n} 是查詢 \cenum{CL_DEVICE_MAX_WORK_ITEM_DIMENSIONS} 時所返回的值。

  對於不是 \cenum{CL_DEVICE_TYPE_CUSTOM} 的\cnglo{device},最小值是$(1, 1, 1)$。
}

\clETD{CL_DEVICE_MAX_WORK_GROUP_SIZE}{size_t}{
  \cnglo{dppm}中,\cnglo{workgrp}中所能容納的\cnglo{workitem}的最大數目(参见 \capi{clEnqueueNDRangeKernel})。
  最小值是1。
}

\clETD{
  CL_DEVICE_PREFERRED_VECTOR_WIDTH_CHAR 
  CL_DEVICE_PREFERRED_VECTOR_WIDTH_SHORT
  CL_DEVICE_PREFERRED_VECTOR_WIDTH_INT
  CL_DEVICE_PREFERRED_VECTOR_WIDTH_LONG
  CL_DEVICE_PREFERRED_VECTOR_WIDTH_FLOAT
  CL_DEVICE_PREFERRED_VECTOR_WIDTH_DOUBLE
  CL_DEVICE_PREFERRED_VECTOR_WIDTH_HALF
}{cl_uint}{
  可以放入矢量中的內建標量類型所期望的原生矢量的寬度。
  矢量寬度定義為可以可以容納標量元素的數目。

  如果不支持雙精度浮點類型,\cenum{CL_DEVICE_PREFERRED_VECTOR_WIDTH_DOUBLE} 必須返回0。

  如果不支持擴展 \cextname{cl_khr_fp16},\cenum{CL_DEVICE_PREFERRED_VECTOR_WIDTH_HALF} 必須返回0。
}

\clETD{
  CL_DEVICE_NATIVE_VECTOR_WIDTH_CHAR
  CL_DEVICE_NATIVE_VECTOR_WIDTH_SHORT
  CL_DEVICE_NATIVE_VECTOR_WIDTH_INT
  CL_DEVICE_NATIVE_VECTOR_WIDTH_LONG
  CL_DEVICE_NATIVE_VECTOR_WIDTH_FLOAT
  CL_DEVICE_NATIVE_VECTOR_WIDTH_DOUBLE
  CL_DEVICE_NATIVE_VECTOR_WIDTH_HALF
}{cl_uint}{
  返回原生 ISA 矢量寬度。
  此矢量寬度定義為所能容納標量元素的數目。

  如果不支持雙精度浮點類型,\cenum{CL_DEVICE_NATIVE_VECTOR_WIDTH_DOUBLE} 必須返回0。

  如果不支持擴展 \cextname{cl_khr_fp16},\cenum{CL_DEVICE_NATIVE_VECTOR_WIDTH_HALF} 必須返回0。
}

\clETD{CL_DEVICE_MAX_CLOCK_FREQUENCY}{cl_uint}{
  \cnglo{device}的時鐘頻率可以配置成的最大值,單位:MHz。
}

\clETD{CL_DEVICE_ADDRESS_BITS}{cl_uint}{
  運算設備的地址空間缺省大小,無符號整形,單位:bit。當前支持 32bit 或 64bit。
}

\clETD{CL_DEVICE_MAX_MEM_ALLOC_SIZE}{cl_ulong}{
  所能分配的\cnglo{memobj}大小的最大值,單位:字節(byte)。
  對於類型不是 \cenum{CL_DEVICE_TYPE_CUSTOM} 的\cnglo{device},此值最小為:

  \math{max(\text{CL\_DEVICE\_GLOBAL\_MEM\_SIZE} * 1/4, 128 * 1024 * 1024)}
}

\clETD{}{}{}

\clETD{CL_DEVICE_IMAGE_SUPPORT}{cl_bool}{
  如果 OpenCL \cnglo{device}支持圖像,則為 \cenum{CL_TRUE},否則為 \cenum{CL_FALSE}。
}

\clETD{CL_DEVICE_MAX_READ_IMAGE_ARGS}{cl_uint}{
  \cnglo{kernel}可以同時讀取多少\cnglo{imgobj}。
  如果 \cenum{CL_DEVICE_IMAGE_SUPPORT} 是 \cenum{CL_TRUE},則此值至少是128。
}

\clETD{CL_DEVICE_MAX_WRITE_IMAGE_ARGS}{cl_uint}{
  \cnglo{kernel}可以同時寫入多少\cnglo{imgobj}。
  如果 \cenum{CL_DEVICE_IMAGE_SUPPORT} 是 \cenum{CL_TRUE},則此值至少是8。
}

\clETD{CL_DEVICE_IMAGE2D_MAX_WIDTH}{size_t}{
  2D 图像或非\cnglo{bufobj}所創建的 1D 圖像的最大寬度,單位:像素。
  如果 \cenum{CL_DEVICE_IMAGE_SUPPORT} 是 \cenum{CL_TRUE},則此值至少是8192。
}

\clETD{CL_DEVICE_IMAGE2D_MAX_HEIGHT}{size_t}{
  2D 圖像的最大高度,單位:像素。
  如果 \cenum{CL_DEVICE_IMAGE_SUPPORT} 是 \cenum{CL_TRUE},則此值至少是8192。
}

\clETD{CL_DEVICE_IMAGE3D_MAX_WIDTH}{size_t}{
  3D 圖像的最大寬度,單位:像素。
  如果 \cenum{CL_DEVICE_IMAGE_SUPPORT} 是 \cenum{CL_TRUE},則此值至少是2048。
}

\clETD{CL_DEVICE_IMAGE3D_MAX_HEIGHT}{size_t}{
  3D 圖像的最大高度,單位:像素。
  如果 \cenum{CL_DEVICE_IMAGE_SUPPORT} 是 \cenum{CL_TRUE},則此值至少是2048。
}

\clETD{CL_DEVICE_IMAGE3D_MAX_DEPTH}{size_t}{
  3D 圖像的最大深度,單位:像素。
  如果 \cenum{CL_DEVICE_IMAGE_SUPPORT} 是 \cenum{CL_TRUE},則此值至少是2048。
}

\clETD{CL_DEVICE_IMAGE_MAX_BUFFER_SIZE}{size_t}{
  由\cnglo{bufobj}所創建的 1D 圖像的最大像素數。
  如果 \cenum{CL_DEVICE_IMAGE_SUPPORT} 是 \cenum{CL_TRUE},則此值至少是65536。
}

\clETD{CL_DEVICE_IMAGE_MAX_ARRAY_SIZE}{size_t}{
  1D 或 2D 圖像數組最大圖像數。
  如果 \cenum{CL_DEVICE_IMAGE_SUPPORT} 是 \cenum{CL_TRUE},則此值至少是2048。
}

\clETD{CL_DEVICE_MAX_SAMPLERS}{cl_uint}{
  一個\cnglo{kernel}內最多可以使用多少個\cnglo{sampler}。
  關於\cnglo{sampler}的細節請參考\todo{section 6.11.13}。

  如果 \cenum{CL_DEVICE_IMAGE_SUPPORT} 是 \cenum{CL_TRUE},則此值至少是16。
}

\clETD{}{}{}

\clETD{CL_DEVICE_MAX_PARAMETER_SIZE}{size_t}{
  \cnglo{kernel}参数的最大字節數。

  如果設備類型不是 \cenum{CL_DEVICE_TYPE_CUSTOM},則此值至少要是1024。
  如果是1024,則\cnglo{kernel}參數最多是128個。
}

\clETD{CL_DEVICE_MEM_BASE_ADDR_ALIGN}{cl_uint}{
  如果設備類型不是 \cenum{CL_DEVICE_TYPE_CUSTOM},其最小值是\cnglo{device}所支持的 OpenCL 內建數據類型中最大的那種的大小,單位:bit。
  ( FULL 規格中是 \ctype{long16},EMBEDDED 規格中是 \ctype{long16} 或 \ctype{int16} )
}

\clETD{}{}{}

\clETD{CL_DEVICE_SINGLE_FP_CONFIG}{cl_device_fp_config}{
  描述\cnglo{device}的單精度浮點能力。這是個位域,支持下列值:
  \startigBase
  \item \cenum{CL_FP_DENORM}——支持非正規化數( denorm )。
  \item \cenum{CL_FP_INF_NAN}——支持 INF 和 quiet NaN。
  \item \cenum{CL_FP_ROUND_TO_NEAREST}——支持舍入到最近偶數( round to nearest even )。
  \item \cenum{CL_FP_ROUND_TO_ZERO}——支持向零舍入( round to zero )。
  \item \cenum{CL_FP_ROUND_TO_INF}——支持向正無窮和負無窮舍入。
  \item \cenum{CL_FP_FMA}——支持 IEEE75-2008 中的積和熔加运算(fused multiply-add, FMA)。
  \item \cenum{CL_FP_CORRECTLY_ROUNDED_DIVIDE_SQRT}——除法和開方可以按 IEEE754 規範進行正確的舍入。
  \item \cenum{CL_FP_SOFT_FLOAT}——軟件中實現了基本的浮點運算(加、減、乘)。
  \stopigBase

  對於非\cenum{CL_DEVICE_TYPE_CUSTOM}類型的\cnglo{device},其浮點能力至少要是:
  \cenum{CL_FP_ROUND_TO_NEAREST} \textbar \cenum{CL_FP_INF_NAN}。
}

\clETD{CL_DEVICE_DOUBLE_FP_CONFIG}{cl_device_fp_config}{
  描述\cnglo{device}的雙精度浮點能力。這是個位域,支持下列值:
  \startigBase
  \item \cenum{CL_FP_DENORM}——支持非正規化數( denorm )。
  \item \cenum{CL_FP_INF_NAN}——支持 INF 和 quiet NaN。
  \item \cenum{CL_FP_ROUND_TO_NEAREST}——支持舍入到最近偶數( round to nearest even )。
  \item \cenum{CL_FP_ROUND_TO_ZERO}——支持向零舍入( round to zero )。
  \item \cenum{CL_FP_ROUND_TO_INF}——支持向正無窮和負無窮舍入。
  \item \cenum{CL_FP_FMA}——支持 IEEE75-2008 中的積和熔加运算(fused multiply-add, FMA)。
  \item \cenum{CL_FP_SOFT_FLOAT}——軟件中實現了基本的浮點運算(加、減、乘)。
  \stopigBase

  由於雙精度浮點是一個可選特性,所以最小的雙精度浮點能力可以是0。

  而如果\cnglo{device}支持雙精度浮點,則其能力至少要是:

  \cenum{CL_FP_FMA} \textbar

  \cenum{CL_FP_ROUND_TO_NEAREST} \textbar

  \cenum{CL_FP_ROUND_TO_ZERO} \textbar

  \cenum{CL_FP_ROUND_TO_INF} \textbar

  \cenum{CL_FP_INF_NAN} \textbar

  \cenum{CL_FP_DENORM}。
}

\clETD{}{}{}

\clETD{CL_DEVICE_GLOBAL_MEM_CACHE_TYPE}{cl_device_mem_cache_type}{
  所支持的全局內存 cache 的類型。其值可以是:
  \startigBase
  \item \cenum{CL_NONE},
  \item \cenum{CL_READ_ONLY_CACHE} 和
  \item \cenum{CL_READ_WRITE_CACHE}。
  \stopigBase
}

\clETD{CL_DEVICE_GLOBAL_MEM_CACHELINE_SIZE}{cl_uint}{
  全局內存 cache line 的字節數。
}

\clETD{CL_DEVICE_GLOBAL_MEM_CACHE_SIZE}{cl_ulong}{
  全局內存 cache 的字節數。
}

\clETD{CL_DEVICE_GLOBAL_MEM_SIZE}{cl_ulong}{
  全局\cnglo{device}內存的字節數。
}

\clETD{}{}{}

\clETD{CL_DEVICE_MAX_CONSTANT_BUFFER_SIZE}{cl_ulong}{
  一次所能分配的常量緩存的最大字節數。
  對於類型不是 \cenum{CL_DEVICE_TYPE_CUSTOM} 的\cnglo{device},最小值是 64KB。
}

\clETD{CL_DEVICE_MAX_CONSTANT_ARGS}{cl_uint}{
  一個\cnglo{kernel}中,最多能有多少個參數在聲明時可以帶有限定符\cqlf{__constant}。
  對於類型不是 \cenum{CL_DEVICE_TYPE_CUSTOM} 的\cnglo{device},最小值是8。
}

\clETD{}{}{}

\clETD{CL_DEVICE_LOCAL_MEM_TYPE}{cl_device_local_mem_type}{
  所支持的\cnglo{locmem}的類型。
  可以是 \cenum{CL_LOCAL}(意指專用的\cnglo{locmem},如 SRAM )或 \cenum{CL_GLOBAL}。

  對於\cnglo{customdev},如果不支持\cnglo{locmem},可以返回 \cenum{CL_NONE}。
}

\clETD{CL_DEVICE_LOCAL_MEM_SIZE}{cl_ulong}{
  \cnglo{locmem}區的字節數。
  對於類型不是 \cenum{CL_DEVICE_TYPE_CUSTOM} 的\cnglo{device},最小值是 32KB。
}

\clETD{CL_DEVICE_ERROR_CORRECTION_SUPPORT}{cl_bool}{
  如果所有對\cnglo{computedevmem}(包括\cnglo{glbmem}和\cnglo{constmem})的訪問,
  都可以由\cnglo{device}進行糾錯,則為 \cenum{CL_TRUE},否則為 \cenum{CL_FALSE}。
}

\clETD{}{}{}

\clETD{CL_DEVICE_HOST_UNIFIED_MEMORY}{cl_bool}{
  如果\cnglo{device}和\cnglo{host}共有一個統一的內存子系統,則為 \cenum{CL_TRUE},否則為 \cenum{CL_FALSE}。
}

\clETD{}{}{}

\clETD{CL_DEVICE_PROFILING_TIMER_RESOLUTION}{size_t}{
  \cnglo{device}定時器的精度。單位是納秒。詳情參見\todo{section 5.12}。
}

\clETD{}{}{}

\clETD{CL_DEVICE_ENDIAN_LITTLE}{cl_bool}{
  如果 OpenCL \cnglo{device}是 little-endian 的,則為 \cenum{CL_TRUE},否則為\cenum{CL_FALSE}。
}

\clETD{CL_DEVICE_AVAILABLE}{cl_bool}{
  如果\cnglo{device}可用,則為 \cenum{CL_TRUE},否則為\cenum{CL_FALSE}。
}

\clETD{}{}{}

\clETD{CL_DEVICE_COMPILER_AVAILABLE}{cl_bool}{
  如果沒有可用的編譯器來編譯程序源碼,則為 \cenum{CL_FALSE},否則為 \cenum{CL_TRUE}。

  只有嵌入式平台的規格才可以是 \cenum{CL_FALSE}。
}

\clETD{CL_DEVICE_LINKER_AVAILABLE}{cl_bool}{
  如果沒有可用的鏈接器,則為 \cenum{CL_FALSE},否則為 \cenum{CL_TRUE}。

  只有嵌入式平台的規格才可以是 \cenum{CL_FALSE}。

  如果 \cenum{CL_DEVICE_COMPILER_AVAILABLE} 是 \cenum{CL_TRUE},則它必須是 \cenum{CL_TRUE}。
}

\clETD{}{}{}

\clETD{CL_DEVICE_EXECUTION_CAPABILITIES}{cl_device_exec_capabilities}{
  描述\cnglo{device}的執行能力。這是個位域,包含以下值:
  \startigBase
  \item \cenum{CL_EXEC_KERNEL}——這個 OpenCL \cnglo{device}可以執行 OpenCL \cnglo{kernel}。
  \item \cenum{CL_EXEC_NATIVE_KERNEL}——這個 OpenCL \cnglo{device}可以執行原生\cnglo{kernel}。
  \stopigBase

  其中 \cenum{CL_EXEC_KERNEL} 是必須的。
}

\clETD{}{}{}

\clETD{CL_DEVICE_QUEUE_PROPERTIES}{cl_command_queue_properties}{
  \cnglo{cmdq}的屬性。這是個位域,包含以下值:
  \startigBase
  \item \cenum{CL_QUEUE_OUT_OF_ORDER_EXEC_MODE_ENABLE}
  \item \cenum{CL_QUEUE_PROFILING_ENABLE}
  \stopigBase

  請參見\todo{table 5.1}。

  其中 \cenum{CL_QUEUE_PROFILING_ENABLE} 是必須的。
}

\clETD{CL_DEVICE_BUILT_IN_KERNELS}{char[]}{
  \cnglo{device}所支持的內建\cnglo{kernel}的清單,以分號分隔。
  如果不支持內建\cnglo{kernel},則返回空字符串。
}

\clETD{}{}{}

\clETD{CL_DEVICE_PLATFORM}{cl_platform_id}{
  此\cnglo{device}所關聯的\cnglo{platform}。
}

\clETD{}{}{}

\clETD{CL_DEVICE_NAME}{char[]}{
  \cnglo{device}的名字。
}

\clETD{CL_DEVICE_VENDOR}{char[]}{
  供應商的名字。
}

\clETD{CL_DEVICE_VERSION}{char[]}{
  OpenCL 軟件驅動的版本,格式為:

  {\ftfmt major\_number.minor\_number}。
}

\clETD{CL_DEVICE_PROFILE}{char[]}{
  OpenCL 規格字符串。
  返回\cnglo{device}所支持的規格名稱。
  可以是下列字符串之一:
  \startigBase
  \item \cenum{FULL_PROFILE}——如果\cnglo{device}支持 OpenCL 規範(核心規格所定義的功能,不要求支持任何擴展)。
  \item \cenum{EMBEDDED_PROFILE}——如果\cnglo{device}支持 OpenCL 嵌入式規格。
  \stopigBase

  返回的是 OpenCL \cnglo{framework}所實現了的規格。
  如果返回的是 \cenum{FULL_PROFILE},則 OpenCL \cnglo{framework}支持符合 \cenum{FULL_PROFILE} 的\cnglo{device},可能也支持符合 \cenum{EMBEDDED_PROFILE} 的 \cnglo{device}。
  所有 \cnglo{device} 都得有可用的編譯器,即 \cenum{CL_DEVICE_COMPILER_AVAILABLE} 必須是 \cenum{CL_TRUE}。
  而如果返回的是 \cenum{EMBEDDED_PROFILE},僅支持符合 \cenum{EMBEDDED_PROFILE} 的\cnglo{device}。
}

\clETD{CL_DEVICE_VERSION}{char[]}{
  OpenCL 版本字符串。返回\cnglo{device}所支持的 OpenCL 版本。
  格式如下:

  {\ftfmt OpenCL<space><major\_version.minor\_version><space><vendor-specific information>}

  所返回的 {\ftfmt major\_version.minor\_version} 的值將是\scver。
}

\clETD{CL_DEVICE_OPENCL_C_VERSION}{char[]}{
  OpenCL C 版本字符串。
  對於類型不是 \cenum{CL_DEVICE_TYPE_CUSTOM} 的 \cnglo{device},返回編譯器在其上所支持的 OpenCL C 的最高版本。
  格式如下:

  {\ftfmt OpenCL<space>C<space><major\_version.minor\_version><space><vendor-specific information>}

  如果 \cenum{CL_DEVICE_VERSION} 是 OpenCL \scver, 則 {\ftfmt major\_version.minor\_version} 必須是 \scver。
  如果 \cenum{CL_DEVICE_VERSION} 是 OpenCL 1.1, 則 {\ftfmt major\_version.minor\_version} 必須是 1.1。
  如果 \cenum{CL_DEVICE_VERSION} 是 OpenCL 1.0, 則 {\ftfmt major\_version.minor\_version} 可以是 1.0 或 1.1。
}

\clETD{CL_DEVICE_EXTENSIONS}{char[]}{
  返回\cnglo{device}所支持的擴展名清單,以空格分隔(擴展名本身不包含空格)。
  所返回的清單可能包含供應商支持的擴展名,也可能是下列已獲 Khronos 批准的擴展名:
  \startigBase
  \item \cextname{cl_khr_int64_base_atomics}
  \item \cextname{cl_khr_int64_extended_atomics}
  \item \cextname{cl_khr_fp16}
  \item \cextname{cl_khr_gl_sharing}
  \item \cextname{cl_khr_gl_event}
  \item \cextname{cl_khr_d3d10_sharing}
  \item \cextname{cl_khr_media_sharing}
  \item \cextname{cl_khr_d3d11_sharing}
  \stopigBase

  對於支持 OpenCL C \scver 的\cnglo{device},所返回的清單中必須包含下列已獲 Khronos 批准的擴展名:
  \startigBase
  \item \cextname{cl_khr_global_int32_base_atomics}
  \item \cextname{cl_khr_global_int32_extended_atomics}
  \item \cextname{cl_khr_local_int32_base_atomics}
  \item \cextname{cl_khr_local_int32_extended_atomics}
  \item \cextname{cl_khr_byte_addressable_store}
  \item \cextname{cl_khr_fp64}(如果支持雙精度浮點,為向後兼容必須支持此擴展)

  \stopigBase

  詳情請參考 OpenCL \scver 擴展規範。
}

\clETD{}{}{}

\clETD{CL_DEVICE_PRINTF_BUFFER_SIZE}{size_t}{
  \cnglo{kernel}調用 printf 時,由一個內部緩衝區存儲其輸出,此區域大小的最大值。
  對於 \cenum{FULL} 規格,最小為 1MB。
}

\clETD{}{}{}

\clETD{CL_DEVICE_PREFERRED_INTEROP_USER_SYNC}{cl_bool}{
  在 OpenCL 和其它 API (如 DirectX )間共享\cnglo{memobj}時,如果\cnglo{device}的偏好是讓用戶自己負責同步,則其值為 \cenum{CL_TRUE};
  而如果\cnglo{device}或實現已經具備有效的方式來進行同步,則其值為 \cenum{CL_FALSE}。
}

\clETD{CL_DEVICE_PARENT_DEVICE}{cl_device_id}{
  返回此\cnglo{subdev}所屬\cnglo{pardev}的 \ctype{cl_device_id}。
  如果 \carg{device} 是\cnglo{rootdev},則返回 \cenum{NULL}。
}

\clETD{CL_DEVICE_PARTITION_MAX_SUB_DEVICES}{cl_uint}{
  當劃分\cnglo{device}時,所能創建的\cnglo{subdev}的最大數目。

  所返回的值不能超過 \cenum{CL_DEVICE_MAX_COMPUTE_UNITS}。
}

\clETD{CL_DEVICE_PARTITION_PROPERTIES}{cl_device_partition_property[]}{
  返回 \carg{device} 所支持的劃分方式。
  這是一個數組,元素類型為 \ctype{cl_device_partition_property},其值可以是:
  \startigBase
  \item \cenum{CL_DEVICE_PARTITION_EQUALLY}
  \item \cenum{CL_DEVICE_PARTITION_BY_COUNTS}
  \item \cenum{CL_DEVICE_PARTITION_BY_AFFINITY_DOMAIN}
  \stopigBase

  如果此\cnglo{device}不支持任何劃分方式,則返回 0。
}

\clETD{CL_DEVICE_PARTITION_AFFINITY_DOMAIN}{cl_device_affinity_domain}{
  用 \cenum{CL_DEVICE_PARTITION_BY_AFFINITY_DOMAIN} 劃分 \carg{device} 時,所支持的相似域( affinity domain )。
  這是個位域,其值如下所示:
  \startigBase
  \item \cenum{CL_DEVICE_AFFINITY_DOMAIN_NUMA}
  \item \cenum{CL_DEVICE_AFFINITY_DOMAIN_L4_CACHE}
  \item \cenum{CL_DEVICE_AFFINITY_DOMAIN_L3_CACHE}
  \item \cenum{CL_DEVICE_AFFINITY_DOMAIN_L2_CACHE}
  \item \cenum{CL_DEVICE_AFFINITY_DOMAIN_L1_CACHE}
  \item \cenum{CL_DEVICE_AFFINITY_DOMAIN_NEXT_PARTITIONABLE}
  \stopigBase

  如果以上都不支持,就返回 0。
}

\clETD{CL_DEVICE_PARTITION_TYPE}{cl_device_partition_property[]}{
  如果 \carg{device} 是\cnglo{subdev},則會返回調用 \capi{clCreateSubDevices} 時所指定的參數 \carg{properties}。
  否則可能返回的 \carg{param_value_size_ret} 是 0 即不存在任何劃分方式;
  或者 \carg{param_value} 所指內存中的屬性值是 0 ( 0用來終止屬性清單)。
}

\clETD{CL_DEVICE_REFERENCE_COUNT}{cl_uint}{
  返回 \carg{device} 的引用計數。
  如果是\cnglo{rootdev},則返回 1。
}

\stopETD

}

如果执行成功,\capi{clGetDeviceInfo}会返回\cenum{CL_SUCCESS};否则,返回下列错误之一:

\startigBase
\item \cenum{CL_INVALID_DEVICE}——如果\carg{device}无效。
\item \cenum{CL_INVALID_VALUE}——如果\cenum{param_name}的值不受支持,或者\cenum{param_value_size}的值<\reftab{cldevquery}中返回类型的大小并且\cenum{param_value}不是\cenum{NULL}。
\item \cenum{CL_OUT_OF_RESOURCES}——如果\scdevfailres。
\item \cenum{CL_OUT_OF_HOST_MEMORY}——如果\schostfailres。
\stopigBase




