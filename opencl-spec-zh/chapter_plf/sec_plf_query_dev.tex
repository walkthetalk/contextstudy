\startbuffer[sectitlequerydevice]
查询\cnglo{device}
\stopbuffer
\section{\getbuffer[sectitlequerydevice]}
%%%%%%%%%%%%%%%%%%%%%%%%%%%%%%%%%%%%%%%%%%%%%clGetDeviceIDs
可以用如下函数获得一个\cnglo{platform}上可用\cnglo{device}的清单\footnote{\capi{clGetDeviceIDs}可能返回\carg{platform}中所有与\carg{device_type}匹配的真正的物理设备,也可能只是其中一个子集。}。
\startclc
cl_int clGetDeviceIDs(
		cl_platform_id platform,
		cl_device_type device_type,
		cl_uint num_entries,
		cl_device_id *devices,
		cl_uint *num_devices)
\stopclc

\carg{platform}用来指明要查询哪个平台的\cnglo{device},它可能是\capi{clGetPlatformIDs}所返回的,也可能是\cenum{NULL}。如果是\cenum{NULL},则行为\cnglo{impdef}。

\carg{device_type}是位域,用来指明要查询哪些类型的\scopencl\cnglo{device},可以仅查询某一种,也可以查询所有的。其有效值如\reftab{cldevctgr}。

\startbuffer[tblcapdevctgrlist]
\scopencl\cnglo{device}种类清单
\stopbuffer
\cltable
{\placetable[here,force][tab:cldevctgr]{\getbuffer[tblcapdevctgrlist]}}
{%%%%%%%%%%%%%%%%%%%%%%%%%%%%%%%%%%%%%%%%%head%%%%%%%%%%%%%%%%%%%%%%%%%%%%%%%%%%%
\bTABLEhead
\bTR[background=color,backgroundcolor=gray]
  \bTH cl\_device\_type \eTH
  \bTH 描述 \eTH
  \eTR
\eTABLEhead

%%%%%%%%%%%%%%%%%%%%%%%%%%%%%%%%%%%%%%%  body  %%%%%%%%%%%%%%%%%%%%%%%%%%%%%%%%%
\bTABLEbody

\clenumdesc{CL_DEVICE_TYPE_CPU}{
    一种\scopencl\cnglo{device},\cnglo{host}处理器。\scopencl实现运行其上,是单核或多核CPU。
}

\clenumdesc{CL_DEVICE_TYPE_GPU}{
    一种\scopencl\cnglo{device},GPU。这意味着此\cnglo{device}也可以用来加速3D
API(如OpenGL或DirectX)。
}

\clenumdesc{CL_DEVICE_TYPE_ACCELERATOR}{
    \scopencl专用加速器(如IBM CELL Blade)。这些设备通过外围内联方式(如PCIe)与\cnglo{host}处理器通信。
}

\clenumdesc{CL_DEVICE_TYPE_DEFAULT}{
    系统中默认的\scopencl\cnglo{device}。
}

\clenumdesc{CL_DEVICE_TYPE_ALL}{
    系统中所有可用的\scopencl\cnglo{device}。
}

\eTABLEbody

}

\carg{num_entries}是可以加入\carg{devices}中的\ctype{cl_device}表项的数目。如果\carg{devices}不是\cenum{NULL},则\carg{num_entries}必须大于0。

\carg{devices}用来返回所找到的\scopencl\cnglo{device}的清单。\carg{devices}中的\ctype{cl_device_id}的值可以用来标识一个特定的\scopencl\cnglo{device}。如果参数\carg{devices}是NULL,则忽略此参数。所返回的\scopencl\cnglo{device}的数目是如下两个数目中较小的一个:\carg{num_entries},类型为\ctype{device_type}的\scopencl\cnglo{device}的数目。

\carg{num_devices}返回与\ctype{device_type}相匹配的可用\scopencl\cnglo{device}的数目。如果\carg{num_devices}是NULL,则忽略此参数。

如果执行成功,则\capi{clGetDeviceIDs}会返回\cenum{CL_SUCCESS}。否则,返回下列错误码之一:
\startigBase
\item \cenum{CL_INVALID_PLATFORM},如果\carg{platform}无效。
\item \cenum{CL_INVALID_DEVICE_TYPE},如果\carg{device_type}无效。
\item \cenum{CL_INVALID_VALUE},如果\carg{num_entries}等于零且\carg{devices}不是\cenum{NULL},或者\carg{num_devices}和\carg{devices}都是\cenum{NULL}。
\item \cenum{CL_DEVICE_NOT_FOUND},如果没有找到任何与\carg{device_type}匹配的\scopencl\cnglo{device}。
\item \cenum{CL_OUT_OF_RESOURCES},如果\scdevfailres。
\item \cenum{CL_OUT_OF_HOST_MEMORY},如果\schostfailres。
\stopigBase

对于\capi{clGetDeviceIDs}返回的\scopencl\cnglo{device},\cnglo{app}可以查询其\sccapability。\cnglo{app}可以据其来决定使用哪些\cnglo{device}。

%%%%%%%%%%%%%%%%%%%%%%%%%%%%%%%%%%%%%%%%%clGetDeviceInfo
可以用函数\capi{clGetDeviceInfo}获取一个\scopencl\cnglo{device}的特定信息,这些信息如表 4.3所示。
\startclc
cl_int clGetDeviceInfo(
		cl_device_id device,
		cl_device_info param_name,
		size_t param_value_size,
		void *param_value,
		size_t *param_value_size_ret)
\stopclc

\carg{device}是\capi{clGetDeviceIDs}所返回的一个\cnglo{device}。

\carg{param_name}是一个枚举常量,用来指名要查询那种信息,其值可以在表4.3所列数值中选取。

\carg{param_value}是一个指针,所指内存中存储有\carg{param_name}所对应的值。如果\carg{param_value}是\cenum{NULL},则忽略。

\carg{param_value_size}的值就是\carg{param_value}所指内存的字节数,其值必须>=表 4.3所列的返回类型的大小。
 
\carg{param_value_size_ret}返回\carg{param_value}所对应数据的实际大小。如果\carg{param_value_size_ret}是\cenum{NULL},则忽略。

\startbuffer[tblcapdevquery]
\scopencl\cnglo{device}查询
\stopbuffer

\startbuffer[footnoteprofile]
平台\scprofile返回\scopencl\cnglo{framework}所实现的\scprofile。如果返回的是\cenum{FULL_PROFILE},则\scopencl\cnglo{framework}支持\cenum{FULL_PROFILE}的\cnglo{device},可能也支持\cenum{EMBEDDED_PROFILE}的\cnglo{device}。编译器必须对所有\cnglo{device}可用,即\cenum{CL_DEVICE_COMPILER_AVAILABLE}必须是\cenum{CL_TRUE}。如果\cnglo{platform}\scprofile是\cenum{EMBEDDED_PROFILE},则只支持\cenum{EMBEDDED_PROFILE}的\cnglo{device}。
\stopbuffer

\cltable
{\placetable[here,force][tab:cldevquery]{\getbuffer[tblcapdevquery]}}
{%%%%%%%%%%%%%%%%%%%%%%%%%%%%%%%%%%%%%%%%%head%%%%%%%%%%%%%%%%%%%%%%%%%%%%%%%%%%%
\bTABLEhead
\bTR[background=color,backgroundcolor=gray]
  \bTH \ctype{cl_device_info} \eTH
  \bTH 返回类型 \eTH
  \bTH 描述 \eTH
  \eTR
\eTABLEhead

%%%%%%%%%%%%%%%%%%%%%%%%%%%%%%%%%%%%%%%  body  %%%%%%%%%%%%%%%%%%%%%%%%%%%%%%%%%
\bTABLEbody

\clenumretdesc{CL_DEVICE_TYPE}{cl_device_type}{
  \scopencl\cnglo{device}类型。当前所支持的值有:
  \startigBase
  \item \cenum{CL_DEVICE_TYPE_CPU},
  \item \cenum{CL_DEVICE_TYPE_GPU},
  \item \cenum{CL_DEVICE_TYPE_ACCELERATOR},
  \item \cenum{CL_DEVICE_TYPE_DEFAULT},或者
  \item 以上值的组合。
  \stopigBase
}

\clenumretdesc{CL_DEVICE_VENDOR_ID}{cl_uint}{
  唯一设备供应商标识符。例如可以是PCIe ID。
}

\clenumretdesc{CL_DEVICE_MAX_COMPUTE_UNITS}{cl_uint}{
  \scopencl\cnglo{device}上的并行计算核心的数目。最小值是1。
}

\clenumretdesc{CL_DEVICE_MAX_WORK_ITEM_DIMENSIONS}{cl_uint}{
  数据并行执行模型中所用的全局和局部\cnglo{workitem}ID的最大维数。(参见 \capi{clEnqueueNDRangeKernel})。最小值是3。
}

\clenumretdesc{CL_DEVICE_MAX_WORK_ITEM_SIZES}{size_t[]}{
给\capi{clEnqueueNDRangeKernel}所指定的\cnglo{workgrp}中每个维度上\cnglo{workitem}的最大数目。

返回\carg{n}个\ctype{size_t}表项。其中\carg{n}是查询\cenum{CL_DEVICE_MAX_WORK_ITEM_DIMENSIONS}所返回的值。

最小值是(1, 1, 1)。
}

\clenumretdesc{CL_DEVICE_MAX_WORK_GROUP_SIZE}{size_t}{
一个使用数据并行执行模型执行\cnglo{kernel}的\cnglo{workgrp}中所能存放的\cnglo{workitem}的最大数目。(参见\capi{clEnqueueNDRangeKernel})。最小值是1。
}

\clenumretdesc{
CL_DEVICE_PREFERRED_VECTOR_WIDTH_CHAR 
CL_DEVICE_PREFERRED_VECTOR_WIDTH_SHORT
CL_DEVICE_PREFERRED_VECTOR_WIDTH_INT
CL_DEVICE_PREFERRED_VECTOR_WIDTH_LONG
CL_DEVICE_PREFERRED_VECTOR_WIDTH_FLOAT
CL_DEVICE_PREFERRED_VECTOR_WIDTH_DOUBLE
CL_DEVICE_PREFERRED_VECTOR_WIDTH_HALF
}{cl_uint}{
可以放入矢量中的内建标量类型所期望的原生矢量的宽度。矢量宽度定义为可以存储到矢量中的标量元素的数目。

如果不支持扩展\cextname{cl_khr_fp64},则\cenum{CL_DEVICE_PREFERRED_VECTOR_WIDTH_DOUBLE}必须返回0。

如果不支持扩展\cextname{cl_khr_fp16},则\cenum{CL_DEVICE_PREFERRED_VECTOR_WIDTH_HALF}必须返回0。
}

\clenumretdesc{
CL_DEVICE_NATIVE_VECTOR_WIDTH_CHAR
CL_DEVICE_NATIVE_VECTOR_WIDTH_SHORT
CL_DEVICE_NATIVE_VECTOR_WIDTH_INT
CL_DEVICE_NATIVE_VECTOR_WIDTH_LONG
CL_DEVICE_NATIVE_VECTOR_WIDTH_FLOAT
CL_DEVICE_NATIVE_VECTOR_WIDTH_DOUBLE
CL_DEVICE_NATIVE_VECTOR_WIDTH_HALF
}{cl_uint}{
返回原生ISA矢量宽度。此矢量宽度定义为可以存储到矢量中的标量元素的数目。

如果不支持扩展\cextname{cl_khr_fp64},则\cenum{CL_DEVICE_NATIVE_VECTOR_WIDTH_DOUBLE}必须返回0。

如果不支持扩展\cextname{cl_khr_fp16},则\cenum{CL_DEVICE_NATIVE_VECTOR_WIDTH_HALF}必须返回0。
}

\clenumretdesc{
CL_DEVICE_MAX_CLOCK_FREQUENCY
}{cl_uint}{
\cnglo{device}的时钟频率可以配置成的最大值,单位:MHZ。
}

\clenumretdesc{
CL_DEVICE_ADDRESS_BITS
}{cl_uint}{
运算设备的地址空间的缺省大小,无符号整形,单位:bit。当前所支持的有32bit和64bit。
}

\clenumretdesc{
CL_DEVICE_MAX_MEM_ALLOC_SIZE
}{cl_ulong}{
所能分配的内存对象大小的最大值,单位:字节(byte)。此值最小为$max(CL\_DEVICE\_GLOBAL\_MEM\_SIZE * 1/4, 128 * 1024 * 1024)$
}

\clenumretdesc{}{}{}

\clenumretdesc{
CL_DEVICE_IMAGE_SUPPORT
}{cl_bool}{
如果\scopencl\cnglo{device}支持图像,则为\cenum{CL_TRUE},否则为\cenum{CL_FALSE}。
}

\clenumretdesc{
CL_DEVICE_MAX_READ_IMAGE_ARGS
}{cl_uint}{
\cnglo{kernel}可以读取的同步\cnglo{imgobj}的最大数目。如果\cenum{CL_DEVICE_IMAGE_SUPPORT}是\cenum{CL_TRUE},则此值至少要是128。
}

\clenumretdesc{
CL_DEVICE_MAX_WRITE_IMAGE_ARGS
}{cl_uint}{
\cnglo{kernel}可以写入的同步\cnglo{imgobj}的最大数目。如果\cenum{CL_DEVICE_IMAGE_SUPPORT}是\cenum{CL_TRUE},则此值至少要是128。
}

\clenumretdesc{
CL_DEVICE_IMAGE2D_MAX_WIDTH
}{size_t}{
2D图像的最大宽度,单位:像素。如果\cenum{CL_DEVICE_IMAGE_SUPPORT}是\cenum{CL_TRUE},则此值至少要是8192。
}

\clenumretdesc{
CL_DEVICE_IMAGE2D_MAX_HEIGHT
}{size_t}{
2D图像的最大高度,单位:像素。如果\cenum{CL_DEVICE_IMAGE_SUPPORT}是\cenum{CL_TRUE},则此值至少要是8192。
}

\clenumretdesc{
CL_DEVICE_IMAGE3D_MAX_WIDTH
}{size_t}{
3D图像的最大宽度,单位:像素。如果\cenum{CL_DEVICE_IMAGE_SUPPORT}是\cenum{CL_TRUE},则此值至少要是2048。
}

\clenumretdesc{
CL_DEVICE_IMAGE3D_MAX_HEIGHT
}{size_t}{
3D图像的最大高度,单位:像素。如果\cenum{CL_DEVICE_IMAGE_SUPPORT}是\cenum{CL_TRUE},则此值至少要是2048。
}

\clenumretdesc{
CL_DEVICE_IMAGE3D_MAX_DEPTH
}{size_t}{
3D图像的最大深度,单位:像素。如果\cenum{CL_DEVICE_IMAGE_SUPPORT}是\cenum{CL_TRUE},则此值至少要是2048。
}

\clenumretdesc{
CL_DEVICE_MAX_SAMPLERS
}{cl_uint}{
一个\cnglo{kernel}中可以使用的\cnglo{sampler}的最大数目。关于\cnglo{sampler}的细节描述请参考\todo{section 6.11.13}。

如果\cenum{CL_DEVICE_IMAGE_SUPPORT}是\cenum{CL_TRUE},则此值至少要是16。
}

\clenumretdesc{}{}{}

\clenumretdesc{
CL_DEVICE_MAX_PARAMETER_SIZE
}{size_t}{
可传递给\cnglo{kernel}的参数的最大字节数。

至少要是1024。如果是1024,则最多可传给\cnglo{kernel}128个参数。
}

\clenumretdesc{
CL_DEVICE_MEM_BASE_ADDR_ALIGN
}{cl_uint}{
其最小值是\cnglo{device}所支持的\scopencl内建数据类型中最大的那种数据类型的大小,单位:bit。(\scprofile FULL 中是\ctype{long16},\scprofile EMBEDDED 中是\ctype{long16}或\ctype{int16})
}

\clenumretdesc{
CL_DEVICE_MIN_DATA_TYPE_ALIGN_SIZE
}{cl_uint}{
其最小值是\cnglo{device}所支持的\scopencl内建数据类型中最大的那种数据类型的大小,单位:byte。(\scprofile FULL 中是\ctype{long16},\scprofile EMBEDDED 中是\ctype{long16}或\ctype{int16})
}

\clenumretdesc{}{}{}

\clenumretdesc{
CL_DEVICE_SINGLE_FP_CONFIG
}{cl_device_fp_config}{
描述\cnglo{device}的单精度浮点\sccapability。用位域描述,支持下列值:
\startigBase
\item \cenum{CL_FP_DENORM}——支持\scdenorm。
\item \cenum{CL_FP_INF_NAN}——支持\scinf和\scqnan。
\item \cenum{CL_FP_ROUND_TO_NEAREST}——支持舍入到最近偶数(round to nearest even)。
\item \cenum{CL_FP_ROUND_TO_ZERO}——支持向零舍入(round to zero)。
\item \cenum{CL_FP_ROUND_TO_INF}——支持向正无穷和负无穷舍入。
\item \cenum{CL_FP_FMA}——支持\scieeeqwsellb\scnfma(fused multiply-add, FMA)。
\item \cenum{CL_FP_SOFT_FLOAT}——硬件中实现了基本的浮点运算(加、减、乘)。
\stopigBase

强制性的最小浮点\sccapability为:\cenum{CL_FP_ROUND_TO_NEAREST} \textbar \cenum{CL_FP_INF_NAN}。
}

\clenumretdesc{}{}{}

\clenumretdesc{
CL_DEVICE_GLOBAL_MEM_CACHE_TYPE
}{cl_device_mem_cache_type}{
所支持全局内存\sccache的类型。有效值有:
\startigBase
\item \cenum{CL_NONE},
\item \cenum{CL_READ_ONLY_CACHE}和
\item \cenum{CL_READ_WRITE_CACHE}。
\stopigBase
}

\clenumretdesc{
CL_DEVICE_GLOBAL_MEM_CACHELINE_SIZE
}{cl_uint}{
全局内存\sccacheline的大小,单位:byte。
}

\clenumretdesc{
CL_DEVICE_GLOBAL_MEM_CACHE_SIZE
}{cl_ulong}{
全局内存\sccache的大小,单位:byte。
}

\clenumretdesc{
CL_DEVICE_GLOBAL_MEM_SIZE
}{cl_ulong}{
全局\cnglo{device}内存\sccache的大小,单位:byte。
}

\clenumretdesc{
CL_DEVICE_MAX_CONSTANT_BUFFER_SIZE
}{cl_ulong}{
一次所能分配的常量缓存最大字节数。最小值是64KB。
}

\clenumretdesc{
CL_DEVICE_MAX_CONSTANT_ARGS
}{cl_uint}{
一个\cnglo{kernel}最多能有多少参数带有限定符\cqlf{__constant}。最小值是8。
}

\clenumretdesc{}{}{}

\clenumretdesc{
CL_DEVICE_LOCAL_MEM_TYPE
}{cl_device_local_mem_type}{
所支持的局部内存的类型。可以是\cenum{CL_LOCAL}(意指专用的局部内存,如SRAM)或\cenum{CL_GLOBAL}。
}

\clenumretdesc{
CL_DEVICE_LOCAL_MEM_SIZE
}{cl_ulong}{
局部内存区的大小。最小值为32KB。
}

\clenumretdesc{
CL_DEVICE_ERROR_CORRECTION_SUPPORT
}{cl_bool}{
如果\cnglo{device}可以对运算设备的内存(包括全局内存和常量内存)进行纠错,则为\cenum{CL_TRUE}。否则为\cenum{CL_FALSE}。
}

\clenumretdesc{}{}{}

\clenumretdesc{
CL_DEVICE_HOST_UNIFIED_MEMORY
}{cl_bool}{
如果\cnglo{device}和\cnglo{host}共有一个统一的内存子系统,则为\cenum{CL_TRUE},否则为\cenum{CL_FALSE}。
}

\clenumretdesc{}{}{}

\clenumretdesc{
CL_DEVICE_PROFILING_TIMER_RESOLUTION
}{size_t}{
描述\cnglo{device}定时器的分辨率。单位是纳秒。其细节参见\todo{section 5.9}。
}

\clenumretdesc{}{}{}

\clenumretdesc{
CL_DEVICE_ENDIAN_LITTLE
}{cl_bool}{
如果\scopencl\cnglo{device}是little-endian的,则为\cenum{CL_TRUE},否则为\cenum{CL_FALSE}。
}

\clenumretdesc{
CL_DEVICE_AVAILABLE
}{cl_bool}{
如果\cnglo{device}可用,则为\cenum{CL_TRUE},否则为\cenum{CL_FALSE}。
}

\clenumretdesc{}{}{}

\clenumretdesc{
CL_DEVICE_COMPILER_AVAILABLE
}{cl_bool}{
如果没有可用编译器来编译程序源码,则为\cenum{CL_FALSE},否则是\cenum{CL_TRUE}。只有嵌入式平台\scprofile才能是\cenum{CL_FALSE}。
}

\clenumretdesc{}{}{}

\clenumretdesc{
CL_DEVICE_EXECUTION_CAPABILITIES
}{cl_device_exec_capabilities}{
描述\cnglo{device}的执行\sccapability。这是一个位域,有以下值:
\startigBase
\item \cenum{CL_EXEC_KERNEL}——此\scopencl\cnglo{device}可以执行\scopencl\cnglo{kernel}。
\item \cenum{CL_EXEC_NATIVE_KERNEL}——此\scopencl\cnglo{device}可以执行原生\cnglo{kernel}。
\stopigBase

其\sccapability至少要为:\cenum{CL_EXEC_KERNEL}。
}

\clenumretdesc{}{}{}

\clenumretdesc{
CL_DEVICE_QUEUE_PROPERTIES
}{cl_command_queue_properties}{
描述\cnglo{device}所支持的\cnglo{cmdq}属性。这是一个位域,包含以下值:
\startigBase
\item \cenum{CL_QUEUE_OUT_OF_ORDER_EXEC_MODE_ENABLE}
\item \cenum{CL_QUEUE_PROFILING_ENABLE}
\stopigBase

这些属性在\todo{table 5.1}中有所描述。

其\sccapability至少要为:\cenum{CL_QUEUE_PROFILING_ENABLE}。
}

\clenumretdesc{}{}{}

\clenumretdesc{
CL_DEVICE_PLATFORM
}{cl_platform_id}{
此\cnglo{device}所关联的\cnglo{platform}。
}

\clenumretdesc{}{}{}

\clenumretdesc{
CL_DEVICE_NAME
}{char[]}{
\cnglo{device}的名字。
}

\clenumretdesc{
CL_DEVICE_VENDOR
}{char[]}{
供应商的名字。
}

\clenumretdesc{
CL_DEVICE_VERSION
}{char[]}{
\scopencl软件驱动的版本,格式为:{\ftfmt major\_number.minor\_number}。
}

\clenumretdesc{
CL_DEVICE_PROFILE
}{char[]}{
\scopencl\scprofile字符串。返回\cnglo{device}所支持的\scprofile名称。可以是如下字符串中的一个\footnote{\getbuffer[footnoteprofile]}:p
\startigBase
\item \cenum{FULL_PROFILE}——如果设备支持\scopencl规范(核心规范所定义的功能,不需要支持任何扩展)。
\item \cenum{EMBEDDED_PROFILE}——如果设备支持\scopencl\scembpf。
\stopigBase
}

\clenumretdesc{
CL_DEVICE_VERSION
}{char[]}{
\scopencl版本字符串。返回\cnglo{device}所支持的\scopencl版本。格式如下:

{\ftfmt OpenCL<space><major\_version.minor\_version><space><vendor-specific information>}

所返回的{\ftfmt major\_version.minor\_version}的值将是\scver。
}

\clenumretdesc{
CL_DEVICE_OPENCL_C_VERSION
}{char[]}{
\scopenclc版本字符串。返回此\cnglo{device}的编译器所支持\scopenclc的最高版本。格式如下:

{\ftfmt OpenCL<space>C<space><major\_version.minor\_version><space><vendor-specific information>}

如果\cenum{CL_DEVICE_VERSION}是\scclver, {\ftfmt major\_version.minor\_version}必须是\scver。

如果\cenum{CL_DEVICE_VERSION}是\scopencl 1.0, {\ftfmt major\_version.minor\_version}可以是1.0或\scver。如果返回的是\scver,意味着此\scopencl 1.0\cnglo{device}支持\scclver规范中\todo{第六章}。
}

\clenumretdesc{
CL_DEVICE_EXTENSIONS
}{char[]}{
返回\cnglo{device}所支持的扩展名清单,以空格来分隔(扩展名称本身不包含空格)。
所返回清单当前可能包括供应商支持的扩展名,以及下列
已获Khronos批准的扩展名称中的一个或多个:

\cextname{cl_khr_fp64}

\cextname{cl_khr_int64_base_atomics}

\cextname{cl_khr_int64_extended_atomics}

\cextname{cl_khr_fp16}

\cextname{cl_khr_gl_sharing}

\cextname{cl_khr_gl_event}

\cextname{cl_khr_d3d10_sharing}

对于支持\scopenclc \scver的\cnglo{device},所返回的清单中必须包含下列已经由Khronos批准的扩展名:

\cextname{cl_khr_global_int32_base_atomics}

\cextname{cl_khr_global_int32_extended_atomics}

\cextname{cl_khr_local_int32_base_atomics}

\cextname{cl_khr_local_int32_extended_atomics}

\cextname{cl_khr_byte_addressable_store}

对于这些扩展的详细描述请参见\todo{第九章}。
}


\eTABLEbody

}

如果执行成功,\capi{clGetDeviceInfo}会返回\cenum{CL_SUCCESS};否则,返回下列错误之一:

\startigBase
\item \cenum{CL_INVALID_DEVICE}——如果\carg{device}无效。
\item \cenum{CL_INVALID_VALUE}——如果\cenum{param_name}的值不受支持,或者\cenum{param_value_size}的值<\reftab{cldevquery}中返回类型的大小并且\cenum{param_value}不是\cenum{NULL}。
\item \cenum{CL_OUT_OF_RESOURCES}——如果\scdevfailres。
\item \cenum{CL_OUT_OF_HOST_MEMORY}——如果\schostfailres。
\stopigBase




