\section{上下文}
%clGetDeviceIDs
函式
\startCLFUNC
cl_context clCreateContext(
		const cl_context_properties *properties,
		cl_uint num_devices,
		const cl_device_id *devices,
		void (CL_CALLBACK *pfn_notify)(
			const char *errinfo,
			const void *private_info,
			size_t cb,
			void *user_data),
		void *user_data,
		cl_int *errcode_ret)
\stopCLFUNC
可以創建一個 OpenCL \cnglo{context}。
OpenCL \cnglo{context}是在一個或多個\cnglo{device}上創建的。
OpenCL runtime 使用\cnglo{context}來管理對象,
如\cnglo{cmdq}、\cnglo{memobj}、\cnglo{programobj}和\cnglo{kernelobj};
以及在\cnglo{context}中所指定的\cnglo{device}上執行\cnglo{kernel}。

\carg{properties} 指定了\cnglo{context}的一系列屬性名和對應的值。
每個屬性名後面緊跟相應的期望值。
此清單以 0 結尾。
\reftab{prptForclCreateContext}列出了所支持的屬性。
如果 \carg{properties} 是 \cmacro{NULL},選擇哪個\cnglo{platform}\cnglo{impdef}。

\placetable[here,force][tab:prptForclCreateContext]{clCreateContext 所支持的屬性清單}
{\startETD[cl_context_properties][屬性值]

\clETD{CL_CONTEXT_PLATFORM}{cl_platform_id}{
指定要使用哪個 \cnglo{platform}。
}

\clETD{CL_CONTEXT_INTEROP_USER_SYNC}{cl_bool}{
OpenCL 和其他 API 之間的同步是否由用戶自己負責。
對於其使用限制,請參考《OpenCL \scver 擴展規範》中相應章節。

如果沒有指定此屬性,則缺省值為 \cenum{CL_FALSE}。
}

\stopETD

}

\carg{num_devices} 是參數 \carg{devices} 中\cnglo{device}的數目。

\startbuffer[footnoteuniquedevice]
\carg{devices} 中重複的\cnglo{device}會被忽略。
\stopbuffer
\carg{devices} 指向一個\cnglo{device}清單,其中的設備都是唯一的
\footnote{\getbuffer[footnoteuniquedevice]}
,
都是由 \capi{clGetDeviceIDs} 所返回的,
或者由 \capi{clCreateSubDevices} 所創建的\cnglo{subdev}。

\carg{pfn_notify} 是\cnglo{app}所註冊的一個回調函式。
對於此\cnglo{context},無論創建時還是運行時,只要發生了錯誤,OpenCL 的實作都會調用此函式,但調用可能是異步的。
\cnglo{app}需要保證此函式是線程安全的。
此函式的參數為:
\startigBase
\item \carg{errinfo} 指向一個錯誤字串。
\item \carg{private_info} 指向一塊二進制數據,其大小為 \carg{cb},這塊數據由 OpenCL 實作所返回,可用來記錄一些附加資訊來幫助調試錯誤。
\item \carg{user_data} 指向用戶提供的數據。
\stopigBase

如果 \carg{pfn_notify} 是 \cmacro{NULL},則表示不註冊回調函式。

注意:很多情況下,即使在\cnglo{context}外發生了錯誤,也需要發出錯誤通告。
這種情況 \carg{pfn_notify} 就無能為力了,怎樣發出這種錯誤通告\cnglo{impdef}。

\carg{user_data} 會在調用 \carg{pfn_notify} 時作為參數 \carg{user_data} 使用。
\carg{user_data} 可以是 \cmacro{NULL}。

\carg{errcode_ret} 用來返回相應錯誤碼。
如果 \carg{errcode_ret} 是 \cmacro{NULL},則不會返回錯誤碼。

如果成功創建了\cnglo{context},\capi{clCreateContext} 會將其返回(非零),
並將 \carg{errcode_ret} 置為 \cenum{CL_SUCCESS}。
否則返回 \cmacro{NULL},並將 \carg{errcode_ret} 置為下列錯誤碼之一:
\startigBase
\item \cenum{CL_INVALID_PLATFORM}——如果 \carg{properties} 是 \cmacro{NULL} 且沒有可選的\cnglo{platform},或者 \carg{properties} 中\cnglo{platform}的值無效。
\item \cenum{CL_INVALID_PROPERTY}——如果 \carg{properties} 中的\cnglo{context}屬性名不受支持,或者支持此屬性但其值無效,或者同一屬性名出現多次。
\item \cenum{CL_INVALID_VALUE}——如果 \carg{devices} 是 \cmacro{NULL}。
\item \cenum{CL_INVALID_VALUE}——如果 \carg{num_devices} 等於零。
\item \cenum{CL_INVALID_VALUE}——如果 \carg{pfn_notify} 是 \cmacro{NULL} 但 \carg{user_data} 不是 \cmacro{NULL}。
\item \cenum{CL_INVALID_DEVICE}——如果 \carg{devices} 中有無效的\cnglo{device}。
\item \cenum{CL_DEVICE_NOT_AVAILABLE}——如果 \carg{devices} 中的某個\cnglo{device}當前不可用,即使此\cnglo{device}是由 \capi{clGetDeviceIDs} 返回的。
\item \cenum{CL_OUT_OF_RESOURCES}——如果\scdevfailres。
\item \cenum{CL_OUT_OF_HOST_MEMORY}——如果\schostfailres。
\stopigBase

%clCreateContextFromType
\startbuffer[footnotecccft]
\capi{clCreateContextfromType} 可能返回平台中現有符合 \carg{device_type} 的所有實際物理設備,也可能只返回其中一個子集。
\stopbuffer
函数 \capi{clCreateContextFromType}\footnote{\getbuffer[footnotecccft]}
會由特定類型的\cnglo{device}創建一個 OpenCL \cnglo{context},
此\cnglo{context}不會引用由這些\cnglo{device}所創建的\cnglo{subdev}。
\startCLFUNC
cl_context clCreateContextFromType(
		const cl_context_properties *properties,
		cl_device_type device_type,
		void (CL_CALLBACK *pfn_notify)(
			const char *errinfo,
			const void *private_info,
			size_t cb,
			void *user_data),
		void *user_data,
		cl_int *errcode_ret)
\stopCLFUNC
\carg{properties} 指定了\cnglo{context}的一系列屬性及其相應的值。
每個屬性名後面緊跟其對應的期望值。
\reftab{prptForclCreateContext}列出了所支持的屬性。
如果 \carg{properties} 是 \cmacro{NULL},選擇哪個\cnglo{platform}\cnglo{impdef}。

\carg{device_type} 是位欄,用來標識\cnglo{device}類型,參見中的\reftab{cldevctgr}。

\carg{pfn_notify} 和 \carg{user_data} 跟 \capi{clCreateContext} 中所描述的一樣。

\carg{errcode_ret} 用來返回相應的錯誤碼,如果 \carg{errcode_ret} 是 \cmacro{NULL},則不返回錯誤碼。

如果成功創建了\cnglo{context},\capi{clCreateContextFromType} 會將其返回,並將 \carg{errcode_ret} 置為 \cenum{CL_SUCCESS}。
否則返回 \cmacro{NULL},並將 \carg{errcode_ret} 置為下列錯誤碼之一:
\startigBase
\item \cenum{CL_INVALID_PLATFORM}——如果 \carg{properties} 是 \cmacro{NULL} 並且沒有\cnglo{platform}可選,
  或者 \carg{properties} 中\cnglo{platform}的值無效。
\item \cenum{CL_INVALID_PROPERTY}——如果 \carg{properties} 中的\cnglo{context}屬性名不受支持,或者支持此屬性但其值無效,或者同一屬性名出現多次。
\item \cenum{CL_INVALID_VALUE}——如果 \carg{pfn_notify} 是 \cmacro{NULL} 但 \carg{user_data} 不是 \cmacro{NULL}。
\item \cenum{CL_INVALID_DEVICE_TYPE}——如果 \carg{device_type} 的值無效。
\item \cenum{CL_DEVICE_NOT_AVAILABLE}——如果當前沒有同時符合 \carg{device_type} 以及 \carg{properties} 中的屬性值的\cnglo{device}可用。
\item \cenum{CL_DEVICE_NOT_FOUND}——如果沒有找到同時符合 \carg{device_type} 以及 \carg{properties} 中的屬性值的\cnglo{device}。
\item \cenum{CL_OUT_OF_RESOURCES}——如果\scdevfailres。
\item \cenum{CL_OUT_OF_HOST_MEMORY}——如果\schostfailres。
\stopigBase

%clRetainContext
函式 \capi{clRetainContext} 會使 \carg{context} 的\cnglo{refcnt}增一。
\startCLFUNC
cl_int clRetainContext(cl_context context)
\stopCLFUNC

如果執行成功,\capi{clRetainContext} 會返回 \cenum{CL_SUCCESS}。否則返回下列錯誤碼之一:
\startigBase
\item \cenum{CL_INVALID_CONTEXT}——如果 \carg{context} 不是一個有效的 OpenCL \cnglo{context}。
\item \cenum{CL_OUT_OF_RESOURCES}——如果\scdevfailres。
\item \cenum{CL_OUT_OF_HOST_MEMORY}——如果\schostfailres。
\stopigBase

\capi{clCreateContext} 和 \capi{clCreateContextFromType} 會執行隱式的\cnglo{retain}。
這對第三方庫非常有用,這樣\cnglo{app}可以將\cnglo{context}傳給他們使用。
然而,\cnglo{app}可能會在沒有通知他們的情況下刪除此\cnglo{context}。
通過使用函式\cnglo{retain}或\cnglo{release}\cnglo{context},在庫所使用的\cnglo{context}不再有效時就不會出問題。

%clReleaseContext
函式 \capi{clReleaseContext} 會使 \carg{context} 的\cnglo{refcnt}減一。
\startCLFUNC
cl_int clReleaseContext(cl_context context)
\stopCLFUNC

如果執行成功,\capi{clReleaseContext} 會返回\cenum{CL_SUCCESS}。否則返回下列錯誤碼之一:
\startigBase
\item \cenum{CL_INVALID_CONTEXT}——如果 \carg{context} 不是一個有效的 OpenCL \cnglo{context}。
\item \cenum{CL_OUT_OF_RESOURCES}——如果\scdevfailres。
\item \cenum{CL_OUT_OF_HOST_MEMORY}——如果\schostfailres。
\stopigBase

如果 \carg{context} 的\cnglo{refcnt}變成了零,
且所有附着其上的對象(如\cnglo{memobj}、\cnglo{cmdq})都被釋放了時候,
\carg{context} 就會被刪除。

%clGetContextInfo
函式 \capi{clGetContextInfo} 可用來查詢\cnglo{context}的相關資訊。
\startCLFUNC
cl_int clGetContextInfo(cl_context context,
		cl_context_info param_name,
		size_t param_value_size,
		void *param_value,
		size_t *param_value_size_ret)
\stopCLFUNC

\carg{context} 指定要查詢哪個 OpenCL \cnglo{context}。

\carg{param_name} 是枚舉常量,指定要查詢什麼資訊。

\carg{param_value} 指向的內存用來存儲查詢結果。
如果 \carg{param_value} 是 \cmacro{NULL},則忽略。

\carg{param_value_size} 即 \carg{param_value} 所指內存塊的大小。
必須大於等於\reftab{paramNameclGetContextInfo}中所列返回型別的大小。

\carg{param_value_size_ret} 會返回 \carg{param_value} 所存查詢結果的實際字節數。
如果 \carg{param_value_size_ret} 是 \cmacro{NULL},則忽略。

\reftab{paramNameclGetContextInfo}列出了所支持的 \carg{param_name} 和 \capi{clGetContextInfo} 返回的 \carg{param_value} 中的資訊。

\placetable[here,force][tab:paramNameclGetContextInfo]{\capi{clGetContextInfo}所支持的\carg{param_names}的清单}
{\startETD[cl_context_info][返回型別]

\clETD{CL_CONTEXT_REFERENCE_COUNT}{cl_unit}{
  返回 \carg{context} 的引用計數
  \footnote{在返回的那一刻,此引用計數就已過時。應用中一般不太適用。提供此特性主要是為了檢測內存泄漏。}。
}

\clETD{CL_CONTEXT_NUM_DEVICES}{cl_unit}{
  返回 \carg{context} 中\cnglo{device}的數目。
}

\clETD{CL_CONTEXT_DEVICES}{cl_device_id[]}{
  返回 \carg{context} 中\cnglo{device}的清單。
}

\clETD{CL_CONTEXT_PROPERTIES}{cl_context_properties[]}{
  返回調用 \capi{clCreateContext} 或 \capi{clCreateContextFromType} 時所指定的參數 \carg{properties}。

  對於調用 \capi{clCreateContext} 或 \capi{clCreateContextFromType} 創建 \carg{context} 時所指定的參數 \carg{properties} 而言,
  如果此參數不是 \cenum{NULL},實現必須返回此參數的值。
  而如果此參數是 \cenum{NULL},實現可以選擇將 \carg{param_value_size_ret} 置為 0,表示沒有返回屬性值,即没有返回属性值,
  或者將 \carg{param_value} 的內容置為 0 ( 0 用\cnglo{context}屬性清單的終止標記)。
}

\stopETD

}

如果執行成功,\capi{clGetContextInfo} 會返回 \cenum{CL_SUCCESS}。否則,返回下列錯誤碼之一:
\startigBase
\item \cenum{CL_INVALID_CONTEXT}——如果 \carg{context} 不是一個有效的 OpenCL \cnglo{context}。
\item \cenum{CL_INVALID_VALUE}——如果 \carg{param_name} 的值不受支持,
  或者 \carg{param_value_size} 的值 < \reftab{paramNameclGetContextInfo}中返回型別的大小且 \carg{param_value} 不是 \cmacro{NULL}。
\item \cenum{CL_OUT_OF_RESOURCES}——如果\scdevfailres。
\item \cenum{CL_OUT_OF_HOST_MEMORY}——如果\schostfailres。
\stopigBase

