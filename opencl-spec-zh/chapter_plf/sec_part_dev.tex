\section{劃分設備}
%clCreateSubDevices
函式
\startclc[indentnext=no]
cl_int clCreateSubDevices (cl_device_id in_device,
			   const cl_device_partition_property *properties,
			   cl_uint num_devices,
			   cl_device_id *out_devices,
			   cl_uint *num_devices_ret)
\stopclc
會按照 \carg{properties} 所規定的劃分方案,
在 \carg{in_device} 中創建一個\cnglo{subdev}的陣列,
其中每個\cnglo{subdev}包含一組\cnglo{computeunit},且它們之間沒有交集。
\cnglo{subdev}的使用方式跟\cnglo{rootdev}(或其\cnglo{pardev})一樣,
如可以用來創建\cnglo{context}、構建\cnglo{program}、進一步調用 \capi{clCreateSubDevices} 以及創建\cnglo{cmdq}。
如果用\cnglo{subdev}創建\cnglo{cmdq},則其中的\cnglo{cmd}只能在這個\cnglo{subdev}上執行。

\carg{in_device} 即為要劃分的設備。

\carg{properties} 則指明怎樣劃分。每個劃分方式緊跟相應的值。此清單以 0 終止。
\reftab{partScheme}列出了所支持的劃分策略。\carg{properties}只能從中選擇一個。

\placetable[here,force][tab:partScheme]{\capi{clCreateSubDevices} 所支持的劃分策略}
{\startETD[cl_device_partition_property][partition value]

\clETD{CL_DEVICE_PARTITION_EQUALLY}{unsigned int}{
  將一個大的\cnglo{device}集合分割成許多小的\cnglo{device}集合,每個都含有$n$個\cnglo{computeunit}。
  $n$伴隨此屬性一起傳遞。
  如果不能均勻平分 \cenum{CL_DEVICE_PARTITION_MAX_COMPUTE_UNITS},則不使用剩下的\cnglo{computeunit}。
}

\clETD{CL_DEVICE_PARTITION_BY_COUNTS}{unsigned int}{
  此屬性後面跟着的是\cnglo{computeunit}數目清單,以 \cenum{CL_DEVICE_PARTITION_BY_COUNTS_LIST_END} 終止。
  對於清單中每個非零的$m$,都會創建一個具有$m$個\cnglo{computeunit}的\cnglo{subdev}。
  \cenum{CL_DEVICE_PARTITION_BY_COUNTS_LIST_END}的值是 0。
  清單中的非零數目不能超過 \cenum{CL_DEVICE_PARTITION_MAX_SUB_DEVICES}。
  \cnglo{computeunit}的總數不能超過 \cenum{CL_DEVICE_PARTITION_MAX_COMPUTE_UNITS}。
}

\clETD{CL_DEVICE_PARTITION_BY_AFFINITY_DOMAIN}{cl_device_affinity_domain}{
  將\cnglo{device}分割成許多小的\cnglo{device}集合,每個集合中包含一個或多個\cnglo{computeunit}。
  他們共享部分 cache 層級系統。
  跟隨此屬性的值從下列值中選取:
  \startigBase
  \item \cenum{CL_DEVICE_AFFINITY_DOMAIN_NUMA}——\cnglo{subdev}中的\cnglo{computeunit}之間共享一個 NUMA 節點。
  \item \cenum{CL_DEVICE_AFFINITY_DOMAIN_L4_CACHE}——\cnglo{subdev}中的\cnglo{computeunit}之間共享一個 4 級數據 cache。
  \item \cenum{CL_DEVICE_AFFINITY_DOMAIN_L3_CACHE}——\cnglo{subdev}中的\cnglo{computeunit}之間共享一個 3 級數據 cache。
  \item \cenum{CL_DEVICE_AFFINITY_DOMAIN_L2_CACHE}——\cnglo{subdev}中的\cnglo{computeunit}之間共享一個 2 級數據 cache。
  \item \cenum{CL_DEVICE_AFFINITY_DOMAIN_L1_CACHE}——\cnglo{subdev}中的\cnglo{computeunit}之間共享一個 1 級數據 cache。
  \item \cenum{CL_DEVICE_AFFINITY_DOMAIN_NEXT_PARTITIONABLE}——將\cnglo{device}按下一個可再分的相似域進行分割。
    實作會判定此\cnglo{device}或者\cnglo{subdev}可以怎樣進一步細分,按 NUMA、L4、L3、L2、L1 的順序,取最前面的那個。
    對於所劃分成的\cnglo{subdev},其中的\cnglo{computeunit}間共享這一級的內存子系統。
  \stopigBase

  用戶可以通過對\cnglo{subdev}調用 \capi{clGetDeviceInfo}(\cenum{CL_DEVICE_PARTITION_TYPE}) 來確定發生了什麼。
}

\stopETD
}

\carg{num_devices} 是 \carg{out_devices} 指向的內存塊所能容納 \ctype{cl_device_id} 的數目。

\carg{out_devices} 是一個 buffer,存儲所返回的 OpenCL \cnglo{subdev}。
如果 \carg{out_devices} 是 \cenum{NULL},則忽略。
否則, \carg{num_devices} 必須大於等於(按照 \carg{properties} 所指定的劃分策略)所劃分成的\cnglo{subdev}的數目。

\carg{num_devices_ret} 返回(按照 \carg{properties} 所指定的劃分策略) \carg{device}可以劃分成\cnglo{subdev}的數目。
如果 \carg{num_devices_ret} 是 \cenum{NULL},則忽略。

如果劃分成功, \capi{clCreateSubDevices} 會返回 \cenum{CL_SUCCESS}。
否則,返回 \cenum{NULL},並將 \carg{errcode_ret} 置為下列錯誤碼之一:

\startigBase
\item \cenum{CL_INVALID_DEVICE},如果 \carg{in_devices} 無效。
\item \cenum{CL_INVALID_VALUE},如果 \carg{properties} 中的值無效,或者有效但是此\cnglo{device}不支持。
\item \cenum{CL_INVALID_VALUE},如果 \carg{out_devices} 不是 \cenum{NULL},且 \carg{num_devices} 小於所創建\cnglo{subdev}的數目。
\item \cenum{CL_DEVICE_PARTITION_FAILED},如果實作支持此劃分方式,但是 \carg{in_device} 不能再細分了。
\item \cenum{CL_INVALID_DEVICE_PARTITION_COUNT},如果 \carg{properties} 所指定的劃分策略是 \cenum{CL_DEVICE_PARTITION_BY_COUNTS},
  且所要求的\cnglo{subdev}數目超過了 \cenum{CL_DEVICE_PARTITION_MAX_SUB_DEVICES},
  或者所要求的\cnglo{computeunit}總數超過了 \carg{in_device} 的 \cenum{CL_DEVICE_PARTITION_MAX_COMPUTE_UNITS},
  或者所要求的某個\cnglo{subdev}中\cnglo{computeunit}的數目小於 0 或者超過了 \carg{in_device} 的 \cenum{CL_DEVICE_PARTITION_MAX_COMPUTE_UNITS}。
\item \cenum{CL_OUT_OF_RESOURCES}——如果\scdevfailres。
\item \cenum{CL_OUT_OF_HOST_MEMORY}——如果\schostfailres。
\stopigBase

下面就如何為 \capi{clCreateSubDevices} 指定參數 \carg{properties} 給出了一些例子:

要將一個包含 16 個\cnglo{computeunit}的\cnglo{device}劃分成兩個\cnglo{subdev},
每個包含 8 個\cnglo{computeunit},這樣指定 \carg{properties}:
\startclc
	{ CL_DEVICE_PARTITION_EQUALLY, 8, 0 }
\stopclc

要將一個包含 4 個\cnglo{computeunit}的\cnglo{device}劃分成兩個\cnglo{subdev},
其中一個\cnglo{subdev}包含 3 個\cnglo{computeunit},另一個只包含 1 個\cnglo{computeunit},
這樣指定 \carg{properties}:
\startclc
	{ CL_DEVICE_PARTITION_BY_COUNTS,
	  3, 1, CL_DEVICE_PARTITION_BY_COUNTS_LIST_END, 0 }
\stopclc

按最外層的 cache line (如果有的話)劃分\cnglo{device},這樣指定 \carg{properties}:
\startclc
	{ CL_DEVICE_PARTITION_BY_AFFINITY_DOMAIN,
	  CL_DEVICE_AFFINITY_DOMAIN_NEXT_PARTITIONABLE,
	  0 }
\stopclc

%clRetainDevice
\startclc
cl_int clRetainDevice (cl_device_id device)
\stopclc

如果 \carg{device} 是通過調用 \capi{clCreateSubDevices} 創建的一個\cnglo{subdev},
則 \capi{clRetainDevice} 會使 \carg{device} 的引用計數增一。
如果 \carg{device} 是一個\cnglo{rootdev},即 \capi{clGetDeviceIDs} 所返回的一個 \ctype{cl_device_id},
則 \carg{device} 的引用計數保持不變。
如果執行成功或者 \carg{device} 是一個\cnglo{rootdev},\capi{clRetainDevice} 會返回 \cenum{CL_SUCCESS},
否則,返回下列錯誤碼之一:
\startigBase
\item \cenum{CL_INVALID_DEVICE},如果 \carg{device} 是一個無效的\cnglo{subdev}。
\item \cenum{CL_OUT_OF_RESOURCES}——如果\scdevfailres。
\item \cenum{CL_OUT_OF_HOST_MEMORY}——如果\schostfailres。
\stopigBase

%clReleaseDevice
\startclc
cl_int clReleaseDevice (cl_device_id device)
\stopclc

如果 \carg{device} 是通過調用 \capi{clCreateSubDevices} 創建的一個\cnglo{subdev},
則 \capi{clReleaseDevice} 會使 \carg{device} 的引用計數減一。
如果 \carg{device} 是一個\cnglo{rootdev},即 \capi{clGetDeviceIDs} 所返回的一個 \ctype{cl_device_id},
則 \carg{device} 的引用計數保持不變。
如果執行成功,\capi{clReleaseDevice} 會返回 \cenum{CL_SUCCESS},
否則,返回下列錯誤碼之一:
\startigBase
\item \cenum{CL_INVALID_DEVICE},如果 \carg{device} 是一個無效的\cnglo{subdev}。
\item \cenum{CL_OUT_OF_RESOURCES}——如果\scdevfailres。
\item \cenum{CL_OUT_OF_HOST_MEMORY}——如果\schostfailres。
\stopigBase

在 \carg{device} 的引用計數降為 0 並且所有附着其上的對象(如\cnglo{cmdq})都釋放後,就會刪除 \carg{device} 對象。

