\section{平台模型}

OpenCL 平台模型的定義可以查看\reffig{plfmodel}。
此模型由一個\empglo{host}構成,這個\refglo{host}連接到了一個或多個 {\ftEmpha OpenCL} \cnglo{device}上。
OpenCL \cnglo{device}被劃分成一個或多個\cnglo{computeunit}(CU),每個\cnglo{computeunit}又被劃分成一個或多個\cnglo{prcele}(PE)。
\cnglo{device}上的計算發生在\cnglo{prcele}中。

\startbuffer[buffigplfmodelcaption]
平台模型 …… 一個\cnglo{host}加上一個或多個計算\cnglo{device},每個\cnglo{device}具有一個或多個\cnglo{computeunit},每個\cnglo{computeunit}具有一個或多個\cnglo{prcele}。
\stopbuffer
\placefigure
[here,force][fig:plfmodel]
{\getbuffer[buffigplfmodelcaption]}
{\useMPgraphic{box}}

OpenCL \cnglo{app}會按照\cnglo{host}平台的原生模型在這個\cnglo{host}上運行。
OpenCL \cnglo{app}會從\cnglo{host}上提交\empglo{cmd}給\cnglo{device}中的\cnglo{prcele}以執行計算任務。
\cnglo{computeunit}中的\cnglo{prcele}會作為 SIMD 單元(執行指令流的步伐一致)或 SPMD 單元(每個 PE 維護自己的程序計數器)執行指令流。

\subsection{平台對多版本的支持}
OpenCL 的設計目標就是要支持同一\cnglo{platform}中多種具有不同能力的設備。
這些設備可以遵循不同版本的 OpenCL 規範。
對於一個 OpenCL 系統,有三個重要的版本 ID 要考慮:
\cnglo{platform}版本、\cnglo{device}版本、設備所支持的 OpenCL C 語言的版本。

\cnglo{platform}版本表明了所支持的 OpenCL runtime 的版本。
這包括\cnglo{host}用來與 OpenCL runtime 交互的所有 API,像 \englo{context}、 \englo{memobj}、 \englo{device}、 \englo{cmdq}。

\cnglo{device}版本表明了\cnglo{device}的能力,獨立於 runtime 和編譯器,由 {\ftCEmpha clGetDeviceInfo} 所返回的\cnglo{device}信息來描述。
有很多特性都與\cnglo{device}版本有關,如資源限制和擴展功能。
所返回的版本號即為此\cnglo{device}所遵循的 OpenCL 規範的最高版本號,但不會高於\cnglo{platform}版本。

對於\cnglo{device}而言,語言的版本可以讓開發人員知道此\cnglo{device}所支持的 OpenCL 編程語言具備哪些特性。
此版本會是所支持語言的最高版本號。

OpenCL C 被設計為向後兼容的,因此對於一個\cnglo{device}而言,只要支持語言的某一個版本,就可以說它和標準是兼容的。
如果某個\cnglo{device}支持語言的多個版本,編譯器缺省使用最高的那個版本。
語言的版本不會高於\cnglo{platform}的版本,但可能高於\cnglo{device}的版本\todo{see section 5.6.4.5}。

