\section{內存對象}

\cnglo{memobj}分成兩類:\refglo{bufobj}和\refglo{imgobj}。
\refglo{bufobj}中所存儲的元素是一維的,
而\refglo{imgobj}則用來存儲二維或三維的材質、幀緩衝(frame-buffer)或圖像。

\refglo{bufobj}中的元素可以是標量數據型別(如 int、float)、
矢量數據型別或用戶自定義的結構體。
\refglo{imgobj}用來表示材質、幀緩衝等緩衝區( buffer )。
\cnglo{imgobj}中元素的格式必須從預定義格式中選取。
\cnglo{memobj}中至少要有一個元素。

\refglo{bufobj}和\refglo{imgobj}的根本區別是:
\startigBase
\item \refglo{bufobj}中的元素是順序存儲的,
\cnglo{kernel}在\cnglo{device}上運行時可以用指針訪問這些元素。
而\refglo{imgobj}中元素的存儲格式對用戶是透明的,不能通過指針直接訪問。
可以使用 OpenCL C 編程語言提供的內建函式來讀寫\cnglo{imgobj}。

\item 對於\refglo{bufobj},\cnglo{kernel}按其存儲格式訪問其中的數據。
而對於\refglo{imgobj},其元素的存儲格式可能與\cnglo{kernel}中使用的數據格式不一樣。
\cnglo{kernel}中的圖像元素始終是四元矢量(每一元都可以是浮點型別或者帶符號/無符號整形)。
內建函式在讀寫圖像元素時會進行相應的格式轉換。
\stopigBase

\cnglo{memobj}是用 \ctype{cl_mem} 來表示的。
\cnglo{kernel}的輸入和輸出都是\cnglo{memobj}。

