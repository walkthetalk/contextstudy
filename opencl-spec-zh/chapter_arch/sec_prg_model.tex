\section{編程模型}
OpenCL 執行模型支持\cnglo{dppm}和\cnglo{tppm},同時也支持這兩種模型的混合體。
對於 OpenCL 而言,驅動其設計的首要模型是數據並行。

\startbuffer[sectitledppm]
\cnglo{dppm}
\stopbuffer
\subsection{\getbuffer[sectitledppm]}

\cnglo{dppm}依據同時應用到\cnglo{memobj}的多個元素上的指令序列來定義計算( computation )。
OpenCL 執行模型所關聯的索引空間定義了\cnglo{workitem},以及數據怎樣映射到\cnglo{workitem}上。
在嚴格的數據並行模型中,\cnglo{workitem}和\cnglo{memobj}的元素間的映射關係為一對一,\cnglo{kernel}可在上面並行執行。
對於\cnglo{dppm},OpenCL 所實現的版本比較寬鬆,不要求嚴格的一對一的映射。

OpenCL 所提供的\cnglo{dppm}是分級的。
有兩種方式來進行分級。
一種是顯式方式,程序員定義可以並行執行的\cnglo{workitem}的總數、以及怎樣將這些\cnglo{workitem}劃分到\cnglo{workgrp}中。
另一種是隱式方式,程序員僅指定前者,後者由 OpenCL 實現來管理。

\startbuffer[sectitletppm]
\cnglo{tppm}
\stopbuffer
\subsection{\getbuffer[sectitletppm]}
在 OpenCL 的\cnglo{tppm}中,\cnglo{kernel}的實例在執行時獨立於任何索引空間。
這在邏輯上等同於:在\cnglo{computeunit}上執行\cnglo{kernel}時,相應\cnglo{workgrp}中只有一個\cnglo{workitem}。
這種模型下,用戶以如下方式表示並行:

\startigBase
\item 使用\cnglo{device}所實現的矢量數據類型;
\item 將多個任務入隊,和/或
\item 將多個原生\cnglo{kernel}入隊(它們是使用一個與 OpenCL 正交的編程模型開發的)。
\stopigBase

\subsection{同步}
在 OpenCL 中,有兩方面的同步:
\startigBase
\item 隸屬同一\cnglo{workgrp}的\cnglo{workitem}之間;
\item \cnglo{cmdq}中處於同一\cnglo{context}中的\cnglo{cmd}之間。
\stopigBase

前者是通過\cnglo{workgrpbarrier}實現的。
對於同一\cnglo{workgrp}中的所有\cnglo{workitem}來說,任意一個要想在\cnglo{barrier}外繼續執行,所有\cnglo{workitem}都必須先執行這個\cnglo{barrier}。
注意,在同一\cnglo{workgrp}中,所有正在執行\cnglo{kernel}的\cnglo{workitem}必須都會去執行這個\cnglo{workgrpbarrier},或者都不會去執行。
\cnglo{workgrp}之間沒有同步機制。

\cnglo{cmdq}中\cnglo{cmd}間的同步點是:
\startigBase
\item \cnglo{cmdqbarrier}。
它保證:所有之前排隊的\cnglo{cmd}都執行完畢,並且對\cnglo{memobj}的所有更新,在後續\cnglo{cmd}開始執行前都是可見的。
它只能在隸屬同一\cnglo{cmdq}的\cnglo{cmd}間進行同步。

\item 等在一個事件上。所有會將\cnglo{cmd}入隊的 OpenCL API 函數都會返回一個事件(用來標識這個\cnglo{cmd})以及將更新的\cnglo{memobj}。
如果某個後續\cnglo{cmd}正在等待那個事件,可以保證在其開始執行前,可以見到對那些\cnglo{memobj}的所有更新。
\stopigBase

