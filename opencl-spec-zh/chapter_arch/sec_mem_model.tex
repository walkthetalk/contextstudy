
\section{內存模型}
\cnglo{workitem}在執行\cnglo{kernel}時可以訪問四塊不同的\cnglo{memregion}:

\startigBase
\item \cngloemp{glbmem}:
所有\cnglo{workgrp}中的所有\cnglo{workitem}都可以對其進行讀寫。
\cnglo{workitem}可以讀寫\cnglo{memobj}中的任意元素。
對\cnglo{glbmem}的讀寫可能會被緩存,這取決於\cnglo{device}的能力。

\item \cngloemp{constmem}:
\cnglo{glbmem}中的一塊區域,在\cnglo{kernel}的執行過程中保持不變。
\cnglo{host}負責對此中\cnglo{memobj}的分配和初始化。

\item \cngloemp{locmem}:
隸屬於某個\cnglo{workgrp}。
可以用來分配一些變量,這些變量由此\cnglo{workgrp}中的所有\cnglo{workitem}共享。
在 OpenCL \cnglo{device}上,可能會將其實現成一塊專用的\cnglo{memregion},
也可能將其映射到\cnglo{glbmem}中。

\item \cngloemp{prvmem}:
隸屬於某個\cnglo{workitem}。
一個\cnglo{workitem}的\cnglo{prvmem}中所定義的變量
對另外一個\cnglo{workitem}而言是不可見的。
\stopigBase

\reftab{memregion}列出了這些信息:
\cnglo{kernel}或\cnglo{host}是否可以從某個\cnglo{memregion}中分配內存、
怎樣分配(靜態編譯時 vs. 動態運行時)
以及允許如何訪問(即\cnglo{kernel}或\cnglo{host}是否可以對其進行讀寫)。

\placetable[here,force][tab:memregion]{內存區域——分配以及訪問}{
\bTABLE

\bTABLEhead
\bTR
\bTD[nc=2] \eTD \bTD\cnglo{glbmem}\eTD \bTD\cnglo{constmem}\eTD \bTD\cnglo{locmem}\eTD \bTD\cnglo{prvmem}\eTD
\eTR
\eTABLEhead

\bTABLEbody
\bTR
\bTD[nr=2]\cnglo{host}\eTD \bTD分配\eTD \bTD動態\eTD \bTD動態\eTD \bTD動態\eTD \bTD NO\eTD
\eTR

\bTR
\bTD 訪問 \eTD \bTD 讀寫 \eTD \bTD 讀寫 \eTD \bTD NO \eTD \bTD NO \eTD
\eTR

\bTR
\bTD[nr=2]\cnglo{kernel}\eTD \bTD 分配 \eTD \bTD NO \eTD \bTD 靜態 \eTD \bTD 靜態 \eTD \bTD 靜態 \eTD
\eTR

\bTR
\bTD 訪問 \eTD \bTD 讀寫 \eTD \bTD 只讀 \eTD \bTD 讀寫 \eTD \bTD 讀寫 \eTD
\eTR
\eTABLEbody

\eTABLE


}

\reffig{openclarch}描述了\cnglo{memregion}以及與\cnglo{platform}模型的關係。
圖中含有\cnglo{prcele}( PE )、\cnglo{computeunit}和\cnglo{device},
但是沒有畫出\cnglo{host}。

\placefigure
[here,force][fig:openclarch]
{OpenCL 設備架構的概念模型}
{\useMPgraphic{box}}

\cnglo{app}在\cnglo{host}上運行時,
使用 OpenCL API 在\cnglo{glbmem}中創建\cnglo{memobj},
並將內存命令(\refsec{exemodel:contextandcmdq}中有所描述)入隊以操作它們。

多數情況下,\cnglo{host}和 OpenCL \cnglo{device}的內存模型是相互獨立的。
其必然性主要在於 OpenCL 沒有囊括\cnglo{host}的定義。
然而,有時它們確實需要交互。
有兩種交互方式:顯式拷貝數據、將\cnglo{memobj}的部分区域映射和解映射。

為了顯式拷貝數據,\cnglo{host}會將一些\cnglo{cmd}插入隊列,
用來在\cnglo{memobj}和\cnglo{host}內存之間傳輸數據。
這些用於傳輸內存的命令可以是阻塞式的,也可以是非阻塞式的。
對於前者,一旦\cnglo{host}上相關內存資源可以被安全的重用,OpenCL 函數調用就會立刻返回。
而對於後者,一旦命令入隊,OpenCL 函數調用就會返回,而不管\cnglo{host}內存是否可以安全使用。

用映射、解映射的方法處理\cnglo{host}和 OpenCL \cnglo{memobj}的交互時,
\cnglo{host}可以將\cnglo{memobj}的某個區域映射到自己的地址空間中。
內存映射命令可能是阻塞的,也可能是非阻塞的。
一旦映射了\cnglo{memobj}的某個區域,\cnglo{host}就可以讀寫這塊區域。
當\cnglo{host}對這塊區域的訪問(讀和/或寫)結束後,就會將其解映射。

% Memory Consistency
\subsection{內存一致性}
OpenCL 所使用的一致性內存模型比較寬鬆;
即,不保證不同\cnglo{workitem}所看到的內存狀態始終一致。

在\cnglo{workitem}內部,內存具有裝載、存儲的一致性。
在隸屬於同一\cnglo{workgrp}的\cnglo{workitem}之間,
\cnglo{locmem}在\cnglo{workgrpbarrier}上是一致的。
\cnglo{glbmem}亦是如此,
但對於執行同一\cnglo{kernel}的不同\cnglo{workgrp},則不保證其內存一致性。

對於已經入隊的\cnglo{cmd}所共享的\cnglo{memobj},其內存一致性由同步點來強制實施。

