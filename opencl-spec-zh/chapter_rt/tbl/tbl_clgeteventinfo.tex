\startETD[cl_event_info][返回型別]

\clETD{CL_EVENT_COMMAND_QUEUE}{cl_command_queue}{
返回 \carg{event} 所關聯的\cnglo{cmdq}。
對於用戶\cnglo{evtobj},會返回 \cmacro{NULL}。
}

\clETD{CL_EVENT_CONTEXT}{cl_context}{
返回 \carg{event} 所關聯的\cnglo{context}。
}

\clETD{CL_EVENT_COMMAND_TYPE}{cl_command_type}{
返回 \carg{event} 所關聯的\cnglo{cmd}。
可以是下列之一:
\startigBase
\item \cenum{CL_COMMAND_NDRANGE_KERNEL}
\item \cenum{CL_COMMAND_TASK}
\item \cenum{CL_COMMAND_NATIVE_KERNEL}
\item \cenum{CL_COMMAND_READ_BUFFER}
\item \cenum{CL_COMMAND_WRITE_BUFFER}
\item \cenum{CL_COMMAND_COPY_BUFFER}
\item \cenum{CL_COMMAND_READ_IMAGE}
\item \cenum{CL_COMMAND_WRITE_IMAGE}
\item \cenum{CL_COMMAND_COPY_IMAGE}
\item \cenum{CL_COMMAND_COPY_BUFFER_TO_IMAGE}
\item \cenum{CL_COMMAND_COPY_IMAGE_TO_BUFFER}
\item \cenum{CL_COMMAND_MAP_BUFFER}
\item \cenum{CL_COMMAND_MAP_IMAGE}
\item \cenum{CL_COMMAND_UNMAP_MEM_OBJECT}
\item \cenum{CL_COMMAND_MARKER}
\item \cenum{CL_COMMAND_ACQUIRE_GL_OBJECTS}
\item \cenum{CL_COMMAND_RELEASE_GL_OBJECTS}
\item \cenum{CL_COMMAND_READ_BUFFER_RECT}
\item \cenum{CL_COMMAND_WRITE_BUFFER_RECT}
\item \cenum{CL_COMMAND_COPY_BUFFER_RECT}
\item \cenum{CL_COMMAND_USER}
\item \cenum{CL_COMMAND_BARRIER}
\item \cenum{CL_COMMAND_MIGRATE_MEM_OBJECTS}
\item \cenum{CL_COMMAND_FILL_BUFFER}
\item \cenum{CL_COMMAND_FILL_IMAGE}
\stopigBase
}

\clETD{CL_EVENT_COMMAND_EXECUTION_STATUS}{cl_int}{
返回 \carg{event} 所標識的\cnglo{cmd}的執行狀態
\footnote{錯誤碼的值是負的,事件狀態的值是正的。
事件狀態的值這樣變化:從第一個或初始狀態,即最大值(\cenum{CL_QUEUED}),
一直到最後一個或完成的狀態,即最小值(\cenum{CL_COMPLETE} 或負整數)。
 \cenum{CL_COMPLETE} 跟 \cenum{CL_SUCCESS} 一樣。
}。
有效值為:
\startigBase
\item \cenum{CL_QUEUED}(\cnglo{cmd}已經入隊);

\item \cenum{CL_SUBMITTED}
(\cnglo{host}已經將所入隊的\cnglo{cmd}提交給了\cnglo{cmdq}所關聯的\cnglo{device});

\item \cenum{CL_RUNNING}(\cnglo{device}正在執行這個\cnglo{cmd});

\item \cenum{CL_COMPLETED}(\cnglo{cmd}已經完成);或

\item 錯誤碼,一個負整數(\cnglo{cmd}異常終止——可能由非法內存存取所導致)。
與\cnglo{platform}或運行時 API 調用所返回的值或 \carg{errcode_ret} 的值使用同一套錯誤碼。

\stopigBase
}

\clETD{CL_EVENT_REFERENCE_COUNT}{cl_uint}{
返回 \carg{event} 的\cnglo{refcnt}
\footnote{在返回的那一刻,此引用計數就已過時。
應用中一般不太適用。提供此特性主要是為了檢測內存泄漏。}。
}
\stopETD
