\startETD[cl_program_info][返回型別]

\clETD{CL_PROGRAM_REFERENCE_COUNT}{cl_uint}{
返回 \carg{program} 的\cnglo{refcnt}。
\footnote{在返回的那一刻,此引用計數就已過時。應用中一般不太適用。提供此特性主要是為了檢測內存泄漏。}
}

\clETD{CL_PROGRAM_CONTEXT}{cl_context}{
返回創建\cnglo{programobj}時所指定的\cnglo{context}。
}

\clETD{CL_PROGRAM_NUM_DEVICES}{cl_uint}{
返回 \carg{program} 所關聯\cnglo{device}的數目。
}

\clETD{CL_PROGRAM_DEVICES}{cl_device_id[]}{
返回 \carg{program} 所關聯的\cnglo{device}。
可以是創建\cnglo{programobj}時所用\cnglo{context}中的\cnglo{device};
也可以是用 \capi{clCreateProgramWithBinary} 創建\cnglo{programobj}時所指定的\cnglo{device}中的一個子集。
}

\clETD{CL_PROGRAM_SOURCE}{char[]}{
返回傳給 \capi{clCreateProgramWithSource} 的\cnglo{program}源碼。
返回的字符串中將所有源碼都串在了一起,並以 null 終止(原有的 null 會被剝離)。

如果 \carg{program} 是用 \capi{clCreateProgramWithBinary} 或 \capi{clCreateProgramWithBuiltinKernels} 創建的,
可能返回空字符串,或者相應的\cnglo{program}源碼,這取決於二元碼中是否包含\cnglo{program}源碼。

\carg{param_value_size_ret} 中會返回源碼中字符的實際數目,包含 null 終止符。
}

\clETD{CL_PROGRAM_BINARY_SIZES}{size_t[]}{
返回一個數組,內含 \carg{program} 所關聯的所有\cnglo{device}對應的\cnglo{program}二元碼(可以是可執行二元碼、編譯過的二元碼或者庫的二元碼)的大小。
數組的大小等於與 \carg{program} 所關聯的\cnglo{device}的個數。
如果任一\cnglo{device}沒有對應的二元碼,則返回零。

如果 \carg{program} 是用 \capi{clCreateProgramWithBuiltinKernels} 創建的,
可能數組中的所有元素都是零。
}

\clETD{CL_PROGRAM_BINARIES}{unsigned char *[]}{
返回一個數組,內含 \carg{program} 所關聯的所有\cnglo{device}對應的\cnglo{program}二元碼(可以是可執行二元碼、編譯過的二元碼或者庫的二元碼)。
對於 \carg{program} 所關聯的每個\cnglo{device}而言,
所返回的二元碼可能是用 \capi{clCreateProgramWithBinary} 創建 \carg{program} 時所指定的二元碼,
或者是用 \capi{clBuildProgram} 或 \capi{clLinkProgram} 所生成的可執行二元碼。
如果是用 \capi{clCreateProgramWithSource} 生成的 \carg{program},
則返回的是 \capi{clBuildProgram}、 \capi{clCompileProgram} 或 \capi{clLinkProgram} 所生成的二元碼。
所返回的可能是特定實現的中間表示(又叫做 IR),或特定設備的可執行二元碼,也可能二者兼有。
至於二元碼中會返回哪種信息由 OpenCL 實現來決定。

\carg{param_value} 指向一個包含\math{n}個指針的數組,所有指針都由調用者分配,
其中\math{n}就是 \carg{program} 所關聯\cnglo{device}的數目。
對於每個指針需要分配多少內存,可以通過此表中的 \cenum{CL_PROGRAM_BINARY_SIZES} 進行查詢。

實現可以使用數組中的元素來存儲特定\cnglo{device}所對應的\cnglo{program}二元碼,如果有的話。
至於數組中的元素都對應於哪個\cnglo{device},可以使用 \cenum{CL_PROGRAM_DEVICES} 進行查詢。
 \cenum{CL_PROGRAM_BINARIES} 和 \cenum{CL_PROGRAM_DEVICES} 所返回的數組具有一對一的關係。
}

\clETD{CL_PROGRAM_NUM_KERNELS}{size_t}{
返回 \carg{program} 中聲明的\cnglo{kernel}總數。
對於 \carg{program} 所關聯的\cnglo{device}而言,至少要為其中之一成功構建了可執行\cnglo{program},
然後才能使用此信息。
}

\clETD{CL_PROGRAM_KERNEL_NAMES}{char[]}{
返回 \carg{program} 中\cnglo{kernel}的名字,以分號間隔。
對於 \carg{program} 所關聯的\cnglo{device}而言,至少要為其中之一成功構建了可執行\cnglo{program},
然後才能使用此信息。
}

\stopETD

