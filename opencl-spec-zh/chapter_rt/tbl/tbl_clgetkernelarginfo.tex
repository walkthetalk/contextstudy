\startETD[cl_kernel_arg_info][返回型別]

\clETD{CL_KERNEL_ARG_ADDRESS_QUALIFIER}{cl_kernel_arg_address_qualifier}{
返回參數的位址限定符。所返回的值可以是下列之一:
\startigBase
\item \cenum{CL_KERNEL_ARG_ADDRESS_GLOBAL}
\item \cenum{CL_KERNEL_ARG_ADDRESS_LOCAL}
\item \cenum{CL_KERNEL_ARG_ADDRESS_CONSTANT}
\item \cenum{CL_KERNEL_ARG_ADDRESS_PRIVATE}
\stopigBase

如果沒有指定位址限定符,則返回缺省的位址限定符 \cenum{CL_KERNEL_ARG_ADDRESS_PRIVATE}。
}

\clETD{CL_KERNEL_ARG_ACCESS_QUALIFIER}{cl_kernel_arg_access_qualifier}{
返回參數的訪問限定符。所返回的值可以是下列之一:
\startigBase
\item \cenum{CL_KERNEL_ARG_ACCESS_READ_ONLY}
\item \cenum{CL_KERNEL_ARG_ACCESS_WRITE_ONLY}
\item \cenum{CL_KERNEL_ARG_ACCESS_READ_WRITE}
\item \cenum{CL_KERNEL_ARG_ACCESS_NONE}
\stopigBase

如果參數型別不是圖像,則返回 \cenum{CL_KERNEL_ARG_ACCESS_NONE}。
如果參數型別是圖像,則會返回所指定的訪問限定符或缺省的訪問限定符。
}

\clETD{CL_KERNEL_ARG_TYPE_NAME}{char[]}{
返回參數的型別名,即所聲明的型別名(會移除空白)。
如果參數是無符號標量型別(即 unsigned char, unsigned short, unsigned int, unsigned long),
則會返回 uchar, ushort, uint 和 ulong。
所返回的型別名中不包括任何型別限定符。
}

\clETD{CL_KERNEL_ARG_TYPE_QUALIFIER}{cl_kernel_arg_type_qualifier}{
返回參數的型別限定符。可能是:
\startigBase
\item \cenum{CL_KERNEL_ARG_TYPE_CONST}
\item \cenum{CL_KERNEL_ARG_TYPE_RESTRICT}
\item \cenum{CL_KERNEL_ARG_TYPE_VOLATILE}
\item 以上三種的組合;或者
\item \cenum{CL_KERNEL_ARG_TYPE_NONE}
\stopigBase

注意:如果參數是指針,且所引用的型別聲明時帶有限定符 \cqlf{volatile},
則會返回 \cenum{CL_KERNEL_ARG_TYPE_VOLATILE}。
例如,如果\cnglo{kernel}參數聲明為 \ccmm{global int volatile *x},
則會返回 \cenum{CL_KERNEL_ARG_TYPE_VOLATILE};
而對於聲明為 \ccmm{global int *volatile x} 的\cnglo{kernel}參數,
則不會返回 \cenum{CL_KERNEL_ARG_TYPE_VOLATILE}。
}

\clETD{CL_KERNEL_ARG_NAME}{char[]}{
返回參數的名字。
}

\stopETD
