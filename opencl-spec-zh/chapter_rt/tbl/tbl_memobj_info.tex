\startETD[cl_mem_info][返回类型]

\clETD{CL_MEM_TYPE}{cl_mem_object_type}{
返回下列值中的一個:
\startigBase
\item \cenum{CL_MEM_OBJECT_BUFFER},
如果 \carg{memobj} 是用 \capi{clCreate{Buffer | SubBuffer}} 創建的。

\item \cvar{cl_image_desc.image_type},
如果 \carg{memobj} 是用 \capi{clCreateImage} 創建的。
\stopigBase
}

\clETD{CL_MEM_FLAGS}{cl_mem_flags}{
返回用 \capi{clCreate{Buffer | SubBuffer | Image}} 創建 \carg{memobj} 時
所指定的引數 \carg{flags}。

如果 \carg{memobj} 是子\cnglo{bufobj},則同時會返回由父對象所繼承的內存訪問限定符。
}

\clETD{CL_MEM_SIZE}{size_t}{
返回 \carg{memobj} 中用於數據存儲的內存的實際大小。單位:字節。
}

\clETD{CL_MEM_HOST_PTR}{void *}{
如果 \carg{memobj} 是由 \capi{clCreate{Buffer | Image}} 創建,
且 \carg{mem_flags} 中設置了 \cenum{CL_MEM_USE_HOST_PTR},
則返回引數 \carg{host_ptr} 的值。否則返回 \cmacro{NULL}。

如果 \carg{memobj} 是由 \capi{clCreateSubBuffer} 創建,則返回 $host\_ptr + origin$。
\carg{host_ptr} 即用 \capi{clCreateBuffer} 創建 \carg{memobj} 的父對象時
(\carg{mem_flags} 中設置了 \cenum{CL_MEM_USE_HOST_PTR})引數的值。
否則返回 \cmacro{NULL}。
}

\clETD{CL_MEM_MAP_COUNT}{cl_uint}{
映射計數。\footnote{%
在返回的那一刻,此映射計數就已過時。應用中一般不太適用。提供此特性主要是為了調試。}
}

\clETD{CL_MEM_REFERENCE_COUNT}{cl_uint}{
返回 \carg{memobj} 的\cnglo{refcnt}。\footnote{%
在返回的那一刻,此引用計數就已過時。應用中一般不太適用。提供此特性主要是為了檢測內存泄漏。}
}

\clETD{CL_MEM_CONTEXT}{cl_context}{
返回創建\cnglo{memobj}時所指定的\cnglo{context}。
如果 \carg{memobj} 是用 \capi{clCreateSubBuffer} 創建的,
則返回引數 \carg{buffer} 所在的\cnglo{context}。
}

\clETD{CL_MEM_ASSOCIATED_MEMOBJECT}{cl_mem}{
返回 \carg{memobj} 的父對象。即 \capi{clCreateSubBuffer} 的引數 \carg{buffer}。

否則返回 \cmacro{NULL}。
}

\clETD{CL_MEM_OFFSET}{size_t}{
如果 \carg{memobj} 是由 \capi{clCreateSubBuffer} 創建的子\cnglo{bufobj},
則返回其偏移量。
否則返回 0。
}

\stopETD

