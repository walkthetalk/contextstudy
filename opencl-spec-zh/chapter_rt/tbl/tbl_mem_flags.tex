\startED[\ctype{cl_mem_flags}]

\clED{CL_MEM_READ_WRITE}{
  表明此\cnglo{memobj}可讀可寫。這是缺省值。
}

\clED{CL_MEM_WRITE_ONLY}{
  只能寫不能讀,對這種\cnglo{memobj}的讀操作是未定義的。
  \cenum{CL_MEM_READ_WRITE} 和 \cenum{CL_MEM_WRITE_ONLY} 是互斥的。
}

\clED{CL_MEM_READ_ONLY}{
  只能讀不能寫,對這種\cnglo{memobj}的寫操作是未定義的。
  \cenum{CL_MEM_READ_WRITE} 或者 \cenum{CL_MEM_WRITE_ONLY} 都與 \cenum{CL_MEM_READ_ONLY} 互斥。
}

\clED{CL_MEM_USE_HOST_PTR}{
  僅當 \carg{host_ptr} 不是 \cenum{NULL} 時才有效。
  它表明\cnglo{app}想讓 OpenCL 實作使用 \carg{host_ptr} 所引用的內存來存儲\cnglo{memobj}的內容。

  OpenCL 實作可以在\cnglo{device}內存中保存一份 \carg{host_ptr} 所引用的內容用作 cache。
  \cnglo{kernel}在\cnglo{device}上執行時可以使用這份用作 cache 的拷貝。

  如果多個\cnglo{bufobj}由同一 \carg{host_ptr} 創建,或者有重疊區域,當 OpenCL \cnglo{cmd}操作這些\cnglo{bufobj}時,其結果未定義。

  用 \cenum{CL_MEM_USE_HOST_PTR} 創建\cnglo{memobj}時,\carg{host_ptr} 的對齊規則請參考\todo{section C.3}。
}

\clED{CL_MEM_ALLOC_HOST_PTR}{
    表明\cnglo{app}想讓 OpenCL 實作在\cnglo{host}可以訪問的內存中分配內存。

    它與 \cenum{CL_MEM_USE_HOST_PTR} 互斥。
}

\clED{CL_MEM_COPY_HOST_PTR}{
  僅當 \carg{host_ptr} 不是 \cenum{NULL} 時才有效。
  它表明\cnglo{app}想讓 OpenCL 實作使用 \carg{host_ptr} 所引用的內存來為\cnglo{memobj}分配內存並拷貝數據。

  它與 \cenum{CL_MEM_USE_HOST_PTR} 互斥。

  它與 \cenum{CL_MEM_ALLOC_HOST_PTR} 一起使用時,可以對由\cnglo{host}可訪問的內存(如 PCIe )分配的 \ctype{cl_mem} 對象進行初始化。
}

\clED{CL_MEM_HOST_WRITE_ONLY}{
  表明\cnglo{host}只會對此\cnglo{memobj}進行寫入。
  可用來對\cnglo{host}的寫操作進行優化(如對於與\cnglo{host}通過系統總線如 PCIe 進行通信的\cnglo{device},分配\cnglo{memobj}時使能 Write-combining)。
}

\clED{CL_MEM_HOST_READ_ONLY}{
  表明\cnglo{host}只會對此\cnglo{memobj}進行讀取。

  它與 \cenum{CL_MEM_HOST_WRITE_ONLY} 互斥。
}

\clED{CL_MEM_HOST_NO_ACCESS}{
  表明\cnglo{host}不會對此\cnglo{memobj}進行讀寫。

  \cenum{CL_MEM_HOST_WRITE_ONLY} 或者 \cenum{CL_MEM_HOST_READ_ONLY} 都與 \cenum{CL_MEM_HOST_NO_ACCESS} 互斥。
}

\stopED

