\subsection{創建內核對象}

% clCreateKernel
使用如下函數創建\cnglo{kernelobj}:
\startclc
cl_kernel clCreateKernel (cl_program program,
			const char *kernel_name,
			cl_int *errcode_ret)
\stopclc

\carg{program} 是一個\cnglo{programobj},帶有成功構建的執行體。

\carg{kernel_name} 是\cnglo{program}中一個函數的名字,聲明時帶有限定符 \cqlf{__kernel}。

\carg{errcode_ret} 會返回相應的錯誤碼。
如果 \carg{errcode_ret} 是 \cenum{NULL},則不會返回錯誤碼。

如果成功創建了\cnglo{kernelobj},則 \capi{clCreateKernel} 會將其返回,
並將 \carg{errcode_ret} 置為 \cenum{CL_SUCCESS}。
否則,返回 \cenum{NULL},並將 \carg{errcode_ret} 置為下列錯誤碼之一:
\startigBase
\item \cenum{CL_INVALID_PROGRAM},如果 \carg{program} 無效。

\item \cenum{CL_INVALID_PROGRAM_EXECUTABLE},如果沒有為 \carg{program} 成功構建的執行體。

\item \cenum{CL_INVALID_KERNEL_NAME},如果 \carg{program} 中找不到 \carg{kernel_name}。

\item \cenum{CL_INVALID_KERNEL_DEFINITION},
如果 \carg{kernel_name} 的函數定義(如參數個數、參數型別)
在已經具有對應執行體的那些\cnglo{device}上有所區別。

\item \cenum{CL_INVALID_VALUE},如果 \carg{kernel_name} 是 \cenum{NULL}。

\item \cenum{CL_OUT_OF_RESOURCES},如果\scdevfailres。

\item \cenum{CL_OUT_OF_HOST_MEMORY},如果\schostfailres。
\stopigBase

% clCreateKernelsInProgram
函數
\startclc
cl_int clCreateKernelsInProgram (cl_program program,
			cl_uint num_kernels,
			cl_kernel *kernels,
			cl_uint *num_kernels_ret)
\stopclc
會為 \carg{program} 中的所有\cnglo{kernel}函數創建對應的\cnglo{kernelobj}。
如果 \cqlf{__kernel} 函數的定義在已經具有執行體的那些\cnglo{device}上不完全一樣,
則不會為其創建\cnglo{kernelobj}。

\carg{program} 是一個\cnglo{programobj},具有成功構建的執行體。

\carg{num_kernels} 即 \carg{kernels} 中 \ctype{cl_kernel} 的數目。

\carg{kernels} 用來存儲所返回的\cnglo{kernelobj}。
如果 \carg{kernels} 是 \cenum{NULL},則忽略;
否則, \carg{num_kernels} 的值必須大於或等於 \carg{program} 中\cnglo{kernel}的數目。

\carg{num_kernels_ret} 即 \carg{program} 中\cnglo{kernel}的數目。
如果 \carg{num_kernels_ret} 是 \cenum{NULL},則忽略。

如果執行成功, \capi{clCreateKernelsInProgram} 會返回 \cenum{CL_SUCCESS}。
否則,返回下列錯誤碼之一:
\startigBase
\item \cenum{CL_INVALID_PROGRAM},如果 \carg{program} 無效。

\item \cenum{CL_INVALID_PROGRAM_EXECUTABLE},
如果 \carg{program} 中的任一\cnglo{device}沒有對應的執行體。

\item \cenum{CL_INVALID_VALUE},
如果 \carg{kernels} 不是 \cenum{NULL},
或者 \carg{num_kernels} 的值小於 \carg{program} 中\cnglo{kernel}的數目。

\item \cenum{CL_OUT_OF_RESOURCES},如果\scdevfailres。

\item \cenum{CL_OUT_OF_HOST_MEMORY},如果\schostfailres。
\stopigBase

只有當已將有效的\cnglo{program}源碼或者二元碼加載到了\cnglo{programobj}中,
並且至少為一個此對象所關聯的\cnglo{device}成功構建了執行體時,才能創建\cnglo{kernelobj}。

對於一個\cnglo{programobj}而言,如果有關聯的\cnglo{kernelobj},
則不允許改變\cnglo{program}執行體,
也就是說此時調用 \capi{clBuildProgram} 或 \capi{clCompileProgram} 會返回 \cenum{CL_INVALID_OPERATION}。
 \carg{program} 所在的 OpenCL \cnglo{context}即為 \carg{kernel} 所在的\cnglo{context}。
 \carg{program} 所關聯的\cnglo{device}即為 \carg{kernel} 所關聯的\cnglo{device}。
對於\cnglo{programobj}中的\cnglo{device}而言,只要有對應的\cnglo{program}執行體,
就可以執行此對象中聲明的\cnglo{kernel}。

% clRetainKernel
函數
\startclc
cl_int clRetainKernel (cl_kernel kernel)
\stopclc
會使 \carg{kernel} 的\cnglo{refcnt}增一。
執行成功後會返回 \cenum{CL_SUCCESS}。
否則,返回下列錯誤的一種:
\startigBase
\item \cenum{CL_INVALID_KERNEL},如果 \carg{kernel} 無效。

\item \cenum{CL_OUT_OF_RESOURCES}——如果\scdevfailres。

\item \cenum{CL_OUT_OF_HOST_MEMORY}——如果\schostfailres。
\stopigBase

\capi{clCreateKernel} 或 \capi{clCreateKernelsInProgram} 會實施隱式的\cnglo{retain}。

% clReleaseKernel
函數
\startclc
cl_int clReleaseKernel (cl_kernel kernel)
\stopclc
會使 \carg{kernel} 的\cnglo{refcnt}減一。
執行成功後會返回 \cenum{CL_SUCCESS}。
否則,返回下列錯誤的一種:
\startigBase
\item \cenum{CL_INVALID_KERNEL},如果 \carg{kernel} 無效。

\item \cenum{CL_OUT_OF_RESOURCES},如果\scdevfailres。

\item \cenum{CL_OUT_OF_HOST_MEMORY},如果\schostfailres。
\stopigBase

當 \carg{kernel} 的\cnglo{refcnt}變成零,隊列中的所有\cnglo{cmd}都不再需要此對象時,
此對象就會被刪除。

