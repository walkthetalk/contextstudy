\subsection{設置內核參數}

要想執行\cnglo{kernel},必須設置\cnglo{kernel}參數。

函式
\startCLFUNC
cl_int clSetKernelArg (cl_kernel kernel,
			cl_uint arg_index,
			size_t arg_size,
			const void *arg_value)
\stopCLFUNC
可用來設置\cnglo{kernel}的某個參數。

\carg{kernel} 是一個\cnglo{kernelobj}。

\carg{arg_index} 是參數索引。
\cnglo{kernel}參數的索引值從最左邊的 0 開始,一直到\math{n - 1},
其中\math{n}是參數的總數。

例如,試想如下內核:
\startclc
kernel void
image_filter (int n, int m,
		__constant float *filter_weights,
		__read_only image2d_t src_image,
		__write_only image2d_t dst_image)
{
	...
}
\stopclc

在 \ccmm{image_filter} 中,參數 \ccmm{n}、 \ccmm{m}、 \ccmm{filter_weights}、
 \ccmm{src_image}、 \ccmm{dst_image} 的索引分別為 0、 1、 2、 3、 4。

\carg{arg_value} 所指數據用作索引為 \carg{arg_index} 的參數的值。
\capi{clSetKernelArg} 返回後, \carg{arg_value} 所指數據已經被複製,
其內存可以由\cnglo{app}重新使用。
所有會將 \carg{kernel} 插入隊列的 API 調用
( \capi{clEnqueueNDRangeKernel} 和 \capi{clEnqueueTask})都會使用這個參數值,
直到再次調用 \capi{clSetKernelArg} 將其改變。

如果此參數是一個\cnglo{memobj}(緩衝、圖像或圖像陣列),
 \carg{arg_value} 會指向相應的對象。
此對象必須是用 \carg{kernel} 所在的\cnglo{context}創建的。
如果參數是一個\cnglo{bufobj}, \carg{arg_value} 可以是 \cmacro{NULL},
也可以指向一個為 \cmacro{NULL} 的值,這時,
如果內核參數聲明時帶有限定符 \cqlf{__global} 或 \cqlf{__constant},
則使用 \cmacro{NULL} 值作為參數值。
而如果參數聲明時所帶限定符為 \cqlf{__local}, \carg{arg_value} 必須是 \cmacro{NULL}。
如果參數型別是 \ctype{sampler_t},則 \carg{arg_value} 必須指向\cnglo{sampler}對象。

如果參數是一個指針,指向的是全局或不變位址空間中的內建標量或矢量型別、或用戶自定義結構體,
則作為參數值的\cnglo{memobj}必須是一個\cnglo{bufobj}(或 \cmacro{NULL})。
如果參數聲明時帶有限定符 \cqlf{__constant},
則\cnglo{memobj}的大小不能超過 \cenum{CL_DEVICE_MAX_CONSTANT_BUFFER_SIZE},
且這種參數的個數不能超過 \cenum{CL_DEVICE_MAX_CONSTANT_ARGS}。

如果參數型別為 \ctype{image2d_t},則作為參數值的\cnglo{memobj}必須是 2D \cnglo{imgobj}。
如果參數型別為 \ctype{image3d_t},則作為參數值的\cnglo{memobj}必須是 3D \cnglo{imgobj}。
如果參數型別為 \ctype{image1d_t},則作為參數值的\cnglo{memobj}必須是 1D \cnglo{imgobj}。
如果參數型別為 \ctype{image1d_buffer_t},則作為參數值的\cnglo{memobj}必須是 1D 圖像\cnglo{bufobj}。
如果參數型別為 \ctype{image1d_array_t},則作為參數值的\cnglo{memobj}必須是 1D 圖像陣列對象。
如果參數型別為 \ctype{image2d_array_t},則作為參數值的\cnglo{memobj}必須是 2D 圖像陣列對象。

對於其他型別的參數, \carg{arg_value} 必須指向作為參數值的實際數據。

\carg{arg_size} 即參數值的大小。
如果參數是\cnglo{memobj},則為\cnglo{bufobj}或\cnglo{imgobj}的大小。
如果參數在聲明時帶有限定符 \cqlf{__local},其值將是為 \cqlf{__local} 參數所分配緩存的大小。
如果參數型別是 \ctype{sampler_t},則 \carg{arg_size} 的值必須是 \ccmm{sizeof(cl_sampler)}。
對於其他型別的參數, \carg{arg_size} 即為參數型別的大小。

注意:

\startbuffer[footnotesetkernelargrefcnt]
實作不能讓 \ctype{cl_kernel} 對象擁有其參數的\cnglo{refcnt},
因為沒有為用戶提供任何機制來告訴\cnglo{kernel}去釋放此所有權。
如果\cnglo{kernel}把持了其參數的所有權,用戶就不可能確切的知道什麼時候可以安全的釋放為其分配的資源,
如用 \cenum{CL_MEM_USE_HOST_PTR} 對 \ctype{cl_mem} 進行回寫。
\stopbuffer

對於 \capi{clSetKernelArg} 所使用的作為參數值的對象(如\cnglo{memobj}、\cnglo{sampler}對象)而言,
\cnglo{kernelobj}不會更新其\cnglo{refcnt}。
用戶不要指望\cnglo{kernelobj}會為\cnglo{kernel}參數中的對象執行\cnglo{retain}操作
\footnote{\getbuffer[footnotesetkernelargrefcnt]}。

如果執行成功, \capi{clSetKernelArg} 會返回 \cenum{CL_SUCCESS}。
否則,返回下列錯誤碼之一:
\startigBase
\item \cenum{CL_INVALID_KERNEL},如果 \carg{kernel} 無效。

\item \cenum{CL_INVALID_ARG_INDEX},如果 \carg{arg_index} 無效。

\item \cenum{CL_INVALID_ARG_VALUE},如果 \carg{arg_value} 無效。

\item \cenum{CL_INVALID_MEM_OBJECT},
如果參數型別是\cnglo{memobj},而 \carg{arg_value} 無效。

\item \cenum{CL_INVALID_SAMPLER},
如果參數型別是 \ctype{sampler_t},而 \carg{arg_value} 無效。

\item \cenum{CL_INVALID_ARG_SIZE},
如果參數不是\cnglo{memobj},而 \carg{arg_size} 的值與參數大小不一致;
或者參數是\cnglo{memobj},而 \ccmm{arg_size != sizeof(cl_mem)};
或者參數聲明時帶有限定符 \cqlf{__local},而 \carg{arg_size} 是零;
或者參數是\cnglo{sampler},而 \ccmm{arg_size != sizeof(cl_sampler)}。

\item \cenum{CL_INVALID_ARG_VALUE},
如果參數是帶有限定符 \cqlf{read_only} 的圖像,而創建\cnglo{imgobj} \carg{arg_value} 時在 \ctype{cl_mem_flags} 中設置了 \cenum{CL_MEM_WRITE}。
或者參數是帶有限定符 \cqlf{write_only} 的圖像,而創建\cnglo{imgobj} \carg{arg_value} 時在 \ctype{cl_mem_flags} 中設置了 \cenum{CL_MEM_READ}。

\item \cenum{CL_OUT_OF_RESOURCES},如果\scdevfailres。

\item \cenum{CL_OUT_OF_HOST_MEMORY},如果\schostfailres。
\stopigBase
