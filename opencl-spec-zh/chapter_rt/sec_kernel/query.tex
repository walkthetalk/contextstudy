\subsection{內核對象的相關查詢}

% clGetKernelInfo
函式
\startclc
cl_int clGetKernelInfo (cl_kernel kernel,
			cl_kernel_info param_name,
			size_t param_value_size,
			void *param_value,
			size_t *param_value_size_ret)
\stopclc
會返回\cnglo{kernelobj}的相關信息。

\carg{kernel} 即所要查詢的\cnglo{kernel}。

\carg{param_name} 指定要查詢什麼信息。
對於所支持的信息類型以及 \carg{param_value} 中返回的信息,請參見\reftab{clGetKernelInfo}。

\carg{param_value} 指向的內存用來存儲查詢結果。
如果 \carg{param_value} 是 \cenum{NULL},則忽略。

\carg{param_value_size} 即 \carg{param_value} 所指內存塊的大小。
其值必須 >= \reftab{clGetKernelInfo}中返回型別的大小。

\carg{param_value_size_ret} 返回查詢結果的實際大小。
如果 \carg{param_value_size_ret} 是 \cenum{NULL},則忽略。

\startbuffer[tblclGetKernelInfo]
\capi{clGetKernelInfo} 所支持的 \carg{param_names}
\stopbuffer
\placetable[here,force][tab:clGetKernelInfo]{\getbuffer[tblclGetKernelInfo]}
{\input{chapter_rt/tbl/tbl_clgetkernelinfo.tex}}

如果執行成功, \capi{clGetKernelInfo} 會返回 \cenum{CL_SUCCESS}。
否則,返回下列錯誤碼之一:
\startigBase
\item \cenum{CL_INVALID_VALUE},如果 \carg{param_name} 無效;
或者 \carg{param_value_size} 的值 < \reftab{clGetKernelInfo}中返回型別的大小,
且 \carg{param_value} 不是 \cenum{NULL}。

\item \cenum{CL_INVALID_KERNEL},如果 \carg{kernel} 無效。

\item \cenum{CL_OUT_OF_RESOURCES},如果\scdevfailres。

\item \cenum{CL_OUT_OF_HOST_MEMORY},如果\schostfailres。
\stopigBase

% clGetKernelWorkGroupInfo
函式
\startclc
cl_int clGetKernelWorkGroupInfo (cl_kernel kernel,
			cl_device_id device,
			cl_kernel_work_group_info param_name,
			size_t param_value_size,
			void *param_value,
			size_t *param_value_size_ret)
\stopclc
會返回\cnglo{kernelobj}針對某個\cnglo{device}的信息。

\carg{kernel} 即所要查詢的\cnglo{kernel}。

\carg{device} 是 \carg{kernel} 所關聯的\cnglo{device}之一。
 \carg{kernel} 所關聯的\cnglo{device}即 \carg{kernel} 所在\cnglo{context}中的\cnglo{device}。
如果 \carg{kernel} 所關聯的\cnglo{device}只有一個,則 \carg{device} 可以是 \cenum{NULL}。

\carg{param_name} 指定要查詢什麼信息。
對於所支持的信息類型以及 \carg{param_value} 中返回的信息,請參見\reftab{clGetKernelWorkGroupInfo}。

\carg{param_value} 指向的內存用來存儲查詢結果。
如果 \carg{param_value} 是 \cenum{NULL},則忽略。

\carg{param_value_size} 即 \carg{param_value} 所指內存塊的大小。
其值必須 >= \reftab{clGetKernelWorkGroupInfo}中返回型別的大小。

\carg{param_value_size_ret} 返回查詢結果的實際大小。
如果 \carg{param_value_size_ret} 是 \cenum{NULL},則忽略。

\startbuffer[tblclGetKernelWorkGroupInfo]
\capi{clGetKernelWorkGroupInfo} 所支持的 \carg{param_names}
\stopbuffer
\splitfloat{
\placetable[here,force][tab:clGetKernelWorkGroupInfo]{\getbuffer[tblclGetKernelWorkGroupInfo]}
}{
{\input{chapter_rt/tbl/tbl_clgetkernelworkgroupinfo.tex}}
}

如果執行成功, \capi{clGetKernelWorkGroupInfo} 會返回 \cenum{CL_SUCCESS}。
否則,返回下列錯誤碼之一:
\startigBase
\item \cenum{CL_INVALID_DEVICE},
如果 \carg{device} 不是 \carg{kernel} 所關聯的\cnglo{device};
或者 \carg{device} 是 \cenum{NULL},但 \carg{kernel} 所關聯的\cnglo{device}多於一個。

\item \cenum{CL_INVALID_VALUE},如果 \carg{param_name} 無效;
或者 \carg{param_value_size} 的值 < \reftab{clGetKernelWorkGroupInfo}中返回型別的大小,
且 \carg{param_value} 不是 \cenum{NULL}。

\item \cenum{CL_INVALID_VALUE},
如果 \carg{param_name} 是 \cenum{CL_KERNEL_GLOBAL_WORK_SIZE},
且 \carg{device} 不是自定義\cnglo{device}或 \carg{kernel} 不是內建\cnglo{kernel}。

\item \cenum{CL_INVALID_KERNEL},如果 \carg{kernel} 無效。

\item \cenum{CL_OUT_OF_RESOURCES},如果\scdevfailres。

\item \cenum{CL_OUT_OF_HOST_MEMORY},如果\schostfailres。
\stopigBase

% clGetKernelArgInfo
函式
\startclc
cl_int clGetKernelArgInfo (cl_kernel kernel,
			cl_uint arg_indx,
			cl_kernel_arg_info param_name,
			size_t param_value_size,
			void *param_value,
			size_t *param_value_size_ret)
\stopclc
會返回\cnglo{kernel}參數的相關信息。
只有滿足下列條件時,\cnglo{kernel}參數信息才可用:
\startigBase
\item \carg{kernel} 所關聯的\cnglo{programobj}是用
 \capi{clCreateProgramWithSource} 創建的;

\item 用 \capi{cl{Build | Compile}Program} 構建\cnglo{program}執行體時,
在參數 \carg{options} 中指定了 \ccmm{-cl-kernel-arg-info}。
\stopigBase

\carg{kernel} 即所要查詢的\cnglo{kernelobj}。

\carg{arg_indx} 即參數的索引。\cnglo{kernel}參數的索引從最左邊的 0 一直到\math{n - 1},
其中\math{n}是\cnglo{kernel}參數的總數。

\carg{param_name} 指定要查詢什麼信息。
對於所支持的信息類型以及 \carg{param_value} 中返回的信息,請參見\reftab{clGetKernelArgInfo}。

\carg{param_value} 指向的內存用來存儲查詢結果。
如果 \carg{param_value} 是 \cenum{NULL},則忽略。

\carg{param_value_size} 即 \carg{param_value} 所指內存塊的大小。
其值必須 >= \reftab{clGetKernelArgInfo}中返回型別的大小。

\carg{param_value_size_ret} 返回查詢結果的實際大小。
如果 \carg{param_value_size_ret} 是 \cenum{NULL},則忽略。

\startbuffer[tblclGetKernelArgInfo]
\capi{clGetKernelArgInfo} 所支持的 \carg{param_names}
\stopbuffer
\splitfloat{
\placetable[here,force][tab:clGetKernelArgInfo]{\getbuffer[tblclGetKernelArgInfo]}
}{
{\startETD[cl_kernel_arg_info][返回類型]

\clETD{CL_KERNEL_ARG_ADDRESS_QUALIFIER}{cl_kernel_arg_address_qualifier}{
返回參數的地址限定符。所返回的值可以是下列之一:
\startigBase
\item \cenum{CL_KERNEL_ARG_ADDRESS_GLOBAL}
\item \cenum{CL_KERNEL_ARG_ADDRESS_LOCAL}
\item \cenum{CL_KERNEL_ARG_ADDRESS_CONSTANT}
\item \cenum{CL_KERNEL_ARG_ADDRESS_PRIVATE}
\stopigBase

如果沒有指定地址限定符,則返回缺省的地址限定符 \cenum{CL_KERNEL_ARG_ADDRESS_PRIVATE}。
}

\clETD{CL_KERNEL_ARG_ACCESS_QUALIFIER}{cl_kernel_arg_access_qualifier}{
返回參數的訪問限定符。所返回的值可以是下列之一:
\startigBase
\item \cenum{CL_KERNEL_ARG_ACCESS_READ_ONLY}
\item \cenum{CL_KERNEL_ARG_ACCESS_WRITE_ONLY}
\item \cenum{CL_KERNEL_ARG_ACCESS_READ_WRITE}
\item \cenum{CL_KERNEL_ARG_ACCESS_NONE}
\stopigBase

如果參數類型不是圖像,則返回 \cenum{CL_KERNEL_ARG_ACCESS_NONE}。
如果參數類型是圖像,則會返回所指定的訪問限定符或缺省的訪問限定符。
}

\clETD{CL_KERNEL_ARG_TYPE_NAME}{char[]}{
返回參數的類型名,即所聲明的類型名(會移除空白)。
如果參數是無符號標量類型(即 unsigned char, unsigned short, unsigned int, unsigned long),
則會返回 uchar, ushort, uint 和 ulong。
所返回的類型名中不包括任何類型限定符。
}

\clETD{CL_KERNEL_ARG_TYPE_QUALIFIER}{cl_kernel_arg_type_qualifier}{
返回參數的類型限定符。可能是:
\startigBase
\item \cenum{CL_KERNEL_ARG_TYPE_CONST}
\item \cenum{CL_KERNEL_ARG_TYPE_RESTRICT}
\item \cenum{CL_KERNEL_ARG_TYPE_VOLATILE}
\item 以上三種的組合;或者
\item \cenum{CL_KERNEL_ARG_TYPE_NONE}
\stopigBase

注意:如果參數是指針,且所引用的類型聲明時帶有限定符 \cqlf{volatile},
則會返回 \cenum{CL_KERNEL_ARG_TYPE_VOLATILE}。
例如,如果\cnglo{kernel}參數聲明為 \ccmm{global int volatile *x},
則會返回 \cenum{CL_KERNEL_ARG_TYPE_VOLATILE};
而對於聲明為 \ccmm{global int *volatile x} 的\cnglo{kernel}參數,
則不會返回 \cenum{CL_KERNEL_ARG_TYPE_VOLATILE}。
}

\clETD{CL_KERNEL_ARG_NAME}{char[]}{
返回參數的名字。
}

\stopETD
}
}

如果執行成功, \capi{clGetKernelArgInfo} 會返回 \cenum{CL_SUCCESS}。
否則,返回下列錯誤碼之一:
\startigBase
\item \cenum{CL_INVALID_ARG_INDEX},如果 \carg{arg_indx} 無效。

\item \cenum{CL_INVALID_VALUE},如果 \carg{param_name} 無效;
或者 \carg{param_value_size} 的值 < \reftab{clGetKernelArgInfo}中返回型別的大小,
且 \carg{param_value} 不是 \cenum{NULL}。

\item \cenum{CL_KERNEL_ARG_INFO_NOT_AVAILABLE},如果沒有可用的參數信息。

\item \cenum{CL_INVALID_KERNEL},如果 \carg{kernel} 無效。
\stopigBase
