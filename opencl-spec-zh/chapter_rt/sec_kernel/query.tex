\subsection{內核對象的相關查詢}

\topclfunc{clGetKernelInfo}

\startCLFUNC
cl_int clGetKernelInfo (cl_kernel kernel,
			cl_kernel_info param_name,
			size_t param_value_size,
			void *param_value,
			size_t *param_value_size_ret)
\stopCLFUNC

此函式會返回\cnglo{kernelobj}的相關資訊。

\carg{kernel} 即所要查詢的\cnglo{kernel}。

\carg{param_name} 指定要查詢什麼資訊。
\reftab{clGetKernelInfo}中列出了所支持的資訊類型以及 \carg{param_value} 中返回的資訊。

\carg{param_value} 指向的內存用來存儲查詢結果。
如果 \carg{param_value} 是 \cmacro{NULL},則忽略。

\carg{param_value_size} 即 \carg{param_value} 所指內存塊的大小。
其值必須 >= \reftab{clGetKernelInfo}中返回型別的大小。

\carg{param_value_size_ret} 返回查詢結果的實際大小。
如果 \carg{param_value_size_ret} 是 \cmacro{NULL},則忽略。

\placetable[here][tab:clGetKernelInfo]
{\capi{clGetKernelInfo} 所支持的 \carg{param_names}}
{\startETD[cl_kernel_info][返回型別]

\clETD{CL_KERNEL_FUNCTION_NAME}{char[]}{
返回\cnglo{kernel}函數的名字。
}

\clETD{CL_KERNEL_NUM_ARGS}{cl_uint}{
返回 \carg{kernel} 的參數個數。
}

\clETD{CL_KERNEL_REFERENCE_COUNT}{cl_uint}{
返回 \carg{kernel} 的\cnglo{refcnt}
\footnote{在返回的那一刻,此引用計數就已過時。
應用中一般不太適用。提供此特性主要是為了檢測內存泄漏。}。
}

\clETD{CL_KERNEL_CONTEXT}{cl_context}{
返回 \carg{kernel} 所在的\cnglo{context}。
}

\clETD{CL_KERNEL_PROGRAM}{cl_program}{
返回 \carg{kernel} 所關聯的\cnglo{programobj}。
}

\clETD{CL_KERNEL_ATTRIBUTES}{char[]}{
返回\cnglo{program}源碼中聲明\cnglo{kernel}函數時
所有通過限定符 \cqlf{__attribute__} 指定的特性。
這些特性包括\todo{6.11.2}中所列特性以及實現所支持的其它特性。

所返回的特性就是聲明時 \ccmm{__attribute__((...))} 中的內容,
但是會移除兩頭的空格以及內嵌的換行。
如果有多個特性,則在所返回的字符串中以空格來分隔。
}

\stopETD
}

如果執行成功, \capi{clGetKernelInfo} 會返回 \cenum{CL_SUCCESS}。
否則,返回下列錯誤碼之一:
\startigBase
\item \cenum{CL_INVALID_VALUE},如果 \carg{param_name} 無效;
或者 \carg{param_value_size} 的值 < \reftab{clGetKernelInfo}中返回型別的大小,
且 \carg{param_value} 不是 \cmacro{NULL}。

\item \cenum{CL_INVALID_KERNEL},如果 \carg{kernel} 無效。

\item \cenum{CL_OUT_OF_RESOURCES},如果\scdevfailres。

\item \cenum{CL_OUT_OF_HOST_MEMORY},如果\schostfailres。
\stopigBase

\topclfunc{clGetKernelWorkGroupInfo}

\startCLFUNC
cl_int clGetKernelWorkGroupInfo (cl_kernel kernel,
			cl_device_id device,
			cl_kernel_work_group_info param_name,
			size_t param_value_size,
			void *param_value,
			size_t *param_value_size_ret)
\stopCLFUNC

此函式會返回\cnglo{kernelobj}針對某個\cnglo{device}的資訊。

\carg{kernel} 即所要查詢的\cnglo{kernel}。

\carg{device} 是 \carg{kernel} 所關聯的\cnglo{device}之一。
\carg{kernel} 所關聯的\cnglo{device}即 \carg{kernel} 所在\cnglo{context}中的\cnglo{device}。
如果 \carg{kernel} 所關聯的\cnglo{device}只有一個,則 \carg{device} 可以是 \cmacro{NULL}。

\carg{param_name} 指定要查詢什麼資訊。
\reftab{clGetKernelWorkGroupInfo}中列出了所支持的資訊類型以及 \carg{param_value} 中返回的資訊。

\carg{param_value} 指向的內存用來存儲查詢結果。
如果 \carg{param_value} 是 \cmacro{NULL},則忽略。

\carg{param_value_size} 即 \carg{param_value} 所指內存塊的大小。
其值必須 >= \reftab{clGetKernelWorkGroupInfo}中返回型別的大小。

\carg{param_value_size_ret} 返回查詢結果的實際大小。
如果 \carg{param_value_size_ret} 是 \cmacro{NULL},則忽略。

\placetable[here][tab:clGetKernelWorkGroupInfo]
{\capi{clGetKernelWorkGroupInfo} 所支持的 \carg{param_names}}
{\startETD[cl_kernel_work_group_info][返回型別]

\clETD{CL_KERNEL_GLOBAL_WORK_SIZE}{size_t[3]}{
利用此機制,
\cnglo{app}可以查詢用來在 \carg{device} 上執行\cnglo{kernel}的全局索引空間的大小
(即 \capi{clEnqueueNDRangeKernel} 的引數 \carg{global_work_size})。

這要求 \carg{device} 是\cnglo{customdev}或所執行的\cnglo{kernel}是內建的,
否則 \capi{clGetKernelWorkGroupInfo} 會返回 \cenum{CL_INVALID_VALUE}。
}

\clETD{CL_KERNEL_WORK_GROUP_SIZE}{size_t}{
利用此機制,
\cnglo{app}可以查詢用來在 \carg{device} 上執行\cnglo{kernel}的\cnglo{workgrp}的大小。
OpenCL 實作可以用\cnglo{kernel}的資源需求(寄存器的使用情況等)來確定\cnglo{workgrp}的大小。
}

\clETD{CL_KERNEL_COMPILE_WORK_GROUP_SIZE}{size_t[3]}{
返回限定符 \ccmm{__attribute__((reqd_work_group_size(X, Y, Z)))} 所指定的
\cnglo{workgrp}的大小。參見\refsec{optAttrQlf}。

如果沒有用上述特性限定符指定\cnglo{workgrp}的大小,則返回\math{(0, 0, 0)}。
}

\clETD{CL_KERNEL_LOCAL_MEM_SIZE}{cl_ulong}{
返回\cnglo{kernel}所使用的\cnglo{locmem}的大小。
他包括實作執行\cnglo{kernel}所需的內存、
\cnglo{kernel}中所聲明的帶有位址限定符 \cqlf{__local} 的變量、
以及為帶有位址限定符 \cqlf{__local} 的指針引數所分配的內存
(其大小由 \capi{clSetKernelArg} 指定)。

對於任一帶有位址限定符 \cqlf{__local} 的指針引數,如果沒有為其指定內存大小,則假定為 0。
}

\clETD{CL_KERNEL_PREFERRED_WORK_GROUP_SIZE_MULTIPLE}{size_t}{
返回所期望的\cnglo{workgrp}大小的粒度。僅為性能建議。
在調用 \capi{clEnqueueNDRangeKernel} 時,
如果給引數 \carg{local_work_size} 所指定的\cnglo{workgrp}大小不是此查詢結果的倍數,
並不會導致函式執行失敗,除非此值超過了\cnglo{device}的數目。
}

\clETD{CL_KERNEL_PRIVATE_MEM_SIZE}{cl_ulong}{
\cnglo{kernel}中每個\cnglo{workitem}至少要使用多少\cnglo{prvmem}。
他包括實作執行\cnglo{kernel}所需的所有\cnglo{prvmem};
包括語言本身所需的\cnglo{prvmem},以及\cnglo{kernel}中聲明的 \cqlf{__private} 變量。
}

\stopETD
}

如果執行成功, \capi{clGetKernelWorkGroupInfo} 會返回 \cenum{CL_SUCCESS}。
否則,返回下列錯誤碼之一:
\startigBase
\item \cenum{CL_INVALID_DEVICE},
如果 \carg{device} 不是 \carg{kernel} 所關聯的\cnglo{device};
或者 \carg{device} 是 \cmacro{NULL},但 \carg{kernel} 所關聯的\cnglo{device}多於一個。

\item \cenum{CL_INVALID_VALUE},如果 \carg{param_name} 無效;
或者 \carg{param_value_size} 的值 < \reftab{clGetKernelWorkGroupInfo}中返回型別的大小,
且 \carg{param_value} 不是 \cmacro{NULL}。

\item \cenum{CL_INVALID_VALUE},
如果 \carg{param_name} 是 \cenum{CL_KERNEL_GLOBAL_WORK_SIZE},
且 \carg{device} 不是自定義\cnglo{device}或 \carg{kernel} 不是內建\cnglo{kernel}。

\item \cenum{CL_INVALID_KERNEL},如果 \carg{kernel} 無效。

\item \cenum{CL_OUT_OF_RESOURCES},如果\scdevfailres。

\item \cenum{CL_OUT_OF_HOST_MEMORY},如果\schostfailres。
\stopigBase

\topclfunc{clGetKernelArgInfo}

\startCLFUNC
cl_int clGetKernelArgInfo (cl_kernel kernel,
			cl_uint arg_indx,
			cl_kernel_arg_info param_name,
			size_t param_value_size,
			void *param_value,
			size_t *param_value_size_ret)
\stopCLFUNC

此函式會返回\cnglo{kernel}引數的相關資訊。
只有滿足下列條件時,\cnglo{kernel}引數資訊才可用:
\startigBase
\item \carg{kernel} 所關聯的\cnglo{programobj}是用
 \capi{clCreateProgramWithSource} 創建的;

\item 用 \capi{cl{Build | Compile}Program} 構建\cnglo{program}執行體時,
在引數 \carg{options} 中指定了 \ccmm{-cl-kernel-arg-info}。
\stopigBase

\carg{kernel} 即所要查詢的\cnglo{kernelobj}。

\carg{arg_indx} 即引數的索引。\cnglo{kernel}引數的索引從最左邊的 0 一直到\math{n - 1},
其中\math{n}是\cnglo{kernel}引數的總數。

\carg{param_name} 指定要查詢什麼資訊。
\reftab{clGetKernelArgInfo}中列出了所支持的資訊類型以及 \carg{param_value} 中返回的資訊。

\carg{param_value} 指向的內存用來存儲查詢結果。
如果 \carg{param_value} 是 \cmacro{NULL},則忽略。

\carg{param_value_size} 即 \carg{param_value} 所指內存塊的大小。
其值必須 >= \reftab{clGetKernelArgInfo}中返回型別的大小。

\carg{param_value_size_ret} 返回查詢結果的實際大小。
如果 \carg{param_value_size_ret} 是 \cmacro{NULL},則忽略。

\placetable[here][tab:clGetKernelArgInfo]
{\capi{clGetKernelArgInfo} 所支持的 \carg{param_names}}
{\startETD[cl_kernel_arg_info][返回型別]

\clETD{CL_KERNEL_ARG_ADDRESS_QUALIFIER}{cl_kernel_arg_address_qualifier}{
返回參數的位址限定符。所返回的值可以是下列之一:
\startigBase
\item \cenum{CL_KERNEL_ARG_ADDRESS_GLOBAL}
\item \cenum{CL_KERNEL_ARG_ADDRESS_LOCAL}
\item \cenum{CL_KERNEL_ARG_ADDRESS_CONSTANT}
\item \cenum{CL_KERNEL_ARG_ADDRESS_PRIVATE}
\stopigBase

如果沒有指定位址限定符,則返回缺省的位址限定符 \cenum{CL_KERNEL_ARG_ADDRESS_PRIVATE}。
}

\clETD{CL_KERNEL_ARG_ACCESS_QUALIFIER}{cl_kernel_arg_access_qualifier}{
返回參數的存取限定符。所返回的值可以是下列之一:
\startigBase
\item \cenum{CL_KERNEL_ARG_ACCESS_READ_ONLY}
\item \cenum{CL_KERNEL_ARG_ACCESS_WRITE_ONLY}
\item \cenum{CL_KERNEL_ARG_ACCESS_READ_WRITE}
\item \cenum{CL_KERNEL_ARG_ACCESS_NONE}
\stopigBase

如果參數型別不是圖像,則返回 \cenum{CL_KERNEL_ARG_ACCESS_NONE}。
如果參數型別是圖像,則會返回所指定的存取限定符或缺省的存取限定符。
}

\clETD{CL_KERNEL_ARG_TYPE_NAME}{char[]}{
返回參數的型別名,即所聲明的型別名(會移除空白)。
如果參數是無符號標量型別(即 unsigned char, unsigned short, unsigned int, unsigned long),
則會返回 uchar, ushort, uint 和 ulong。
所返回的型別名中不包括任何型別限定符。
}

\clETD{CL_KERNEL_ARG_TYPE_QUALIFIER}{cl_kernel_arg_type_qualifier}{
返回參數的型別限定符。可能是:
\startigBase
\item \cenum{CL_KERNEL_ARG_TYPE_CONST}
\item \cenum{CL_KERNEL_ARG_TYPE_RESTRICT}
\item \cenum{CL_KERNEL_ARG_TYPE_VOLATILE}
\item 以上三種的組合;或者
\item \cenum{CL_KERNEL_ARG_TYPE_NONE}
\stopigBase

注意:如果參數是指針,且所引用的型別聲明時帶有限定符 \cqlf{volatile},
則會返回 \cenum{CL_KERNEL_ARG_TYPE_VOLATILE}。
例如,如果\cnglo{kernel}參數聲明為 \ccmm{global int volatile *x},
則會返回 \cenum{CL_KERNEL_ARG_TYPE_VOLATILE};
而對於聲明為 \ccmm{global int *volatile x} 的\cnglo{kernel}參數,
則不會返回 \cenum{CL_KERNEL_ARG_TYPE_VOLATILE}。
}

\clETD{CL_KERNEL_ARG_NAME}{char[]}{
返回參數的名字。
}

\stopETD
}

如果執行成功, \capi{clGetKernelArgInfo} 會返回 \cenum{CL_SUCCESS}。
否則,返回下列錯誤碼之一:
\startigBase
\item \cenum{CL_INVALID_ARG_INDEX},如果 \carg{arg_indx} 無效。

\item \cenum{CL_INVALID_VALUE},如果 \carg{param_name} 無效;
或者 \carg{param_value_size} 的值 < \reftab{clGetKernelArgInfo}中返回型別的大小,
且 \carg{param_value} 不是 \cmacro{NULL}。

\item \cenum{CL_KERNEL_ARG_INFO_NOT_AVAILABLE},如果沒有可用的引數資訊。

\item \cenum{CL_INVALID_KERNEL},如果 \carg{kernel} 無效。
\stopigBase
