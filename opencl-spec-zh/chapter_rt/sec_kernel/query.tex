\subsection{內核對象的相關查詢}

函數
\startclc
cl_int clGetKernelInfo (cl_kernel kernel,
			cl_kernel_info param_name,
			size_t param_value_size,
			void *param_value,
			size_t *param_value_size_ret)
\stopclc
會返回\cnglo{kernelobj}的相關信息。

\carg{kernel} 即所要查詢的\cnglo{kernel}。

\carg{param_name} 指定要查詢什麼信息。
對於所支持的信息類型以及 \carg{param_value} 中返回的信息,請參見\reftab{clGetKernelInfo}。

\carg{param_value} 指向的內存用來存儲查詢結果。
如果 \carg{param_value} 是 \cenum{NULL},則忽略。

\carg{param_value_size} 即 \carg{param_value} 所指內存塊的大小。
其值必須 >= \reftab{clGetKernelInfo}中返回類型的大小。

\carg{param_value_size_ret} 返回查詢結果的實際大小。
如果 \carg{param_value_size_ret} 是 \cenum{NULL},則忽略。

\startbuffer[tblclGetKernelInfo]
\capi{clGetKernelInfo} 所支持的 \carg{param_names}
\stopbuffer
\placetable[here,force][tab:clGetKernelInfo]{\getbuffer[tblclGetKernelInfo]}
{\startETD[cl_kernel_info][返回型別]

\clETD{CL_KERNEL_FUNCTION_NAME}{char[]}{
返回\cnglo{kernel}函數的名字。
}

\clETD{CL_KERNEL_NUM_ARGS}{cl_uint}{
返回 \carg{kernel} 的參數個數。
}

\clETD{CL_KERNEL_REFERENCE_COUNT}{cl_uint}{
返回 \carg{kernel} 的\cnglo{refcnt}
\footnote{在返回的那一刻,此引用計數就已過時。
應用中一般不太適用。提供此特性主要是為了檢測內存泄漏。}。
}

\clETD{CL_KERNEL_CONTEXT}{cl_context}{
返回 \carg{kernel} 所在的\cnglo{context}。
}

\clETD{CL_KERNEL_PROGRAM}{cl_program}{
返回 \carg{kernel} 所關聯的\cnglo{programobj}。
}

\clETD{CL_KERNEL_ATTRIBUTES}{char[]}{
返回\cnglo{program}源碼中聲明\cnglo{kernel}函數時
所有通過限定符 \cqlf{__attribute__} 指定的特性。
這些特性包括\todo{6.11.2}中所列特性以及實現所支持的其它特性。

所返回的特性就是聲明時 \ccmm{__attribute__((...))} 中的內容,
但是會移除兩頭的空格以及內嵌的換行。
如果有多個特性,則在所返回的字符串中以空格來分隔。
}

\stopETD
}

如果執行成功, \capi{clGetKernelInfo} 會返回 \cenum{CL_SUCCESS}。
否則,返回下列錯誤碼之一:
\startigBase
\item \cenum{CL_INVALID_VALUE},如果 \carg{param_value} 無效;
或者 \carg{param_value_size} 的值 < \reftab{clGetKernelInfo}中返回類型的大小,
且 \carg{param_value} 是 \cenum{NULL}。

\item \cenum{CL_INVALID_KERNEL},如果 \carg{kernel} 無效。

\item \cenum{CL_OUT_OF_RESOURCES},如果\scdevfailres。

\item \cenum{CL_OUT_OF_HOST_MEMORY},如果\schostfailres。
\stopigBase
