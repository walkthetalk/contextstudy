\subsection{採樣器對象的相關查詢}

函數
\startclc
cl_int clGetSamplerInfo (cl_sampler sampler,
			cl_sampler_info param_name,
			size_t param_value_size,
			void *param_value,
			size_t *param_value_size_ret)
\stopclc
會返回\cnglo{sampler}對象的相關信息。

\carg{sampler} 即所要查詢的\cnglo{sampler}。

\carg{param_name} 指定要查詢什麼信息。所支持的信息類型以及 \carg{param_value} 中所返回的內容如\reftab{samplerinfo}所示。

\carg{param_value} 指向的內存用來存儲查詢結果。如果是 \cenum{NULL},則忽略。

\carg{param_value_size} 即 \carg{param_value} 所指內存塊的大小(單位:字節)。
其值必須 >= \reftab{samplerinfo}中返回型別的大小。

\carg{param_value_size_ret} 會返回查詢結果的實際大小。如果是 \cenum{NULL},則忽略。

\capi{clGetCommandQueueInfo} 所支持的 \carg{param_name} 的值以及 \carg{param_value} 中所返回的信息如\reftab{samplerinfo}所示。

\placetable[here,force][tab:samplerinfo]{\capi{clGetSamplerInfo}所支持的\carg{param_names}}
{\startETD[cl_sampler_info][返回类型]

\clETD{CL_SAMPLER_REFERENCE_COUNT}{cl_uint}{
返回 \carg{sampler} 的\cnglo{refcnt}。\footnote{%
在返回的那一刻,此\cnglo{refcnt}就已過時。
應用中一般不太適用。提供此特性主要是為了檢測內存泄漏。}
}

\clETD{CL_SAMPLER_CONTEXT}{cl_context}{
返回創建\cnglo{sampler}時所指定的\cnglo{context}。
}

\clETD{CL_SAMPLER_NORMALIZED_COORDS}{cl_bool}{
\carg{sampler} 的坐標是否歸一化。
}

\clETD{CL_SAMPLER_ADDRESSING_MODE}{cl_addressing_mode}{
返回 \carg{sampler} 的尋址模式。
}

\clETD{CL_SAMPLER_FILTER_MODE}{cl_filter_mode}{
返回 \carg{sampler} 的濾波模式。
}

\stopETD

}

如果執行成功,\capi{clGetSamplerInfo} 會返回 \cenum{CL_SUCCESS}。否則,返回下列錯誤碼之一:
\startigBase
\item \cenum{CL_INVALID_VALUE}——如果 \carg{param_name} 不在支持之列,
  或者 \carg{param_value_size} 的值 < \reftab{samplerinfo}中返回型別的大小且 \carg{param_value} 不是 \cenum{NULL}。
\item \cenum{CL_INVALID_SAMPLER}——如果 \carg{sampler} 無效。
\item \cenum{CL_OUT_OF_RESOURCES}——如果\scdevfailres。
\item \cenum{CL_OUT_OF_HOST_MEMORY}——如果\schostfailres。
\stopigBase

