\section{記號、屏障以及對事件的等待}

% clEnqueueMarkerWithWaitList
函式
\startCLFUNC
cl_int clEnqueueMarkerWithWaitList(
			cl_command_queue command_queue,
			cl_uint num_events_in_wait_list,
			const cl_event *event_wait_list,
			cl_event *event)
\stopCLFUNC
會將一個\cnglo{marker}\cnglo{cmd}入隊,
此\cnglo{cmd}會等待 \carg{event_wait_list} 中的事件完成,
如果 \carg{event_wait_list} 為空,
則等待 \carg{command_queue} 中之前入隊的\cnglo{cmd}完成。
此\cnglo{cmd}會返回一個可以等待的事件,
即可以等在這個事件上,來確定 \carg{event_wait_list} 中的事件完成,
或者 \carg{command_queue} 中之前入隊的\cnglo{cmd}完成。

\carg{command_queue} 是一個\cnglo{cmdq}。

\carg{event_wait_list} 和 \carg{num_events_in_wait_list}
 即執行這個\cnglo{cmd}前需要完成的事件。

如果 \carg{event_wait_list} 是 \cmacro{NULL},
 \carg{num_events_in_wait_list} 必須是 0。
如果 \carg{event_wait_list} 不是 \cmacro{NULL},
 \carg{event_wait_list} 中的事件必須有效,
並且 \carg{num_events_in_wait_list} 必須大於 0。
 \carg{event_wait_list} 中的事件充當同步點。
 \carg{event_wait_list} 中的事件與 \carg{command_queue} 必須位於同一\cnglo{context}中。
當此函式返回後,就可以重新使用或者釋放 \carg{event_wait_list} 所關聯的內存了。

如果 \carg{event_wait_list} 是 \cmacro{NULL},
則此\cnglo{cmd}會等到 \carg{command_queue} 中所有之前入隊的\cnglo{cmd}完成。

\carg{event} 會返回一個\cnglo{evtobj},可用來標識此\cnglo{cmd}。
如果 \carg{event} 是 \cmacro{NULL},
那麼\cnglo{app}就無法查詢此\cnglo{cmd}的狀態,或者等待此\cnglo{cmd}的完成。
如果 \carg{event_wait_list} 和 \carg{event} 都不是 \cmacro{NULL},
參數 \carg{event} 不能是 \carg{event_wait_list} 中的事件。

如果執行成功, \capi{clEnqueueMarkerWithWaitList} 會返回 \cenum{CL_SUCCESS}。
否則,返回下列錯誤碼之一:
\startigBase
\item \cenum{CL_INVALID_COMMAND_QUEUE},如果 \carg{command_queue} 無效。

\item \cenum{CL_INVALID_EVENT_WAIT_LIST},
如果 \carg{event_wait_list} 是 \cmacro{NULL},
且 \carg{num_events_in_wait_list} > 0;
或者 \carg{event_wait_list} 不是 \cmacro{NULL},
但 \carg{num_events_in_wait_list} 是 0;
或者 \carg{event_wait_list} 中有無效的事件。

\item \cenum{CL_OUT_OF_RESOURCES},如果\scdevfailres。

\item \cenum{CL_OUT_OF_HOST_MEMORY},如果\schostfailres。
\stopigBase

% clEnqueueBarrierWithWaitList
函式
\startCLFUNC
cl_int clEnqueueBarrierWithWaitList(
			cl_command_queue command_queue,
			cl_uint num_events_in_wait_list,
			const cl_event *event_wait_list,
			cl_event *event)
\stopCLFUNC
會將一個\cnglo{barrier}\cnglo{cmd}入隊,
此\cnglo{cmd}會等待 \carg{event_wait_list} 中的事件完成,
如果 \carg{event_wait_list} 為空,
則等待 \carg{command_queue} 中之前入隊的\cnglo{cmd}完成。
此\cnglo{cmd}會阻塞\cnglo{cmd}的執行,
即,所遇之後入隊的\cnglo{cmd}都得等到他完成後才能執行。
此\cnglo{cmd}會返回一個可以等待的事件,
即可以等在這個事件上,來確定 \carg{event_wait_list} 中的事件完成,
或者 \carg{command_queue} 中之前入隊的\cnglo{cmd}完成。

\carg{command_queue} 是一個\cnglo{cmdq}。

\carg{event_wait_list} 和 \carg{num_events_in_wait_list}
 即執行這個\cnglo{cmd}前需要完成的事件。

如果 \carg{event_wait_list} 是 \cmacro{NULL},
 \carg{num_events_in_wait_list} 必須是 0。
如果 \carg{event_wait_list} 不是 \cmacro{NULL},
 \carg{event_wait_list} 中的事件必須有效,
並且 \carg{num_events_in_wait_list} 必須大於 0。
 \carg{event_wait_list} 中的事件充當同步點。
 \carg{event_wait_list} 中的事件與 \carg{command_queue} 必須位於同一\cnglo{context}中。
當此函式返回後,就可以重新使用或者釋放 \carg{event_wait_list} 所關聯的內存了。

如果 \carg{event_wait_list} 是 \cmacro{NULL},
則此\cnglo{cmd}會等到 \carg{command_queue} 中所有之前入隊的\cnglo{cmd}完成。

\carg{event} 會返回一個\cnglo{evtobj},可用來標識此\cnglo{cmd}。
如果 \carg{event} 是 \cmacro{NULL},
那麼\cnglo{app}就無法查詢此\cnglo{cmd}的狀態,或者等待此\cnglo{cmd}的完成。
如果 \carg{event_wait_list} 和 \carg{event} 都不是 \cmacro{NULL},
參數 \carg{event} 不能是 \carg{event_wait_list} 中的事件。

如果執行成功, \capi{clEnqueueBarrierWithWaitList} 會返回 \cenum{CL_SUCCESS}。
否則,返回下列錯誤碼之一:
\startigBase
\item \cenum{CL_INVALID_COMMAND_QUEUE},如果 \carg{command_queue} 無效。

\item \cenum{CL_INVALID_EVENT_WAIT_LIST},
如果 \carg{event_wait_list} 是 \cmacro{NULL},
且 \carg{num_events_in_wait_list} > 0;
或者 \carg{event_wait_list} 不是 \cmacro{NULL},
但 \carg{num_events_in_wait_list} 是 0;
或者 \carg{event_wait_list} 中有無效的事件。

\item \cenum{CL_OUT_OF_RESOURCES},如果\scdevfailres。

\item \cenum{CL_OUT_OF_HOST_MEMORY},如果\schostfailres。
\stopigBase
