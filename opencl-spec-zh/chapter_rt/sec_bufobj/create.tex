\subsection{創建緩衝對象}

%%%%%%%%%%%%%%%%%%%%%%%%%%%%%%%%%%%%%%% clCreateBuffer
可用如下函數創建\cngloemp{bufobj}。
\startclc
cl_mem clCreateBuffer(
		cl_context context,
		cl_mem_flags flags,
		size_t size,
		void *host_ptr,
		cl_int *errcode_ret)
\stopclc

\carg{context} 是 OpenCL \cnglo{context},用來創建\cnglo{bufobj}。

\carg{flags} 是位域,用來表明如何分配(如使用哪個內存區域分配\cnglo{bufobj})以及怎樣使用。
\reftab{clmemflags}列出了 \carg{flags} 可能的值。
如果 \carg{flags} 的值是 0,則使用缺省值 \cenum{CL_MEM_READ_WRITE}。

\splitfloat{
  \placetable[here,force][tab:clmemflags]{\carg{cl_mem_flags}的值的清单}
}{
  {\startED[\ctype{cl_mem_flags}]

\clED{CL_MEM_READ_WRITE}{
  表明此\cnglo{memobj}可讀可寫。這是缺省值。
}

\clED{CL_MEM_WRITE_ONLY}{
  只能寫不能讀,對這種\cnglo{memobj}的讀操作是未定義的。
  \cenum{CL_MEM_READ_WRITE} 和 \cenum{CL_MEM_WRITE_ONLY} 是互斥的。
}

\clED{CL_MEM_READ_ONLY}{
  只能讀不能寫,對這種\cnglo{memobj}的寫操作是未定義的。
  \cenum{CL_MEM_READ_WRITE} 或者 \cenum{CL_MEM_WRITE_ONLY} 都與 \cenum{CL_MEM_READ_ONLY} 互斥。
}

\clED{CL_MEM_USE_HOST_PTR}{
  僅當 \carg{host_ptr} 不是 \cenum{NULL} 時才有效。
  它表明\cnglo{app}想讓 OpenCL 實作使用 \carg{host_ptr} 所引用的內存來存儲\cnglo{memobj}的內容。

  OpenCL 實作可以在\cnglo{device}內存中保存一份 \carg{host_ptr} 所引用的內容用作 cache。
  \cnglo{kernel}在\cnglo{device}上執行時可以使用這份用作 cache 的拷貝。

  如果多個\cnglo{bufobj}由同一 \carg{host_ptr} 創建,或者有重疊區域,當 OpenCL \cnglo{cmd}操作這些\cnglo{bufobj}時,其結果未定義。

  用 \cenum{CL_MEM_USE_HOST_PTR} 創建\cnglo{memobj}時,\carg{host_ptr} 的對齊規則請參考\todo{section C.3}。
}

\clED{CL_MEM_ALLOC_HOST_PTR}{
    表明\cnglo{app}想讓 OpenCL 實作在\cnglo{host}可以存取的內存中分配內存。

    它與 \cenum{CL_MEM_USE_HOST_PTR} 互斥。
}

\clED{CL_MEM_COPY_HOST_PTR}{
  僅當 \carg{host_ptr} 不是 \cenum{NULL} 時才有效。
  它表明\cnglo{app}想讓 OpenCL 實作使用 \carg{host_ptr} 所引用的內存來為\cnglo{memobj}分配內存並拷貝數據。

  它與 \cenum{CL_MEM_USE_HOST_PTR} 互斥。

  它與 \cenum{CL_MEM_ALLOC_HOST_PTR} 一起使用時,可以對由\cnglo{host}可存取的內存(如 PCIe )分配的 \ctype{cl_mem} 對象進行初始化。
}

\clED{CL_MEM_HOST_WRITE_ONLY}{
  表明\cnglo{host}只會對此\cnglo{memobj}進行寫入。
  可用來對\cnglo{host}的寫操作進行優化(如對於與\cnglo{host}通過系統總線如 PCIe 進行通信的\cnglo{device},分配\cnglo{memobj}時使能 Write-combining)。
}

\clED{CL_MEM_HOST_READ_ONLY}{
  表明\cnglo{host}只會對此\cnglo{memobj}進行讀取。

  它與 \cenum{CL_MEM_HOST_WRITE_ONLY} 互斥。
}

\clED{CL_MEM_HOST_NO_ACCESS}{
  表明\cnglo{host}不會對此\cnglo{memobj}進行讀寫。

  \cenum{CL_MEM_HOST_WRITE_ONLY} 或者 \cenum{CL_MEM_HOST_READ_ONLY} 都與 \cenum{CL_MEM_HOST_NO_ACCESS} 互斥。
}

\stopED

}
}

\carg{size} 即所要分配的\cnglo{bufobj}的大小,單位:字節。

\carg{host_ptr} 指向可能已經由\cnglo{app}分配好了的緩衝數據。其大小必須 >= \carg{size}。

\carg{errcode_ret} 用來返回錯誤碼。如果是 \cenum{NULL},則不會返回錯誤碼。

如果成功創建了\cnglo{bufobj},\capi{clCreateBuffer} 會將其返回,並將 \carg{errcode_ret} 置為 \cenum{CL_SUCCESS}。
否則返回 \cenum{NULL},並將 \carg{errcode_ret} 置為下列錯誤碼之一:
\startigBase
\item \cenum{CL_INVALID_CONTEXT},如果 \carg{context} 無效。
\item \cenum{CL_INVALID_VALUE},如果 \carg{flags} 的值不是\reftab{clmemflags}中定義的。

\startbuffer[footnoteshi]
如果 \carg{size} 比 \carg{context} 中所有\cnglo{device}的 \cenum{CL_DEVICE_MAX_MEM_ALLOC_SIZE}(参见\reftab{cldevquery})都大,实现可能返回 \cenum{CL_INVALID_BUFFER_SIZE}。
\stopbuffer
\item \cenum{CL_INVALID_BUFFER_SIZE},如果 \carg{size} 是 0\footnote{\getbuffer[footnoteshi]}。

\item \cenum{CL_INVLAID_HOST_PTR},
  如果 \carg{host_ptr} 是 \cenum{NULL},并且 \carg{flags} 中设置了 \cenum{CL_MEM_USE_HOST_PTR} 或 \cenum{CL_MEM_COPY_HOST_PTR};
  或者 \carg{host_ptr} 不是 \cenum{NULL},但是 \carg{flags} 中沒有設置 \cenum{CL_MEM_USE_HOST_PTR} 或 \cenum{CL_MEM_COPY_HOST_PTR}。

\item \cenum{CL_MEM_OBJECT_ALLOCATION_FAILURE},如果為\cnglo{bufobj}分配內存失敗。

\item \cenum{CL_OUT_OF_RESOURCES},如果\scdevfailres。

\item \cenum{CL_OUT_OF_HOST_MEMORY},如果\schostfailres。

\stopigBase

%%%%%%%%%%%%%%%%%%%%%%%%%%%%%%%%%%%% clCreateSubBuffer
函數
\startclc
cl_mem clCreateSubBuffer(cl_mem buffer,
			cl_mem_flags flags,
			cl_buffer_create_type buffer_create_type,
			const void *buffer_create_info,
			cl_int *errcode_ret)
\stopclc
可以由一個現有的\cnglo{bufobj}創建一個新的\cnglo{bufobj}(叫做子\cnglo{bufobj}, sub-buffer object)。

\carg{buffer} 必須是一個有效的\cnglo{bufobj},且不能是子\cnglo{bufobj}。

\carg{flags} 是位域,用來表明如何分配以及怎樣使用\cnglo{memobj},參見\reftab{clmemflags}。
  如果 \carg{flags} 中沒有設置 \cenum{CL_MEM_READ_WRITE}、 \cenum{CL_MEM_READ_ONLY} 或 \cenum{CL_MEM_WRITE_ONLY},會從 \carg{buffer} 中繼承這些屬性。
  而 \carg{flags} 中不能設置 \cenum{CL_MEM_USE_HOST_PTR}、 \cenum{CL_MEM_ALLOC_HOST_PTR} 和 \cenum{CL_MEM_COPY_HOST_PTR},這些也很由 \carg{buffer} 繼承。
  即使 \carg{buffer} 的內存訪問限定符中有 \cenum{CL_MEM_COPY_HOST_PTR},也並不意味着創建 sub-buffer 時會有額外的拷貝。
  如果 \carg{flags} 中沒有設置 \cenum{CL_MEM_HOST_WRITE_ONLY}、 \cenum{CL_MEM_HOST_READ_ONLY} 或 \cenum{CL_MEM_HOST_NO_ACCESS},則會從 \carg{buffer} 中繼承這些屬性。

\carg{buffer_create_type} 和 \carg{buffer_create_info} 表明了所要創建的\cnglo{bufobj}的類型。
\reftab{clcreatesubbuffer}列出了所支持的 \carg{buffer_create_type} 以及 \carg{buffer_create_info} 中相應的內容。

\placetable[here,force][tab:clcreatesubbuffer]{\capi{clCreateSubBuffer} 所支持的創建類型}
{\startED[\ctype{cl_buffer_create_type}]

\clED{CL_BUFFER_CREATE_TYPE_REGION}{
  用 \carg{buffer} 中的特定區域創建\cnglo{bufobj}。

  \carg{buffer_create_info} 指向如下數據結構\todo{clcintable}:

\type{struct _cl_buffer_region \{}\crlf
\type{        size_t origin;}\crlf
\type{        size_t size;}\crlf
\type{\} cl_buffer_region;}

  $(origin, size)$ 就是在 \carg{buffer} 中的偏移量和字節數。

  如果 \carg{buffer} 是用 \cenum{CL_MEM_USE_HOST_PTR} 創建的,所返回\cnglo{bufobj}的 \carg{host_ptr} 就是 $host\_ptr+origin$。

  所返回的\cnglo{bufobj}引用了為 \carg{buffer} 分配的數據存儲空間,並指向其中的特定區域 $(origin,size)$。

  如果在 \carg{buffer} 中,區域 $(origin,size)$ 越界了,則會在 \carg{errcode_ret} 中返回 \cenum{CL_INVALID_VALUE}。

  如果 \carg{size} 是 0,則返回 \cenum{CL_INVALID_BUFFER_SIZE}。

  如果與 \carg{buffer} 相關聯的\cnglo{context}中沒有一個設備的 \cenum{CL_DEVICE_MEM_BASE_ADDR_ALIGN} 與 $origin$ 對齊,
  則會在 \carg{errcode_ret} 中返回 \cenum{CL_MISALIGNED_SUB_BUFFER_OFFSET}。
}

\stopED

}

如果執行成功,\capi{clCreateSubBuffer} 會返回 \cenum{CL_SUCCESS}。
否則會將 \carg{errcode_ret} 置為下列錯誤碼之一:
\startigBase
\item \cenum{CL_INVALID_MEM_OBJECT},如果 \carg{buffer} 無效或者是一個子\cnglo{bufobj}。

\item \cenum{CL_INVALID_VALUE},
  如果 \carg{buffer} 是用 \cenum{CL_MEM_WRITE_ONLY} 創建的,但 \carg{flags} 中設置了 \cenum{CL_MEM_READ_WRITE} 或 \cenum{CL_MEM_READ_ONLY};
  或者 \carg{buffer} 是用 \cenum{CL_MEM_READ_ONLY} 創建的,但 \carg{flags} 中設置了 \cenum{CL_MEM_READ_WRITE} 或 \cenum{CL_MEM_WRITE_ONLY};
  或者 \carg{flags} 中設置了 \cenum{CL_MEM_USE_HOST_PRT} 或 \cenum{CL_MEM_ALLOC_HOST_PTR} 或 \cenum{CL_MEM_COPY_HOST_PTR}。

\item \cenum{CL_INVALID_VALUE},
  如果 \carg{buffer} 是用 \cenum{CL_MEM_HOST_WRITE_ONLY} 創建的,但 \carg{flags} 中設置了 \cenum{CL_MEM_HOST_READ_ONLY};
  或者 \carg{buffer} 是用 \cenum{CL_MEM_HOST_READ_ONLY} 創建的,但 \carg{flags} 中設置了 \cenum{CL_MEM_HOST_WRITE_ONLY};
  或者 \carg{buffer} 使用 \cenum{CL_MEM_HOST_NO_ACCESS} 創建的,但 \carg{flags} 中設置了 \cenum{CL_MEM_HOST_READ_ONLY} 或 \cenum{CL_MEM_HOST_WRITE_ONLY}。

\item \cenum{CL_INVALID_VALUE},如果 \carg{buffer_create_type} 的值無效。

\item \cenum{CL_INVALID_VALUE},
  如果 \carg{buffer_create_info} 中的值無效(對應於 \carg{buffer_create_type} ),
  或者 \carg{buffer_create_info} 是 \cenum{NULL}。

\item \cenum{CL_INVALID_BUFFER_SIZE},如果 \carg{size} 是 0。

\item \cenum{CL_MEM_OBJECT_ALLOCATION_FAILURE},如果為子\cnglo{bufobj}分配內存失敗。

\item \cenum{CL_OUT_OF_RESOURCES},如果\scdevfailres。

\item \cenum{CL_OUT_OF_HOST_MEMORY},如果\schostfailres。

\stopigBase

注意:

對一個\cnglo{bufobj}及其子\cnglo{bufobj}的並發讀、寫、拷貝是未定義的。
對於由同一\cnglo{bufobj}創建的互相重疊的子\cnglo{bufobj}的並發讀、寫、拷貝也是未定義的。
只有讀操作是定義了的。



