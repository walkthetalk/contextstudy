\subsection{創建緩衝對象}

%%%%%%%%%%%%%%%%%%%%%%%%%%%%%%%%%%%%%%% clCreateBuffer
可用如下函數創建\cngloemp{bufobj}。
\startclc
cl_mem clCreateBuffer(
		cl_context context,
		cl_mem_flags flags,
		size_t size,
		void *host_ptr,
		cl_int *errcode_ret)
\stopclc

\carg{context} 是 OpenCL \cnglo{context},用來創建\cnglo{bufobj}。

\carg{flags} 是位域,用來表明如何分配(如使用哪個內存區域分配\cnglo{bufobj})以及怎樣使用。
\reftab{clmemflags}列出了 \carg{flags} 可能的值。
如果 \carg{flags} 的值是 0,則使用缺省值 \cenum{CL_MEM_READ_WRITE}。

\splitfloat{
  \placetable[here,force][tab:clmemflags]{\carg{cl_mem_flags}的值的清单}
}{
  {%%%%%%%%%%%%%%%%%%%%%%%%%%%%%%%%%%%%%%%%%head%%%%%%%%%%%%%%%%%%%%%%%%%%%%%%%%%%%
\bTABLEhead
\bTR[background=color,backgroundcolor=gray]
  \bTH cl\_mem\_flags \eTH
  \bTH 描述 \eTH \eTR
\eTABLEhead

%%%%%%%%%%%%%%%%%%%%%%%%%%%%%%%%%%%%%%%  body  %%%%%%%%%%%%%%%%%%%%%%%%%%%%%%%%%
\bTABLEbody

\clenumdesc{CL_MEM_READ_WRITE}{
    表明此\cnglo{memobj}可读可写。这是默认值。
}

\clenumdesc{CL_MEM_WRITE_ONLY}{
    只能写不能读,对这种\cnglo{memobj}的读操作是未定义的。
}

\clenumdesc{CL_MEM_READ_ONLY}{
    只能读不能写,对这种\cnglo{memobj}的写操作是未定义的。
}

\clenumdesc{CL_MEM_USE_HOST_PTR}{
    只有当\carg{host_ptr}不是\cenum{NULL}时,此flag才有效。它表明\cnglo{app}想让\scopencl实现使用\carg{host_ptr}所引用的内存来存储\cnglo{memobj}的内容。

    允许\scopencl实现在\cnglo{device}内存中保存一份\carg{host_ptr}所指内容的拷贝。\cnglo{kernel}在\cnglo{device}上执行时可以使用这份作为cache的拷贝。

    如果多个\cnglo{bufobj}由同一个\carg{host_ptr}创建,或者有重叠区域,当\scopencl\cnglo{cmd}操作这些\cnglo{bufobj}时,其结果未定义。
}

\clenumdesc{CL_MEM_ALLOC_HOST_PTR}{
    此flag表明\cnglo{app}想让\scopencl实现在\cnglo{host}可以访问的内存中分配内存。

    它与\cenum{CL_MEM_USE_HOST_PTR}是互斥的。
}

\clenumdesc{CL_MEM_COPY_HOST_PTR}{
    只有\carg{host_ptr}不是\cenum{NULL}时此flag才有效。它表明\cnglo{app}想让\scopencl实现使用\carg{host_ptr}所引用的内存来为\cnglo{memobj}分配内存,并拷贝数据。

    它与\cenum{CL_MEM_USE_HOST_PTR}互斥。

    它可以与\cenum{CL_MEM_ALLOC_HOST_PTR}一起使用,对由\cnglo{host}可访问的内存(如PCIe)分配的\ctype{cl_mem}对象进行初始化。
}

\eTABLEbody

}
}

\carg{size} 即所要分配的\cnglo{bufobj}的大小,單位:字節。

\carg{host_ptr} 指向可能已經由\cnglo{app}分配好了的緩衝數據。其大小必須 >= \carg{size}。

\carg{errcode_ret} 用來返回錯誤碼。如果是 \cenum{NULL},則不會返回錯誤碼。

如果成功創建了\cnglo{bufobj},\capi{clCreateBuffer} 會將其返回,並將 \carg{errcode_ret} 置為 \cenum{CL_SUCCESS}。
否則返回 \cenum{NULL},並將 \carg{errcode_ret} 置為下列錯誤碼之一:
\startigBase
\item \cenum{CL_INVALID_CONTEXT},如果 \carg{context} 無效。
\item \cenum{CL_INVALID_VALUE},如果 \carg{flags} 的值不是\reftab{clmemflags}中定義的。

\startbuffer[footnoteshi]
如果 \carg{size} 比 \carg{context} 中所有\cnglo{device}的 \cenum{CL_DEVICE_MAX_MEM_ALLOC_SIZE}(参见\reftab{cldevquery})都大,实现可能返回 \cenum{CL_INVALID_BUFFER_SIZE}。
\stopbuffer
\item \cenum{CL_INVALID_BUFFER_SIZE},如果 \carg{size} 是 0\footnote{\getbuffer[footnoteshi]}。

\item \cenum{CL_INVLAID_HOST_PTR},
  如果 \carg{host_ptr} 是 \cenum{NULL},并且 \carg{flags} 中设置了 \cenum{CL_MEM_USE_HOST_PTR} 或 \cenum{CL_MEM_COPY_HOST_PTR};
  或者 \carg{host_ptr} 不是 \cenum{NULL},但是 \carg{flags} 中沒有設置 \cenum{CL_MEM_USE_HOST_PTR} 或 \cenum{CL_MEM_COPY_HOST_PTR}。

\item \cenum{CL_MEM_OBJECT_ALLOCATION_FAILURE},如果為\cnglo{bufobj}分配內存失敗。

\item \cenum{CL_OUT_OF_RESOURCES},如果\scdevfailres。

\item \cenum{CL_OUT_OF_HOST_MEMORY},如果\schostfailres。

\stopigBase

%%%%%%%%%%%%%%%%%%%%%%%%%%%%%%%%%%%% clCreateSubBuffer
函數
\startclc
cl_mem clCreateSubBuffer(cl_mem buffer,
			cl_mem_flags flags,
			cl_buffer_create_type buffer_create_type,
			const void *buffer_create_info,
			cl_int *errcode_ret)
\stopclc
可以由一個現有的\cnglo{bufobj}創建一個新的\cnglo{bufobj}(叫做子\cnglo{bufobj}, sub-buffer object)。

\carg{buffer} 必須是一個有效的\cnglo{bufobj},且不能是子\cnglo{bufobj}。

\carg{flags} 是位域,用來表明如何分配以及怎樣使用\cnglo{memobj},參見\reftab{clmemflags}。
  如果 \carg{flags} 中沒有設置 \cenum{CL_MEM_READ_WRITE}、 \cenum{CL_MEM_READ_ONLY} 或 \cenum{CL_MEM_WRITE_ONLY},會從 \carg{buffer} 中繼承這些屬性。
  而 \carg{flags} 中不能設置 \cenum{CL_MEM_USE_HOST_PTR}、 \cenum{CL_MEM_ALLOC_HOST_PTR} 和 \cenum{CL_MEM_COPY_HOST_PTR},這些也很由 \carg{buffer} 繼承。
  即使 \carg{buffer} 的內存訪問限定符中有 \cenum{CL_MEM_COPY_HOST_PTR},也並不意味着創建 sub-buffer 時會有額外的拷貝。
  如果 \carg{flags} 中沒有設置 \cenum{CL_MEM_HOST_WRITE_ONLY}、 \cenum{CL_MEM_HOST_READ_ONLY} 或 \cenum{CL_MEM_HOST_NO_ACCESS},則會從 \carg{buffer} 中繼承這些屬性。

\carg{buffer_create_type} 和 \carg{buffer_create_info} 表明了所要創建的\cnglo{bufobj}的類型。
\reftab{clcreatesubbuffer}列出了所支持的 \carg{buffer_create_type} 以及 \carg{buffer_create_info} 中相應的內容。

\placetable[here,force][tab:clcreatesubbuffer]{\capi{clCreateSubBuffer} 所支持的創建類型}
{%%%%%%%%%%%%%%%%%%%%%%%%%%%%%%%%%%%%%%%%%head%%%%%%%%%%%%%%%%%%%%%%%%%%%%%%%%%%%
\bTABLEhead
\bTR[background=color,backgroundcolor=gray]
  \bTH \ctype{cl_buffer_create_type} \eTH
  \bTH 描述 \eTH
  \eTR
\eTABLEhead

%%%%%%%%%%%%%%%%%%%%%%%%%%%%%%%%%%%%%%%  body  %%%%%%%%%%%%%%%%%%%%%%%%%%%%%%%%%
\bTABLEbody

\clenumdesc{CL_BUFFER_CREATE_TYPE_REGION}{
  创建一个用来描述\carg{buffer}中特定区块的\cnglo{bufobj}。

  \carg{buffer_create_info}指向如下数据结构:
\todo{clcintable}
%\startclc
%typedef struct _cl_buffer_region {
%	size_t origin;
%	size_t size;
%} cl_buffer_region;
%\stopclc
$(origin, size)$就是在\carg{buffer}中的偏移量和大小。

如果\carg{buffer}是用\cenum{CL_MEM_USE_HOST_PTR}创建的,所返回\cnglo{bufobj}的
\carg{host_ptr}就是$host\_ptr+origin$。

所返回的\cnglo{bufobj}引用了为\carg{buffer}分配的数据存储空间,并指向其中的特定区域
$(origin,size)$。

如果在\carg{buffer}中,区域$(origin,size)$越界了,则会在\carg{errcode_ret}中返回
\cenum{CL_INVALID_VALUE}。

如果\carg{size}是0,则返回\cenum{CL_INVALID_BUFFER_SIZE}。

如果与\carg{buffer}像关联的\cnglo{context}中没有一个设备的\cenum{CL_DEVICE_MEM_BASE_ADDR_ALIGN}与$origin$对齐,则会在\carg{errcode_ret}中返回
\cenum{CL_MISALIGNED_SUB_BUFFER_OFFSET}。

}

\eTABLEbody

}

如果執行成功,\capi{clCreateSubBuffer} 會返回 \cenum{CL_SUCCESS}。
否則會將 \carg{errcode_ret} 置為下列錯誤碼之一:
\startigBase
\item \cenum{CL_INVALID_MEM_OBJECT},如果 \carg{buffer} 無效或者是一個子\cnglo{bufobj}。

\item \cenum{CL_INVALID_VALUE},
  如果 \carg{buffer} 是用 \cenum{CL_MEM_WRITE_ONLY} 創建的,但 \carg{flags} 中設置了 \cenum{CL_MEM_READ_WRITE} 或 \cenum{CL_MEM_READ_ONLY};
  或者 \carg{buffer} 是用 \cenum{CL_MEM_READ_ONLY} 創建的,但 \carg{flags} 中設置了 \cenum{CL_MEM_READ_WRITE} 或 \cenum{CL_MEM_WRITE_ONLY};
  或者 \carg{flags} 中設置了 \cenum{CL_MEM_USE_HOST_PRT} 或 \cenum{CL_MEM_ALLOC_HOST_PTR} 或 \cenum{CL_MEM_COPY_HOST_PTR}。

\item \cenum{CL_INVALID_VALUE},
  如果 \carg{buffer} 是用 \cenum{CL_MEM_HOST_WRITE_ONLY} 創建的,但 \carg{flags} 中設置了 \cenum{CL_MEM_HOST_READ_ONLY};
  或者 \carg{buffer} 是用 \cenum{CL_MEM_HOST_READ_ONLY} 創建的,但 \carg{flags} 中設置了 \cenum{CL_MEM_HOST_WRITE_ONLY};
  或者 \carg{buffer} 使用 \cenum{CL_MEM_HOST_NO_ACCESS} 創建的,但 \carg{flags} 中設置了 \cenum{CL_MEM_HOST_READ_ONLY} 或 \cenum{CL_MEM_HOST_WRITE_ONLY}。

\item \cenum{CL_INVALID_VALUE},如果 \carg{buffer_create_type} 的值無效。

\item \cenum{CL_INVALID_VALUE},
  如果 \carg{buffer_create_info} 中的值無效(對應於 \carg{buffer_create_type} ),
  或者 \carg{buffer_create_info} 是 \cenum{NULL}。

\item \cenum{CL_INVALID_BUFFER_SIZE},如果 \carg{size} 是 0。

\item \cenum{CL_MEM_OBJECT_ALLOCATION_FAILURE},如果為子\cnglo{bufobj}分配內存失敗。

\item \cenum{CL_OUT_OF_RESOURCES},如果\scdevfailres。

\item \cenum{CL_OUT_OF_HOST_MEMORY},如果\schostfailres。

\stopigBase

注意:

對一個\cnglo{bufobj}及其子\cnglo{bufobj}的並發讀、寫、拷貝是未定義的。
對於由同一\cnglo{bufobj}創建的互相重疊的子\cnglo{bufobj}的並發讀、寫、拷貝也是未定義的。
只有讀操作是定義了的。



