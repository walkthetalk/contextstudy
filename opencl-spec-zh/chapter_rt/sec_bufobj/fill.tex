\subsection{填充緩衝對象}

函式
\startCLFUNC
cl_int clEnqueueFillBuffer (cl_command_queue command_queue,
			cl_mem buffer,
			const void *pattern,
			size_t pattern_size,
			size_t offset,
			size_t size,
			cl_uint num_events_in_wait_list,
			const cl_event *event_wait_list,
			cl_event *event)
\stopCLFUNC
會將一個\cnglo{cmd}插入隊列,此\cnglo{cmd}可以按給定的樣式( pattern )填充\cnglo{bufobj}。
\carg{buffer} 的一些用法資訊,如此\cnglo{memobj}是否可讀、是否可寫,
還有創建 \carg{buffer} 時所指定的參數 \ctype{cl_mem_flags},
都會被 \capi{clEnqueueFillBuffer} 忽略。

\carg{command_queue} 即此填充\cnglo{cmd}要插入的隊列。
\carg{command_queue} 和 \carg{buffer} 必須位於用一個 OpenCL \cnglo{context}中。

\carg{buffer} 是一個\cnglo{bufobj}。

\carg{pattern} 指向數據樣式( data pattern ),其大小為 \carg{pattern_size}。
用 \carg{pattern} 填充的區域在 \carg{buffer} 中的偏移量為 \carg{offset},大小是 \carg{size}。
數據樣式必須是 OpenCL 所支持的標量或矢量的整數或浮點數型別,參見\todo{section 6.1.1 and 6.1.2}。
例如,如果數據樣式是 \ctype{float4},則 \carg{pattern} 指向一個型別為 \ctype{cl_float4} 的值,
且 \carg{pattern_size} 是 \ctype{sizeof(cl_float4)}。
\carg{pattern_size} 的最大值就是 OpenCL \cnglo{device}所支持的最大的整數或浮點數類型的矢量數據型別的大小。
函式返回後,就可以重用或者釋放 \carg{pattern} 所指向的內存了。

\carg{offset} 即填充區域在 \carg{buffer} 中的偏移量,單位:字節,他必須是 \carg{pattern_size} 的整數倍。

\carg{event_wait_list} 和 \carg{num_events_in_wait_list} 中列出了執行此\cnglo{cmd}前要等待的事件。
如果 \carg{event_wait_list} 是 \cmacro{NULL},則無須等待任何事件,並且 \carg{num_events_in_wait_list} 必須是0。
如果 \carg{event_wait_list} 不是 \cmacro{NULL},則其中所有事件都必須是有效的,並且 \carg{num_events_in_wait_list} 必須大於 0。
\carg{event_wait_list} 中的事件充當同步點,並且必須與 \carg{command_queue} 位於同一個\cnglo{context}中。
此函式返回後,可以回收並重新使用 \carg{event_wait_list} 所關聯的內存。

\carg{event} 會返回一個\cnglo{evtobj},用來標識此\cnglo{cmd},可用來查詢或等待此\cnglo{cmd}完成。
而如果 \carg{event} 是 \cmacro{NULL},就沒辦法查詢此\cnglo{cmd}的狀態或等待其完成了。
不過可以用 \capi{clEnqueueBarrierWithWaitList} 代替。
如果 \carg{event_wait_list} 和 \carg{event} 都不是 \cmacro{NULL}, \carg{event} 不能屬於 \carg{event_wait_list}。

如果執行成功, \capi{clEnqueueFillBuffer} 會返回 \cenum{CL_SUCCESS}。
否則,返回下列錯誤碼之一:
\startigBase
\item \cenum{CL_INVALID_COMMAND_QUEUE},如果 \carg{command_queue} 無效。

\item \cenum{CL_INVALID_CONTEXT},
  如果 \carg{command_queue} 和 \carg{buffer} 位於不同的\cnglo{context}中,
  或者 \carg{command_queue} 和 \carg{event_wait_list} 中的事件位於不同的\cnglo{context}中。

\item \cenum{CL_INVALID_MEM_OBJECT},如果 \carg{buffer} 無效。

\item \cenum{CL_INVALID_VALUE},如果 $(offset,size)$ 所指定的區域越限。

\item \cenum{CL_INVALID_VALUE},如果 \carg{pattern} 是 \cmacro{NULL},
  或者 \carg{pattern_size} 是 0,
  或者 \carg{pattern_size} 不屬於 $\{1,2,4,8,16,32,64,128\}$。

\item \cenum{CL_INVALID_VALUE},如果 \carg{offset} 和 \carg{size} 不是 \carg{pattern_size} 的整數倍。

\item \cenum{CL_INVALID_EVENT_WAIT_LIST},
  如果 \carg{event_wait_list} 是 \cmacro{NULL},且 \carg{num_events_in_wait_list} > 0;
  或者 \carg{event_wait_list} 不是 \cmacro{NULL},但 \carg{num_events_in_wait_list} 是 0;
  或者 \carg{event_wait_list} 中的事件無效。

\item \cenum{CL_MISALIGNED_SUB_BUFFER_OFFSET},如果 \carg{buffer} 是子\cnglo{bufobj},
  且創建此對象時所指定的 \carg{offset} 沒有與 \carg{queue} 所關聯\cnglo{device}的 \cenum{CL_DEVICE_MEM_BASE_ADDR_ALIGN} 對齊。

\item \cenum{CL_MEM_OBJECT_ALLOCATION_FAILURE},如果為 \carg{buffer} 分配內存失敗。

\item \cenum{CL_OUT_OF_RESOURCES}——如果\scdevfailres。
\item \cenum{CL_OUT_OF_HOST_MEMORY}——如果\schostfailres。
\stopigBase


