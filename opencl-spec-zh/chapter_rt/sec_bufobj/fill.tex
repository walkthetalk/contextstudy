\subsection{填充緩衝對象}

\topclfunc{clEnqueueFillBuffer}

\startCLFUNC
cl_int clEnqueueFillBuffer (
			cl_command_queue command_queue,
			cl_mem buffer,
			const void *pattern,
			size_t pattern_size,
			size_t offset,
			size_t size,
			cl_uint num_events_in_wait_list,
			const cl_event *event_wait_list,
			cl_event *event)
\stopCLFUNC

此函式所入隊的\cnglo{cmd}可以按給定的範式( pattern )填充\cnglo{bufobj}。
\carg{buffer} 的一些用法資訊(如是否可由\cnglo{kernel}或\cnglo{host}讀寫、
以及創建 \carg{buffer} 時所指定的引數 \ctype{cl_mem_flags})
會被 \capi{clEnqueueFillBuffer} 忽略。

\carg{command_queue} 即這個填充\cnglo{cmd}要插入的隊列。
\carg{command_queue} 和 \carg{buffer} 必須位於用一個 OpenCL \cnglo{context}中。

\carg{buffer} 是一個\cnglo{bufobj}。

\carg{pattern} 指向數據範式( data pattern ),其大小為 \carg{pattern_size}。
用 \carg{pattern} 填充的區域在 \carg{buffer} 中的偏移量為 \carg{offset},
大小是 \carg{size}。
數據範式必須是 OpenCL 所支持的標量或矢量的整數或浮點數型別,
參見\refsec{bisdt}和\refsec{bivdt}。
例如,如果數據範式是 \ctype{float4},
則 \carg{pattern} 指向一個型別為 \ctype{cl_float4} 的值,
且 \carg{pattern_size} 是 \ctype{sizeof(cl_float4)}。
\carg{pattern_size} 的最大值就是
OpenCL \cnglo{device}所支持的最大的整數或浮點數矢量數據型別的大小。
函式返回後,就可以重用或者釋放 \carg{pattern} 所指向的內存了。

\carg{offset} 即填充區域在 \carg{buffer} 中的偏移量,單位:字節,
他必須是 \carg{pattern_size} 的整數倍。

\carg{event_wait_list} 和 \carg{num_events_in_wait_list}
中列出了執行此\cnglo{cmd}前要等待的事件。
如果 \carg{event_wait_list} 是 \cmacro{NULL},
則無須等待任何事件,並且 \carg{num_events_in_wait_list} 必須是 0。
如果 \carg{event_wait_list} 不是 \cmacro{NULL},
則其中所有事件都必須是有效的,並且 \carg{num_events_in_wait_list} 必須大於 0。
\carg{event_wait_list} 中的事件充當同步點,
並且必須與 \carg{command_queue} 位於同一個\cnglo{context}中。
此函式返回後,就可以回收並重用 \carg{event_wait_list} 所關聯的內存了。

\carg{event} 會返回一個\cnglo{evtobj},
用來標識此\cnglo{cmd},可用來查詢或等待此\cnglo{cmd}完成。
而如果 \carg{event} 是 \cmacro{NULL},就沒辦法查詢此\cnglo{cmd}的狀態或等待其完成了。
不過可以用 \capi{clEnqueueBarrierWithWaitList} 代替。
如果 \carg{event_wait_list} 和 \carg{event} 都不是 \cmacro{NULL},
\carg{event} 不能屬於 \carg{event_wait_list}。

如果執行成功, \capi{clEnqueueFillBuffer} 會返回 \cenum{CL_SUCCESS}。
否則,返回下列錯誤碼之一:
\startigBase
\item \cenum{CL_INVALID_COMMAND_QUEUE},如果 \carg{command_queue} 無效。

\item \cenum{CL_INVALID_CONTEXT},
如果 \carg{command_queue} 和 \carg{buffer} 位於不同的\cnglo{context}中,
或者 \carg{command_queue} 和 \carg{event_wait_list} 中的事件
位於不同的\cnglo{context}中。

\item \cenum{CL_INVALID_MEM_OBJECT},如果 \carg{buffer} 無效。

\item \cenum{CL_INVALID_VALUE},如果 \math{(\marg{offset}, \marg{size})} 所指定的區域越限。

\item \cenum{CL_INVALID_VALUE},如果 \carg{pattern} 是 \cmacro{NULL},
或者 \carg{pattern_size} 是 0,
或者 \carg{pattern_size} 不屬於 \ccmm{{1, 2, 4, 8, 16, 32, 64, 128}}。

\item \cenum{CL_INVALID_VALUE},
如果 \carg{offset} 和 \carg{size} 不是 \carg{pattern_size} 的整數倍。

\startitem
\cenum{CL_INVALID_EVENT_WAIT_LIST},如果滿足下列條件中的任一項:
\startigBase
\item \carg{event_wait_list} 是 \cmacro{NULL},但 \carg{num_events_in_wait_list} > 0;
\item 或者 \carg{event_wait_list} 不是 \cmacro{NULL},但 \carg{num_events_in_wait_list} 是 0;
\item 或者 \carg{event_wait_list} 中有無效的事件。
\stopigBase
\stopitem

\item \cenum{CL_MISALIGNED_SUB_BUFFER_OFFSET},
如果 \carg{buffer} 是子\cnglo{bufobj},
且創建此對象時所指定的 \carg{offset} 沒有與 \carg{queue} 所關聯\cnglo{device}
的 \cenum{CL_DEVICE_MEM_BASE_ADDR_ALIGN} 對齊。

\item \cenum{CL_MEM_OBJECT_ALLOCATION_FAILURE},如果為 \carg{buffer} 分配內存失敗。

\item \cenum{CL_OUT_OF_RESOURCES},如果\scdevfailres。
\item \cenum{CL_OUT_OF_HOST_MEMORY},如果\schostfailres。
\stopigBase
