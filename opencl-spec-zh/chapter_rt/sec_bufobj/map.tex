\subsection{映射緩衝對象}

函數
\startclc
void * clEnqueueMapBuffer (cl_command_queue command_queue,
			cl_mem buffer,
			cl_bool blocking_map,
			cl_map_flags map_flags,
			size_t offset,
			size_t size,
			cl_uint num_events_in_wait_list,
			const cl_event *event_wait_list,
			cl_event *event,
			cl_int *errcode_ret)
\stopclc
所要插入隊列的\cnglo{cmd}會將 \carg{buffer} 的某個區域映射到\cnglo{host}的地址空間中,並返回新地址。

\carg{command_queue} 必須是一個有效的\cnglo{cmd}。

\carg{blocking_map} 表明此映射操作是{\ftRef 阻塞的}還是{\ftRef 非阻塞的}。

如果 \carg{blocking_map} 是 \cenum{CL_TRUE},即映射\cnglo{cmd}是阻塞的,
直到映射完成,\cnglo{app}可以用返回的指針訪問所映射區域的內容時, \capi{clEnqueueMapBuffer} 才會返回。

如果 \carg{blocking_map} 是 \cenum{CL_FALSE},即映射\cnglo{cmd}是非阻塞的,
直到映射\cnglo{cmd}完成後,才能使用 \capi{clEnqueueMapBuffer} 所返回的指針。
參數 \carg{event} 會返回一個\cnglo{evtobj},可用來查詢映射\cnglo{cmd}的執行情況。
映射\cnglo{cmd}完成後,\cnglo{app}就可以使用 \capi{clEnqueueMapBuffer} 所返回的指針訪問映射區域的內容了。

\carg{map_flags} 是位域,參見\reftab{clmapflags}。

\carg{buffer} 是一個\cnglo{bufobj},它必須與 \carg{command_queue} 位於同一 OpenCL \cnglo{context}中。

\carg{offset} 和 \carg{size} 即所要映射的區域在\cnglo{bufobj}中的偏移量和大小,單位都是字節。

\carg{event_wait_list} 和 \carg{num_events_in_wait_list} 中列出了執行此\cnglo{cmd}前要等待的事件。
如果 \carg{event_wait_list} 是 \cenum{NULL},則無須等待任何事件,並且 \carg{num_events_in_wait_list} 必須是0。
如果 \carg{event_wait_list} 不是 \cenum{NULL},則其中所有事件都必須是有效的,並且 \carg{num_events_in_wait_list} 必須大於 0。
\carg{event_wait_list} 中的事件充當同步點,並且必須與 \carg{command_queue} 位於同一個\cnglo{context}中。
此函數返回後,可以回收並重新使用 \carg{event_wait_list} 所關聯的內存。

\carg{event} 會返回一個\cnglo{evtobj},用來標識此\cnglo{cmd},可用來查詢或等待此\cnglo{cmd}完成。
而如果 \carg{event} 是 \cenum{NULL},就沒辦法查詢此\cnglo{cmd}的狀態或等待其完成了。
如果 \carg{event_wait_list} 和 \carg{event} 都不是 \cenum{NULL}, \carg{event} 不能屬於 \carg{event_wait_list}。

\carg{errcode_ret} 用來返回錯誤碼。如果是 \cenum{NULL},則不會返回錯誤碼。

如果執行成功, \capi{clEnqueueMapBuffer} 會返回映射區域的指針,並將 \carg{errcode_ret} 置為 \cenum{CL_SUCCESS}。
否則,返回 \cenum{NULL},並將 \carg{errcode_ret} 置為下列錯誤碼之一:
\startigBase
\item \cenum{CL_INVALID_COMMAND_QUEUE},如果 \carg{command_queue} 無效。

\item \cenum{CL_INVALID_CONTEXT},
  如果 \carg{command_queue} 和 \carg{buffer} 位於不同的\cnglo{context}中,
  或者 \carg{command_queue} 和 \carg{event_wait_list} 中的事件位於不同的\cnglo{context}中。

\item \cenum{CL_INVALID_MEM_OBJECT},如果 \carg{buffer} 無效。

\item \cenum{CL_INVALID_VALUE},如果 $(offset,size)$ 所指定的區域越限,
  或者 \carg{size} 是 0,
  或者 \carg{map_flags} 的值無效。

\item \cenum{CL_INVALID_EVENT_WAIT_LIST},
  如果 \carg{event_wait_list} 是 \cenum{NULL},且 \carg{num_events_in_wait_list} > 0;
  或者 \carg{event_wait_list} 不是 \cenum{NULL},但 \carg{num_events_in_wait_list} 是 0;
  或者 \carg{event_wait_list} 中的事件無效。

\item \cenum{CL_MISALIGNED_SUB_BUFFER_OFFSET},如果 \carg{buffer} 是子\cnglo{bufobj},
  且創建此對象時所指定的 \carg{offset} 沒有與 \carg{queue} 所關聯\cnglo{device}的 \cenum{CL_DEVICE_MEM_BASE_ADDR_ALIGN} 對齊。

\item \cenum{CL_MAP_FAILURE},如果映射失敗。
  對於用 \cenum{CL_MEM_USE_HOST_PTR} 或 \cenum{CL_MEM_ALLOC_HOST_PTR} 創建的\cnglo{bufobj}不能出現此錯誤。

\item \cenum{CL_INVALID_OPERATION},
  如果創建 \carg{buffer} 時在 \carg{map_flags} 中指定了 \cenum{CL_MEM_HOST_WRITE_ONLY} 或 \cenum{CL_MEM_HOST_NO_ACCESS} 以及 \cenum{CL_MAP_READ},
  或者創建 \carg{buffer} 時在 \carg{map_flags} 中指定了 \cenum{CL_MEM_HOST_READ_ONLY} 或 \cenum{CL_MEM_HOST_NO_ACCESS} 以及 \cenum{CL_MAP_WRITE} 或 \cenum{CL_MAP_WRITE_INVALIDATE_REGION}。

\item \cenum{CL_EXEC_STATUS_ERROR_FOR_EVENTS_IN_WAIT_LIST},
  如果映射操作是阻塞的,且 \carg{event_wait_list} 中任一事件的執行狀態是負整數。

\item \cenum{CL_MEM_OBJECT_ALLOCATION_FAILURE},如果為 \carg{buffer} 分配內存失敗。

\item \cenum{CL_OUT_OF_RESOURCES}——如果\scdevfailres。
\item \cenum{CL_OUT_OF_HOST_MEMORY}——如果\schostfailres。
\stopigBase

所返回的指針僅可用於訪問所映射的區域 $(offset,size)$,雖然實現可能映射更大的區域。
對於訪問此區域外的內容,其結果未定義。

注意:

如果創建\cnglo{bufobj}時在 \carg{mem_flags} 中指定了 \cenum{CL_MEM_USE_HOST_PTR},
則:
\startigBase
\item 當 \capi{clEnqueueMapBuffer} 的\cnglo{cmd}完成後,
  保證 \capi{clCreateBuffer} 的 \carg{host_ptr} 中所映射的區域的內容是最新的。

\item \capi{clEnqueueMapBuffer} 所返回的指針得自創建\cnglo{bufobj}的 \carg{host_ptr}。
\stopigBase

被映射過的\cnglo{bufobj}使用 \capi{clEnqueueUnmapMemObject} 進行解映射。參見\todo{section 5.4.2}。

\placetable[here,force][tab:clmapflags]{所支持的 \ctype{cl_map_flags}}
{\startED[\ctype{cl_map_flags}]

\clED{CL_MAP_READ}{
  表明所映射區域是用來讀的。
  當 \capi{clEnqueueMap{Buffer | Image}} 的\cnglo{cmd}完成時,保證所返回的指針中所映射區域的內容是最新的。
}

\clED{CL_MAP_WRITE}{
  表明所映射區域是用來寫的。
  當 \capi{clEnqueueMap{Buffer | Image}} 的\cnglo{cmd}完成時,保證所返回的指針中所映射區域的內容是最新的。
}

\clED{CL_MAP_WRITE_INVALIDATE_REGION}{
  表明所映射區域是用來寫的。
  所映射區域中的原有內容會被丟棄。一種典型的情況就是\cnglo{host}會修改( overwrite )其內容。
  此標記告訴實作可以不再保證 \capi{clEnqueueMap{Buffer | Image}} 所返回的指針中所映射區域的內容是最新的,
  這會極大提升性能。

  它和 \cenum{CL_MAP_WRITE} 是互斥的。
}

\stopED

}

