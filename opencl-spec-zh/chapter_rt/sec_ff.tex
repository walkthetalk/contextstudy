\section{刷新和完結}

% clFlush
函數
\startclc
cl_int clFlush (cl_command_queue command_queue)
\stopclc
會將 \carg{command_queue} 中所有之前入隊的 OpenCL \cnglo{cmd}
都提交給 \carg{command_queue} 所在\cnglo{device}去執行。
 \capi{clFlush} 返回時僅保證\cnglo{cmd}已經提交,不保證執行完畢。

如果執行成功, \capi{clFlush} 會返回 \cenum{CL_SUCCESS}。
否則,返回下列錯誤碼之一:
\startigBase
\item \cenum{CL_INVALID_COMMAND_QUEUE},如果 \carg{command_queue} 無效。

\item \cenum{CL_OUT_OF_RESOURCES},如果\scdevfailres。

\item \cenum{CL_OUT_OF_HOST_MEMORY},如果\schostfailres。
\stopigBase

\cnglo{cmdq}中所有阻塞的\cnglo{cmd}
以及 \capi{clReleaseCommandQueue} 都會對\cnglo{cmdq}實施隱式的刷新(flush)。
這些阻塞的\cnglo{cmd}包括:
\startigBase
\item \capi{clEnqueueReadBuffer}、 \capi{clEnqueueReadBufferRect}、
 \capi{clEnqueueReadImage},其中參數 \carg{blocking_read} 設置成 \cenum{CL_TRUE}。

\item \capi{clEnqueueWriteBuffer}、 \capi{clEnqueueWriteBufferRect}、
 \capi{clEnqueueWriteImage},其中參數 \carg{blocking_write} 設置成 \cenum{CL_TRUE}。

\item \capi{clEnqueueMapBuffer}、 \capi{clEnqueueMapImage},
其中參數 \carg{blocking_map} 設置成 \cenum{CL_TRUE}。

\item \capi{clWaitForEvents}。
\stopigBase

如果\cnglo{cmdq}中有一個\cnglo{cmd} A,
而另一個\cnglo{cmdq}中有\cnglo{cmd} B 在等待用來指代\cnglo{cmd} A 的\cnglo{evtobj},
那麼\cnglo{app}必須對\cnglo{cmd} A 所在的\cnglo{cmdq}調用 \capi{clFlush},
或者任何會實施隱式刷新的的阻塞\cnglo{cmd}。

% clFinish
函數
\startclc
cl_int clFinish (cl_command_queue command_queue)
\stopclc
是阻塞的,等到 \carg{command_queue} 中所有之前入隊的\cnglo{cmd}
都被提交給相應的\cnglo{device}並執行完畢後才會返回。
 \capi{clFinish} 也是一個同步點。

如果執行成功, \capi{clFinish} 會返回 \cenum{CL_SUCCESS}。
否則,返回下列錯誤碼之一:
\startigBase
\item \cenum{CL_INVALID_COMMAND_QUEUE},如果 \carg{command_queue} 無效。

\item \cenum{CL_OUT_OF_RESOURCES},如果\scdevfailres。

\item \cenum{CL_OUT_OF_HOST_MEMORY},如果\schostfailres。
\stopigBase

