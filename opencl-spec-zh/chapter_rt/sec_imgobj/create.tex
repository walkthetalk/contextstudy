\subsection{創建圖像對象}
如下函數可以用來創建 1D 圖像、 1D 圖像緩衝( image buffer )、 1D 圖像陣列( image array )、
2D 圖像、 2D 圖像陣列以及 3D 圖像對象:
\startclc
cl_mem clCreateImage (cl_context context,
		cl_mem_flags flags,
		const cl_image_format *image_format,
		const cl_image_desc *image_desc,
		void *host_ptr,
		cl_int *errcode_ret)
\stopclc

\carg{context} 即所要創建的\cnglo{imgobj}所位於的 OpenCL \cnglo{context}。

\carg{flags} 是位於,用來指明所要創建的\cnglo{imgobj}如何分配以及怎樣使用,參見\reftab{clmemflags}。

對於所有類型的\cnglo{imgobj},除了 \cenum{CL_MEM_OBJECT_IMAGE1D_BUFFER},
如果 \carg{flags} 的值為 0,則使用缺省值 \cenum{CL_MEM_READ_WRITE}。

對於類型為 \cenum{CL_MEM_OBJECT_IMAGE1D_BUFFER} 的圖像對象,
如果 \carg{flags} 中沒有設置 \cenum{CL_MEM_READ_WRITE}、 \cenum{CL_MEM_READ_ONLY} 或 \cenum{CL_MEM_WRITE_ONLY},會從 \carg{buffer} 中繼承這些屬性。
而 \carg{flags} 中不能設置 \cenum{CL_MEM_USE_HOST_PTR}、 \cenum{CL_MEM_ALLOC_HOST_PTR} 和 \cenum{CL_MEM_COPY_HOST_PTR},這些也很由 \carg{buffer} 繼承。
即使 \carg{buffer} 的內存訪問限定符中有 \cenum{CL_MEM_COPY_HOST_PTR},也並不意味着創建 sub-buffer 時會有額外的拷貝。
如果 \carg{flags} 中沒有設置 \cenum{CL_MEM_HOST_WRITE_ONLY}、 \cenum{CL_MEM_HOST_READ_ONLY} 或 \cenum{CL_MEM_HOST_NO_ACCESS},則會從 \carg{buffer} 中繼承這些屬性。

\carg{image_format} 指明圖像的格式。參見\reftit{image format desc}。

\carg{image_desc} 指明圖像的類型以及維數。參見\reftit{image desc}。

\carg{host_ptr} 指向圖像數據(可能已由\cnglo{app}分配好)。
下表列出了 \carg{host_ptr} 所指向的 buffer 大小的一些限制。

\input{chapter_rt/tbl/tbl_imgtypesize.tex}

\subsubsection[tit:image format desc]{圖像格式描述符}

\subsubsection[tit:image desc]{圖像描述符}

