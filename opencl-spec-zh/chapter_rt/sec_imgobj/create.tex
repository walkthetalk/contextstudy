\subsection{創建圖像對象}
如下函式可以用來創建 1D 圖像、 1D 圖像緩衝( image buffer )、 1D 圖像陣列( image array )、
2D 圖像、 2D 圖像陣列以及 3D 圖像對象:
\startclc
cl_mem clCreateImage (cl_context context,
		cl_mem_flags flags,
		const cl_image_format *image_format,
		const cl_image_desc *image_desc,
		void *host_ptr,
		cl_int *errcode_ret)
\stopclc

\carg{context} 即所要創建的\cnglo{imgobj}所位於的 OpenCL \cnglo{context}。

\carg{flags} 是位於,用來指明所要創建的\cnglo{imgobj}如何分配以及怎樣使用,參見\reftab{clmemflags}。

對於所有類型的\cnglo{imgobj},除了 \cenum{CL_MEM_OBJECT_IMAGE1D_BUFFER},
如果 \carg{flags} 的值為 0,則使用缺省值 \cenum{CL_MEM_READ_WRITE}。

對於類型為 \cenum{CL_MEM_OBJECT_IMAGE1D_BUFFER} 的圖像對象,
如果 \carg{flags} 中沒有設置 \cenum{CL_MEM_READ_WRITE}、 \cenum{CL_MEM_READ_ONLY} 或 \cenum{CL_MEM_WRITE_ONLY},會從 \carg{buffer} 中繼承這些屬性。
而 \carg{flags} 中不能設置 \cenum{CL_MEM_USE_HOST_PTR}、 \cenum{CL_MEM_ALLOC_HOST_PTR} 和 \cenum{CL_MEM_COPY_HOST_PTR},這些也很由 \carg{buffer} 繼承。
即使 \carg{buffer} 的內存訪問限定符中有 \cenum{CL_MEM_COPY_HOST_PTR},也並不意味着創建 sub-buffer 時會有額外的拷貝。
如果 \carg{flags} 中沒有設置 \cenum{CL_MEM_HOST_WRITE_ONLY}、 \cenum{CL_MEM_HOST_READ_ONLY} 或 \cenum{CL_MEM_HOST_NO_ACCESS},則會從 \carg{buffer} 中繼承這些屬性。

\carg{image_format} 指明圖像的格式。參見\refsec{imgFmtDsc}。

\carg{image_desc} 指明圖像的類型以及維數。參見\refsec{imgDsc}。

\carg{host_ptr} 指向圖像數據(可能已由\cnglo{app}分配好)。
下表列出了 \carg{host_ptr} 所指向的 buffer 大小的一些限制。

\bTABLE[option=stretch]

\bTABLEhead
  \bTR[background=color,backgroundcolor=gray]
    \bTH 圖像類型 \eTH
    \bTH \carg{host_ptr} 指向的 buffer 的大小 \eTH
  \eTR
\eTABLEhead

\bTABLEbody
  \bTR
    \bTD \cenum{CL_MEM_OBJECT_IMAGE1D} \eTD
    \bTD >= image_row_pitch \eTD
  \eTR
  \bTR
    \bTD \cenum{CL_MEM_OBJECT_IMAGE1D_BUFFER} \eTD
    \bTD >= image_row_pitch \eTD
  \eTR
  \bTR
    \bTD \cenum{CL_MEM_OBJECT_IMAGE2D} \eTD
    \bTD >= image_row_pitch * image_height \eTD
  \eTR
  \bTR
    \bTD \cenum{CL_MEM_OBJECT_IMAGE3D} \eTD
    \bTD >= image_slice_pitch * image_depth \eTD
  \eTR
  \bTR
    \bTD \cenum{CL_MEM_OBJECT_IMAGE1D_ARRAY} \eTD
    \bTD >= image_slice_pitch * image_array_size \eTD
  \eTR
  \bTR
    \bTD \cenum{CL_MEM_OBJECT_IMAGE2D_ARRAY} \eTD
    \bTD >= image_slice_pitch * image_array_size \eTD
  \eTR
\eTABLEbody

\eTABLE


對於 3D 圖像 或 2D 圖像陣列, \carg{host_ptr} 所指向的圖像數據分別按相鄰的 2D 圖像平面或 2D 圖像線性序列進行存儲。
每個 2D 圖像都是相鄰掃描線( scanline )的線性序列。
每個掃描線都是相鄰圖像元素的線性序列。

對於 2D 圖像, \carg{host_ptr} 所指向的圖像數據按相鄰掃描線( scanline )的線性序列進行存儲。
每個掃描線都是相鄰圖像元素的線性序列。

對於 1D 圖像陣列, \carg{host_ptr} 所指向的圖像數據按相鄰 1D 圖像的線性序列進行存儲。
每個 1D 圖像就是一個掃描線,是相鄰圖像元素的線性序列。

\carg{errcode_ret} 會返回相應的錯誤碼。
如果 \carg{errcode_ret} 是 \cenum{NULL},不會返回錯誤碼。

如果成功創建了\cnglo{imgobj}, \capi{clCreateImage} 會將其返回,並將 \carg{errcode_ret} 置為 \cenum{CL_SUCCESS}。
否則,返回 \cenum{NULL} 並將 \carg{errcode_ret} 置為下列錯誤碼之一:
\startigBase
\item \cenum{CL_INVALID_CONTEXT},如果 \carg{context} 無效。

\item \cenum{CL_INVALID_VALUE},如果 \carg{flags} 的值無效。

\item \cenum{CL_INVALID_IMAGE_FORMAT_DESCRIPTOR},
  如果 \carg{image_format} 中的值無效,
  或者 \carg{image_format} 是 \cenum{NULL}。

\item \cenum{CL_INVALID_IMAGE_DESCRIPTOR},
  如果 \carg{image_desc} 中的值無效,
  或者 \carg{image_desc} 是 \cenum{NULL}。

\item \cenum{CL_INVALID_IMAGE_SIZE},
  如果對於 \carg{context} 中的所有\cnglo{device}而言,
 \carg{image_desc} 中圖像任一維度的值超過了\reftab{cldevquery}中相應維度的最大值。

\item \cenum{CL_INVALID_HOST_PTR},
  如果 \carg{image_desc} 中 \carg{host_ptr} 是 \cenum{NULL},且 \carg{flags} 中設置了 \cenum{CL_MEM_USE_HOST_PTR} 或 {CL_MEM_COPY_HOST_PTR};
  或者 \carg{host_ptr} 不是 \cenum{NULL},但 \carg{flags} 中設置了 \cenum{CL_MEM_COPY_HOST_PTR} 或 {CL_MEM_USE_HOST_PTR}。

\item \cenum{CL_INVALID_VALUE},如果要創建的是 1D \cnglo{imgobj},
  為其指定 \cenum{CL_MEM_WRITE_ONLY} 的同時 \carg{flags} 中設置了 \cenum{CL_MEM_READ_WRITE} 或 {CL_MEM_READ_ONLY};
  或者為其指定 \cenum{CL_MEM_READ_ONLY} 的同時 \carg{flags} 中設置了 \cenum{CL_MEM_READ_WRITE} 或 {CL_MEM_WRITE_ONLY};
  或者 \carg{flags} 中設置了 \cenum{CL_MEM_USE_HOST_PTR} 或 {CL_MEM_ALLOC_HOST_PTR} 或 \cenum{CL_MEM_COPY_HOST_PTR}。

\item \cenum{CL_INVALID_VALUE},如果要創建的是 1D \cnglo{imgobj},
  為其指定 \cenum{CL_MEM_HOST_WRITE_ONLY} 的同時 \carg{flags} 中設置了 \cenum{CL_MEM_HOST_READ_ONLY};
  或者為其指定 \cenum{CL_MEM_HOST_READ_ONLY} 的同時 \carg{flags} 中設置了 \cenum{CL_MEM_HOST_WRITE_ONLY};
  或者為其指定 \cenum{CL_MEM_HOST_NO_ACCESS} 的同時 \carg{flags} 中設置了 \cenum{CL_MEM_HOST_READ_ONLY} 或 \cenum{CL_MEM_HOST_WRITE_ONLY}。

\item \cenum{CL_IMAGE_FORMAT_NOT_SUPPORTED},如果 \carg{image_format} 的值不受支持。

\item \cenum{CL_MEM_OBJECT_ALLOCATION_FAILURE},如果為\cnglo{imgobj}分配內存失敗。

\item \cenum{CL_INVALID_OPERATION},如果 \carg{context} 中的所有\cnglo{device}都不支持圖像
  (即\reftab{cldevquery}中的 \cenum{CL_DEVICE_IMAGE_SUPPORT} 是 \cenum{CL_FALSE} )。

\item \cenum{CL_OUT_OF_RESOURCES}——如果\scdevfailres。

\item \cenum{CL_OUT_OF_HOST_MEMORY}——如果\schostfailres。

\stopigBase

\subsubsection[sec:imgFmtDsc]{圖像格式描述符}

圖像格式描述符的結構定義如下:
\startclc
typedef struct _cl_image_format {
	cl_channel_order	image_channel_order;
	cl_channel_type		image_channel_data_type;
} cl_image_format;
\stopclc

\cvar{image_channel_order} 指定了通道( channel )數目以及通道布局,
即圖像中通道存儲的內存布局。其有效值參見\reftab{imgChannelOrder}。

\cvar{image_channel_data_type} 即通道的數據類型。
其有效值參見\reftab{imgChannelDataType}。

由以上兩者所確定的圖像元素的 bit 數目必須是 2 的指數。

\placetable[here,force][tab:imgChannelOrder]{圖像通道順序}
{\bTABLE[option=stretch]

\bTABLEhead
\bTR[background=color,backgroundcolor=gray]
  \bTH channel_order 的值 \eTH
  \bTH 只有通道數據類型為下列值時才能使用此格式 \eTH
\eTR
\eTABLEhead

\bTABLEbody

\bTR
  \bTD \cenum{CL_R}、 \cenum{CL_Rx}、 \cenum{CL_A} \eTD
  \bTD \eTD
\eTR

\bTR
  \bTD \cenum{CL_INTENSITY} \eTD
  \bTD \cenum{CL_UNORM_INT8}、 \cenum{CL_UNORM_INT16}、 \cenum{CL_SNORM_INT8}、 \cenum{CL_SNORM_INT16}、 \cenum{CL_HALF_FLOAT}、 \cenum{CL_FLOAT} \eTD
\eTR

\bTR
  \bTD \cenum{CL_LUMINANCE} \eTD
  \bTD \cenum{CL_UNORM_INT8}、 \cenum{CL_UNORM_INT16}、 \cenum{CL_SNORM_INT8}、 \cenum{CL_SNORM_INT16}、 \cenum{CL_HALF_FLOAT}、 \cenum{CL_FLOAT} \eTD
\eTR

\bTR
  \bTD \cenum{CL_RG}、 \cenum{CL_RGx}、 \cenum{CL_RA} \eTD
  \bTD \eTD
\eTR

\bTR
  \bTD \cenum{CL_RGB} 或者 \cenum{CL_RGBx} \eTD
  \bTD \cenum{CL_UNORM_SHORT_565}、 \cenum{CL_UNORM_SHORT_555}、 \cenum{CL_UNORM_INT_101010} \eTD
\eTR

\bTR
  \bTD \cenum{CL_RGBA} \eTD
  \bTD \cenum{CL_UNORM_INT8}、 \cenum{CL_SNORM_INT8}、 \cenum{CL_SIGNED_INT8}、 \cenum{CL_UNSIGNED_INT8} \eTD
\eTR

\eTABLEbody

\eTABLE

}

\placetable[here,force][tab:imgChannelDataType]{圖像通道數據類型}
{\bTABLE[option=stretch]

\bTABLEhead
\bTR[background=color,backgroundcolor=gray]
  \bTH 圖像通道數據類型 \eTH
  \bTH 描述 \eTH
\eTR
\eTABLEhead

\bTABLEbody

\bTR
  \bTD \cenum{CL_SNORM_INT8} \eTD
  \bTD 各通道均為規範化帶符號 8 位整數 \eTD
\eTR

\bTR
  \bTD \cenum{CL_SNORM_INT16} \eTD
  \bTD 各通道均為規範化帶符號 16 位整數 \eTD
\eTR

\bTR
  \bTD \cenum{CL_UNORM_INT8} \eTD
  \bTD 各通道均為規範化無符號 8 位整數 \eTD
\eTR

\bTR
  \bTD \cenum{CL_UNORM_INT16} \eTD
  \bTD 各通道均為規範化無符號 16 位整數 \eTD
\eTR

\bTR
  \bTD \cenum{CL_UNORM_SHORT_565} \eTD
  \bTD 規範化 5-6-6 3 通道 RGB 圖像。通道順序必須是 \cenum{CL_RGB} 或 \cenum{CL_RGBx}。 \eTD
\eTR

\bTR
  \bTD \cenum{CL_UNORM_INT_101010} \eTD
  \bTD 規範化 x-10-10-10 4 通道 xRGB 圖像。通道順序必須是 \cenum{CL_RGB} 或 \cenum{CL_RGBx}。 \eTD
\eTR

\bTR
  \bTD \cenum{CL_SIGNED_INT8} \eTD
  \bTD 各通道均為非規範化帶符號 8 位整數 \eTD
\eTR

\bTR
  \bTD \cenum{CL_SIGNED_INT16} \eTD
  \bTD 各通道均為非規範化帶符號 16 位整數 \eTD
\eTR

\bTR
  \bTD \cenum{CL_SIGNED_INT32} \eTD
  \bTD 各通道均為非規範化帶符號 32 位整數 \eTD
\eTR

\bTR
  \bTD \cenum{CL_UNSIGNED_INT8} \eTD
  \bTD 各通道均為非規範化無符號 8 位整數 \eTD
\eTR

\bTR
  \bTD \cenum{CL_UNSIGNED_INT16} \eTD
  \bTD 各通道均為非規範化無符號 16 位整數 \eTD
\eTR

\bTR
  \bTD \cenum{CL_UNSIGNED_INT32} \eTD
  \bTD 各通道均為非規範化無符號 32 位整數 \eTD
\eTR

\bTR
  \bTD \cenum{CL_HALF_FLOAT} \eTD
  \bTD 個通道均為 16 位半浮點數 \eTD
\eTR

\bTR
  \bTD \cenum{CL_FLOAT} \eTD
  \bTD 個通道均為單精度浮點數 \eTD
\eTR

\eTABLEbody

\eTABLE

}

例如,如果 \cvar{image_channel_order} = \cenum{CL_RGBA}, \cvar{image_channel_data_type} = \cenum{CL_UNORM_INT8},
此格式的內存布局為:

\bTABLE[frame=off]
\bTR[align={middle,middle}]
\bTD[width=.3\textwidth] \eTD
\bTD[frame=on,width=.1\textwidth] R \eTD
\bTD[frame=on,width=.1\textwidth] G \eTD
\bTD[frame=on,width=.1\textwidth] B \eTD
\bTD[frame=on,width=.1\textwidth] A \eTD
\bTD[frame=on,width=.3\textwidth] …………………… \eTD
\eTR
\bTR[align={hi}]
\bTD[align=left] 字節偏移$\rightarrow$ \eTD \bTD 0 \eTD \bTD 1 \eTD \bTD 2 \eTD \bTD 3 \eTD \bTD \eTD
\eTR
\eTABLE

類似,如果 \cvar{image_channel_order} = \cenum{CL_RGBA}, \cvar{image_channel_data_type} = \cenum{CL_SIGNED_INT16},
此格式的內存布局為:

\bTABLE
\bTR[frame=on,align={middle,middle}]
\bTD[frame=off,width=.3\textwidth] \eTD
\bTD[width=.1\textwidth] R \eTD
\bTD[width=.1\textwidth] G \eTD
\bTD[width=.1\textwidth] B \eTD
\bTD[width=.1\textwidth] A \eTD
\bTD[width=.3\textwidth] …………………… \eTD
\eTR
\bTR[frame=off,align={hi}]
\bTD[align=left] 字節偏移$\rightarrow$ \eTD \bTD 0 \eTD \bTD 2 \eTD \bTD 4 \eTD \bTD 6 \eTD \bTD \eTD
\eTR
\eTABLE

對於 \cenum{CL_UNORM_SHORT_565}、 \cenum{CL_UNORM_SHORT_555} 和 \cenum{CL_UNORM_INT_101010} 這三種通道數據類型比較特殊,
他們都是壓縮過的圖像格式,每個元素的所有通道數據被壓縮到一個無符號短整形或無符號整形中。
在壓縮時,一般第一個通道佔據最高位( most significant bit ),後續通道相繼佔據次高位。
對於 \cenum{CL_UNORM_SHORT_565}, \ccmm{R} 佔據比特 \ccmm{15:11};
 \ccmm{G} 佔據比特 \ccmm{10:5}; \ccmm{B} 佔據比特 \ccmm{4:0}。
對於 \cenum{CL_UNORM_SHORT_555},比特 \ccmm{15} 未定義;
 \ccmm{R} 佔據比特 \ccmm{14:10}; \ccmm{G} 佔據比特 \ccmm{9:5}; \ccmm{B} 佔據比特 \ccmm{4:0}。
對於 \cenum{CL_UNORM_INT_101010}, 比特 \ccmm{31:30} 未定義;
 \ccmm{R} 佔據比特 \ccmm{29:20}; \ccmm{G} 佔據比特 \ccmm{19:10}; \ccmm{B} 佔據比特 \ccmm{9:0}。

OpenCL 實作必須保持 \cvar{image_channel_data_type} 中比特數所指定的最小精度。
如果 OpenCL 實作不支持由 \cvar{image_channel_order} 和 \cvar{image_channel_data_type} 所定義的圖像格式,
則 \capi{clCreateImage} 會返回 \cmacro{NULL}。


\subsubsection[sec:imgDsc]{圖像描述符}

圖像描述符描述了圖像或圖像陣列的類型和維數,其定義如下:
\startclc
typedef struct _cl_image_desc {
	cl_mem_object_type	image_type,
	size_t			image_width;
	size_t			image_height;
	size_t			image_depth;
	size_t			image_array_size;
	size_t			image_row_pitch;
	size_t			image_slice_pitch;
	cl_uint			num_mip_levels;
	cl_uint			num_samples;
	cl_mem			buffer;
} cl_image_desc;
\stopclc

\cvar{image_type} 即圖像類型,其值必須是 \cenum{CL_MEM_OBJECT_IMAGE1D}、
 \cenum{CL_MEM_OBJECT_IMAGE1D_BUFFER}、 \cenum{CL_MEM_OBJECT_IMAGE1D_ARRAY}、
 \cenum{CL_MEM_OBJECT_IMAGE2D}、 \cenum{CL_MEM_OBJECT_IMAGE2D_ARRAY}、
 \cenum{CL_MEM_OBJECT_IMAGE3D}。

\cvar{image_width} 即圖像寬度,單位:像素。
對於 2D 圖像或圖像陣列,其寬度必須 <= \cenum{CL_DEVICE_IMAGE2D_MAX_WIDTH}。
對於 3D 圖像,其寬度必須 <= \cenum{CL_DEVICE_IMAGE3D_MAX_WIDTH}。
對於 1D 圖像 buffer,其寬度必須 <= \cenum{CL_DEVICE_IMAGE_MAX_BUFFER_SIZE}。
對於 1D 圖像和 1D 圖像陣列,其寬度必須 <= \cenum{CL_DEVICE_IMAGE2D_MAX_WIDTH}。

\cvar{image_height} 即圖像高度,單位:像素。只有圖像為 2D、 3D 或 2D 圖像陣列時才有效。
對於 2D 圖像或圖像陣列,其寬度必須 <= \cenum{CL_DEVICE_IMAGE2D_MAX_HEIGHT}。
對於 3D 圖像,其寬度必須 <= \cenum{CL_DEVICE_IMAGE3D_MAX_HEIGHT}。

\cvar{image_depth} 即圖像深度,單位:像素。
只有圖像為 3D 圖像時才有效,其值必須 >= 1 且 <= \cenum{CL_DEVICE_IMAGE3D_MAX_DEPTH}。

\startbuffer[footnoteimagearraysize]
對於\cnglo{kernel}而言,以 $image\_array\_size = 1$ 讀寫 2D 圖像陣列時,其性能可能會比讀寫 2D 圖像要低。
\stopbuffer
\cvar{image_array_size}\footnote{\getbuffer[footnoteimagearraysize]} 即圖像陣列中圖像個數。
只有圖像為 1D 或 2D 圖像陣列時才有效。
如果設置了 \cvar{image_array_size},其值必須 >= 1 且 <= \cenum{CL_DEVICE_IMAGE_MAX_ARRAY_SIZE}。

\cvar{image_row_pitch} 即掃描線間隔,單位字節。
如果 \carg{host_ptr} 是 \cmacro{NULL},其值必須是 0;
如果 \carg{host_ptr} 不是 \cmacro{NULL},則可以是 0 或者 >= \cvar{image_width} * 元素大小。
如果 \carg{host_ptr} 不是 \cmacro{NULL},並且 \cvar{image_row_pitch} = 0,則用 \cvar{image_width} * 元素大小 代替 \cvar{image_row_pitch}。
如果 \cvar{imag_row_pitch} 不是 0,則必須是圖像元素大小的整數倍。

\cvar{image_slice_pitch} 即 3D 圖像中每個 2D 平面的大小,或者 1D、 2D 圖像陣列中每個圖像的大小。單位字節。
如果 \carg{host_ptr} 是 \cmacro{NULL},其值必須是 0;
如果 \carg{host_ptr} 不是 \cmacro{NULL},
  對於 2D 圖像陣列或 3D 圖像,可以是 0 或者 >= \cvar{image_row_pitch} * \cvar{image_height};
  對於 1D 圖像陣列,可以是 0 或者 >= \cvar{image_row_pitch}。
如果 \carg{host_ptr} 不是 \cmacro{NULL},並且 \cvar{image_slice_pitch} = 0,
  對於 2D 圖像陣列或 3D 圖像,則用 \cvar{image_row_pitch} * \cvar{image_height} 代替 \cvar{image_slice_pitch};
  對於 1D 圖像陣列,則用 \cvar{image_row_pitch}。
如果 \cvar{imag_slice_pitch} 不是 0,則必須是 \cvar{image_row_pitch} 的整數倍。

\cvar{num_mip_levels} 和 \cvar{num_samples} 必須是 0。

如果 \cvar{image_type} 是 \cenum{CL_MEM_OBJECT_IMAGE1D_BUFFER},則 \cvar{buffer} 引用一個\cnglo{bufobj},否則 \cvar{buffer} 必須是 \cmacro{NULL}。
對於 1D 圖像\cnglo{bufobj},圖像的所有像素均取自\cnglo{bufobj}的數據。
在相應的同步點上,對\cnglo{bufobj}中數據的改動也會反映到 1D 圖像的內容上,反之亦然。
 \cvar{image_width} * 元素大小 必須 <= \cnglo{bufobj}中的數據大小。

注意:

對一個\cnglo{bufobj} 和 與其關聯的 1D 圖像\cnglo{bufobj} 的並發讀、寫以及拷貝是未定義的。
只有讀是定義了的。



