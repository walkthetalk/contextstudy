\subsection{圖像對象和緩衝對象間的拷貝}

%% clEnqueueCopyImageToBuffer
函式
\startclc
cl_int clEnqueueCopyImageToBuffer (cl_command_queue command_queue,
			cl_mem src_image,
			cl_mem dst_buffer,
			const size_t *src_origin,
			const size_t *region,
			size_t dst_offset,
			cl_uint num_events_in_wait_list,
			const cl_event *event_wait_list,
			cl_event *event)
\stopclc
會在隊列中插入一個\cnglo{cmd},用來將\cnglo{imgobj}拷貝到\cnglo{bufobj}中。

\carg{command_queue} 就是拷貝\cnglo{cmd}要進入的隊列。
\carg{command_queue}、 \carg{src_image}、 \carg{dst_buffer} 必須位於同一 OpenCL \cnglo{context}中。

\carg{src_image} 是一個\cnglo{imgobj}。
\carg{dst_buffer} 是一個\cnglo{bufobj}。

\carg{src_origin} 定義了源區域的起始位置 $(x, y, z)$,單位像素;
對於 2D 圖像陣列,即 $(x, y)$ 以及圖像索引;
對於 1D 圖像陣列,即 $(x)$ 以及圖像索引。
如果 \carg{src_image} 是 2D \cnglo{imgobj}, $src\_origin[2]$ 必須是 0。
如果 \carg{src_image} 是 1D \cnglo{imgobj},$src\_origin[1]$ 和 $src\_origin[2]$ 都必須是 0。
如果 \carg{src_image} 是 1D 圖像陣列對象, $src\_origin[1]$ 即圖像索引。
如果 \carg{src_image} 是 2D 圖像陣列對象, $src\_origin[2]$ 即圖像索引。

\carg{region} 定義了所拷貝區域的大小 $(width, height, depth)$,單位像素;
對於 2D 圖像陣列,即 $(width, height)$ 以及圖像索引;
對於 1D 圖像陣列,即 $(width)$ 以及圖像索引。
如果 \carg{src_image} 是 2D \cnglo{imgobj} 或 1D 圖像陣列對象, $region[2]$ 必須是 1。
如果 \carg{src_image} 是 1D 圖像或 1D 圖像 buffer 對象, $src\_origin[1]$ 和 $region[2]$ 必須是 1。

\carg{dst_offset} 即拷貝時目的區域在 \carg{dst_buffer} 中的起始位置。
目的區域的大小,即 \carg{dst_cb} 這樣計算:
如果 \carg{src_image} 是 3D \cnglo{imgobj},則為 $width * height * depth * bytes/image element$;
如果 \carg{src_image} 是 2D \cnglo{imgobj},則為 $width * height * bytes/image element$;
如果 \carg{src_image} 是 2D 圖像陣列對象,則為 $width * height * arraysize * bytes/image element$;
如果 \carg{src_image} 是 1D 圖像或 1D 圖像 buffer 對象,則為 $width * bytes/image element$;
如果 \carg{src_image} 是 1D 圖像陣列對象,則為 $width * arraysize * bytes/image element$;

\carg{event_wait_list} 和 \carg{num_events_in_wait_list} 中列出了執行此\cnglo{cmd}前要等待的事件。
如果 \carg{event_wait_list} 是 \cenum{NULL},則無須等待任何事件,並且 \carg{num_events_in_wait_list} 必須是0。
如果 \carg{event_wait_list} 不是 \cenum{NULL},則其中所有事件都必須是有效的,並且 \carg{num_events_in_wait_list} 必須大於 0。
\carg{event_wait_list} 中的事件充當同步點,並且必須與 \carg{command_queue} 位於同一個\cnglo{context}中。
此函式返回後,即可回收並重新使用 \carg{event_wait_list} 所關聯的內存。

\carg{event} 會返回一個\cnglo{evtobj},用來標識此讀、寫\cnglo{cmd},可用來查詢或等待此\cnglo{cmd}完成。
而如果 \carg{event} 是 \cenum{NULL},就沒辦法查詢此\cnglo{cmd}的狀態或等待其完成了。
這時可以用 \capi{clEnqueueBarrierWithWaitList} 來代替。
如果 \carg{event_wait_list} 和 \carg{event} 都不是 \cenum{NULL}, \carg{event} 不能屬於 \carg{event_wait_list}。

如果執行成功, \capi{clEnqueueCopyImageToBuffer} 會返回 \cenum{CL_SUCCESS}。
否則,返回下列錯誤碼之一:
\startigBase
\item \cenum{CL_INVALID_COMMAND_QUEUE},如果 \carg{command_queue} 無效。

\item \cenum{CL_INVALID_CONTEXT},
  如果 \carg{command_queue}、 \carg{src_image} 和 \carg{dst_buffer} 位於不同的\cnglo{context}中,
  或者 \carg{command_queue} 和 \carg{event_wait_list} 中的事件位於不同的\cnglo{context}中。

\item \cenum{CL_INVALID_MEM_OBJECT},如果 \carg{src_image} 或 \carg{dst_buffer} 無效;
  或者 \carg{src_image} 是由 \carg{dst_buffer} 創建的 1D 圖像 buffer 對象。

\item \cenum{CL_INVALID_VALUE},
  如果 $(src\_origin, region)$ 所指定的區域在 \carg{src_image} 中越限,
  或者 $(dst\_offset, dst\_cb)$ 所指定的區域在 \carg{dst_buffer} 中越限。

\item \cenum{CL_INVALID_VALUE},如果 \carg{src_origin} 和 \carg{region} 沒有遵守上面所描述的相應規則。

\item \cenum{CL_INVALID_EVENT_WAIT_LIST},如果 \carg{event_wait_list} 是 \cenum{NULL},
  且 \carg{num_events_in_wait_list} > 0;
  或者 \carg{event_wait_list} 不是 \cenum{NULL},但 \carg{num_events_in_wait_list} 是 0;
  或者 \carg{event_wait_list} 中的事件無效。

\item \cenum{CL_MISALIGNED_SUB_BUFFER_OFFSET},
  如果 \carg{dst_buffer} 是一個子\cnglo{bufobj},
  並且創建他時所指定的參數 \carg{offset} 沒有和 \carg{queue} 所在\cnglo{device}的 \cenum{CL_DEVICE_MEM_BASE_ADDR_ALIGN} 對齊。

\item \cenum{CL_INVALID_IMAGE_SIZE},如果圖像的大小(寬度、高度、指定的或自動計算的行間距或平面間距)不被 \carg{queue} 所在\cnglo{device} 支持。

\item \cenum{CL_IMAGE_FORMAT_NOT_SUPPORTED},如果 \carg{src_image} 的圖像格式(圖像通道順序和數據類型)不被 \carg{queue} 所在\cnglo{device} 支持。

\item \cenum{CL_MEM_OBJECT_ALLOCATION_FAILURE},如果為 \carg{src_image} 或 \carg{dst_buffer} 分配內存失敗。

\item \cenum{CL_INVALID_OPERATION},如果 \carg{command_queue} 所在\cnglo{device}不支持圖像
(即,\reftab{cldevquery}中的 \cenum{CL_DEVICE_IMAGE_SUPPORT} 是 \cenum{CL_FALSE})。)。

\item \cenum{CL_OUT_OF_RESOURCES}——如果\scdevfailres。
\item \cenum{CL_OUT_OF_HOST_MEMORY}——如果\schostfailres。
\stopigBase

%% clEnqueueCopyBufferToImage
函式
\startclc
cl_int clEnqueueCopyBufferToImage (cl_command_queue command_queue,
			cl_mem src_buffer,
			cl_mem dst_image,
			size_t src_offset,
			const size_t *dst_origin,
			const size_t *region,
			cl_uint num_events_in_wait_list,
			const cl_event *event_wait_list,
			cl_event *event)
\stopclc
會在隊列中插入一個\cnglo{cmd},用來將\cnglo{bufobj}拷貝到\cnglo{imgobj}中。

\carg{command_queue} 就是拷貝\cnglo{cmd}要進入的隊列。
\carg{command_queue}、 \carg{src_buffer}、 \carg{dst_image} 必須位於同一 OpenCL \cnglo{context}中。

\carg{src_buffer} 是一個\cnglo{bufobj}。
\carg{dst_image} 是一個\cnglo{imgobj}。

\carg{src_offset} 即拷貝時源區域在 \carg{src_buffer} 中的起始位置。

\carg{dst_origin} 定義了目的區域的起始位置 $(x, y, z)$,單位像素;
對於 2D 圖像陣列,即 $(x, y)$ 以及圖像索引;
對於 1D 圖像陣列,即 $(x)$ 以及圖像索引。
如果 \carg{dst_image} 是 2D \cnglo{imgobj}, $dst\_origin[2]$ 必須是 0。
如果 \carg{dst_image} 是 1D \cnglo{imgobj},$dst\_origin[1]$ 和 $dst\_origin[2]$ 都必須是 0。
如果 \carg{dst_image} 是 1D 圖像陣列對象, $dst\_origin[1]$ 即圖像索引。
如果 \carg{dst_image} 是 2D 圖像陣列對象, $dst\_origin[2]$ 即圖像索引。

\carg{region} 定義了所拷貝區域的大小 $(width, height, depth)$,單位像素;
對於 2D 圖像陣列,即 $(width, height)$ 以及圖像索引;
對於 1D 圖像陣列,即 $(width)$ 以及圖像索引。
如果 \carg{dst_image} 是 2D \cnglo{imgobj} 或 1D 圖像陣列對象, $region[2]$ 必須是 1。
如果 \carg{dst_image} 是 1D 圖像或 1D 圖像 buffer 對象, $dst\_origin[1]$ 和 $region[2]$ 必須是 1。

所拷貝的源區域的大小,即 \carg{src_cb} 這樣計算:
如果 \carg{dst_image} 是 3D \cnglo{imgobj},則為 $width * height * depth * bytes/image element$;
如果 \carg{dst_image} 是 2D \cnglo{imgobj},則為 $width * height * bytes/image element$;
如果 \carg{dst_image} 是 2D 圖像陣列對象,則為 $width * height * arraysize * bytes/image element$;
如果 \carg{dst_image} 是 1D 圖像或 1D 圖像 buffer 對象,則為 $width * bytes/image element$;
如果 \carg{dst_image} 是 1D 圖像陣列對象,則為 $width * arraysize * bytes/image element$;

\carg{event_wait_list} 和 \carg{num_events_in_wait_list} 中列出了執行此\cnglo{cmd}前要等待的事件。
如果 \carg{event_wait_list} 是 \cenum{NULL},則無須等待任何事件,並且 \carg{num_events_in_wait_list} 必須是0。
如果 \carg{event_wait_list} 不是 \cenum{NULL},則其中所有事件都必須是有效的,並且 \carg{num_events_in_wait_list} 必須大於 0。
\carg{event_wait_list} 中的事件充當同步點,並且必須與 \carg{command_queue} 位於同一個\cnglo{context}中。
此函式返回後,即可回收並重新使用 \carg{event_wait_list} 所關聯的內存。

\carg{event} 會返回一個\cnglo{evtobj},用來標識此讀、寫\cnglo{cmd},可用來查詢或等待此\cnglo{cmd}完成。
而如果 \carg{event} 是 \cenum{NULL},就沒辦法查詢此\cnglo{cmd}的狀態或等待其完成了。
這時可以用 \capi{clEnqueueBarrierWithWaitList} 來代替。
如果 \carg{event_wait_list} 和 \carg{event} 都不是 \cenum{NULL}, \carg{event} 不能屬於 \carg{event_wait_list}。

如果執行成功, \capi{clEnqueueCopyBufferToImage} 會返回 \cenum{CL_SUCCESS}。
否則,返回下列錯誤碼之一:
\startigBase
\item \cenum{CL_INVALID_COMMAND_QUEUE},如果 \carg{command_queue} 無效。

\item \cenum{CL_INVALID_CONTEXT},
  如果 \carg{command_queue}、 \carg{src_buffer} 和 \carg{dst_image} 位於不同的\cnglo{context}中,
  或者 \carg{command_queue} 和 \carg{event_wait_list} 中的事件位於不同的\cnglo{context}中。

\item \cenum{CL_INVALID_MEM_OBJECT},如果 \carg{src_buffer} 或 \carg{dst_image} 無效;
  或者 \carg{dst_image} 是由 \carg{src_buffer} 創建的 1D 圖像 buffer 對象。

\item \cenum{CL_INVALID_VALUE},
  如果 $(dst\_origin, region)$ 所指定的區域在 \carg{dst_image} 中越限,
  或者 $(src\_offset, src\_cb)$ 所指定的區域在 \carg{src_buffer} 中越限。

\item \cenum{CL_INVALID_VALUE},如果 \carg{dst_origin} 和 \carg{region} 沒有遵守參數說明中描述的相應規則。

\item \cenum{CL_INVALID_EVENT_WAIT_LIST},如果 \carg{event_wait_list} 是 \cenum{NULL},
  且 \carg{num_events_in_wait_list} > 0;
  或者 \carg{event_wait_list} 不是 \cenum{NULL},但 \carg{num_events_in_wait_list} 是 0;
  或者 \carg{event_wait_list} 中的事件無效。

\item \cenum{CL_MISALIGNED_SUB_BUFFER_OFFSET},
  如果 \carg{src_buffer} 是一個子\cnglo{bufobj},
  並且創建他時所指定的參數 \carg{offset} 沒有和 \carg{queue} 所在\cnglo{device}的 \cenum{CL_DEVICE_MEM_BASE_ADDR_ALIGN} 對齊。

\item \cenum{CL_INVALID_IMAGE_SIZE},如果 \carg{dst_image} 的圖像大小(寬度、高度、指定的或自動計算的行間距或平面間距)不被 \carg{queue} 所在\cnglo{device} 支持。

\item \cenum{CL_IMAGE_FORMAT_NOT_SUPPORTED},如果 \carg{dst_image} 的圖像格式(圖像通道順序和數據類型)不被 \carg{queue} 所在\cnglo{device} 支持。

\item \cenum{CL_MEM_OBJECT_ALLOCATION_FAILURE},如果為 \carg{src_buffer} 或 \carg{dst_image} 分配內存失敗。

\item \cenum{CL_INVALID_OPERATION},如果 \carg{command_queue} 所在\cnglo{device}不支持圖像
(即,\reftab{cldevquery}中的 \cenum{CL_DEVICE_IMAGE_SUPPORT} 是 \cenum{CL_FALSE})。)。

\item \cenum{CL_OUT_OF_RESOURCES}——如果\scdevfailres。
\item \cenum{CL_OUT_OF_HOST_MEMORY}——如果\schostfailres。
\stopigBase

