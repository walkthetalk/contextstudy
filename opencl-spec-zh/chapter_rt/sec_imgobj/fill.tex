\subsection{填充圖像對象}

函數
\startclc
cl_int clEnqueueFillImage (cl_command_queue command_queue,
			cl_mem image,
			const void *fill_color,
			const size_t *origin,
			const size_t *region,
			cl_uint num_events_in_wait_list,
			const cl_event *event_wait_list,
			cl_event *event)
\stopclc
會將一個\cnglo{cmd}插入隊列,此\cnglo{cmd}可以將\cnglo{imgobj}填充成某種顏色。
 \carg{image} 的一些使用信息,如是否可讀、是否可寫,
還有創建 \carg{image} 時所指定的參數 \ctype{cl_mem_flags},
都會被 \capi{clEnqueueFillImage} 忽略。

\carg{command_queue} 即填充\cnglo{cmd}所要插入的隊列。
\carg{command_queue} 和 \carg{image} 必須位於用一個 OpenCL \cnglo{context}中。

\carg{image} 是一個\cnglo{imgobj}。

\carg{fill_color} 即填充色。
如果 \carg{image} 通道數據型別不是非規範化帶符號或無符號整形,則它是四元 RGBA 浮點顏色值;
如果 \carg{image} 通道數據型別是非規範化帶符號整形,則它是四元帶符號整形值;
如果 \carg{image} 通道數據型別是非規範化無符號整形,則它是四元無符號整形值。
填充時會根據 \carg{image} 的圖像通道格式以及順序對此顏色進行轉換,參見\todo{6.12.14 和 8.3}。

\carg{origin} 定義了所要填充區域的起始位置 $(x, y, z)$,單位像素;
對於 2D 圖像陣列是 $(x, y)$ 以及圖像索引;
對於 1D 圖像陣列是 $(x)$ 以及圖像索引。
如果 \carg{image} 是 2D \cnglo{imgobj}, $origin[2]$ 必須是 0。
如果 \carg{image} 是 1D 圖像 或 1D 圖像 buffer 對象, $origin[1]$ 和 $origin[2]$ 都必須是 0。
如果 \carg{image} 是 1D 圖像陣列對象, $origin[2]$ 必須是 0, $origin[1]$ 是圖像在陣列中的索引。
如果 \carg{image} 是 2D 圖像陣列對象, $origin[2]$ 是圖像在陣列中的索引。

\carg{region} 定義了要填充區域的大小 $(width, height, depth)$,單位像素。
對於 2D 圖像陣列對象, 即 $(width, height)$ 以及圖像個數。
對於 1D 圖像陣列對象, 即 $(width)$ 以及圖像個數。
如果 \carg{image} 是 2D 圖像對象 或 1D 圖像陣列對象, $region[2]$ 必須是 1。
如果 \carg{image} 是 1D 圖像或 1D 圖像 buffer 對象, $region[1]$ 和 $region[2]$ 都必須是 0。

\carg{event_wait_list} 和 \carg{num_events_in_wait_list} 中列出了執行此\cnglo{cmd}前要等待的事件。
如果 \carg{event_wait_list} 是 \cenum{NULL},則無須等待任何事件,並且 \carg{num_events_in_wait_list} 必須是0。
如果 \carg{event_wait_list} 不是 \cenum{NULL},則其中所有事件都必須是有效的,並且 \carg{num_events_in_wait_list} 必須大於 0。
\carg{event_wait_list} 中的事件充當同步點,並且必須與 \carg{command_queue} 位於同一個\cnglo{context}中。
此函數返回後,可以回收並重新使用 \carg{event_wait_list} 所關聯的內存。

\carg{event} 會返回一個\cnglo{evtobj},用來標識此\cnglo{cmd},可用來查詢或等待此\cnglo{cmd}完成。
而如果 \carg{event} 是 \cenum{NULL},就沒辦法查詢此\cnglo{cmd}的狀態或等待其完成了。
不過可以用 \capi{clEnqueueBarrierWithWaitList} 代替。
如果 \carg{event_wait_list} 和 \carg{event} 都不是 \cenum{NULL}, \carg{event} 不能屬於 \carg{event_wait_list}。

如果執行成功, \capi{clEnqueueFillImage} 會返回 \cenum{CL_SUCCESS}。
否則,返回下列錯誤碼之一:
\startigBase
\item \cenum{CL_INVALID_COMMAND_QUEUE},如果 \carg{command_queue} 無效。

\item \cenum{CL_INVALID_CONTEXT},
  如果 \carg{command_queue} 和 \carg{image} 位於不同的\cnglo{context}中,
  或者 \carg{command_queue} 和 \carg{event_wait_list} 中的事件位於不同的\cnglo{context}中。

\item \cenum{CL_INVALID_MEM_OBJECT},如果 \carg{image} 無效。

\item \cenum{CL_INVALID_VALUE},如果 \carg{fill_color} 是 \cenum{NULL}。

\item \cenum{CL_INVALID_VALUE},如果 $(origin, region)$ 所指定的區域越限或者 \carg{ptr} 是 \cenum{NULL}。

\item \cenum{CL_INVALID_VALUE},如果 \carg{origin} 和 \carg{region} 沒有遵守前面參數描述中所列規則。

\item \cenum{CL_INVALID_EVENT_WAIT_LIST},
  如果 \carg{event_wait_list} 是 \cenum{NULL},且 \carg{num_events_in_wait_list} > 0;
  或者 \carg{event_wait_list} 不是 \cenum{NULL},但 \carg{num_events_in_wait_list} 是 0;
  或者 \carg{event_wait_list} 中的事件無效。

\item \cenum{CL_INVALID_IMAGE_SIZE},如果圖像的大小(寬度、高度、指定的或自動計算的行間距或平面間距)不被 \carg{queue} 所關聯\cnglo{device} 支持。

\item \cenum{CL_IMAGE_FORMAT_NOT_SUPPORTED},如果 \carg{image} 的圖像格式(圖像通道順序和數據型別)不被 \carg{queue} 所關聯\cnglo{device} 支持。

\item \cenum{CL_MEM_OBJECT_ALLOCATION_FAILURE},如果為 \carg{image} 分配內存失敗。

\item \cenum{CL_OUT_OF_RESOURCES}——如果\scdevfailres。
\item \cenum{CL_OUT_OF_HOST_MEMORY}——如果\schostfailres。
\stopigBase


