\subsection{圖像對象查詢}

對於所有\cnglo{memobj}都適用的資訊可以使用 \capi{clGetMemObjectInfo} 來獲取,參見\todo{5.4.5}。

而對於由 \capi{clCreateImage} 創建的\cnglo{imgobj}而言,一些特定資訊使用如下函式獲取:
\startCLFUNC
cl_int clGetImageInfo (cl_mem image,
			cl_image_info param_name,
			size_t param_value_size,
			void *param_value,
			size_t *param_value_size_ret)
\stopCLFUNC

\carg{image} 即被查詢的\cnglo{imgobj}。

\carg{param_name} 指明要查詢什麼資訊。他所支持的類型以及在 \carg{param_value} 中返回的資訊如\reftab{clgetimginfo}所示。

\carg{param_value} 用來存儲查詢結果。如果 \carg{param_value} 是 \cenum{NULL},則忽略。

\carg{param_value_size},即 \carg{param_value} 內存塊的大小。
其值必須 >= \reftab{clgetimginfo}中返回型別的大小。

\carg{param_value_size_ret},即查詢結果的實際大小。
如果 \carg{param_value_size_ret} 是 \cenum{NULL},則忽略。

如果執行成功, \capi{clGetImageInfo} 會返回 \cenum{CL_SUCCESS}。
否則,返回下列錯誤碼之一:
\startigBase
\item \cenum{CL_INVALID_VALUE},如果 \cenum{param_name} 的值無效,
或者 \cenum{param_value_size} 的值 < \reftab{clgetimginfo}中返回型別的大小並且 \cenum{param_value} 不是 \cenum{NULL},
或者 \carg{param_name} 指的是某個擴展,但是此\cnglo{device}不支持相應擴展。

\item \cenum{CL_INVALID_MEM_OBJECT},如果 \carg{image} 無效。

\item \cenum{CL_OUT_OF_RESOURCES},如果\scdevfailres。
\item \cenum{CL_OUT_OF_HOST_MEMORY},如果\schostfailres。
\stopigBase

\placetable[here,force,split][tab:clgetimginfo]
{\capi{clGetImageInfo} 所支持的 \carg{param_names}}
{\startETD[cl_image_info][返回型別]

\clETD{CL_IMAGE_FORMAT}{cl_image_format}{
返回用 \capi{clCreateImage} 創建 \carg{image} 時所指定的圖像格式描述符。
}

\clETD{CL_IMAGE_ELEMENT_SIZE}{size_t}{
返回 \carg{image} 中每個元素的大小。
一個元素由 $n$ 個通道組成,其中 $n$ 由 \ctype{cl_image_format} 給出。
}

\clETD{CL_IMAGE_ROW_PITCH}{size_t}{
返回 \carg{image} 中圖像元素的行間距。
}

\clETD{CL_IMAGE_SLICE_PITCH}{size_t}{
對於 3D \cnglo{imgobj},返回 2D 面間距;
對於 1D 或 2D 圖像陣列,返回每個圖像的大小;
對於 1D \cnglo{imgobj}、 1D 圖像緩衝對象以及 2D \cnglo{imgobj},返回 0。
}

\clETD{CL_IMAGE_WIDTH}{size_t}{
返回圖像寬度,單位像素。
}

\clETD{CL_IMAGE_HEIGHT}{size_t}{
返回圖像高度,單位像素。
對於 1D \cnglo{imgobj}、 1D 圖像緩衝對象以及 1D 圖像陣列對象,高度為 0。
}

\clETD{CL_IMAGE_DEPTH}{size_t}{
返回圖像深度,單位像素。
對於 1D \cnglo{imgobj}、 1D 圖像緩衝對象、 2D \cnglo{imgobj},
以及 1D 或 2D 圖像陣列對象,深度為 0。
}

\clETD{CL_IMAGE_ARRAY_SIZE}{size_t}{
返回圖像陣列中圖像的個數。
如果 \carg{image} 不是圖像陣列,則返回 0。
}

\clETD{CL_IMAGE_BUFFER}{cl_mem}{
返回 \carg{image} 所關聯的\cnglo{bufobj}。
}

\clETD{CL_IMAGE_NUM_MIP_LEVELS}{cl_uint}{
返回 \carg{image} 所關聯的 \carg{num_mip_levels}。
}

\clETD{CL_IMAGE_NUM_SAMPLES}{cl_uint}{
返回 \carg{image} 所關聯的 \carg{num_samples}。
}

\stopETD
}

