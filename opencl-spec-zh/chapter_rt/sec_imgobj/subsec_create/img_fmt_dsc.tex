\subsubsection[tit:imgFmtDsc]{圖像格式描述符}

圖像格式描述符的結構定義如下:
\startclc
typedef struct _cl_image_format {
	cl_channel_order	image_channel_order;
	cl_channel_type		image_channel_data_type;
} cl_image_format;
\stopclc

\cvar{image_channel_order} 指定了通道( channel )數目以及通道布局,
即圖像中通道存儲的內存布局。其有效值參見\reftab{imgChannelOrder}。

\cvar{image_channel_data_type} 即通道的數據類型。
其有效值參見\reftab{imgChannelDataType}。

由以上兩者所確定的圖像元素的 bit 數目必須是 2 的指數。

\placetable[here,force][tab:imgChannelOrder]{圖像通道順序}
{\input{chapter_rt/tbl/tbl_img_channel_order.tex}}

\placetable[here,force][tab:imgChannelDataType]{圖像通道數據類型}
{\input{chapter_rt/tbl/tbl_img_channel_data_type.tex}}

例如,如果 \cvar{image_channel_order} = \cenum{CL_RGBA}, \cvar{image_channel_data_type} = \cenum{CL_UNORM_INT8},
此格式的內存布局為:

\bTABLE[frame=off]
\bTR[align=middle]
\bTD[width=.3\textwidth] \eTD
\bTD[frame=on,width=.1\textwidth] R \eTD
\bTD[frame=on,width=.1\textwidth] G \eTD
\bTD[frame=on,width=.1\textwidth] B \eTD
\bTD[frame=on,width=.1\textwidth] A \eTD
\bTD[frame=on,width=.3\textwidth] \eTD
\eTR
\bTR[align={hi}]
\bTD[align=left] 字節偏移$\rightarrow$ \eTD \bTD 0 \eTD \bTD 1 \eTD \bTD 2 \eTD \bTD 3 \eTD \bTD \eTD
\eTR
\eTABLE



