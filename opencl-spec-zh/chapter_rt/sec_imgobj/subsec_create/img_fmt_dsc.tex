\subsubsection[tit:imgFmtDsc]{圖像格式描述符}

圖像格式描述符的結構定義如下:
\startclc
typedef struct _cl_image_format {
	cl_channel_order	image_channel_order;
	cl_channel_type		image_channel_data_type;
} cl_image_format;
\stopclc

\cvar{image_channel_order} 指定了通道( channel )數目以及通道布局,
即圖像中通道存儲的內存布局。其有效值參見\reftab{imgChannelOrder}。

\cvar{image_channel_data_type} 即通道的數據類型。
其有效值參見\reftab{imgChannelDataType}。

由以上兩者所確定的圖像元素的 bit 數目必須是 2 的指數。

\placetable[here,force][tab:imgChannelOrder]{圖像通道順序}
{\bTABLE[option=stretch]

\bTABLEhead
\bTR[background=color,backgroundcolor=gray]
  \bTH channel_order 的值 \eTH
  \bTH 只有通道數據類型為下列值時才能使用此格式 \eTH
\eTR
\eTABLEhead

\bTABLEbody

\bTR
  \bTD \cenum{CL_R}、 \cenum{CL_Rx}、 \cenum{CL_A} \eTD
  \bTD \eTD
\eTR

\bTR
  \bTD \cenum{CL_INTENSITY} \eTD
  \bTD \cenum{CL_UNORM_INT8}、 \cenum{CL_UNORM_INT16}、 \cenum{CL_SNORM_INT8}、 \cenum{CL_SNORM_INT16}、 \cenum{CL_HALF_FLOAT}、 \cenum{CL_FLOAT} \eTD
\eTR

\bTR
  \bTD \cenum{CL_LUMINANCE} \eTD
  \bTD \cenum{CL_UNORM_INT8}、 \cenum{CL_UNORM_INT16}、 \cenum{CL_SNORM_INT8}、 \cenum{CL_SNORM_INT16}、 \cenum{CL_HALF_FLOAT}、 \cenum{CL_FLOAT} \eTD
\eTR

\bTR
  \bTD \cenum{CL_RG}、 \cenum{CL_RGx}、 \cenum{CL_RA} \eTD
  \bTD \eTD
\eTR

\bTR
  \bTD \cenum{CL_RGB} 或者 \cenum{CL_RGBx} \eTD
  \bTD \cenum{CL_UNORM_SHORT_565}、 \cenum{CL_UNORM_SHORT_555}、 \cenum{CL_UNORM_INT_101010} \eTD
\eTR

\bTR
  \bTD \cenum{CL_RGBA} \eTD
  \bTD \cenum{CL_UNORM_INT8}、 \cenum{CL_SNORM_INT8}、 \cenum{CL_SIGNED_INT8}、 \cenum{CL_UNSIGNED_INT8} \eTD
\eTR

\eTABLEbody

\eTABLE

}

\placetable[here,force][tab:imgChannelDataType]{圖像通道數據類型}
{\bTABLE[option=stretch]

\bTABLEhead
\bTR[background=color,backgroundcolor=gray]
  \bTH 圖像通道數據類型 \eTH
  \bTH 描述 \eTH
\eTR
\eTABLEhead

\bTABLEbody

\bTR
  \bTD \cenum{CL_SNORM_INT8} \eTD
  \bTD 各通道均為規範化帶符號 8 位整數 \eTD
\eTR

\bTR
  \bTD \cenum{CL_SNORM_INT16} \eTD
  \bTD 各通道均為規範化帶符號 16 位整數 \eTD
\eTR

\bTR
  \bTD \cenum{CL_UNORM_INT8} \eTD
  \bTD 各通道均為規範化無符號 8 位整數 \eTD
\eTR

\bTR
  \bTD \cenum{CL_UNORM_INT16} \eTD
  \bTD 各通道均為規範化無符號 16 位整數 \eTD
\eTR

\bTR
  \bTD \cenum{CL_UNORM_SHORT_565} \eTD
  \bTD 規範化 5-6-6 3 通道 RGB 圖像。通道順序必須是 \cenum{CL_RGB} 或 \cenum{CL_RGBx}。 \eTD
\eTR

\bTR
  \bTD \cenum{CL_UNORM_INT_101010} \eTD
  \bTD 規範化 x-10-10-10 4 通道 xRGB 圖像。通道順序必須是 \cenum{CL_RGB} 或 \cenum{CL_RGBx}。 \eTD
\eTR

\bTR
  \bTD \cenum{CL_SIGNED_INT8} \eTD
  \bTD 各通道均為非規範化帶符號 8 位整數 \eTD
\eTR

\bTR
  \bTD \cenum{CL_SIGNED_INT16} \eTD
  \bTD 各通道均為非規範化帶符號 16 位整數 \eTD
\eTR

\bTR
  \bTD \cenum{CL_SIGNED_INT32} \eTD
  \bTD 各通道均為非規範化帶符號 32 位整數 \eTD
\eTR

\bTR
  \bTD \cenum{CL_UNSIGNED_INT8} \eTD
  \bTD 各通道均為非規範化無符號 8 位整數 \eTD
\eTR

\bTR
  \bTD \cenum{CL_UNSIGNED_INT16} \eTD
  \bTD 各通道均為非規範化無符號 16 位整數 \eTD
\eTR

\bTR
  \bTD \cenum{CL_UNSIGNED_INT32} \eTD
  \bTD 各通道均為非規範化無符號 32 位整數 \eTD
\eTR

\bTR
  \bTD \cenum{CL_HALF_FLOAT} \eTD
  \bTD 個通道均為 16 位半浮點數 \eTD
\eTR

\bTR
  \bTD \cenum{CL_FLOAT} \eTD
  \bTD 個通道均為單精度浮點數 \eTD
\eTR

\eTABLEbody

\eTABLE

}

例如,如果 \cvar{image_channel_order} = \cenum{CL_RGBA}, \cvar{image_channel_data_type} = \cenum{CL_UNORM_INT8},
此格式的內存布局為:

\bTABLE[frame=off]
\bTR[align=middle]
\bTD[width=.3\textwidth] \eTD
\bTD[frame=on,width=.1\textwidth] R \eTD
\bTD[frame=on,width=.1\textwidth] G \eTD
\bTD[frame=on,width=.1\textwidth] B \eTD
\bTD[frame=on,width=.1\textwidth] A \eTD
\bTD[frame=on,width=.3\textwidth] \eTD
\eTR
\bTR[align={hi}]
\bTD[align=left] 字節偏移$\rightarrow$ \eTD \bTD 0 \eTD \bTD 1 \eTD \bTD 2 \eTD \bTD 3 \eTD \bTD \eTD
\eTR
\eTABLE



