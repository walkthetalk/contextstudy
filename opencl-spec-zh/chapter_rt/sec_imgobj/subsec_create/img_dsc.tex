\subsubsection[sec:imgDsc]{圖像描述符}

圖像描述符描述了圖像或圖像陣列的類型和維數,其定義如下:
\startclc
typedef struct _cl_image_desc {
	cl_mem_object_type	image_type,
	size_t			image_width;
	size_t			image_height;
	size_t			image_depth;
	size_t			image_array_size;
	size_t			image_row_pitch;
	size_t			image_slice_pitch;
	cl_uint			num_mip_levels;
	cl_uint			num_samples;
	cl_mem			buffer;
} cl_image_desc;
\stopclc

\cvar{image_type} 即圖像類型,其值必須是 \cenum{CL_MEM_OBJECT_IMAGE1D}、
\cenum{CL_MEM_OBJECT_IMAGE1D_BUFFER}、 \cenum{CL_MEM_OBJECT_IMAGE1D_ARRAY}、
\cenum{CL_MEM_OBJECT_IMAGE2D}、 \cenum{CL_MEM_OBJECT_IMAGE2D_ARRAY}
或 \cenum{CL_MEM_OBJECT_IMAGE3D}。

\cvar{image_width} 即圖像寬度,單位:像素。
對於 2D 圖像或圖像陣列,其寬度必須 <= \cenum{CL_DEVICE_IMAGE2D_MAX_WIDTH}。
對於 3D 圖像,其寬度必須 <= \cenum{CL_DEVICE_IMAGE3D_MAX_WIDTH}。
對於 1D 圖像緩衝,其寬度必須 <= \cenum{CL_DEVICE_IMAGE_MAX_BUFFER_SIZE}。
對於 1D 圖像和 1D 圖像陣列,其寬度必須 <= \cenum{CL_DEVICE_IMAGE2D_MAX_WIDTH}。

\cvar{image_height} 即圖像高度,單位:像素。
只有圖像為 2D、 3D 或 2D 圖像陣列時才有效。
對於 2D 圖像或圖像陣列,其寬度必須 <= \cenum{CL_DEVICE_IMAGE2D_MAX_HEIGHT}。
對於 3D 圖像,其寬度必須 <= \cenum{CL_DEVICE_IMAGE3D_MAX_HEIGHT}。

\cvar{image_depth} 即圖像深度,單位:像素。
只有圖像為 3D 圖像時才有效,其值必須 >= 1 且 <= \cenum{CL_DEVICE_IMAGE3D_MAX_DEPTH}。

\cvar{image_array_size}\footnote{%
對於\cnglo{kernel}而言,以 \math{\marg{image_array_size} = 1} 讀寫 2D 圖像陣列時,
其性能可能會比直接讀寫 2D 圖像要低。%
} 即圖像陣列中圖像個數。
只有圖像為 1D 或 2D 圖像陣列時才有效。
如果設置了 \cvar{image_array_size},
其值必須 >= 1 且 <= \cenum{CL_DEVICE_IMAGE_MAX_ARRAY_SIZE}。

\cvar{image_row_pitch} 即掃描列間隔,單位字節。
如果 \carg{host_ptr} 是 \cmacro{NULL},其值必須是 0;
如果 \carg{host_ptr} 不是 \cmacro{NULL},
則可以是 0 或者 >= \cvar{image_width} * 元素大小。
如果 \carg{host_ptr} 不是 \cmacro{NULL},並且 \cvar{image_row_pitch} = 0,
則用 \cvar{image_width} * 元素大小 取代 \cvar{image_row_pitch}。
如果 \cvar{imag_row_pitch} 不是 0,則必須是圖像元素大小的整數倍。

\cvar{image_slice_pitch} 即 3D 圖像中每個 2D 平面的大小,
或者 1D、 2D 圖像陣列中每個圖像的大小。單位字節。
如果 \carg{host_ptr} 是 \cmacro{NULL},其值必須是 0;
如果 \carg{host_ptr} 不是 \cmacro{NULL},
對於 2D 圖像陣列或 3D 圖像,可以是 0 或者 >= \cvar{image_row_pitch} * \cvar{image_height};
對於 1D 圖像陣列,可以是 0 或者 >= \cvar{image_row_pitch}。
如果 \carg{host_ptr} 不是 \cmacro{NULL},並且 \cvar{image_slice_pitch} = 0,
對於 2D 圖像陣列或 3D 圖像,
則用 \cvar{image_row_pitch} * \cvar{image_height} 取代 \cvar{image_slice_pitch};
對於 1D 圖像陣列,則用 \cvar{image_row_pitch}。
如果 \cvar{imag_slice_pitch} 不是 0,則必須是 \cvar{image_row_pitch} 的整數倍。

\cvar{num_mip_levels} 和 \cvar{num_samples} 必須是 0。

如果 \cvar{image_type} 是 \cenum{CL_MEM_OBJECT_IMAGE1D_BUFFER},
則 \cvar{buffer} 引用一個\cnglo{bufobj},否則 \cvar{buffer} 必須是 \cmacro{NULL}。
對於 1D 圖像\cnglo{bufobj},圖像的所有像素均取自\cnglo{bufobj}的數據。
在相應的同步點上,
對\cnglo{bufobj}中數據的改動也會反映到 1D 圖像\cnglo{bufobj}的內容上,反之亦然。
\cvar{image_width} * 元素大小 必須 <= \cnglo{bufobj}中的數據大小。

\startnotepar
對一個\cnglo{bufobj}以及與其關聯的 1D 圖像\cnglo{bufobj}
的並發讀、寫以及拷貝是\cnglo{undef}的。
只有讀是定義了的。
\stopnotepar
