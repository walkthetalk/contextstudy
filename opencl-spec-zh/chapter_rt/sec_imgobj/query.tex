\subsection{查詢所支持的圖像格式}

\topclfunc{clGetSupportedImageFormats}

\startCLFUNC
cl_int clGetSupportedImageFormats (cl_context context,
			cl_mem_flags flags,
			cl_mem_object_type image_type,
			cl_uint num_entries,
			cl_image_format *image_formats,
			cl_uint *num_image_formats)
\stopCLFUNC

此函式可用來查詢 OpenCL 實作所支持的圖像格式,對於\cnglo{imgobj},需要指定如下資訊:
\startigBase
\item \cnglo{context}
\item 圖像類型——1D、 2D 或 3D 圖像, 1D 圖像緩衝, 1D 或 2D 圖像陣列
\item \cnglo{imgobj}的分配資訊
\stopigBase

\capi{clGetSupportedImageFormats} 會返回一個聯合體(union),
其中的圖像格式是 \carg{contex} 中所有\cnglo{device}都支持的。

\carg{context} 是 OpenCL \cnglo{context},將在其上創建\cnglo{imgobj}。

\carg{flags} 是位欄,
用來描述將要創建的\cnglo{imgobj}的分配和用法資訊。參見\reftab{clmapflags}。

\carg{image_type} 即圖像類型,其值可以是:
\cenum{CL_MEM_OBJECT_IMAGE1D}、 \cenum{CL_MEM_OBJECT_IMAGE1D_BUFFER}、
\cenum{CL_MEM_OBJECT_IMAGE2D}、 \cenum{CL_MEM_OBJECT_IMAGE3D}、
\cenum{CL_MEM_OBJECT_IMAGE1D_ARRAY} 或 \cenum{CL_MEM_OBJECT_IMAGE2D_ARRAY}。

\carg{num_entries} 即 \carg{image_formats} 所能容納的表項數目。

\carg{image_formats} 用來存儲所返回的圖像格式。
每一項都是 \ctype{cl_image_format} 結構體。
如果 \carg{image_formats} 是 \cmacro{NULL},則忽略。

\carg{num_image_formats} 即所返回的圖像格式的實際數目。如果是 \cmacro{NULL},則忽略。

如果執行成功,則 \capi{clGetSupportedImageFormats} 會返回 \cenum{CL_SUCCESS}。
否則,返回下列錯誤碼之一:
\startigBase
\item \cenum{CL_INVALID_CONTEXT},如果 \carg{context} 無效。

\item \cenum{CL_INVALID_VALUE},如果 \carg{flags} 或 \carg{image_type} 無效,
或者 \carg{num_entries} 是 0 且 \carg{image_formats} 是 \cmacro{NULL}。

\item \cenum{CL_OUT_OF_RESOURCES},如果\scdevfailres。

\item \cenum{CL_OUT_OF_HOST_MEMORY},如果\schostfailres。
\stopigBase

如果 \cenum{CL_DEVICE_IMAGE_SUPPORT}
(參見\reftab{cldevquery})是 \cenum{CL_TRUE},
則 OpenCL 實作賦予下列變量的值必須大於或等於\reftab{cldevquery}中所規定的最小值:
\startigBase
\item \cenum{CL_DEVICE_MAX_READ_IMAGE_ARGS}
\item \cenum{CL_DEVICE_MAX_WRITE_IMAGE_ARGS}
\item \cenum{CL_DEVICE_IMAGE2D_MAX_WIDTH}
\item \cenum{CL_DEVICE_IMAGE2D_MAX_HEIGHT}
\item \cenum{CL_DEVICE_IMAGE3D_MAX_WIDTH}
\item \cenum{CL_DEVICE_IMAGE3D_MAX_HEIGHT}
\item \cenum{CL_DEVICE_IMAGE3D_MAX_DEPTH}
\item \cenum{CL_DEVICE_MAX_SAMPLERS}
\stopigBase

\subsubsection{必須要支持的圖像格式}

對於所有支持圖像的\cnglo{device}而言,
必須要支持的 2D 和 3D \cnglo{imgobj}的格式(無論讀寫)如\reftab{minImgFmt}所示。

\placetable[here][tab:minImgFmt]
{必須要支持的圖像格式}
{\bTABLE[option=stretch]

\bTABLEhead
\bTR[background=color,backgroundcolor=gray]
  \bTH \cvar{image_num_channels} \eTH
  \bTH \cvar{image_channel_order} \eTH
  \bTH \cvar{image_channel_data_type} \eTH
\eTR
\eTABLEhead

\bTABLEbody

\bTR
  \bTD 4 \eTD
  \bTD \cenum{CL_RGBA} \eTD
  \bTD \cenum{CL_UNORM_INT8}\par
\cenum{CL_UNORM_INT16}\par
\cenum{CL_SIGNED_INT8}\par
\cenum{CL_SIGNED_INT16}\par
\cenum{CL_SIGNED_INT32}\par
\cenum{CL_UNSIGNED_INT8}\par
\cenum{CL_UNSIGNED_INT16}\par
\cenum{CL_UNSIGNED_INT32}\par
\cenum{CL_HALF_FLOAT}\par
\cenum{CL_FLOAT}
\eTD
\eTR

\bTR
  \bTD 4 \eTD
  \bTD \cenum{CL_BGRA} \eTD
  \bTD \cenum{CL_UNORM_INT8} \eTD
\eTR

\eTABLEbody

\eTABLE

}
