\subsection{讀、寫以及拷貝圖像對象}

%% clEnqueueReadImage / clEnqueueWriteImage
下列函式會將一個用於讀寫\cnglo{imgobj}的\cnglo{cmd}插入隊列,
此\cnglo{cmd}可以將圖像或圖像陣列的內容讀到\cnglo{host}內存中,
或者由\cnglo{host}內存寫入到圖像或圖像陣列中。
\startCLFUNC
cl_int clEnqueueReadImage (cl_command_queue command_queue,
			cl_mem image,
			cl_bool blocking_read,
			const size_t *origin,
			const size_t *region,
			size_t row_pitch,
			size_t slice_pitch,
			void *ptr,
			cl_uint num_events_in_wait_list,
			const cl_event *event_wait_list,
			cl_event *event)

cl_int clEnqueueWriteImage (cl_command_queue command_queue,
			cl_mem image,
			cl_bool blocking_write,
			const size_t *origin,
			const size_t *region,
			size_t input_row_pitch,
			size_t input_slice_pitch,
			const void * ptr,
			cl_uint num_events_in_wait_list,
			const cl_event *event_wait_list,
			cl_event *event)
\stopCLFUNC

\carg{command_queue} 就是\cnglo{cmd}要進入的隊列。
\carg{command_queue} 和 \carg{image} 必須是由同一個 OpenCL \cnglo{context}創建的。

\carg{image} 是一個圖像或圖像陣列對象。

\carg{blocking_read} 和 \carg{blocking_write} 表明讀寫操作是{\ftRef 阻塞}的還是{\ftRef 非阻塞}的。

如果 \carg{blocking_read} 是 \cenum{CL_TRUE},即讀\cnglo{cmd}是阻塞的,
直到將圖像數據完全拷貝到 \carg{ptr} 所指內存中後, \capi{clEnqueueReadImage} 才會返回。

如果 \carg{blocking_read} 是 \cenum{CL_FALSE},即讀\cnglo{cmd}是非阻塞的, \capi{clEnqueueReadBuffer} 將此\cnglo{cmd}插入隊列後就會返回。
只有等到讀\cnglo{cmd}執行完畢,才能繼續使用 \carg{ptr} 所指向的內容。
參數 \carg{event} 會返回一個\cnglo{evtobj},可用來查詢讀\cnglo{cmd}的執行情況。
讀\cnglo{cmd}完成後,\cnglo{app}就可以繼續使用 \carg{ptr} 所指向的內容了。

如果 \carg{blocking_write} 是 \cenum{CL_TRUE}, OpenCL 實作會拷貝 \carg{ptr} 所指向的數據,並將寫\cnglo{cmd}插入隊列。
當 \capi{clEnqueueWriteBuffer} 返回後,\cnglo{app}就可以重新使用 \carg{ptr} 所指向的內存了。

如果 \carg{blocking_write} 是 \cenum{CL_FALSE}, OpenCL 實作會用 \carg{ptr} 實施非阻塞的寫操作。
既然非阻塞,實作就可以立即返回。返回後,\cnglo{app}還不能立刻就使用 \carg{ptr} 所指內存。
參數 \carg{event} 會返回一個\cnglo{evtobj},可用來查詢寫\cnglo{cmd}的執行情況。
寫\cnglo{cmd}完成後,\cnglo{app}就可以重新使用 \carg{ptr} 所指向的內存了。

\carg{origin} 定義了所要讀寫區域的起始位置 $(x, y, z)$,單位像素;
對於 2D 圖像陣列是 $(x, y)$ 以及圖像索引;
對於 1D 圖像陣列是 $(x)$ 以及圖像索引。
如果 \carg{image} 是 2D \cnglo{imgobj}, $origin[2]$ 必須是 0。
如果 \carg{image} 是 1D 圖像 或 1D 圖像 buffer 對象, $origin[1]$ 和 $origin[2]$ 都必須是 0。
如果 \carg{image} 是 1D 圖像陣列對象, $origin[2]$ 必須是 0, $origin[1]$ 是圖像在陣列中的索引。
如果 \carg{image} 是 2D 圖像陣列對象, $origin[2]$ 是圖像在陣列中的索引。

\carg{region} 定義了要讀寫區域的大小 $(width, height, depth)$,單位像素。
對於 2D 圖像陣列對象, 即 $(width, height)$ 以及圖像個數。
對於 1D 圖像陣列對象, 即 $(width)$ 以及圖像個數。
如果 \carg{image} 是 2D 圖像對象 或 1D 圖像陣列對象, $region[2]$ 必須是 1。
如果 \carg{image} 是 1D 圖像或 1D 圖像 buffer 對象, $region[1]$ 和 $region[2]$ 都必須是 0。

\capi{clEnqueueReadImage} 中的 \carg{row_pitch} 和 \capi{clEnqueueWriteImage} 中的 \carg{input_row_pitch} 即每行的長度,單位字節。
其值必須大於或等於 $元素大小 * width$。
如果 \carg{row_pitch} 或 \carg{input_row_pitch} 是 0,則會根據 $元素大小 * width$ 自動計算行間距。

\capi{clEnqueueReadImage} 中的 \carg{slice_pitch} 和 \capi{clEnqueueWriteImage} 中的 \carg{input_slice_pitch} 
即 3D 圖像的 3D 區域中 2D 平面大小,或者 1D、 2D 圖像陣列中每個圖像的大小,單位字節。
如果 \carg{image} 是 1D 或 2D 圖像,其值必須是 0。否則必須大於或等於 $row\_pitch * height$。
如果 \carg{slice_pitch} 或 \carg{input_slice_pitch} 是 0,則會根據 $row\_pitch * height$ 自動計算平面間距。

\carg{ptr} 指向\cnglo{host}中的一塊內存,用作讀\cnglo{cmd}的數據源以及寫\cnglo{cmd}的目的地。

\carg{event_wait_list} 和 \carg{num_events_in_wait_list} 中列出了執行此\cnglo{cmd}前要等待的事件。
如果 \carg{event_wait_list} 是 \cmacro{NULL},則無須等待任何事件,並且 \carg{num_events_in_wait_list} 必須是0。
如果 \carg{event_wait_list} 不是 \cmacro{NULL},則其中所有事件都必須是有效的,並且 \carg{num_events_in_wait_list} 必須大於 0。
\carg{event_wait_list} 中的事件充當同步點,並且必須與 \carg{command_queue} 位於同一個\cnglo{context}中。
此函式返回後,可以回收並重新使用 \carg{event_wait_list} 所關聯的內存。

\carg{event} 會返回一個\cnglo{evtobj},用來標識此讀、寫\cnglo{cmd},可用來查詢或等待此\cnglo{cmd}完成。
而如果 \carg{event} 是 \cmacro{NULL},就沒辦法查詢此\cnglo{cmd}的狀態或等待其完成了。
如果 \carg{event_wait_list} 和 \carg{event} 都不是 \cmacro{NULL}, \carg{event} 不能屬於 \carg{event_wait_list}。

如果執行成功, \capi{clEnqueueReadImage} 和 \capi{clEnqueueWriteImage} 會返回 \cenum{CL_SUCCESS}。
否則,返回下列錯誤碼之一:
\startigBase
\item \cenum{CL_INVALID_COMMAND_QUEUE},如果 \carg{command_queue} 無效。

\item \cenum{CL_INVALID_CONTEXT},
  如果 \carg{command_queue} 和 \carg{image} 位於不同的\cnglo{context}中,
  或者 \carg{command_queue} 和 \carg{event_wait_list} 中的事件位於不同的\cnglo{context}中。

\item \cenum{CL_INVALID_MEM_OBJECT},如果 \carg{image} 無效。

\item \cenum{CL_INVALID_VALUE},如果 $(origin, region)$ 所指定的區域越限,
  或者 \carg{ptr} 是 \cmacro{NULL}。

\item \cenum{CL_INVALID_VALUE},如果 \carg{origin} 和 \carg{region} 沒有遵守上面所描述的相應規則。

\item \cenum{CL_INVALID_EVENT_WAIT_LIST},
  如果 \carg{event_wait_list} 是 \cmacro{NULL},且 \carg{num_events_in_wait_list} > 0;
  或者 \carg{event_wait_list} 不是 \cmacro{NULL},但 \carg{num_events_in_wait_list} 是 0;
  或者 \carg{event_wait_list} 中的事件無效。

\item \cenum{CL_INVALID_IMAGE_SIZE},如果圖像的大小(寬度、高度、指定的或自動計算的行間距或平面間距)不被 \carg{queue} 所關聯\cnglo{device} 支持。

\item \cenum{CL_IMAGE_FORMAT_NOT_SUPPORTED},如果 \carg{image} 的圖像格式(圖像通道順序和數據類型)不被 \carg{queue} 所關聯\cnglo{device} 支持。

\item \cenum{CL_MEM_OBJECT_ALLOCATION_FAILURE},如果為 \carg{image} 分配內存失敗。

\item \cenum{CL_INVALID_OPERATION},如果 \carg{command_queue} 所關聯\cnglo{device}不支持圖像
(即,\reftab{cldevquery}中的 \cenum{CL_DEVICE_IMAGE_SUPPORT} 是 \cenum{CL_FALSE})。

\item \cenum{CL_INVALID_OPERATION},如果調用的是 \capi{clEnqueueReadImage},
  但創建 \carg{image} 時指定了 \cenum{CL_MEM_HOST_WRITE_ONLY} 或 \cenum{CL_MEM_HOST_NO_ACCESS}。

\item \cenum{CL_INVALID_OPERATION},如果調用的是 \capi{clEnqueueWriteImage},
  但創建 \carg{image} 時指定了 \cenum{CL_MEM_HOST_READ_ONLY} 或 \cenum{CL_MEM_HOST_NO_ACCESS}。

\item \cenum{CL_EXEC_STATUS_ERROR_FOR_EVENTS_IN_WAIT_LIST},
  如果讀寫操作是阻塞的,並且 \carg{event_wait_list} 中任一事件的執行結果是負整數。

\item \cenum{CL_OUT_OF_RESOURCES},如果\scdevfailres。
\item \cenum{CL_OUT_OF_HOST_MEMORY},如果\schostfailres。
\stopigBase

注意:

如果調用 \capi{clEnqueueReadImage} 時,參數 \carg{ptr} 的值為
 $host\_ptr + (origin[2] * image\_slice\_pitch + origin[1] * image\_row\_pitch + origin[0] * bytes per pixel$,
其中 \carg{host_ptr} 是用 \cenum{CL_MEM_USE_HOST_PTR} 創建 \carg{image} 時指定的,
為避免未定義行為必須滿足下列需求:
\startigBase
\item 在讀\cnglo{cmd}開始執行前,所有使用此\cnglo{imgobj}的\cnglo{cmd}都已執行完畢。

\item \capi{clEnqueueReadImage} 的 \carg{row_pitch} 和 \carg{slice_pitch} 必須設置成圖像的行間距和平面間距。

\item 此\cnglo{imgobj}沒有被映射。

\item 在讀\cnglo{cmd}執行完以前,任何\cnglo{cmdq}都不能使用此\cnglo{imgobj}。
\stopigBase

如果調用 \capi{clEnqueueWriteImage} 時,參數 \carg{ptr} 的值為
 $host\_ptr + (origin[2] * image\_slice\_pitch + origin[1] * image\_row\_pitch + origin[0] * bytes per pixel$,
其中 \carg{host_ptr} 是用 \cenum{CL_MEM_USE_HOST_PTR} 創建 \carg{image} 時指定的,
為避免未定義行為必須滿足下列需求:
\startigBase
\item 在寫\cnglo{cmd}開始執行前,\cnglo{host}內存中的內容已更新的最新。

\item \capi{clEnqueueWriteImage} 的 \carg{input_row_pitch} 和 \carg{input_slice_pitch} 必須設置成圖像的行間距和平面間距。

\item 此\cnglo{imgobj}沒有被映射。

\item 在寫\cnglo{cmd}執行完以前,任何\cnglo{cmdq}都不能使用此\cnglo{imgobj}。
\stopigBase

%% clEnqueueCopyImage
函式
\startCLFUNC
cl_int clEnqueueCopyImage (cl_command_queue command_queue,
			cl_mem src_image,
			cl_mem dst_image,
			const size_t *src_origin,
			const size_t *dst_origin,
			const size_t *region,
			cl_uint num_events_in_wait_list,
			const cl_event *event_wait_list,
			cl_event *event)
\stopCLFUNC
會將一個拷貝\cnglo{cmd}插入隊列。
 \carg{src_image} 和 \carg{dst_image} 可以是 1D、 2D、 3D 圖像或者 1D、 2D 圖像陣列對象,
允許我們實施下列拷貝動作:
\startigBase
\item 由 1D \cnglo{imgobj}拷貝到 1D \cnglo{imgobj}。
\item 由 1D \cnglo{imgobj}拷貝到 2D \cnglo{imgobj} 的掃描列,反之亦可。
\item 由 1D \cnglo{imgobj}拷貝到 3D \cnglo{imgobj} 的 2D 平面的掃描列,反之亦可。
\item 由 1D \cnglo{imgobj}拷貝到 1D 或 2D 圖像陣列對象中某個圖像的掃描列,反之亦可。
\item 由 2D \cnglo{imgobj}拷貝到 2D \cnglo{imgobj}。
\item 由 2D \cnglo{imgobj}拷貝到 3D \cnglo{imgobj} 的 2D 平面,反之亦可。
\item 由 2D \cnglo{imgobj}拷貝到 2D 圖像陣列對象中的某個圖像,反之亦可。
\item 由 1D 圖像陣列對象拷貝到 1D 圖像陣列對象,反之亦可。
\item 由 2D 圖像陣列對象拷貝到 2D 圖像陣列對象,反之亦可。
\item 由 3D \cnglo{imgobj}拷貝到 3D \cnglo{imgobj}。
\stopigBase

\carg{command_queue} 就是拷貝\cnglo{cmd}要進入的隊列。
\carg{command_queue}、 \carg{src_image}、 \carg{dst_image} 必須位於同一 OpenCL \cnglo{context}中。

\carg{src_origin} 定義了源區域的偏移量。
對於 1D、 2D 或 3D 圖像,即 $(x, y, z)$,單位像素;
對於 2D 圖像陣列,即 $(x, y)$ 以及圖像索引;
對於 1D 圖像陣列,即 $(x)$ 以及圖像索引。
如果 \carg{src_image} 是 2D \cnglo{imgobj}, $src\_origin[2]$ 必須是 0。
如果 \carg{src_image} 是 1D \cnglo{imgobj},$src\_origin[1]$ 和 $src\_origin[2]$ 都必須是 0。
如果 \carg{src_image} 是 1D 圖像陣列對象, $src\_origin[1]$ 即圖像索引。
如果 \carg{src_image} 是 2D 圖像陣列對象, $src\_origin[2]$ 即圖像索引。

\carg{dst_origin} 除了對應的是 \carg{dst_image},其他都跟 \carg{src_origin} 一樣。

\carg{region} 定義了要拷貝的 1D、 2D 或 3D 區域 $(width, height, depth)$,單位像素;
對於 2D 圖像陣列,即 2D 區域 $(width, height)$ 以及圖像索引;
對於 1D 圖像陣列,即 1D 區域 $(width)$ 以及圖像索引。
如果 \carg{src_image} 或 \carg{dst_image} 是 2D \cnglo{imgobj} 或 1D 圖像陣列對象, $region[2]$ 必須是 1。
如果 \carg{src_image} 或 \carg{dst_image} 是 1D 圖像或 1D 圖像 buffer 對象, $src\_origin[1]$ 和 $region[2]$ 必須是 1。

\carg{event_wait_list} 和 \carg{num_events_in_wait_list} 中列出了執行此\cnglo{cmd}前要等待的事件。
如果 \carg{event_wait_list} 是 \cmacro{NULL},則無須等待任何事件,並且 \carg{num_events_in_wait_list} 必須是0。
如果 \carg{event_wait_list} 不是 \cmacro{NULL},則其中所有事件都必須是有效的,並且 \carg{num_events_in_wait_list} 必須大於 0。
\carg{event_wait_list} 中的事件充當同步點,並且必須與 \carg{command_queue} 位於同一個\cnglo{context}中。
此函式返回後,可以回收並重新使用 \carg{event_wait_list} 所關聯的內存。

\carg{event} 會返回一個\cnglo{evtobj},用來標識此讀、寫\cnglo{cmd},可用來查詢或等待此\cnglo{cmd}完成。
而如果 \carg{event} 是 \cmacro{NULL},就沒辦法查詢此\cnglo{cmd}的狀態或等待其完成了。
這時可以用 \capi{clEnqueueBarrierWithWaitList} 來代替。
如果 \carg{event_wait_list} 和 \carg{event} 都不是 \cmacro{NULL}, \carg{event} 不能屬於 \carg{event_wait_list}。

目前而言,要求 \carg{src_image} 和 \carg{dst_image} 具有完全相同的圖像格式
(即創建 \carg{src_image} 和 \carg{dst_image} 時所指定的 \ctype{cl_image_format} 必須相同)。

如果執行成功, \capi{clEnqueueCopyImage} 會返回 \cenum{CL_SUCCESS}。
否則,返回下列錯誤碼之一:
\startigBase
\item \cenum{CL_INVALID_COMMAND_QUEUE},如果 \carg{command_queue} 無效。

\item \cenum{CL_INVALID_CONTEXT},
  如果 \carg{command_queue}、 \carg{src_image} 和 \carg{dst_image} 位於不同的\cnglo{context}中,
  或者 \carg{command_queue} 和 \carg{event_wait_list} 中的事件位於不同的\cnglo{context}中。

\item \cenum{CL_INVALID_MEM_OBJECT},如果 \carg{src_image} 或 \carg{dst_image} 無效。

\item \cenum{CL_IMAGE_FORMAT_MISMATCH},如果 \carg{src_image} 和 \carg{dst_image} 所使用的圖像格式不同。

\item \cenum{CL_INVALID_VALUE},
  如果 $(src\_origin, region)$ 所指定的區域在 \carg{src_image} 中越限,
  或者 $(dst\_origin, region)$ 所指定的區域在 \carg{dst_image} 中越限。

\item \cenum{CL_INVALID_VALUE},如果 \carg{src_origin}、 \carg{dst_origin} 和 \carg{region} 沒有遵守上面所描述的相應規則。

\item \cenum{CL_INVALID_VALUE},如果 \carg{src_buffer} 和 \carg{dst_buffer} 是同一個\cnglo{bufobj},
  且 \carg{src_slice_pitch} 不等於 \carg{dst_slice_pitch}
  或者 \carg{src_row_pitch} 不等於 \carg{dst_row_pitch}。

\item \cenum{CL_INVALID_EVENT_WAIT_LIST},如果 \carg{event_wait_list} 是 \cmacro{NULL},
  且 \carg{num_events_in_wait_list} > 0;
  或者 \carg{event_wait_list} 不是 \cmacro{NULL},但 \carg{num_events_in_wait_list} 是 0;
  或者 \carg{event_wait_list} 中的事件無效。

\item \cenum{CL_INVALID_IMAGE_SIZE},如果圖像的大小(寬度、高度、指定的或自動計算的行間距或平面間距)不被 \carg{queue} 所關聯\cnglo{device} 支持。

\item \cenum{CL_IMAGE_FORMAT_NOT_SUPPORTED},如果 \carg{src_image} 或 \carg{dst_image} 的圖像格式(圖像通道順序和數據類型)不被 \carg{queue} 所關聯\cnglo{device} 支持。

\item \cenum{CL_MEM_OBJECT_ALLOCATION_FAILURE},如果為 \carg{src_image} 或 \carg{dst_image} 分配內存失敗。

\item \cenum{CL_OUT_OF_RESOURCES}——如果\scdevfailres。
\item \cenum{CL_OUT_OF_HOST_MEMORY}——如果\schostfailres。

\item \cenum{CL_INVALID_OPERATION},如果調用的是 \capi{clEnqueueReadBufferRect},
  但創建 \carg{buffer} 時指定了 \cenum{CL_MEM_HOST_WRITE_ONLY} 或 \cenum{CL_MEM_HOST_NO_ACCESS}。

\item \cenum{CL_INVALID_OPERATION},如果 \carg{command_queue} 所關聯\cnglo{device}不支持圖像
(即,\reftab{cldevquery}中的 \cenum{CL_DEVICE_IMAGE_SUPPORT} 是 \cenum{CL_FALSE})。)。

\item \cenum{CL_MEM_COPY_OVERLAP},
  如果 \carg{src_image} 和 \carg{dst_image} 是同一個\cnglo{imgobj},並且源區域和宿區域重疊。
\stopigBase
