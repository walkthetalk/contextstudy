\subsection{映射圖像對象}

函式
\startclc
void * clEnqueueMapImage (cl_command_queue command_queue,
			cl_mem image,
			cl_bool blocking_map,
			cl_map_flags map_flags,
			const size_t *origin,
			const size_t *region,
			size_t *image_row_pitch,
			size_t *image_slice_pitch,
			cl_uint num_events_in_wait_list,
			const cl_event *event_wait_list,
			cl_event *event,
			cl_int *errcode_ret)
\stopclc
所要插入隊列的\cnglo{cmd}會將 \carg{image} 的某個區域映射到\cnglo{host}的地址空間中,並返回新地址。

\carg{command_queue} 必須是一個有效的\cnglo{cmd}。

\carg{image} 是一個\cnglo{imgobj}。
\carg{command_queue} 與 \carg{image} 必須位於同一 OpenCL \cnglo{context}中。

\carg{blocking_map} 表明此映射操作是{\ftRef 阻塞的}還是{\ftRef 非阻塞的}。

如果 \carg{blocking_map} 是 \cenum{CL_TRUE},即映射\cnglo{cmd}是阻塞的,
直到映射完成, \capi{clEnqueueMapImage} 才會返回,\cnglo{app}可以用返回的指針訪問所映射區域的內容。

如果 \carg{blocking_map} 是 \cenum{CL_FALSE},即映射\cnglo{cmd}是非阻塞的,
直到映射\cnglo{cmd}完成後,才能使用 \capi{clEnqueueMapImage} 所返回的指針。
參數 \carg{event} 會返回一個\cnglo{evtobj},可用來查詢映射\cnglo{cmd}的執行情況。
映射\cnglo{cmd}完成後,\cnglo{app}就可以使用 \capi{clEnqueueMapImage} 所返回的指針訪問映射區域的內容了。

\carg{map_flags} 是位域,參見\reftab{clmapflags}。

\carg{origin} 定義了所要映射區域的起始位置 $(x, y, z)$,單位像素;
對於 2D 圖像陣列是 $(x, y)$ 以及圖像索引;
對於 1D 圖像陣列是 $(x)$ 以及圖像索引。
如果 \carg{image} 是 2D \cnglo{imgobj}, $origin[2]$ 必須是 0。
如果 \carg{image} 是 1D 圖像 或 1D 圖像 buffer 對象, $origin[1]$ 和 $origin[2]$ 都必須是 0。
如果 \carg{image} 是 1D 圖像陣列對象, $origin[2]$ 必須是 0, $origin[1]$ 是圖像在陣列中的索引。
如果 \carg{image} 是 2D 圖像陣列對象, $origin[2]$ 是圖像在陣列中的索引。

\carg{region} 定義了映射區域的大小 $(width, height, depth)$,單位像素。
對於 2D 圖像陣列對象, 即 $(width, height)$ 以及圖像個數。
對於 1D 圖像陣列對象, 即 $(width)$ 以及圖像個數。
如果 \carg{image} 是 2D 圖像對象 或 1D 圖像陣列對象, $region[2]$ 必須是 1。
如果 \carg{image} 是 1D 圖像或 1D 圖像 buffer 對象, $region[1]$ 和 $region[2]$ 都必須是 0。

\carg{image_row_pitch} 即所映射區域的掃描線間距,單位字節,不能是 \cenum{NULL}。

\carg{image_slice_pitch} 對於 3D 圖像而言是 2D 平面的大小,
對於 1D 或 2D 圖像陣列而言分別是每個 1D 或 2D 圖像的大小。
對於 1D 或 2D 圖像,如果此參數不是 \cenum{NULL} 則返回零。
對於 3D 圖像、 1D 或 2D 圖像陣列, \carg{image_slice_pitch} 不能是 \cenum{NULL}。

\carg{event_wait_list} 和 \carg{num_events_in_wait_list} 中列出了執行此\cnglo{cmd}前要等待的事件。
如果 \carg{event_wait_list} 是 \cenum{NULL},則無須等待任何事件,並且 \carg{num_events_in_wait_list} 必須是0。
如果 \carg{event_wait_list} 不是 \cenum{NULL},則其中所有事件都必須是有效的,並且 \carg{num_events_in_wait_list} 必須大於 0。
\carg{event_wait_list} 中的事件充當同步點,並且必須與 \carg{command_queue} 位於同一個\cnglo{context}中。
此函式返回後,可以回收並重新使用 \carg{event_wait_list} 所關聯的內存。

\carg{event} 會返回一個\cnglo{evtobj},用來標識此\cnglo{cmd},可用來查詢或等待此\cnglo{cmd}完成。
而如果 \carg{event} 是 \cenum{NULL},就沒辦法查詢此\cnglo{cmd}的狀態或等待其完成了。
如果 \carg{event_wait_list} 和 \carg{event} 都不是 \cenum{NULL}, \carg{event} 不能屬於 \carg{event_wait_list}。

\carg{errcode_ret} 用來返回錯誤碼。如果是 \cenum{NULL},則不會返回錯誤碼。

如果執行成功, \capi{clEnqueueMapImage} 會返回所映射區域的指針,並將 \carg{errcode_ret} 置為 \cenum{CL_SUCCESS}。
否則,返回 \cenum{NULL},並將 \carg{errcode_ret} 置為下列錯誤碼之一:
\startigBase
\item \cenum{CL_INVALID_COMMAND_QUEUE},如果 \carg{command_queue} 無效。

\item \cenum{CL_INVALID_CONTEXT},
  如果 \carg{command_queue} 和 \carg{image} 位於不同的\cnglo{context}中,
  或者 \carg{command_queue} 和 \carg{event_wait_list} 中的事件位於不同的\cnglo{context}中。

\item \cenum{CL_INVALID_MEM_OBJECT},如果 \carg{image} 無效。

\item \cenum{CL_INVALID_VALUE},如果 $(origin, region)$ 所指定的區域越限,
  或者 \carg{map_flags} 的值無效。

\item \cenum{CL_INVALID_VALUE},如果 \carg{image_row_pitch} 是 \cenum{NULL}。

\item \cenum{CL_INVALID_VALUE},如果 \carg{image} 是 3D 圖像、 1D 或 2D 圖像陣列對象,
  而 \carg{image_slice_pitch} 是 \cenum{NULL}。

\item \cenum{CL_INVALID_EVENT_WAIT_LIST},
  如果 \carg{event_wait_list} 是 \cenum{NULL},且 \carg{num_events_in_wait_list} > 0;
  或者 \carg{event_wait_list} 不是 \cenum{NULL},但 \carg{num_events_in_wait_list} 是 0;
  或者 \carg{event_wait_list} 中的事件無效。

\item \cenum{CL_INVALID_IMAGE_SIZE},如果 \carg{image} 的大小
(圖像寬度、高度、指定的或自動計算的行間距或平面間距)不被 \carg{queue} 所在\cnglo{device}支持。

\item \cenum{CL_IMAGE_FORMAT_NOT_SUPPORTED},
如果 \carg{image} 的圖像格式(圖像通道順序和數據類型)不被 \carg{queue} 所在\cnglo{device} 支持。

\item \cenum{CL_MAP_FAILURE},如果映射失敗。
  對於用 \cenum{CL_MEM_USE_HOST_PTR} 或 \cenum{CL_MEM_ALLOC_HOST_PTR} 創建的\cnglo{imgobj}不能出現此錯誤。

\item \cenum{CL_EXEC_STATUS_ERROR_FOR_EVENTS_IN_WAIT_LIST},
  如果映射操作是阻塞的,且 \carg{event_wait_list} 中任一事件的執行狀態是負整數。

\item \cenum{CL_MEM_OBJECT_ALLOCATION_FAILURE},如果為 \carg{buffer} 分配內存失敗。

\item \cenum{CL_INVALID_OPERATION},如果 \carg{command_queue} 所在\cnglo{device}不支持圖像
(即 \cenum{CL_DEVICE_IMAGE_SUPPORT} 是 \cenum{CL_FALSE},參見\reftab{cldevquery})。

\item \cenum{CL_INVALID_OPERATION},
如果創建 \carg{image} 時指定了 \cenum{CL_MEM_HOST_WRITE_ONLY} 或 \cenum{CL_MEM_HOST_NO_ACCESS},
而在 \carg{map_flags} 中設置了 \cenum{CL_MAP_READ};
或者創建 \carg{image} 時指定了 \cenum{CL_MEM_HOST_READ_ONLY} 或 \cenum{CL_MEM_HOST_NO_ACCESS},
而在 \carg{map_flags} 中設置了 \cenum{CL_MAP_WRITE} 或 \cenum{CL_MAP_WRITE_INVALIDATE_REGION}。

\item \cenum{CL_OUT_OF_RESOURCES}——如果\scdevfailres。
\item \cenum{CL_OUT_OF_HOST_MEMORY}——如果\schostfailres。
\stopigBase

所映射的區域起自 \carg{origin},
對於 1D 圖像、1D 圖像 buffer 或 1D 圖像陣列,其大小至少為 $region[0]$;
對於 2D 圖像或 2D 圖像陣列,其大小至少為 $(image\_row\_pitch * region[1])$;
對於 3D 圖像,其大小至少為 $(image\_slice\_pitch * region[1])$。
對於訪問此區域外的內容,其結果未定義。

注意:

如果創建\cnglo{bufobj}時在 \carg{mem_flags} 中指定了 \cenum{CL_MEM_USE_HOST_PTR},
則:
\startigBase
\item 當 \capi{clEnqueueMapImage} 的\cnglo{cmd}完成後,
  保證 \capi{clCreateImage} 的 \carg{host_ptr} 中所映射區域的內容是最新的。

\item \capi{clEnqueueMapImage} 所返回的指針得自創建\cnglo{imgobj}時所用的 \carg{host_ptr}。
\stopigBase

被映射過的\cnglo{imgobj}使用 \capi{clEnqueueUnmapMemObject} 進行解映射。參見\todo{section 5.4.2}。

