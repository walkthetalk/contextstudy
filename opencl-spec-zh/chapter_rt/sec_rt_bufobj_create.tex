\startbuffer[sectitlecreatbufobj]
创建\cnglo{bufobj}
\stopbuffer
\subsection{\getbuffer[sectitlecreatbufobj]}

%%%%%%%%%%%%%%%%%%%%%%%%%%%%%%%%%%%%%%% clCreateBuffer
可以使用函数\capi{clCreateBuffer}创建\cnglo{bufobj}。
\startclc
cl_mem clCreateBuffer(
		cl_context context,
		cl_mem_flags flags,
		size_t size,
		void *host_ptr,
		cl_int *errcode_ret)
\stopclc

\carg{context}是一个有效的\scopencl\cnglo{context},用来创建\cnglo{bufobj}。

\carg{flags}是位域,用来指定分配和使用信息,如在哪个内存区域中分配\cnglo{bufobj}以及怎样使用它。\reftab{clmemflags}描述了\carg{flags}可能的值:

\cltable
{\placetable[here,force][tab:clmemflags]{\carg{cl_mem_flags}的值的清单}}
{\startED[\ctype{cl_mem_flags}]

\clED{CL_MEM_READ_WRITE}{
  表明此\cnglo{memobj}可讀可寫。這是缺省值。
}

\clED{CL_MEM_WRITE_ONLY}{
  只能寫不能讀,對這種\cnglo{memobj}的讀操作是未定義的。
  \cenum{CL_MEM_READ_WRITE} 和 \cenum{CL_MEM_WRITE_ONLY} 是互斥的。
}

\clED{CL_MEM_READ_ONLY}{
  只能讀不能寫,對這種\cnglo{memobj}的寫操作是未定義的。
  \cenum{CL_MEM_READ_WRITE} 或者 \cenum{CL_MEM_WRITE_ONLY} 都與 \cenum{CL_MEM_READ_ONLY} 互斥。
}

\clED{CL_MEM_USE_HOST_PTR}{
  僅當 \carg{host_ptr} 不是 \cenum{NULL} 時才有效。
  它表明\cnglo{app}想讓 OpenCL 實作使用 \carg{host_ptr} 所引用的內存來存儲\cnglo{memobj}的內容。

  OpenCL 實作可以在\cnglo{device}內存中保存一份 \carg{host_ptr} 所引用的內容用作 cache。
  \cnglo{kernel}在\cnglo{device}上執行時可以使用這份用作 cache 的拷貝。

  如果多個\cnglo{bufobj}由同一 \carg{host_ptr} 創建,或者有重疊區域,當 OpenCL \cnglo{cmd}操作這些\cnglo{bufobj}時,其結果未定義。

  用 \cenum{CL_MEM_USE_HOST_PTR} 創建\cnglo{memobj}時,\carg{host_ptr} 的對齊規則請參考\todo{section C.3}。
}

\clED{CL_MEM_ALLOC_HOST_PTR}{
    表明\cnglo{app}想讓 OpenCL 實作在\cnglo{host}可以存取的內存中分配內存。

    它與 \cenum{CL_MEM_USE_HOST_PTR} 互斥。
}

\clED{CL_MEM_COPY_HOST_PTR}{
  僅當 \carg{host_ptr} 不是 \cenum{NULL} 時才有效。
  它表明\cnglo{app}想讓 OpenCL 實作使用 \carg{host_ptr} 所引用的內存來為\cnglo{memobj}分配內存並拷貝數據。

  它與 \cenum{CL_MEM_USE_HOST_PTR} 互斥。

  它與 \cenum{CL_MEM_ALLOC_HOST_PTR} 一起使用時,可以對由\cnglo{host}可存取的內存(如 PCIe )分配的 \ctype{cl_mem} 對象進行初始化。
}

\clED{CL_MEM_HOST_WRITE_ONLY}{
  表明\cnglo{host}只會對此\cnglo{memobj}進行寫入。
  可用來對\cnglo{host}的寫操作進行優化(如對於與\cnglo{host}通過系統總線如 PCIe 進行通信的\cnglo{device},分配\cnglo{memobj}時使能 Write-combining)。
}

\clED{CL_MEM_HOST_READ_ONLY}{
  表明\cnglo{host}只會對此\cnglo{memobj}進行讀取。

  它與 \cenum{CL_MEM_HOST_WRITE_ONLY} 互斥。
}

\clED{CL_MEM_HOST_NO_ACCESS}{
  表明\cnglo{host}不會對此\cnglo{memobj}進行讀寫。

  \cenum{CL_MEM_HOST_WRITE_ONLY} 或者 \cenum{CL_MEM_HOST_READ_ONLY} 都與 \cenum{CL_MEM_HOST_NO_ACCESS} 互斥。
}

\stopED

}

\carg{size}是所分配的\cnglo{bufobj}的大小。

\carg{host_ptr}指向由\cnglo{app}所分配的缓冲数据。其大小必须大于等于\carg{size}。

\carg{errcode_ret}用来返回错误码。如果是\cenum{NULL},不会返回错误码。

如果执行成功,\capi{clCreateBuffer}会返回一个非零的\cnglo{bufobj},并将
\carg{errcode_ret}置为\cenum{CL_SUCCESS}。否则,返回\cenum{NULL},并将
\carg{errcode_ret}置为下列错误值之一:
\startigBase
\itemenumdesc{CL_INVALID_CONTEXT}{如果\carg{context}无效。}
\itemenumdesc{CL_INVALID_VALUE}{如果\carg{flags}的值无效。}

\startbuffer[footnoteshi]
如果\carg{size}比\carg{context}中所有\cnglo{device}的
\cenum{CL_DEVICE_MAX_MEM_ALLOC_SIZE}(参见\reftab{cldevquery})都大,实现可能返回
\cenum{CL_INVALID_BUFFER_SIZE}。
\stopbuffer
\itemenumdesc{CL_INVALID_BUFFER_SIZE}{
如果\carg{size}是0\footnote{\getbuffer[footnoteshi]}。
}

\itemenumdesc{CL_INVLAID_HOST_PTR}{
如果\carg{host_ptr}是\cenum{NULL},并且\carg{flags}中设置了\cenum{CL_MEM_USE_HOST_PTR}
或\cenum{CL_MEM_COPY_HOST_PTR};或者\carg{host_ptr}不是\cenum{NULL},但是\carg{flags}
中没有设置\cenum{CL_MEM_USE_HOST_PTR}或\cenum{CL_MEM_COPY_HOST_PTR}。
}

\itemenumdesc{CL_MEM_OBJECT_ALLOCATION_FAILURE}{
如果为\cnglo{bufobj}分配内存失败。
}

\itemenumdesc{CL_MEM_OBJECT_ALLOCATION_FAILURE}{
如果为\cnglo{bufobj}分配内存失败。
}

\itemenumdesc{CL_OUT_OF_RESOURCES}{如果\scdevfailres。}
\itemenumdesc{CL_OUT_OF_HOST_MEMORY}{如果\schostfailres。}

\stopigBase

%%%%%%%%%%%%%%%%%%%%%%%%%%%%%%%%%%%% clCreateSubBuffer
可以使用函数\capi{clCreateSubBuffer}由一个现有的\cnglo{bufobj}创建一个新的\cnglo{bufobj}
(叫做子\cnglo{bufobj})。
\startclc
cl_mem clCreateSubBuffer(cl_mem buffer,
			cl_mem_flags flags,
			cl_buffer_create_type buffer_create_type,
			const void *buffer_create_info,
			cl_int *errcode_ret)
\stopclc

\carg{buffer}必须是一个有效的\cnglo{bufobj},并且不能是子\cnglo{bufobj}。

\carg{flags}是位域,用来指定创建\cnglo{memobj}时的一些分配和使用信息,参见\reftab{clmemflags}。

\carg{buffer_create_type}和\carg{buffer_create_info}描述了所创建\cnglo{bufobj}的类型。
\carg{buffer_create_type}的支持清单以及对应的\carg{buffer_create_info}如
\reftab{clcreatesubbuffer}所示。

\cltable
{\placetable[here,force][tab:clcreatesubbuffer]
{\capi{clCreateSubBuffer}中所支持的名字及值的清单}}
{\startED[\ctype{cl_buffer_create_type}]

\clED{CL_BUFFER_CREATE_TYPE_REGION}{
  用 \carg{buffer} 中的特定區域創建\cnglo{bufobj}。

  \carg{buffer_create_info} 指向如下數據結構\todo{clcintable}:

\type{struct _cl_buffer_region \{}\crlf
\type{        size_t origin;}\crlf
\type{        size_t size;}\crlf
\type{\} cl_buffer_region;}

  $(origin, size)$ 就是在 \carg{buffer} 中的偏移量和字節數。

  如果 \carg{buffer} 是用 \cenum{CL_MEM_USE_HOST_PTR} 創建的,所返回\cnglo{bufobj}的 \carg{host_ptr} 就是 $host\_ptr+origin$。

  所返回的\cnglo{bufobj}引用了為 \carg{buffer} 分配的數據存儲空間,並指向其中的特定區域 $(origin,size)$。

  如果在 \carg{buffer} 中,區域 $(origin,size)$ 越界了,則會在 \carg{errcode_ret} 中返回 \cenum{CL_INVALID_VALUE}。

  如果 \carg{size} 是 0,則返回 \cenum{CL_INVALID_BUFFER_SIZE}。

  如果與 \carg{buffer} 相關聯的\cnglo{context}中沒有一個設備的 \cenum{CL_DEVICE_MEM_BASE_ADDR_ALIGN} 與 $origin$ 對齊,
  則會在 \carg{errcode_ret} 中返回 \cenum{CL_MISALIGNED_SUB_BUFFER_OFFSET}。
}

\stopED

}

如果执行成功,\capi{clCreateSubBuffer}会返回\cenum{CL_SUCCESS}。否则,会将\carg{errcode_ret}
置为下列错误码之一:

\startigBase
\itemenumdesc{CL_INVALID_MEM_OBJECT}{
如果\carg{buffer}不是一个有效的\cnglo{bufobj}或者是一个子\cnglo{bufobj}。
}

\itemenumdesc{CL_INVALID_VALUE}{
如果\carg{buffer}是用\cenum{CL_MEM_WRITE_ONLY}创建的,可是\carg{flags}中却指定了
\cenum{CL_MEM_READ_WRITE}或\cenum{CL_MEM_READ_ONLY};或者\carg{buffer}是用
\cenum{CL_MEM_READ_ONLY}创建的,可是\carg{flags}中却指定了
\cenum{CL_MEM_READ_WRITE}或\cenum{CL_MEM_WRITE_ONLY};或者\carg{flags}中指定了
\cenum{CL_MEM_USE_HOST_PRT}或\cenum{CL_MEM_ALLOC_HOST_PTR}或
\cenum{CL_MEM_COPY_HOST_PTR}。
}

\itemenumdesc{CL_INVALID_VALUE}{
如果\carg{buffer_create_type}中的值无效。
}

\itemenumdesc{CL_INVALID_VALUE}{
如果\carg{buffer_create_info}(对应于\carg{buffer_create_type})中的值无效或者是
\cenum{NULL}。
}

\itemenumdesc{CL_INVALID_BUFFER_SIZE}{如果\carg{size}是0。}

\itemenumdesc{CL_MEM_OBJECT_ALLOCATION_FAILURE}{
如果为子\cnglo{bufobj}分配内存失败。
}

\itemenumdesc{CL_OUT_OF_RESOURCES}{如果\scdevfailres。}
\itemenumdesc{CL_OUT_OF_HOST_MEMORY}{如果\schostfailres。}

\stopigBase

NOTE:对一个\cnglo{bufobj}及其子\cnglo{bufobj}的并行读写都是未定义的。对由同一\cnglo{bufobj}
创建的互相重叠的子\cnglo{bufobj}的并行读写是未定义的。当然对这些对象的读操作都是定义了的。



