\startbuffer[sectitlecreatbufobj]
创建\cnglo{bufobj}
\stopbuffer
\subsection{\getbuffer[sectitlecreatbufobj]}

%%%%%%%%%%%%%%%%%%%%%%%%%%%%%%%%%%%%%%% clCreateBuffer
可以使用函数\capi{clCreateBuffer}创建\cnglo{bufobj}。
\startclc
cl_mem clCreateBuffer(
		cl_context context,
		cl_mem_flags flags,
		size_t size,
		void *host_ptr,
		cl_int *errcode_ret)
\stopclc

\carg{context}是一个有效的\scopencl\cnglo{context},用来创建\cnglo{bufobj}。

\carg{flags}是位域,用来指定分配和使用信息,如在哪个内存区域中分配\cnglo{bufobj}以及怎样使用它。\reftab{clmemflags}描述了\carg{flags}可能的值:

\cltable
{\placetable[here,force][tab:clmemflags]{\carg{cl_mem_flags}的值的清单}}
{%%%%%%%%%%%%%%%%%%%%%%%%%%%%%%%%%%%%%%%%%head%%%%%%%%%%%%%%%%%%%%%%%%%%%%%%%%%%%
\bTABLEhead
\bTR[background=color,backgroundcolor=gray]
  \bTH cl\_mem\_flags \eTH
  \bTH 描述 \eTH \eTR
\eTABLEhead

%%%%%%%%%%%%%%%%%%%%%%%%%%%%%%%%%%%%%%%  body  %%%%%%%%%%%%%%%%%%%%%%%%%%%%%%%%%
\bTABLEbody

\clenumdesc{CL_MEM_READ_WRITE}{
    表明此\cnglo{memobj}可读可写。这是默认值。
}

\clenumdesc{CL_MEM_WRITE_ONLY}{
    只能写不能读,对这种\cnglo{memobj}的读操作是未定义的。
}

\clenumdesc{CL_MEM_READ_ONLY}{
    只能读不能写,对这种\cnglo{memobj}的写操作是未定义的。
}

\clenumdesc{CL_MEM_USE_HOST_PTR}{
    只有当\carg{host_ptr}不是\cenum{NULL}时,此flag才有效。它表明\cnglo{app}想让\scopencl实现使用\carg{host_ptr}所引用的内存来存储\cnglo{memobj}的内容。

    允许\scopencl实现在\cnglo{device}内存中保存一份\carg{host_ptr}所指内容的拷贝。\cnglo{kernel}在\cnglo{device}上执行时可以使用这份作为cache的拷贝。

    如果多个\cnglo{bufobj}由同一个\carg{host_ptr}创建,或者有重叠区域,当\scopencl\cnglo{cmd}操作这些\cnglo{bufobj}时,其结果未定义。
}

\clenumdesc{CL_MEM_ALLOC_HOST_PTR}{
    此flag表明\cnglo{app}想让\scopencl实现在\cnglo{host}可以访问的内存中分配内存。

    它与\cenum{CL_MEM_USE_HOST_PTR}是互斥的。
}

\clenumdesc{CL_MEM_COPY_HOST_PTR}{
    只有\carg{host_ptr}不是\cenum{NULL}时此flag才有效。它表明\cnglo{app}想让\scopencl实现使用\carg{host_ptr}所引用的内存来为\cnglo{memobj}分配内存,并拷贝数据。

    它与\cenum{CL_MEM_USE_HOST_PTR}互斥。

    它可以与\cenum{CL_MEM_ALLOC_HOST_PTR}一起使用,对由\cnglo{host}可访问的内存(如PCIe)分配的\ctype{cl_mem}对象进行初始化。
}

\eTABLEbody

}

\carg{size}是所分配的\cnglo{bufobj}的大小。

\carg{host_ptr}指向由\cnglo{app}所分配的缓冲数据。其大小必须大于等于\carg{size}。

\carg{errcode_ret}用来返回错误码。如果是\cenum{NULL},不会返回错误码。

如果执行成功,\capi{clCreateBuffer}会返回一个非零的\cnglo{bufobj},并将
\carg{errcode_ret}置为\cenum{CL_SUCCESS}。否则,返回\cenum{NULL},并将
\carg{errcode_ret}置为下列错误值之一:
\startigBase
\itemenumdesc{CL_INVALID_CONTEXT}{如果\carg{context}无效。}
\itemenumdesc{CL_INVALID_VALUE}{如果\carg{flags}的值无效。}

\startbuffer[footnoteshi]
如果\carg{size}比\carg{context}中所有\cnglo{device}的
\cenum{CL_DEVICE_MAX_MEM_ALLOC_SIZE}(参见\reftab{cldevquery})都大,实现可能返回
\cenum{CL_INVALID_BUFFER_SIZE}。
\stopbuffer
\itemenumdesc{CL_INVALID_BUFFER_SIZE}{
如果\carg{size}是0\footnote{\getbuffer[footnoteshi]}。
}

\itemenumdesc{CL_INVLAID_HOST_PTR}{
如果\carg{host_ptr}是\cenum{NULL},并且\carg{flags}中设置了\cenum{CL_MEM_USE_HOST_PTR}
或\cenum{CL_MEM_COPY_HOST_PTR};或者\carg{host_ptr}不是\cenum{NULL},但是\carg{flags}
中没有设置\cenum{CL_MEM_USE_HOST_PTR}或\cenum{CL_MEM_COPY_HOST_PTR}。
}

\itemenumdesc{CL_MEM_OBJECT_ALLOCATION_FAILURE}{
如果为\cnglo{bufobj}分配内存失败。
}

\itemenumdesc{CL_MEM_OBJECT_ALLOCATION_FAILURE}{
如果为\cnglo{bufobj}分配内存失败。
}

\itemenumdesc{CL_OUT_OF_RESOURCES}{如果\scdevfailres。}
\itemenumdesc{CL_OUT_OF_HOST_MEMORY}{如果\schostfailres。}

\stopigBase

%%%%%%%%%%%%%%%%%%%%%%%%%%%%%%%%%%%% clCreateSubBuffer
可以使用函数\capi{clCreateSubBuffer}由一个现有的\cnglo{bufobj}创建一个新的\cnglo{bufobj}
(叫做子\cnglo{bufobj})。
\startclc
cl_mem clCreateSubBuffer(cl_mem buffer,
			cl_mem_flags flags,
			cl_buffer_create_type buffer_create_type,
			const void *buffer_create_info,
			cl_int *errcode_ret)
\stopclc

\carg{buffer}必须是一个有效的\cnglo{bufobj},并且不能是子\cnglo{bufobj}。

\carg{flags}是位域,用来指定创建\cnglo{memobj}时的一些分配和使用信息,参见\reftab{clmemflags}。

\carg{buffer_create_type}和\carg{buffer_create_info}描述了所创建\cnglo{bufobj}的类型。
\carg{buffer_create_type}的支持清单以及对应的\carg{buffer_create_info}如
\reftab{clcreatesubbuffer}所示。

\cltable
{\placetable[here,force][tab:clcreatesubbuffer]
{\capi{clCreateSubBuffer}中所支持的名字及值的清单}}
{%%%%%%%%%%%%%%%%%%%%%%%%%%%%%%%%%%%%%%%%%head%%%%%%%%%%%%%%%%%%%%%%%%%%%%%%%%%%%
\bTABLEhead
\bTR[background=color,backgroundcolor=gray]
  \bTH \ctype{cl_buffer_create_type} \eTH
  \bTH 描述 \eTH
  \eTR
\eTABLEhead

%%%%%%%%%%%%%%%%%%%%%%%%%%%%%%%%%%%%%%%  body  %%%%%%%%%%%%%%%%%%%%%%%%%%%%%%%%%
\bTABLEbody

\clenumdesc{CL_BUFFER_CREATE_TYPE_REGION}{
  创建一个用来描述\carg{buffer}中特定区块的\cnglo{bufobj}。

  \carg{buffer_create_info}指向如下数据结构:
\todo{clcintable}
%\startclc
%typedef struct _cl_buffer_region {
%	size_t origin;
%	size_t size;
%} cl_buffer_region;
%\stopclc
$(origin, size)$就是在\carg{buffer}中的偏移量和大小。

如果\carg{buffer}是用\cenum{CL_MEM_USE_HOST_PTR}创建的,所返回\cnglo{bufobj}的
\carg{host_ptr}就是$host\_ptr+origin$。

所返回的\cnglo{bufobj}引用了为\carg{buffer}分配的数据存储空间,并指向其中的特定区域
$(origin,size)$。

如果在\carg{buffer}中,区域$(origin,size)$越界了,则会在\carg{errcode_ret}中返回
\cenum{CL_INVALID_VALUE}。

如果\carg{size}是0,则返回\cenum{CL_INVALID_BUFFER_SIZE}。

如果与\carg{buffer}像关联的\cnglo{context}中没有一个设备的\cenum{CL_DEVICE_MEM_BASE_ADDR_ALIGN}与$origin$对齐,则会在\carg{errcode_ret}中返回
\cenum{CL_MISALIGNED_SUB_BUFFER_OFFSET}。

}

\eTABLEbody

}

如果执行成功,\capi{clCreateSubBuffer}会返回\cenum{CL_SUCCESS}。否则,会将\carg{errcode_ret}
置为下列错误码之一:

\startigBase
\itemenumdesc{CL_INVALID_MEM_OBJECT}{
如果\carg{buffer}不是一个有效的\cnglo{bufobj}或者是一个子\cnglo{bufobj}。
}

\itemenumdesc{CL_INVALID_VALUE}{
如果\carg{buffer}是用\cenum{CL_MEM_WRITE_ONLY}创建的,可是\carg{flags}中却指定了
\cenum{CL_MEM_READ_WRITE}或\cenum{CL_MEM_READ_ONLY};或者\carg{buffer}是用
\cenum{CL_MEM_READ_ONLY}创建的,可是\carg{flags}中却指定了
\cenum{CL_MEM_READ_WRITE}或\cenum{CL_MEM_WRITE_ONLY};或者\carg{flags}中指定了
\cenum{CL_MEM_USE_HOST_PRT}或\cenum{CL_MEM_ALLOC_HOST_PTR}或
\cenum{CL_MEM_COPY_HOST_PTR}。
}

\itemenumdesc{CL_INVALID_VALUE}{
如果\carg{buffer_create_type}中的值无效。
}

\itemenumdesc{CL_INVALID_VALUE}{
如果\carg{buffer_create_info}(对应于\carg{buffer_create_type})中的值无效或者是
\cenum{NULL}。
}

\itemenumdesc{CL_INVALID_BUFFER_SIZE}{如果\carg{size}是0。}

\itemenumdesc{CL_MEM_OBJECT_ALLOCATION_FAILURE}{
如果为子\cnglo{bufobj}分配内存失败。
}

\itemenumdesc{CL_OUT_OF_RESOURCES}{如果\scdevfailres。}
\itemenumdesc{CL_OUT_OF_HOST_MEMORY}{如果\schostfailres。}

\stopigBase

NOTE:对一个\cnglo{bufobj}及其子\cnglo{bufobj}的并行读写都是未定义的。对由同一\cnglo{bufobj}
创建的互相重叠的子\cnglo{bufobj}的并行读写是未定义的。当然对这些对象的读操作都是定义了的。



