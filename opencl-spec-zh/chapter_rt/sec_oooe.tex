\section[sec:oooe]{內核和內存對象命令的亂序執行}

提交給\cnglo{cmdq}的 OpenCL 函式按調用順序入隊,
但執行順序卻是可配置的,既可以\cnglo{inordexec},也可以\cnglo{outordexec}。
\capi{clCreateCommandQueue} 的參數 \carg{properties} 可用來指定執行順序。

如果沒有為\cnglo{cmdq}設置屬性 \cenum{CL_QUEUE_OUT_OF_ORDER_EXEC_MODE_ENABLE},
則其中的\cnglo{cmd}\cnglo{inordexec}。
例如,如果\cnglo{app}先調用 \capi{clEnqueueNDRangeKernel} 執行\cnglo{kernel} A,
然後調用 \capi{clEnqueueNDRangeKernel} 執行\cnglo{kernel} B,
則\cnglo{app}可以假定 A 執行完後 B 才開始執行。
如果\cnglo{kernel} A 所輸出的\cnglo{memobj}是\cnglo{kernel} B 的輸入,
那麼就通過執行\cnglo{kernel} A 所產生的\cnglo{memobj}而言,
\cnglo{kernel} B 就可以看到其中正確的數據。
如果為\cnglo{cmdq}設置了屬性 \cenum{CL_QUEUE_OUT_OF_ORDER_EXEC_MODE_ENABLE},
則不保證\cnglo{kernel} B 開始執行前,\cnglo{kernel} A 一定執行完畢。

\cnglo{app}可以通過設置\cnglo{cmdq}的屬性 \cenum{CL_QUEUE_OUT_OF_ORDER_EXEC_MODE_ENABLE},
將其中的\cnglo{cmd}配置成\cnglo{outordexec}。
創建\cnglo{cmdq}時就可以設置此屬性。
在\cnglo{outordexec}模式下,不保證\cnglo{cmd}按入隊的順序執行完畢。
由於不保證\cnglo{kernel}會\cnglo{inordexec}
(這裡的順序是指調用 \capi{clEnqueueNDRangeKernel} 的順序),
因此很可能先入隊的\cnglo{kernel} A (用事件 A 來標識)
開始執行和/或執行完畢要比後入隊的\cnglo{kernel} B 晚。
要想保證\cnglo{kernel}的執行順序,可以等待某個事件(這裡就是事件 A)。
對事件 A 的等待可以通過為\cnglo{kernel} B 調用 \capi{clEnqueueNDRangeKernel} 時
設置其參數 \carg{event_wait_list} 來實現。

另外, OpenCL 還提供下列\cnglo{cmd}:
\startigBase[indentnext=no]
\item \cnglo{marker},由 \capi{clEnqueueMarkerWithWaitList} 入隊,用於等待一些事件;
\item \cnglo{barrier},由 \capi{clEnqueueBarrierWithWaitList} 入隊;
\stopigBase
等待事件的\cnglo{cmd}可以保證事件所標識的\cnglo{cmd}完成後,下一批\cnglo{cmd}才開始執行。
而\cnglo{barrier}\cnglo{cmd}則保證\cnglo{cmdq}中所有之前入隊的\cnglo{cmd}都執行完畢後,
下一批\cnglo{cmd}才開始執行。

類似的,如果設置了屬性 \cenum{CL_QUEUE_OUT_OF_ORDER_EXEC_MODE_ENABLE},
那麼在 \capi{clEnqueueNDRangeKernel}、
\capi{clEnqueueTask} 或 \capi{clEnqueueNativeKernel}
之後入隊的一些讀、寫、拷貝或映射\cnglo{memobj}的\cnglo{cmd}
可能不會等待\cnglo{kernel}執行完畢。
要想保證\cnglo{cmd}的正確順序,
可以使用\cnglo{marker}\cnglo{cmd}等待 \capi{clEnqueueNDRangeKernel}、
\capi{clEnqueueTask} 或 \capi{clEnqueueNativeKernel}
所返回的\cnglo{evtobj},
或者使用\cnglo{barrier}\cnglo{cmd},
這樣只有在\cnglo{kernel}執行完畢後,才會開始讀寫\cnglo{memobj}。
