\section{執行內核}

\topclfunc{clEnqueueNDRangeKernel}

\startCLFUNC
cl_int clEnqueueNDRangeKernel (
			cl_command_queue command_queue,
			cl_kernel kernel,
			cl_uint work_dim,
			const size_t *global_work_offset,
			const size_t *global_work_size,
			const size_t *local_work_size,
			cl_uint num_events_in_wait_list,
			const cl_event *event_wait_list,
			cl_event *event)
\stopCLFUNC

此函式所入隊的\cnglo{cmd}可以在\cnglo{device}上執行\cnglo{kernel}。

\carg{command_queue} 是一個\cnglo{cmdq}。
排隊的\cnglo{kernel}會在 \carg{command_queue} 所關聯的\cnglo{device}上執行。

\carg{kernel} 是一個\cnglo{kernelobj}。
\carg{kernel} 和 \carg{command_queue} 必須位於同一 OpenCL \cnglo{context}中。

\carg{work_dim} 用來指定全局\cnglo{workitem}以及\cnglo{workgrp}中\cnglo{workitem}的維數。
其值必須大於零並且小於等於 \cenum{CL_DEVICE_MAX_WORK_ITEM_DIMENSIONS}。

\carg{global_work_offset} 是一個陣列,有 \carg{work_dim} 個元素,且元素為無符號值;
所描述的偏移量可用來計算\cnglo{workitem}的\cnglo{glbid}。
如果 \carg{global_work_offset} 是 \cmacro{NULL},
則\cnglo{glbid}起自偏移量\math{(0, 0, ... 0)}。

\carg{global_work_size} 也是一個陣列,有 \carg{work_dim} 個元素,且元素均為無符號值;
對於將要執行 \carg{kernel} 的全局\cnglo{workitem}而言,
他在某個維度上的數目由陣列中的相應元素表示,
總數為:
\startformula
\math{\prod_{i=0}^{\marg{work_dim} - 1}\marg{global_work_size}[i]}。
\stopformula

\carg{local_work_size} 也是一個陣列,有 \carg{work_dim} 個元素,且元素均為無符號值;
對於將要執行 \carg{kernel} 的\cnglo{workgrp}而言,
其中\cnglo{workitem}的數目由陣列中的相應元素表示,
總數為:
\startformula
\prod_{i=0}^{\marg{work_dim} - 1}\marg{local_work_size}[i]
\stopformula
這個總數必須小於或等於\reftab{cldevquery}中的 \cenum{CL_DEVICE_MAX_WORK_GROUP_SIZE},
而且 \math{\marg{local_work_size}[i]} 必須小於或等於相應的
\math{\menum{CL_DEVICE_MAX_WORK_ITEM_SIZES}[i]},
其中 \math{0\leq i \leq \marg{work_dim} -1}。
顯式指定的 \carg{local_work_size} 可用來確定
怎樣將 \carg{global_work_size} 所指定的全局\cnglo{workitem}劃分成多個\cnglo{workgrp}實體。
如果指定了 \carg{local_work_size},
\math{\marg{global_work_size}[i]} 必須能被相應的
\math{\menum{local_work_size}[i]} 整除,
其中 \math{0\leq i \leq \marg{work_dim} -1}。

也可以在\cnglo{program}源碼中通過限定符
\ccmm{__attribute__((reqd_work_group_size(X, Y, Z)))} (參見\refsec{optAttrQlf})
為 \carg{kernel} 指定\cnglo{workgrp}的大小。
這種情況下, \carg{local_work_size} 的值必須與此特性限定符所指定的值相匹配。

\carg{local_work_size} 也可以是 \cmacro{NULL},
這樣的話 OpenCL 實作將自己決定如何將全局\cnglo{workitem}劃分成多個\cnglo{workgrp}實體。

這些\cnglo{workgrp}實體將在多個\cnglo{computeunit}上並行執行,
或在單個\cnglo{computeunit}上並發執行。

每個\cnglo{workitem}都有一個唯一的\cnglo{glbid}。
在\cnglo{kernel}中,
可以通過對 \carg{global_work_size} 和 \carg{global_work_offset} 的運算得到這個\cnglo{glbid}。
另外,每個\cnglo{workitem}在\cnglo{workgrp}中還有一個唯一的\cnglo{locid}。
在\cnglo{kernel}中,
可以通過對 \carg{local_work_size} 的運算得到這個\cnglo{locid}。
\cnglo{locid}始終起自 \math{(0, 0, ... 0)}。

\carg{event_wait_list} 和 \carg{num_events_in_wait_list} 中
列出了執行此\cnglo{cmd}前要等待的事件。
如果 \carg{event_wait_list} 是 \cmacro{NULL},
則無須等待任何事件,並且 \carg{num_events_in_wait_list} 必須是0。
如果 \carg{event_wait_list} 不是 \cmacro{NULL},
則其中所有事件都必須是有效的,並且 \carg{num_events_in_wait_list} 必須大於 0。
\carg{event_wait_list} 中的事件充當同步點,
並且必須與 \carg{command_queue} 位於同一個\cnglo{context}中。
此函式返回後,即可回收並重新使用 \carg{event_wait_list} 所關聯的內存。

\carg{event} 會返回一個\cnglo{evtobj},
用來標識此拷貝\cnglo{cmd},可用來查詢或等待此\cnglo{cmd}完成。
而如果 \carg{event} 是 \cmacro{NULL},就沒辦法查詢此\cnglo{cmd}的狀態或等待其完成了。
不過可以用 \capi{clEnqueueBarrierWithWaitList} 來代替。
如果 \carg{event_wait_list} 和 \carg{event} 都不是 \cmacro{NULL},
\carg{event} 不能屬於 \carg{event_wait_list}。

如果\cnglo{kernel}成功排隊,
\capi{clEnqueueNDRangeKernel} 會返回 \cenum{CL_SUCCESS}。
否則,返回下列錯誤碼之一:
\startigBase
\item \cenum{CL_INVALID_PROGRAM_EXECUTABLE},
如果 \carg{command_queue} 所關聯的\cnglo{device} 沒有對應的\cnglo{program}執行體。

\item \cenum{CL_INVALID_COMMAND_QUEUE},如果 \carg{command_queue} 無效。

\item \cenum{CL_INVALID_KERNEL},如果 \carg{kernel} 無效。

\item \cenum{CL_INVALID_CONTEXT},
如果 \carg{command_queue} 和 \carg{kernel} 位於不同的\cnglo{context}中,
或者 \carg{command_queue} 和 \carg{event_wait_list} 中的事件位於不同的\cnglo{context}中。

\item \cenum{CL_INVALID_KERNEL_ARGS},如果還未指定\cnglo{kernel}參數。

\item \cenum{CL_INVALID_WORK_DIMENSION},
如果 \carg{work_dim} 的值無效(即不在 1 到 3 的範圍內)。

\item \cenum{CL_INVALID_GLOBAL_WORK_SIZE},
如果 \carg{global_work_size} 是 \cmacro{NULL};
或者 \carg{global_work_size} 中的任意一個是 0
或超過了用來執行\cnglo{kernel}的\cnglo{device}的 \ccmm{sizeof(size_t)} 所給定的範圍。

\item \cenum{CL_INVALID_GLOBAL_OFFSET},
如果 \math{\marg{global_work_size}[i] + \marg{global_work_offset}[i]}
(其中 \math{0 \leq i \leq \marg{work_dim} - 1})中的任意一個超過了
用來執行\cnglo{kernel}的\cnglo{device}的 \ccmm{sizeof(size_t)} 所給定的範圍。

\item \cenum{CL_INVALID_WORK_GROUP_SIZE},
如果指定了 \carg{local_work_size},
並且 \carg{global_work_size} 不能被 \carg{local_work_size} 所整除
或者與\cnglo{program}源碼中的特性 \ccmm{__attribute__((reqd_work_group_size(X, Y, Z)))} 不匹配。

\item \cenum{CL_INVALID_WORK_GROUP_SIZE},
如果指定了 \carg{local_work_size},
並且\cnglo{workgrp}中\cnglo{workitem}的總數
(即 \math{\prod_{i=0}^{\marg{work_dim} - 1}\marg{local_work_size}[i]})
大於\reftab{cldevquery}中的 \cenum{CL_DEVICE_MAX_WORK_GROUP_SIZE}。

\item \cenum{CL_INVALID_WORK_GROUP_SIZE},
如果 \carg{local_work_size} 是 \cmacro{NULL},
並且 \cnglo{program}源碼用限定符 \ccmm{__attribute__((reqd_work_group_size(X, Y, Z)))}
聲明了\cnglo{workgrp}的大小。

\item \cenum{CL_INVALID_WORK_ITEM_SIZE},
如果 \math{\marg{local_work_size}[i]}
(其中 \math{0 \leq i \leq \marg{work_dim} - 1})
中的任意一個超過了對應的 \math{\menum{CL_DEVICE_MAX_WORK_ITEM_SIZES}[i]}。

\item \cenum{CL_MISALIGNED_SUB_BUFFER_OFFSET},
如果給型別為\cnglo{bufobj}的參數指定了一個子\cnglo{bufobj}作為其值,
並且創建此子\cnglo{bufobj}時所指定的 \carg{offset} 沒有按 \carg{command_queue}
所關聯\cnglo{device}的 \cenum{CL_DEVICE_MEM_BASE_ADDR_ALIGN} 進行對齊。

\item \cenum{CL_INVALID_IMAGE_SIZE},
如果參數是\cnglo{imgobj},但是圖像的大小
(圖像的寬度、高度、指定的或計算出來的行間距和/或面間距)
不被 \carg{command_queue} 所關聯的\cnglo{device}所支持。

\item \cenum{CL_IMAGE_FORMAT_NOT_SUPPORTED},
如果參數是\cnglo{imgobj},但是圖像的格式
(圖像通道順序和數據型別)
不被 \carg{command_queue} 所關聯的\cnglo{device}所支持。

\item \cenum{CL_OUT_OF_RESOURCES},
如果由於執行\cnglo{kernel}所需的資源不足,而無法將 \carg{kernel} 的執行實體入隊。
例如,顯式指定了 \carg{local_work_size},
但是沒有足夠的資源(如寄存器、\cnglo{locmem})從而導致失敗。
另一個例子, \carg{kernel}中的只讀圖像參數的數目
超過了\cnglo{device}的 \cenum{CL_DEVICE_MAX_READ_IMAGE_ARGS},
或者 \carg{kernel}中的只寫圖像參數的數目
超過了\cnglo{device}的 \cenum{CL_DEVICE_MAX_WRITE_IMAGE_ARGS},
或者 \carg{kernel}中所使用的\cnglo{sampler}的數目
超過了\cnglo{device}的 \cenum{CL_DEVICE_MAX_SAMPLERS}。

\item \cenum{CL_MEM_OBJECT_ALLOCATION_FAILURE},
如果為 \carg{kernel} 參數中的\cnglo{imgobj}或\cnglo{bufobj}分配內存時失敗。

\item \cenum{CL_INVALID_EVENT_WAIT_LIST},
如果 \carg{event_wait_list} 是 \cmacro{NULL},且 \carg{num_events_in_wait_list} > 0;
或者 \carg{event_wait_list} 不是 \cmacro{NULL},但 \carg{num_events_in_wait_list} 是 0;
或者 \carg{event_wait_list} 中的事件無效。

\item \cenum{CL_OUT_OF_RESOURCES},如果\scdevfailres。

\item \cenum{CL_OUT_OF_HOST_MEMORY},如果\schostfailres。
\stopigBase

\topclfunc{clEnqueueTask}

\startCLFUNC
cl_int clEnqueueTask (
		cl_command_queue command_queue,
		cl_kernel kernel,
		cl_uint num_events_in_wait_list,
		const cl_event *event_wait_list,
		cl_event *event)
\stopCLFUNC

此函式所入隊的\cnglo{cmd}可以在\cnglo{device}上執行\cnglo{kernel}。
執行\cnglo{kernel}時使用單個\cnglo{workitem}。

\carg{command_queue} 是一個\cnglo{cmdq}。
所排隊的\cnglo{kernel}會在 \carg{command_queue} 所關聯的\cnglo{device}上執行。

\carg{kernel} 是一個\cnglo{kernelobj}。
\carg{kernel} 和 \carg{command_queue} 必須位於同一個 OpenCL \cnglo{context}中。

\carg{event_wait_list} 和 \carg{num_events_in_wait_list} 中
列出了執行此\cnglo{cmd}前要等待的事件。
如果 \carg{event_wait_list} 是 \cmacro{NULL},
則無須等待任何事件,並且 \carg{num_events_in_wait_list} 必須是0。
如果 \carg{event_wait_list} 不是 \cmacro{NULL},
則其中所有事件都必須是有效的,並且 \carg{num_events_in_wait_list} 必須大於 0。
\carg{event_wait_list} 中的事件充當同步點,
並且必須與 \carg{command_queue} 位於同一個\cnglo{context}中。
此函式返回後,即可回收並重新使用 \carg{event_wait_list} 所關聯的內存。

\carg{event} 會返回一個\cnglo{evtobj},
用來標識此拷貝\cnglo{cmd},可用來查詢或等待此\cnglo{cmd}完成。
而如果 \carg{event} 是 \cmacro{NULL},就沒辦法查詢此\cnglo{cmd}的狀態或等待其完成了。
不過可以用 \capi{clEnqueueBarrierWithWaitList} 來代替。
如果 \carg{event_wait_list} 和 \carg{event} 都不是 \cmacro{NULL},
\carg{event} 不能屬於 \carg{event_wait_list}。

\capi{clEnqueueTask} 相當於在調用 \capi{clEnqueueNDRangeKernel} 時,
參數 \carg{work_dim} 為 1, \carg{global_work_offset} 為 \cmacro{NULL},
\math{\marg{global_work_size}[0]} 為 1,
\math{\marg{local_work_size}[0]} 為 1。

如果\cnglo{kernel}成功排隊, \capi{clEnqueueTask} 會返回 \cenum{CL_SUCCESS}。
否則,返回下列錯誤碼之一:
\startigBase
\item \cenum{CL_INVALID_PROGRAM_EXECUTABLE},
如果 \carg{command_queue} 所關聯的\cnglo{device} 沒有對應的\cnglo{program}執行體。

\item \cenum{CL_INVALID_COMMAND_QUEUE},如果 \carg{command_queue} 無效。

\item \cenum{CL_INVALID_KERNEL},如果 \carg{kernel} 無效。

\item \cenum{CL_INVALID_CONTEXT},
如果 \carg{command_queue} 和 \carg{kernel} 位於不同的\cnglo{context}中,
或者 \carg{command_queue} 和 \carg{event_wait_list} 中的事件位於不同的\cnglo{context}中。

\item \cenum{CL_INVALID_KERNEL_ARGS},如果還未指定\cnglo{kernel}參數。

\item \cenum{CL_INVALID_WORK_GROUP_SIZE},
如果通過限定符 \ccmm{__attribute__((reqd_work_group_size(X, Y, Z)))} 為 \carg{kernel}
指定的\cnglo{workgrp}大小不是 \math{(1, 1, 1)}。

\item \cenum{CL_MISALIGNED_SUB_BUFFER_OFFSET},
如果給型別為\cnglo{bufobj}的參數指定了一個子\cnglo{bufobj}作為其值,
並且創建此子\cnglo{bufobj}時所指定的 \carg{offset} 沒有按 \carg{command_queue}
所關聯\cnglo{device}的 \cenum{CL_DEVICE_MEM_BASE_ADDR_ALIGN} 進行對齊。

\item \cenum{CL_INVALID_IMAGE_SIZE},
如果參數是\cnglo{imgobj},但是圖像的大小
(圖像的寬度、高度、指定的或計算出來的行間距和/或面間距)
不被 \carg{command_queue} 所關聯的\cnglo{device}所支持。

\item \cenum{CL_IMAGE_FORMAT_NOT_SUPPORTED},
如果參數是\cnglo{imgobj},但是圖像的格式
(圖像通道順序和數據型別)
不被 \carg{command_queue} 所關聯的\cnglo{device}所支持。

\item \cenum{CL_OUT_OF_RESOURCES},
如果由於執行\cnglo{kernel}所需的資源不足,而無法將 \carg{kernel} 的執行實體入隊。

\item \cenum{CL_MEM_OBJECT_ALLOCATION_FAILURE},
如果為 \carg{kernel} 參數中的\cnglo{imgobj}或\cnglo{bufobj}分配內存時失敗。

\item \cenum{CL_INVALID_EVENT_WAIT_LIST},
如果 \carg{event_wait_list} 是 \cmacro{NULL},且 \carg{num_events_in_wait_list} > 0;
或者 \carg{event_wait_list} 不是 \cmacro{NULL},但 \carg{num_events_in_wait_list} 是 0;
或者 \carg{event_wait_list} 中的事件無效。

\item \cenum{CL_OUT_OF_RESOURCES},如果\scdevfailres。

\item \cenum{CL_OUT_OF_HOST_MEMORY},如果\schostfailres。
\stopigBase

\topclfunc{clEnqueueNativeKernel}

\startCLFUNC
cl_int clEnqueueNativeKernel (
			cl_command_queue command_queue,
			void (CL_CALLBACK *user_func)(void *)
			void *args,
			size_t cb_args,
			cl_uint num_mem_objects,
			const cl_mem *mem_list,
			const void **args_mem_loc,
			cl_uint num_events_in_wait_list,
			const cl_event *event_wait_list,
			cl_event *event)
\stopCLFUNC

此函式所入隊的\cnglo{cmd}將執行一個原生的 C/C++ 函式(所謂原生的即不是用 OpenCL 編譯器編譯的)。

\carg{command_queue} 是一個\cnglo{cmdq}。
原生的用戶函式要想在其中執行,
此隊列所在\cnglo{device}必須具有 \cenum{CL_EXEC_NATIVE_KERNEL} 的能力
(參見\reftab{cldevquery}中的 \cenum{CL_DEVICE_EXECUTION_CAPABILITIES})。

\carg{user_func} 指向一個可以被\cnglo{host}調用的用戶函式。

\carg{args} 指向調用 \carg{user_func} 時所需的參數。

\carg{cb_args} 即 \carg{args} 所指參數的大小。

\carg{args} 所指的數據(大小為 \carg{cb_args})會被拷貝一份,
這份拷貝的指針將被傳遞給 \carg{user_func}。
之所以要拷貝,是因為 \carg{args} 可能包含有\cnglo{memobj}(\ccmm{cl_mem}),
而且需要通過一個指向\cnglo{glbmem}的指針對其進行修改和取代。
一旦 \capi{clEnqueueNativeKernel} 返回,
\cnglo{app}就可以重新使用 \carg{args} 所指向的內存區域了。

\carg{num_mem_objects} 即 \carg{args} 中\cnglo{bufobj}的數目。

\carg{mem_list} 指向一組\cnglo{bufobj},要求 \carg{num_mem_objects} > 0。
其中的\cnglo{bufobj}可能是 \capi{clCreateBuffer} 所返回的\cnglo{memobj}的句柄
(即 \ccmm{cl_mem}),也可能是 \cmacro{NULL}。

\carg{args_mem_loc} 指向 \carg{args} 中存儲\cnglo{memobj}句柄(即 \ccmm{cl_mem})的位置。
在執行用戶函式之前,會用執行\cnglo{glbmem}的指針取代這些句柄。

\carg{event_wait_list}、 \carg{num_events_in_wait_list} 和 \carg{event} 與
\capi{clEnqueueNDRangeKernel} 中描述的一樣。

如果用戶函式的執行實體成功排隊, \capi{clEnqueueNativeKernel} 會返回 \cenum{CL_SUCCESS}。
否則,返回下列錯誤碼之一:
\startigBase
\item \cenum{CL_INVALID_COMMAND_QUEUE},如果 \carg{command_queue} 無效。

\item \cenum{CL_INVALID_CONTEXT},
如果 \carg{command_queue} 和 \carg{event_wait_list} 中的事件位於不同的\cnglo{context}中。

\item \cenum{CL_INVALID_VALUE},
如果 \carg{user_func} 是 \cmacro{NULL}。

\item \cenum{CL_INVALID_VALUE},
如果 \carg{args} 是 \cmacro{NULL},並且 \carg{cb_args} > 0;
或者 \carg{args} 是一個 \cmacro{NULL} 值,並且 \carg{num_mem_objects} > 0。

\item \cenum{CL_INVALID_VALUE},
如果 \carg{args} 是 \cmacro{NULL},並且 \carg{cb_args} 是 0。

\item \cenum{CL_INVALID_VALUE},
如果 \carg{num_mem_objects} > 0,
並且 \carg{mem_list} 或 \carg{args_mem_loc} 是 \cmacro{NULL}。

\item \cenum{CL_INVALID_VALUE},
如果 \carg{num_mem_objects} = 0,
並且 \carg{mem_list} 或 \carg{args_mem_loc} 不是 \cmacro{NULL}。

\item \cenum{CL_INVALID_OPERATION},
如果 \carg{command_queue} 所關聯的\cnglo{device} 不能執行這個原生\cnglo{kernel}。

\item \cenum{CL_INVALID_MEM_OBJECT},
如果 \carg{mem_list} 中的任一\cnglo{memobj}無效或者不是\cnglo{bufobj}。

\item \cenum{CL_OUT_OF_RESOURCES},
如果由於執行\cnglo{kernel}所需的資源不足,而無法將\cnglo{kernel}的執行實體入隊。

\item \cenum{CL_MEM_OBJECT_ALLOCATION_FAILURE},
如果為\cnglo{kernel}參數中的\cnglo{bufobj}分配內存時失敗。

\item \cenum{CL_INVALID_EVENT_WAIT_LIST},
如果 \carg{event_wait_list} 是 \cmacro{NULL},
但 \carg{num_events_in_wait_list} > 0;
或者 \carg{event_wait_list} 不是 \cmacro{NULL},
但 \carg{num_events_in_wait_list} 是 0;
或者 \carg{event_wait_list} 中的事件無效。

\item \cenum{CL_OUT_OF_RESOURCES},如果\scdevfailres。

\item \cenum{CL_OUT_OF_HOST_MEMORY},如果\schostfailres。
\stopigBase

\startnotepar
\cnglo{kernel}參數中只讀圖像的數目不能超過 \cenum{CL_DEVICE_MAX_READ_IMAGE_ARGS}。
\cnglo{kernel}參數中帶有限定符 \cqlf{read_only} 的 2D 圖像陣列記為一個圖像。

\cnglo{kernel}參數中只寫圖像的數目不能超過 \cenum{CL_DEVICE_MAX_WRITE_IMAGE_ARGS}。
\cnglo{kernel}參數中帶有限定符 \cqlf{write_only} 的 2D 圖像陣列記為一個圖像。
\stopnotepar
