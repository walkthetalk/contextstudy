\section{事件對象}

\cnglo{evtobj}可用來指代執行\cnglo{kernel}的\cnglo{cmd}
(\capi{clEnqueueNDRangeKernel}、
 \capi{clEnqueueTask}、 \capi{clEnqueueNativeKernel}),
讀、寫、映射以及拷貝\cnglo{memobj}的命令
(\capi{clEnqueue{Read | Write | Map}Buffer}、 \capi{clEnqueueUnmapMemObject}、
 \capi{clEnqueue{Read | Write}BufferRect}、 \capi{clEnqueue{Read | Write | Map}Image}、
 \capi{clEnqueueCopy{Buffer | Image}}、 \capi{clEnqueueCopyBufferRect}、
 \capi{clEnqueueCopyBufferToImage}、 \capi{clEnqueueCopyImageToBuffer}),
 \capi{clEnqueueMarkerWithWaitList}、
 \capi{clEnqueueBarrierWithWaitList}
 或用戶事件。

\cnglo{evtobj}可用來跟蹤\cnglo{cmd}的執行狀態。
那些會將\cnglo{cmd}入隊的 API 調用會創建一個新的\cnglo{evtobj},
並在參數 \carg{event} 中將其返回;
如果將\cnglo{cmd}插入隊列時出現了錯誤,則不會返回\cnglo{evtobj}。

在任意給定的時間點上,所入隊的\cnglo{cmd}的執行狀態將是下列之一:
\startigBase
\item \cenum{CL_QUEUED},這表示\cnglo{cmd}已經入隊。
這是所有事件的初始狀態(用戶事件除外)。

\item \cenum{CL_SUBMITTED},這是所有用戶事件的初始狀態。
對於其它事件,這表明\cnglo{host}已經將\cnglo{cmd}提交給了\cnglo{device}。

\item \cenum{CL_RUNNING},這表明\cnglo{device}已經開始執行\cnglo{cmd}。
\cnglo{cmd}的執行狀態要想從 \cenum{CL_SUBMITTED} 變成 \cenum{CL_RUNNING},
它所等待的所有事件必須都已經成功完成,即它們的執行狀態必須是 \cenum{CL_COMPLETE}。

\item \cenum{CL_COMPLETE},這表明\cnglo{cmd}已經成功完成。

\item 錯誤碼,錯誤碼是一個負整數,表明\cnglo{cmd}異常終止。
異常終止的原因有很多,如非法的內存訪問。
\stopigBase

注意:\cnglo{cmd}的執行狀態是 \cenum{CL_COMPLETE} 或者負整數都表示已經完成。

如果\cnglo{cmd}被終止執行,那麼它所關聯的\cnglo{cmdq}、\cnglo{context}
(包括其中的其它\cnglo{cmdq})就不再可用。
這時如果 OpenCL API 調用要使用這個\cnglo{context}(和其中的\cnglo{cmdq}),
則其行為\cnglo{impdef}。
創建\cnglo{context}時用戶所註冊的回調函數可用來報告相應的錯誤信息。

% clCreateUserEvent
函數
\startclc
cl_event clCreateUserEvent (cl_context context, cl_int *errcode_ret)
\stopclc
可用來創建用戶自己的\cnglo{evtobj}。
有了用戶事件,
\cnglo{app}就可以讓所入隊的\cnglo{cmd}先等待用戶事件完成,然後再由\cnglo{device}來執行。

\carg{context} 是 OpenCL \cnglo{contex}。

\carg{errcode_ret} 會返回相應的錯誤碼。
如果 \carg{errcode_ret} 是 \cenum{NULL},則不會返回錯誤碼。

如果成功創建了用戶\cnglo{evtobj}, \capi{clCreateUserEvent} 會將其返回,
並將 \carg{errcode_ret} 置為 \cenum{CL_SUCCESS}。
否則,返回 \cenum{NULL},並將 \carg{errcode_ret} 置為下列錯誤碼之一:
\startigBase
\item \cenum{CL_INVALID_CONTEXT},如果 \carg{context} 無效。

\item \cenum{CL_OUT_OF_RESOURCES},如果\scdevfailres。

\item \cenum{CL_OUT_OF_HOST_MEMORY},如果\schostfailres。
\stopigBase

用戶\cnglo{evtobj}在創建後,其執行狀態缺省為 \cenum{CL_SUBMITTED}。

% clSetUserEventStatus
函數
\startclc
cl_int clSetUserEventStatus (cl_event event, cl_int execution_status)
\stopclc
可用來設置用戶\cnglo{evtobj}的執行狀態。

\carg{event} 即用 \capi{clCreateUserEvent} 創建的用戶\cnglo{evtobj}。

\carg{execution_status} 即將要設置的新的執行狀態,
可以是 \cenum{CL_COMPLETE},或者一個用來表示錯誤的負整數。
負整數會導致所有已經入隊、並且等待此事件的\cnglo{cmd}被終止。
要想改變 \carg{event} 的執行狀態, \capi{clSetUserEventStatus} 只能被調用一次。

如果執行成功, \capi{clSetUserEventStatus} 會返回 \cenum{CL_SUCCESS}。
否則,返回下列錯誤碼之一:
\startigBase
\item \cenum{CL_INVALID_EVENT},如果 \carg{event} 無效。

\item \cenum{CL_INVALID_VALUE},
如果 \carg{execution_status} 既不是 \cenum{CL_COMPLETE},也不是負整數。

\item \cenum{CL_INVALID_OPERATION},
如果之前已經調用 \capi{clSetUserEventStatus} 改變過 \carg{event} 的執行狀態。

\item \cenum{CL_OUT_OF_RESOURCES},如果\scdevfailres。

\item \cenum{CL_OUT_OF_HOST_MEMORY},如果\schostfailres。
\stopigBase

注意:
使用 \capi{clEnqueue***} 時,如果參數 \carg{event_wait_list} 中有用戶事件,
那麼對於所入隊的\cnglo{cmd}而言,
在調用會釋放 OpenCL 對象(\cnglo{evtobj})的 OpenCL API 之前,
必須保證已經用 \capi{clSetUserEventStatus} 設置過這些用戶事件的狀態;
否則其行為未定義。

例如,下列代碼序列會導致 \capi{clReleaseMemObject} 的未定義行為:
\startclc
ev1 = clCreateUserEvent(ctx, NULL);
clEnqueueWriteBuffer(cq, buf1, CL_FALSE, ...,
				1, &ev1, NULL);
clEnqueueWriteBuffer(cq, buf2, CL_FALSE,...);
clReleaseMemObject(buf2);
clSetUserEventStatus(ev1, CL_COMPLETE);
\stopclc

而下列代碼序列則可以正確工作:
\startclc
ev1 = clCreateUserEvent(ctx, NULL);
clEnqueueWriteBuffer(cq, buf1, CL_FALSE, ...,
				1, &ev1, NULL);
clEnqueueWriteBuffer(cq, buf2, CL_FALSE,...);
clSetUserEventStatus(ev1, CL_COMPLETE);
clReleaseMemObject(buf2);
\stopclc

% clWaitForEvents
函數
\startclc
cl_int clWaitForEvents (cl_uint num_events, const cl_event *event_list)
\stopclc
會使\cnglo{host}線程等待 \carg{event_list} 中的\cnglo{evtobj}所標識的\cnglo{cmd}完成。
對於一個\cnglo{cmd}而言,如果其執行狀態是 \cenum{CL_COMPLETE} 或負整數,則任務已經完成了。
 \carg{event_list} 中的事件充當同步點。

如果 \carg{event_list} 中的所有事件的執行狀態都是 \cenum{CL_COMPLETE},
則 \capi{clWaitForEvents} 會返回 \cenum{CL_SUCCESS}。
否則,返回下列錯誤碼之一:
\startigBase
\item \cenum{CL_INVALID_VALUE},
如果 \carg{num_events} 是零或者 \carg{event_list} 是 \cenum{NULL}。

\item \cenum{CL_INVALID_CONTEXT},
如果 \carg{event_list} 中的事件分屬不同的\cnglo{context}。

\item \cenum{CL_INVALID_EVENT},
如果 \carg{event_list} 中的\cnglo{evtobj}無效。

\item \cenum{CL_EXEC_STATUS_ERROR_FOR_EVENTS_IN_WAIT_LIST},
如果 \carg{event_list} 中的任一事件的執行狀態是負整數。

\item \cenum{CL_OUT_OF_RESOURCES},如果\scdevfailres。

\item \cenum{CL_OUT_OF_HOST_MEMORY},如果\schostfailres。
\stopigBase

% clGetEventInfo
函數
\startclc
cl_int clGetEventInfo (cl_event event,
			cl_event_info param_name,
			size_t param_value_size,
			void *param_value,
			size_t *param_value_size_ret)
\stopclc
會返回\cnglo{evtobj}的相關信息。

\carg{event} 即所要查詢的\cnglo{evtobj}。

\carg{param_name} 指定要查詢什麼信息。
對於所支持的信息類型以及 \carg{param_value} 中返回的信息,請參見\reftab{clGetEventInfo}。

\carg{param_value} 指向的內存用來存儲查詢結果。
如果 \carg{param_value} 是 \cenum{NULL},則忽略。

\carg{param_value_size} 即 \carg{param_value} 所指內存塊的大小。
其值必須 >= \reftab{clGetEventInfo}中返回類型的大小。

\carg{param_value_size_ret} 返回查詢結果的實際大小。
如果 \carg{param_value_size_ret} 是 \cenum{NULL},則忽略。

\startbuffer[tblclgeteventinfo]
\capi{clGetEventInfo} 所支持的 \carg{param_names}
\stopbuffer
\splitfloat{
\placetable[here,force][tab:clGetEventInfo]{\getbuffer[tblclgeteventinfo]}
}{
{\startETD[cl_event_info][返回型別]

\clETD{CL_EVENT_COMMAND_QUEUE}{cl_command_queue}{
返回 \carg{event} 所關聯的\cnglo{cmdq}。
對於用戶\cnglo{evtobj},會返回 \cenum{NULL}。
}

\clETD{CL_EVENT_CONTEXT}{cl_context}{
返回 \carg{event} 所關聯的\cnglo{context}。
}

\clETD{CL_EVENT_COMMAND_TYPE}{cl_command_type}{
返回 \carg{event} 所關聯的\cnglo{cmd}。
可以是下列之一:
\startigBase
\item \cenum{CL_COMMAND_NDRANGE_KERNEL}
\item \cenum{CL_COMMAND_TASK}
\item \cenum{CL_COMMAND_NATIVE_KERNEL}
\item \cenum{CL_COMMAND_READ_BUFFER}
\item \cenum{CL_COMMAND_WRITE_BUFFER}
\item \cenum{CL_COMMAND_COPY_BUFFER}
\item \cenum{CL_COMMAND_READ_IMAGE}
\item \cenum{CL_COMMAND_WRITE_IMAGE}
\item \cenum{CL_COMMAND_COPY_IMAGE}
\item \cenum{CL_COMMAND_COPY_BUFFER_TO_IMAGE}
\item \cenum{CL_COMMAND_COPY_IMAGE_TO_BUFFER}
\item \cenum{CL_COMMAND_MAP_BUFFER}
\item \cenum{CL_COMMAND_MAP_IMAGE}
\item \cenum{CL_COMMAND_UNMAP_MEM_OBJECT}
\item \cenum{CL_COMMAND_MARKER}
\item \cenum{CL_COMMAND_ACQUIRE_GL_OBJECTS}
\item \cenum{CL_COMMAND_RELEASE_GL_OBJECTS}
\item \cenum{CL_COMMAND_READ_BUFFER_RECT}
\item \cenum{CL_COMMAND_WRITE_BUFFER_RECT}
\item \cenum{CL_COMMAND_COPY_BUFFER_RECT}
\item \cenum{CL_COMMAND_USER}
\item \cenum{CL_COMMAND_BARRIER}
\item \cenum{CL_COMMAND_MIGRATE_MEM_OBJECTS}
\item \cenum{CL_COMMAND_FILL_BUFFER}
\item \cenum{CL_COMMAND_FILL_IMAGE}
\stopigBase
}

\clETD{CL_EVENT_COMMAND_EXECUTION_STATUS}{cl_int}{
返回 \carg{event} 所標識的\cnglo{cmd}的執行狀態
\footnote{錯誤碼的值是負的,事件狀態的值是正的。
事件狀態的值這樣變化:從第一個或初始狀態,即最大值(\cenum{CL_QUEUED}),
一直到最後一個或完成的狀態,即最小值(\cenum{CL_COMPLETE} 或負整數)。
 \cenum{CL_COMPLETE} 跟 \cenum{CL_SUCCESS} 一樣。
}。
有效值為:
\startigBase
\item \cenum{CL_QUEUED}(\cnglo{cmd}已經入隊);

\item \cenum{CL_SUBMITTED}
(\cnglo{host}已經將所入隊的\cnglo{cmd}提交給了\cnglo{cmdq}所關聯的\cnglo{device});

\item \cenum{CL_RUNNING}(\cnglo{device}正在執行這個\cnglo{cmd});

\item \cenum{CL_COMPLETED}(\cnglo{cmd}已經完成);或

\item 錯誤碼,一個負整數(\cnglo{cmd}異常終止——可能由非法內存訪問所導致)。
與\cnglo{platform}或運行時 API 調用所返回的值或 \carg{errcode_ret} 的值使用同一套錯誤碼。

\stopigBase
}

\clETD{CL_EVENT_REFERENCE_COUNT}{cl_uint}{
返回 \carg{event} 的\cnglo{refcnt}
\footnote{在返回的那一刻,此引用計數就已過時。
應用中一般不太適用。提供此特性主要是為了檢測內存泄漏。}。
}
\stopETD
}
}

可以使用 \capi{clGetEventInfo} 來確定 \carg{event} 所標識的命令是否執行完畢
(即 \cenum{CL_EVENT_COMMAND_EXECUTION_STATUS} 返回 \cenum{CL_COMPLETE}),
但這不是同步點。
 \carg{event} 所關聯的\cnglo{cmd}可能會對\cnglo{memobj}做一些修改,
不保證這些修改對其它已經入隊的\cnglo{cmd}是可見的。

如果執行成功, \capi{clGetEventInfo} 會返回 \cenum{CL_SUCCESS}。
否則,返回下列錯誤碼之一:
\startigBase
\item \cenum{CL_INVALID_VALUE},如果 \carg{param_name} 無效,
或者 \carg{param_value_size} < \reftab{clGetEventInfo}中返回類型的大小,
且 \carg{param_value} 不是 \cenum{NULL}。

\item \cenum{CL_INVALID_VALUE},
如果對於 \carg{event} 而言,所要查詢的信息還不能被查詢。

\item \cenum{CL_INVALID_EVENT},如果 \carg{event} 無效。

\item \cenum{CL_OUT_OF_RESOURCES}——如果\scdevfailres。

\item \cenum{CL_OUT_OF_HOST_MEMORY}——如果\schostfailres。
\stopigBase
