\section{事件對象}

\cnglo{evtobj}可用來指代執行\cnglo{kernel}的\cnglo{cmd}
(\capi{clEnqueueNDRangeKernel}、
 \capi{clEnqueueTask}、 \capi{clEnqueueNativeKernel}),
讀、寫、映射以及拷貝\cnglo{memobj}的命令
(\capi{clEnqueue{Read | Write | Map}Buffer}、 \capi{clEnqueueUnmapMemObject}、
 \capi{clEnqueue{Read | Write}BufferRect}、 \capi{clEnqueue{Read | Write | Map}Image}、
 \capi{clEnqueueCopy{Buffer | Image}}、 \capi{clEnqueueCopyBufferRect}、
 \capi{clEnqueueCopyBufferToImage}、 \capi{clEnqueueCopyImageToBuffer}),
 \capi{clEnqueueMarkerWithWaitList}、
 \capi{clEnqueueBarrierWithWaitList}
 或用戶事件。

\cnglo{evtobj}可用來跟蹤\cnglo{cmd}的執行狀態。
那些會將\cnglo{cmd}入隊的 API 調用會創建一個新的\cnglo{evtobj},
並在參數 \carg{event} 中將其返回;
如果將\cnglo{cmd}插入隊列時出現了錯誤,則不會返回\cnglo{evtobj}。

在任意給定的時間點上,所入隊的\cnglo{cmd}的執行狀態將是下列之一:
\startigBase
\item \cenum{CL_QUEUED},這表示\cnglo{cmd}已經入隊。
這是所有事件的初始狀態(用戶事件除外)。

\item \cenum{CL_SUBMITTED},這是所有用戶事件的初始狀態。
對於其他事件,這表明\cnglo{host}已經將\cnglo{cmd}提交給了\cnglo{device}。

\item \cenum{CL_RUNNING},這表明\cnglo{device}已經開始執行\cnglo{cmd}。
\cnglo{cmd}的執行狀態要想從 \cenum{CL_SUBMITTED} 變成 \cenum{CL_RUNNING},
他所等待的所有事件必須都已經成功完成,即他們的執行狀態必須是 \cenum{CL_COMPLETE}。

\item \cenum{CL_COMPLETE},這表明\cnglo{cmd}已經成功完成。

\item 錯誤碼,錯誤碼是一個負整數,表明\cnglo{cmd}異常終止。
異常終止的原因有很多,如非法的內存存取。
\stopigBase

注意:\cnglo{cmd}的執行狀態是 \cenum{CL_COMPLETE} 或者負整數都表示已經完成。

如果\cnglo{cmd}被終止執行,那麼他所關聯的\cnglo{cmdq}、\cnglo{context}
(包括其中的其他\cnglo{cmdq})就不再可用。
這時如果 OpenCL API 調用要使用這個\cnglo{context}(和其中的\cnglo{cmdq}),
則其行為\cnglo{impdef}。
創建\cnglo{context}時用戶所註冊的回調函式可用來報告相應的錯誤資訊。

% clCreateUserEvent
函式
\startCLFUNC
cl_event clCreateUserEvent (cl_context context, cl_int *errcode_ret)
\stopCLFUNC
可用來創建用戶自己的\cnglo{evtobj}。
有了用戶事件,
\cnglo{app}就可以讓所入隊的\cnglo{cmd}先等待用戶事件完成,然後再由\cnglo{device}來執行。

\carg{context} 是 OpenCL \cnglo{context}。

\carg{errcode_ret} 會返回相應的錯誤碼。
如果 \carg{errcode_ret} 是 \cmacro{NULL},則不會返回錯誤碼。

如果成功創建了用戶\cnglo{evtobj}, \capi{clCreateUserEvent} 會將其返回,
並將 \carg{errcode_ret} 置為 \cenum{CL_SUCCESS}。
否則,返回 \cmacro{NULL},並將 \carg{errcode_ret} 置為下列錯誤碼之一:
\startigBase
\item \cenum{CL_INVALID_CONTEXT},如果 \carg{context} 無效。

\item \cenum{CL_OUT_OF_RESOURCES},如果\scdevfailres。

\item \cenum{CL_OUT_OF_HOST_MEMORY},如果\schostfailres。
\stopigBase

用戶\cnglo{evtobj}在創建後,其執行狀態缺省為 \cenum{CL_SUBMITTED}。

% clSetUserEventStatus
函式
\startCLFUNC
cl_int clSetUserEventStatus (cl_event event, cl_int execution_status)
\stopCLFUNC
可用來設置用戶\cnglo{evtobj}的執行狀態。

\carg{event} 即用 \capi{clCreateUserEvent} 創建的用戶\cnglo{evtobj}。

\carg{execution_status} 即將要設置的新的執行狀態,
可以是 \cenum{CL_COMPLETE},或者一個用來表示錯誤的負整數。
負整數會導致所有已經入隊、並且等待此事件的\cnglo{cmd}被終止。
要想改變 \carg{event} 的執行狀態, \capi{clSetUserEventStatus} 只能被調用一次。

如果執行成功, \capi{clSetUserEventStatus} 會返回 \cenum{CL_SUCCESS}。
否則,返回下列錯誤碼之一:
\startigBase
\item \cenum{CL_INVALID_EVENT},如果 \carg{event} 無效。

\item \cenum{CL_INVALID_VALUE},
如果 \carg{execution_status} 既不是 \cenum{CL_COMPLETE},也不是負整數。

\item \cenum{CL_INVALID_OPERATION},
如果之前已經調用 \capi{clSetUserEventStatus} 改變過 \carg{event} 的執行狀態。

\item \cenum{CL_OUT_OF_RESOURCES},如果\scdevfailres。

\item \cenum{CL_OUT_OF_HOST_MEMORY},如果\schostfailres。
\stopigBase

注意:
使用 \capi{clEnqueue***} 時,如果參數 \carg{event_wait_list} 中有用戶事件,
那麼對於所入隊的\cnglo{cmd}而言,
在調用會\cnglo{release} OpenCL 對象(\cnglo{evtobj})的 OpenCL API 之前,
必須保證已經用 \capi{clSetUserEventStatus} 設置過這些用戶事件的狀態;
否則其行為未定義。

例如,下列代碼序列會導致 \capi{clReleaseMemObject} 的未定義行為:
\startCLFUNC
ev1 = clCreateUserEvent(ctx, NULL);
clEnqueueWriteBuffer(cq, buf1, CL_FALSE, ...,
				1, &ev1, NULL);
clEnqueueWriteBuffer(cq, buf2, CL_FALSE,...);
clReleaseMemObject(buf2);
clSetUserEventStatus(ev1, CL_COMPLETE);
\stopCLFUNC

而下列代碼序列則可以正確工作:
\startCLFUNC
ev1 = clCreateUserEvent(ctx, NULL);
clEnqueueWriteBuffer(cq, buf1, CL_FALSE, ...,
				1, &ev1, NULL);
clEnqueueWriteBuffer(cq, buf2, CL_FALSE,...);
clSetUserEventStatus(ev1, CL_COMPLETE);
clReleaseMemObject(buf2);
\stopCLFUNC

% clWaitForEvents
函式
\startCLFUNC
cl_int clWaitForEvents (cl_uint num_events, const cl_event *event_list)
\stopCLFUNC
會使\cnglo{host}線程等待 \carg{event_list} 中的\cnglo{evtobj}所標識的\cnglo{cmd}完成。
對於一個\cnglo{cmd}而言,如果其執行狀態是 \cenum{CL_COMPLETE} 或負整數,則任務已經完成了。
 \carg{event_list} 中的事件充當同步點。

如果 \carg{event_list} 中的所有事件的執行狀態都是 \cenum{CL_COMPLETE},
則 \capi{clWaitForEvents} 會返回 \cenum{CL_SUCCESS}。
否則,返回下列錯誤碼之一:
\startigBase
\item \cenum{CL_INVALID_VALUE},
如果 \carg{num_events} 是零或者 \carg{event_list} 是 \cmacro{NULL}。

\item \cenum{CL_INVALID_CONTEXT},
如果 \carg{event_list} 中的事件分屬不同的\cnglo{context}。

\item \cenum{CL_INVALID_EVENT},
如果 \carg{event_list} 中的\cnglo{evtobj}無效。

\item \cenum{CL_EXEC_STATUS_ERROR_FOR_EVENTS_IN_WAIT_LIST},
如果 \carg{event_list} 中的任一事件的執行狀態是負整數。

\item \cenum{CL_OUT_OF_RESOURCES},如果\scdevfailres。

\item \cenum{CL_OUT_OF_HOST_MEMORY},如果\schostfailres。
\stopigBase

% clGetEventInfo
函式
\startCLFUNC
cl_int clGetEventInfo (cl_event event,
			cl_event_info param_name,
			size_t param_value_size,
			void *param_value,
			size_t *param_value_size_ret)
\stopCLFUNC
會返回\cnglo{evtobj}的相關資訊。

\carg{event} 即所要查詢的\cnglo{evtobj}。

\carg{param_name} 指定要查詢什麼資訊。
對於所支持的資訊類型以及 \carg{param_value} 中返回的資訊,請參見\reftab{clGetEventInfo}。

\carg{param_value} 指向的內存用來存儲查詢結果。
如果 \carg{param_value} 是 \cmacro{NULL},則忽略。

\carg{param_value_size} 即 \carg{param_value} 所指內存塊的大小。
其值必須 >= \reftab{clGetEventInfo}中返回型別的大小。

\carg{param_value_size_ret} 返回查詢結果的實際大小。
如果 \carg{param_value_size_ret} 是 \cmacro{NULL},則忽略。

\placetable[here,force,split][tab:clGetEventInfo]
{\capi{clGetEventInfo} 所支持的 \carg{param_names}}
{\startETD[cl_event_info][返回型別]

\clETD{CL_EVENT_COMMAND_QUEUE}{cl_command_queue}{
返回 \carg{event} 所關聯的\cnglo{cmdq}。
對於用戶\cnglo{evtobj},會返回 \cenum{NULL}。
}

\clETD{CL_EVENT_CONTEXT}{cl_context}{
返回 \carg{event} 所關聯的\cnglo{context}。
}

\clETD{CL_EVENT_COMMAND_TYPE}{cl_command_type}{
返回 \carg{event} 所關聯的\cnglo{cmd}。
可以是下列之一:
\startigBase
\item \cenum{CL_COMMAND_NDRANGE_KERNEL}
\item \cenum{CL_COMMAND_TASK}
\item \cenum{CL_COMMAND_NATIVE_KERNEL}
\item \cenum{CL_COMMAND_READ_BUFFER}
\item \cenum{CL_COMMAND_WRITE_BUFFER}
\item \cenum{CL_COMMAND_COPY_BUFFER}
\item \cenum{CL_COMMAND_READ_IMAGE}
\item \cenum{CL_COMMAND_WRITE_IMAGE}
\item \cenum{CL_COMMAND_COPY_IMAGE}
\item \cenum{CL_COMMAND_COPY_BUFFER_TO_IMAGE}
\item \cenum{CL_COMMAND_COPY_IMAGE_TO_BUFFER}
\item \cenum{CL_COMMAND_MAP_BUFFER}
\item \cenum{CL_COMMAND_MAP_IMAGE}
\item \cenum{CL_COMMAND_UNMAP_MEM_OBJECT}
\item \cenum{CL_COMMAND_MARKER}
\item \cenum{CL_COMMAND_ACQUIRE_GL_OBJECTS}
\item \cenum{CL_COMMAND_RELEASE_GL_OBJECTS}
\item \cenum{CL_COMMAND_READ_BUFFER_RECT}
\item \cenum{CL_COMMAND_WRITE_BUFFER_RECT}
\item \cenum{CL_COMMAND_COPY_BUFFER_RECT}
\item \cenum{CL_COMMAND_USER}
\item \cenum{CL_COMMAND_BARRIER}
\item \cenum{CL_COMMAND_MIGRATE_MEM_OBJECTS}
\item \cenum{CL_COMMAND_FILL_BUFFER}
\item \cenum{CL_COMMAND_FILL_IMAGE}
\stopigBase
}

\clETD{CL_EVENT_COMMAND_EXECUTION_STATUS}{cl_int}{
返回 \carg{event} 所標識的\cnglo{cmd}的執行狀態
\footnote{錯誤碼的值是負的,事件狀態的值是正的。
事件狀態的值這樣變化:從第一個或初始狀態,即最大值(\cenum{CL_QUEUED}),
一直到最後一個或完成的狀態,即最小值(\cenum{CL_COMPLETE} 或負整數)。
 \cenum{CL_COMPLETE} 跟 \cenum{CL_SUCCESS} 一樣。
}。
有效值為:
\startigBase
\item \cenum{CL_QUEUED}(\cnglo{cmd}已經入隊);

\item \cenum{CL_SUBMITTED}
(\cnglo{host}已經將所入隊的\cnglo{cmd}提交給了\cnglo{cmdq}所關聯的\cnglo{device});

\item \cenum{CL_RUNNING}(\cnglo{device}正在執行這個\cnglo{cmd});

\item \cenum{CL_COMPLETED}(\cnglo{cmd}已經完成);或

\item 錯誤碼,一個負整數(\cnglo{cmd}異常終止——可能由非法內存訪問所導致)。
與\cnglo{platform}或運行時 API 調用所返回的值或 \carg{errcode_ret} 的值使用同一套錯誤碼。

\stopigBase
}

\clETD{CL_EVENT_REFERENCE_COUNT}{cl_uint}{
返回 \carg{event} 的\cnglo{refcnt}
\footnote{在返回的那一刻,此引用計數就已過時。
應用中一般不太適用。提供此特性主要是為了檢測內存泄漏。}。
}
\stopETD
}

可以使用 \capi{clGetEventInfo} 來確定 \carg{event} 所標識的命令是否執行完畢
(即 \cenum{CL_EVENT_COMMAND_EXECUTION_STATUS} 返回 \cenum{CL_COMPLETE}),
但這不是同步點。
 \carg{event} 所關聯的\cnglo{cmd}可能會對\cnglo{memobj}做一些修改,
不保證這些修改對其他已經入隊的\cnglo{cmd}是可見的。

如果執行成功, \capi{clGetEventInfo} 會返回 \cenum{CL_SUCCESS}。
否則,返回下列錯誤碼之一:
\startigBase
\item \cenum{CL_INVALID_VALUE},如果 \carg{param_name} 無效,
或者 \carg{param_value_size} < \reftab{clGetEventInfo}中返回型別的大小,
且 \carg{param_value} 不是 \cmacro{NULL}。

\item \cenum{CL_INVALID_VALUE},
如果對於 \carg{event} 而言,所要查詢的資訊還不能被查詢。

\item \cenum{CL_INVALID_EVENT},如果 \carg{event} 無效。

\item \cenum{CL_OUT_OF_RESOURCES}——如果\scdevfailres。

\item \cenum{CL_OUT_OF_HOST_MEMORY}——如果\schostfailres。
\stopigBase

% clSetEventCallback
函式
\startCLFUNC
cl_int clSetEventCallback (cl_event event,
			cl_int command_exec_callback_type,
			void (CL_CALLBACK *pfn_event_notify)(cl_event event,
					cl_int event_command_exec_status,
					void *user_data),
			void *user_data)
\stopCLFUNC
可以為\cnglo{cmd}的某個執行狀態註冊一個用戶回調函式。
當 \carg{event} 所關聯的\cnglo{cmd}的執行狀態變成或越過 \carg{command_exec_status} 時,
所註冊的回調函式就會被調用。

調用 \capi{clSetEventCallback} 會將用戶回調函式註冊到 \carg{event} 的回調棧上。
而對於所註冊用戶回調函式的調用順序,則沒有定義。

\carg{event} 是一個\cnglo{evtobj}。

\carg{command_exec_callback_type} 指定\cnglo{cmd}的執行狀態,回調就註冊到此狀態上。
能註冊回調的狀態為:
 \cenum{CL_SUBMITTED}、 \cenum{CL_RUNNING} 或 \cenum{CL_COMPLETE}
\footnote{如果 \carg{command_exec_callback} 是 \cenum{CL_COMPLETE},
當命令執行成功或者異常終止時都會調用所註冊的回調函式。}。
不保證按執行狀態的變化順序來調用相應的回調函式。
進而,需要注意的是,調用回調時,事件的狀態可能不是 \cenum{CL_COMPLETE},
但這絕不意味着 OpenCL 規範所定義的內存模型或執行模型發生了變化。
例如,除非事件的狀態是 \cenum{CL_COMPLETE},否則不能假設相應的內存遷移已經完成。

\carg{pfn_event_notify} 即\cnglo{app}所註冊的事件回調函式。
此函式可能會被 OpenCL 實作異步調用。
\cnglo{app}要保證此函式是線程安全的。
此函式的參數為:
\startigBase
\item \carg{event} 即此函式所關注的事件。

\item \carg{event_command_exec_status} 即此函式所關注的\cnglo{cmd}執行狀態。
關於\cnglo{cmd}的執行狀態請參見\reftab{clGetEventInfo}。
如果是由於\cnglo{cmd}異常終止而調用的此函式,
那麼傳給 \carg{event_command_exec_status} 的將是相應的錯誤碼。

\item \carg{user_data} 指向用戶提供的數據。
\stopigBase

\carg{user_data} 將在調用 \carg{pfn_event_notify} 時作為其參數 \carg{user_data} 傳入。
 \carg{user_data} 可以是 \cmacro{NULL}。

註冊到一個\cnglo{evtobj}上所有回調都必須被調用,而且是在銷毀\cnglo{evtobj}之前被調用。
回調必須儘快返回。
如果在回調中調用了一些代價高昂的系統例程,
如創建\cnglo{context}或\cnglo{cmdq}的 OpenCL API,或者下列會阻塞的 OpenCL 操作,
則其行為是未定義的。
\startigBase
\item \capi{clFinish}

\item \capi{clWaitForEvents}

\item 對 \capi{clEnqueueReadBuffer}、 \capi{clEnqueueReadBufferRect}、
 \capi{clEnqueueWriteBuffer}、 \capi{clEnqueueWriteBufferRect} 的阻塞式調用

\item 對 \capi{clEnqueueReadImage}、 \capi{clEnqueueWriteImage} 的阻塞式調用

\item 對 \capi{clEnqueueMapBuffer}、 \capi{clEnqueueMapImage} 的阻塞式調用

\item 對 \capi{clBuildProgram}、 \capi{clCompileProgram}、 \capi{clLinkProgram} 的阻塞式調用
\stopigBase

如果\cnglo{app}想等待上述例程的完成,請使用其非阻塞的形式,並設置一個回調來完成剩餘的工作。
注意,如果回調(或者其他代碼)入隊了\cnglo{cmd},
可能在刷新隊列時才開始執行這些\cnglo{cmd}。
在標準用法中,阻塞的入隊調用通過顯式的刷新隊列來達到此效果。
由於回調中不運行有阻塞的調用,
那些會入隊\cnglo{cmd}的回調應該在返回前調用 \capi{clFlush}
 或者將 \capi{clFlush} 安排在另一個線程中晚些時候再調用。

如果執行成功, \capi{clSetEventCallback} 會返回 \cenum{CL_SUCCESS}。
否則,返回下列錯誤碼之一:
\startigBase
\item \cenum{CL_INVALID_EVENT},如果 \carg{event} 無效。

\item \cenum{CL_INVALID_VALUE},如果 \carg{pfn_event_notify} 是 \cmacro{NULL}
 或者 \carg{command_exec_callback_type} 不是 \cenum{CL_COMPLETE}。

\item \cenum{CL_OUT_OF_RESOURCES}——如果\scdevfailres。

\item \cenum{CL_OUT_OF_HOST_MEMORY}——如果\schostfailres。
\stopigBase

% clRetainEvent
函式
\startCLFUNC
cl_int clRetainEvent (cl_event event)
\stopCLFUNC
會使 \carg{memobj} 的\cnglo{refcnt}增一。
那些會返回事件的 OpenCL \cnglo{cmd}會實施隱式的\cnglo{retain}。

如果執行成功, \capi{clRetainEvent} 會返回 \cenum{CL_SUCCESS}。
否則,返回下列錯誤的一種:
\startigBase
\item \cenum{CL_INVALID_EVENT},如果 \carg{event} 無效。

\item \cenum{CL_OUT_OF_RESOURCES}——如果\scdevfailres。

\item \cenum{CL_OUT_OF_HOST_MEMORY}——如果\schostfailres。
\stopigBase

% clReleaseEvent
函式
\startCLFUNC
cl_int clReleaseEvent (cl_event event)
\stopCLFUNC
會使 \carg{event} 的\cnglo{refcnt}減一。

如果執行成功, \capi{clReleaseEvent} 會返回 \cenum{CL_SUCCESS}。
否則,返回下列錯誤的一種:
\startigBase
\item \cenum{CL_INVALID_EVENT},如果 \carg{event} 無效。

\item \cenum{CL_OUT_OF_RESOURCES}——如果\scdevfailres。

\item \cenum{CL_OUT_OF_HOST_MEMORY}——如果\schostfailres。
\stopigBase

一旦 \carg{event} 的\cnglo{refcnt}變成零,
並且他所標識的\cnglo{cmd}執行完畢(或終止),
同時沒有正在等待他完成的\cnglo{cmd},
則此事件就會被刪除。

注意:

對於用 \capi{clCreateUserEvent} 創建的事件而言,
如果還未將其狀態置為 \cenum{CL_COMPLETE} 或者一個錯誤碼,
這時要\cnglo{release}他的最後一個\cnglo{refcnt},開發人員就要小心了。
如果用戶事件作為參數 \carg{event_wait_list} 傳給了 \capi{clEnqueue***},
或者另一個\cnglo{host}線程正在用 \capi{clWaitForEvents} 等待他,
那麼即使用戶\cnglo{release}了此對象,
這些\cnglo{cmd}和\cnglo{host}線程也會等待其狀態變為 \cenum{CL_COMPLETE} 或錯誤。
這種情況下,如果開發人員\cnglo{release}了用戶事件的最後一個\cnglo{refcnt},
理論上他就不能改變事件的狀態以解除其他部分的阻塞。
結果就是正在等待的任務會永遠等下去,
相關的事件、 \ctype{cl_mem} 對象、\cnglo{cmdq}和\cnglo{context}多半就泄漏了。
順序執行的\cnglo{cmdq}碰到這種死鎖就會停止做任何事情。

