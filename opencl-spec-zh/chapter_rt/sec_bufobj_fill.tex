\subsection{填充緩衝對象}

函數
\startclc
cl_int clEnqueueFillBuffer (cl_command_queue command_queue,
			cl_mem buffer,
			const void *pattern,
			size_t pattern_size,
			size_t offset,
			size_t size,
			cl_uint num_events_in_wait_list,
			const cl_event *event_wait_list,
			cl_event *event)
\stopclc
會將一個\cnglo{cmd}插入隊列,此\cnglo{cmd}可以按給定的模式( pattern )填充\cnglo{bufobj}。
\carg{buffer} 的一些使用信息,如此\cnglo{memobj}是否可讀、是否可寫,
還有創建 \carg{buffer} 時所指定的參數 \ctype{cl_mem_flags},
都會被 \capi{clEnqueueFillBuffer} 忽略。

\carg{command_queue} 即此填充\cnglo{cmd}要插入的隊列。
\carg{command_queue} 和 \carg{buffer} 必須位於用一個 OpenCL \cnglo{context}中。

\carg{buffer} 是一個\cnglo{bufobj}。

\carg{pattern} 指向數據模式( data pattern ),其大小為 \carg{pattern_size}。
用 \carg{pattern} 填充的區域在 \carg{buffer} 中的偏移量為 \carg{offset},大小是 \carg{size}。
數據模式必須是 OpenCL 所支持的標量或矢量的整形或浮點類型,參見\todo{section 6.1.1 and 6.1.2}。
例如,如果數據模式是 \ctype{float4},則 \carg{pattern} 指向一個 \ctype{cl_float4} 類型的值,
且 \carg{pattern_size} 是 \ctype{sizeof(cl_float4)}。
\carg{pattern_size} 的最大值就是 OpenCL \cnglo{device}所支持的最大的整形或浮點類型的矢量數據類型的大小。
函數返回後,就可以重用或者釋放 \carg{pattern} 所指向的內存了。

