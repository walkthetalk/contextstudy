\subsection{訪問所映射的區域}

本節將介紹 OpenCL \cnglo{cmd}是如何訪問所映射區域的。

如果某\cnglo{memobj}的某區域映射為可寫
(即 \capi{clEnqueueMap{Buffer | Image}} 的參數 \carg{map_flags} 中
設置了 \cenum{CL_MAP_WRITE} 或 \cenum{CL_MAP_WIRTE_INVALIDATE_REGION} ),
在將其解映射之前,
對於此\cnglo{memobj}以及與其相關的\cnglo{memobj}
(與此區域有重疊的子\cnglo{bufobj}或 1D 圖像\cnglo{bufobj})而言,
此區域的內容是\cnglo{undef}的。

對於一個\cnglo{memobj}以及與其相關的\cnglo{memobj}
(與此區域有重疊的子\cnglo{bufobj}或 1D 圖像\cnglo{bufobj})而言,
\cnglo{cmdq}中的多個\cnglo{cmd}都可以將其中某個區域或重疊區域映射為可讀
(即 \carg{mem_flags} = \cenum{CL_MAP_READ})。
對於\cnglo{memobj}中映射為可讀的區域,
\cnglo{kernel}以及其他 OpenCL \cnglo{cmd}(像 \capi{clEnqueueCopyBuffer})都可以讀取其內容。

對於一個\cnglo{memobj}以及與其相關的\cnglo{memobj}
(與此區域有重疊的子\cnglo{bufobj}或 1D 圖像\cnglo{bufobj})而言,
如果映射(或解映射)的區域有重疊,則認為是錯誤,
會導致 \capi{clEnqueueMap{Buffer | Image}} 返回 \cenum{CL_INVALID_OPERATION}。

如果\cnglo{memobj}映射為可寫,
則\cnglo{app}要保證在其他會讀寫其內容的\cnglo{kernel}或\cnglo{cmd}開始執行前將其解映射,
否則其行為\cnglo{undef}。
這些行為包括讀寫此\cnglo{memobj},
讀寫與其相關的\cnglo{memobj}(子\cnglo{bufobj}或 1D 圖像\cnglo{bufobj}),
讀寫其父對象(如果此\cnglo{memobj}本身是子\cnglo{bufobj}或 1D 圖像\cnglo{bufobj})。

而如果\cnglo{memobj}映射為可讀,
則要保證其他會改寫其內容的\cnglo{kernel}或\cnglo{cmd}開始執行前將其解映射,
否則其行為\cnglo{undef}。
這些行為包括對此\cnglo{memobj}的寫操作,
對與其相關的\cnglo{memobj}(子\cnglo{bufobj}或 1D 圖像\cnglo{bufobj})的寫操作,
對其父對象(如果此\cnglo{memobj}本身是子\cnglo{bufobj}或 1D 圖像\cnglo{bufobj})的寫操作。

在解映射後,如果繼續使用之前映射後的指針來訪問所映射的區域,其行為\cnglo{undef}。

在遵守上面所列規則的前提下,
可以將 \capi{clEnqueueMap{Buffer | Image}} 所返回的指針作為引數 \carg{ptr} 調用
\capi{clEnqueue{Read | Write}Buffer}、\capi{clEnqeue{Read | Write}BufferRect}
或 \capi{clEnqueue{Read | Write}Image}。
