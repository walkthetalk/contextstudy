\subsection{內存對象的遷移}

本節所描述的機制可用來將\cnglo{memobj}指派給某個\cnglo{device}。
在創建\cnglo{memobj}時,用戶可能希望顯式的控制其位置。
此機制可用來:
\startigBase
\item 在使用前,保證在某個特定\cnglo{device}上分配對象。

\item 將\cnglo{memobj}從一個\cnglo{device}遷移到另一個上面。
\stopigBase

\topclfunc{clEnqueueMigrateMemObjects}

\startCLFUNC
cl_int clEnqueueMigrateMemObjects(
			cl_command_queue command_queue,
			cl_uint num_mem_objects,
			const cl_mem *mem_objects,
			cl_mem_migration_flags flags,
			cl_uint num_events_in_wait_list,
			const cl_event *event_wait_list,
			cl_event *event)
\stopCLFUNC

此函式所入隊的\cnglo{cmd}可以將一組\cnglo{memobj}強制關聯到某個\cnglo{device}上。
典型的,在\cnglo{device}上執行\cnglo{cmd}時,
會將所使用的\cnglo{memobj}隱式的遷移到此\cnglo{device}上。
\capi{clEnqueueMigrateMemObjects} 使得可以在相應\cnglo{cmd}執行前實施顯式的遷移。
這使得用戶可以通過\cnglo{cmdq}的正規調度,提前改變\cnglo{memobj}的關聯,為後續命令做準備,
同時可以避免隱式遷移所帶來的延遲。
一旦 \capi{clEnqueueMigrateMemObjects} 所返回的事件被標註為 \cenum{CL_COMPLETE},
\carg{mem_objects} 中的所有\cnglo{memobj}
就都成功遷移到了 \carg{command_queue} 所在\cnglo{device}上。
這些對象會一直在這個\cnglo{device}上,直到另一\cnglo{cmd}顯式或隱式地將其再次遷移。

\capi{clEnqueueMigrateMemObjects} 也可以用來控制\cnglo{memobj}在創建後的初始位置,
這樣可以避免第一條\cnglo{cmd}將其實體化時可能的開銷。

為了避免對\cnglo{memobj}的存取相互衝突,
與此\cnglo{cmd}相關的所有事件間的依賴關係由用戶負責。
如果依賴關係有問題,則 \capi{clEnqueueMigrateMemObjects} 會導致\cnglo{undef}的結果。

\carg{command_queue} 是一個\cnglo{cmdq}。
\carg{mem_objects} 中的\cnglo{memobj}會被遷移到 \carg{command_queue} 所在的\cnglo{device}上。
如果設置了 \cenum{CL_MIGRATE_MEM_OBJECT_HOST},則會遷移到\cnglo{host}上。

\carg{num_mem_objects} 即 \carg{mem_objects} 中\cnglo{memobj}的數目。

\carg{mem_objects} 指向一組\cnglo{memobj}。

\carg{flags} 是位欄,用來指定遷移選項。\reftab{clMemMigration}中列出了可能的值。

\placetable[here][tab:clMemMigration]
{\ctype{cl_mem_migration} 的值}
{\startED[\ctype{cl_mem_migration}]

\clED{CL_MIGRATE_MEM_OBJECT_HOST}{
  表明要將\cnglo{memobj}遷移到\cnglo{host}上,無視\cnglo{cmdq}。
}

\clED{CL_MIGRATE_MEM_OBJECT_CONTENT_UNDEFINED}{
  表明遷移後,\cnglo{memobj}的內容是未定義的。
  遷移時不會招致遷移內容所帶來的開銷。
}

\stopED
}

\carg{event_wait_list} 和 \carg{num_events_in_wait_list} 中
列出了執行此\cnglo{cmd}前要等待的事件。
如果 \carg{event_wait_list} 是 \cmacro{NULL},
則無須等待任何事件,並且 \carg{num_events_in_wait_list} 必須是0。
如果 \carg{event_wait_list} 不是 \cmacro{NULL},
則其中所有事件都必須是有效的,並且 \carg{num_events_in_wait_list} 必須大於 0。
\carg{event_wait_list} 中的事件充當同步點,
並且必須與 \carg{command_queue} 位於同一個\cnglo{context}中。
此函式返回後,即可回收並重用 \carg{event_wait_list} 所關聯的內存。

\carg{event} 會返回一個\cnglo{evtobj},
用來識別此\cnglo{cmd},可用來查詢或等待此\cnglo{cmd}完成。
而如果 \carg{event} 是 \cmacro{NULL},就沒辦法查詢此\cnglo{cmd}的狀態或等待其完成了。
不過可以用 \capi{clEnqueueBarrierWithWaitList} 來代替。
如果 \carg{event_wait_list} 和 \carg{event} 都不是 \cmacro{NULL},
\carg{event} 不能屬於 \carg{event_wait_list}。

如果執行成功, \capi{clEnqueueMigrateMemObjects} 會返回 \cenum{CL_SUCCESS}。
否則,返回下列錯誤碼之一:
\startigBase
\item \cenum{CL_INVALID_COMMAND_QUEUE},如果 \carg{command_queue} 無效。

\item \cenum{CL_INVALID_CONTEXT},
如果 \carg{command_queue} 和 \carg{mem_objects} 位於不同的\cnglo{context}中,
或者 \carg{command_queue} 和 \carg{event_wait_list} 中的事件位於不同的\cnglo{context}中。

\item \cenum{CL_INVALID_MEM_OBJECT},
如果 \carg{mem_objects} 中的任一\cnglo{memobj}無效。

\item \cenum{CL_INVALID_VALUE},
如果 \carg{num_mem_objects} 是零或者 \carg{mem_objects} 是 \cmacro{NULL}。

\item \cenum{CL_INVALID_VALUE},如果 \carg{flags} 的值無效。

\startitem
\cenum{CL_INVALID_EVENT_WAIT_LIST},如果滿足下列條件中的任一項:
\startigBase
\item \carg{event_wait_list} 是 \cmacro{NULL},但 \carg{num_events_in_wait_list} > 0;
\item 或者 \carg{event_wait_list} 不是 \cmacro{NULL},但 \carg{num_events_in_wait_list} 是 0;
\item 或者 \carg{event_wait_list} 中有無效的事件。
\stopigBase
\stopitem

\item \cenum{CL_MEM_OBJECT_ALLOCATION_FAILURE},
如果為 \carg{mem_objects} 中的\cnglo{memobj}分配內存失敗。

\item \cenum{CL_OUT_OF_RESOURCES},如果\scdevfailres。

\item \cenum{CL_OUT_OF_HOST_MEMORY},如果\schostfailres。
\stopigBase
