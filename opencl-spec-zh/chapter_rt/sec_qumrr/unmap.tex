\subsection[sec:unmap]{內存對象的解映射}

\topclfunc{clEnqueueUnmapMemObject}

\startCLFUNC
cl_int clEnqueueUnmapMemObject (cl_command_queue command_queue,
			cl_mem memobj,
			void *mapped_ptr,
			cl_uint num_events_in_wait_list,
			const cl_event *event_wait_list,
			cl_event *event)
\stopCLFUNC

此函式所入隊的\cnglo{cmd}可以將\cnglo{memobj}之前映射的區域解映射。
假定使用 \capi{clEnqueueMapBuffer} 或 \capi{clEnqueueMapImage} 所返回的指針
對 \carg{memobj} 所進行的讀寫已經完畢。

\carg{command_queue} 就是解映射\cnglo{cmd}要進入的隊列。

\carg{memobj} 是一個\cnglo{memobj},
他必須與 \carg{command_queue} 位於同一 OpenCL \cnglo{context}中。

\carg{mapped_ptr} 即之前調用 \capi{clEnqueueMapBuffer}
或 \capi{clEnqueueMapImage} 所返回的\cnglo{host}位址。

\carg{event_wait_list} 和 \carg{num_events_in_wait_list} 中
列出了執行此\cnglo{cmd}前要等待的事件。
如果 \carg{event_wait_list} 是 \cmacro{NULL},
則無須等待任何事件,並且 \carg{num_events_in_wait_list} 必須是0。
如果 \carg{event_wait_list} 不是 \cmacro{NULL},
則其中所有事件都必須是有效的,並且 \carg{num_events_in_wait_list} 必須大於 0。
\carg{event_wait_list} 中的事件充當同步點,
並且必須與 \carg{command_queue} 位於同一個\cnglo{context}中。
此函式返回後,即可回收並重用 \carg{event_wait_list} 所關聯的內存。

\carg{event} 會返回一個\cnglo{evtobj},
用來標識此\cnglo{cmd},可用來查詢或等待此\cnglo{cmd}完成。
而如果 \carg{event} 是 \cmacro{NULL},就沒辦法查詢此\cnglo{cmd}的狀態或等待其完成了。
不過可以用 \capi{clEnqueueBarrierWithWaitList} 來代替。
如果 \carg{event_wait_list} 和 \carg{event} 都不是 \cmacro{NULL},
\carg{event} 不能屬於 \carg{event_wait_list}。

如果執行成功, \capi{clEnqueueUnmapMemObject} 會返回 \cenum{CL_SUCCESS}。
否則,返回下列錯誤碼之一:
\startigBase
\item \cenum{CL_INVALID_COMMAND_QUEUE},如果 \carg{command_queue} 無效。
\item \cenum{CL_INVALID_MEM_OBJECT},如果 \carg{memobj} 無效。

\item \cenum{CL_INVALID_VALUE},如果 \carg{mapped_ptr} 無效。

\startitem
\cenum{CL_INVALID_EVENT_WAIT_LIST},如果滿足下列條件中的任一項:
\startigBase
\item \carg{event_wait_list} 是 \cmacro{NULL},但 \carg{num_events_in_wait_list} > 0;
\item 或者 \carg{event_wait_list} 不是 \cmacro{NULL},但 \carg{num_events_in_wait_list} 是 0;
\item 或者 \carg{event_wait_list} 中有無效的事件。
\stopigBase
\stopitem

\item \cenum{CL_OUT_OF_RESOURCES},如果\scdevfailres。

\item \cenum{CL_OUT_OF_HOST_MEMORY},如果\schostfailres。

\item \cenum{CL_INVALID_CONTEXT},
如果 \carg{command_queue} 和 \carg{memobj} 位於不同的\cnglo{context}中,
或者 \carg{command_queue} 和 \carg{event_wait_list} 中的事件位於不同的\cnglo{context}中。
\stopigBase

\capi{clEnqueueMapBuffer} 和 \capi{clEnqueueMapImage} 會使\cnglo{memobj}的映射計數增一。
映射計數的初始值是零。
對同一\cnglo{memobj}多次調用 \capi{clEnqueueMapBuffer} 或 \capi{clEnqueueMapImage},
會使其映射計數增加相應的數目。
\capi{clEnqueueUnmapMemObject} 會使\cnglo{memobj}的映射計數減一。

對於\cnglo{bufobj}中被映射的那部分而言,
\capi{clEnqueueMapBuffer} 和 \capi{clEnqueueMapImage} 充當同步點。
