\subsection{保留以及釋放內存對象}

函數
\startclc
cl_int clRetainMemObject (cl_mem memobj)
\stopclc
會使 \carg{memobj} 的\cnglo{refcnt}增一。執行成功後會返回 \cenum{CL_SUCCESS}。
否則,返回下列錯誤的一種:
\startigBase
\item \cenum{CL_INVALID_MEM_OBJECT},如果 \carg{memobj} 無效。

\item \cenum{CL_OUT_OF_RESOURCES}——如果\scdevfailres。

\item \cenum{CL_OUT_OF_HOST_MEMORY}——如果\schostfailres。
\stopigBase

\capi{clCreateBuffer}、 \capi{clCreateSubBuffer} 以及 \capi{clCreateImage} 會實施隱式的\cnglo{retain}。

函數
\startclc
cl_int clReleaseMemObject (cl_mem memobj)
\stopclc
會使 \carg{memobj} 的\cnglo{refcnt}減一。執行成功後會返回 \cenum{CL_SUCCESS}。
否則,返回下列錯誤的一種:
\startigBase
\item \cenum{CL_INVALID_MEM_OBJECT},如果 \carg{memobj} 無效。

\item \cenum{CL_OUT_OF_RESOURCES}——如果\scdevfailres。

\item \cenum{CL_OUT_OF_HOST_MEMORY}——如果\schostfailres。
\stopigBase

一旦 \carg{memobj} 的\cnglo{refcnt}變成零,並且所有使用它的\cnglo{cmd}都執行完畢,此\cnglo{memobj}就會被刪除。
如果 \carg{memobj} 是一個\cnglo{bufobj},只有等到其所有子\cnglo{bufobj}都被刪除後,才能將其刪除。

函數
\startclc
cl_int clSetMemObjectDestructorCallback (cl_mem memobj,
		void (CL_CALLBACK *pfn_notify)(cl_mem memobj,
			void *user_data),
		void *user_data)
\stopclc
會向 \carg{memobj} 註冊一個回調函數。
每次調用 \capi{clSetMemObjectDestructorCallback} 都會將回調函數註冊到 \carg{memobj} 的回調棧上。
回調函數的調用順序與其註冊順序相反。
對於一個\cnglo{memobj}而言,這些回調函數是在其資源被釋放以及將其刪除之前調用的。
這樣就提供了一種機制,
當 \carg{host_ptr} (創建 \carg{memobj} 時指定)所引用的內存(用來存儲\cnglo{memobj}的內容)可以重新使用或是被釋放時,
正在使用 \carg{memobj} 的\cnglo{app}(以及庫)可以收到通知。

