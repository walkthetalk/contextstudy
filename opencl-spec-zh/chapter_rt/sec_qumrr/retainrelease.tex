\subsection{保留以及釋放內存對象}

函式
\startCLFUNC
cl_int clRetainMemObject (cl_mem memobj)
\stopCLFUNC
會使 \carg{memobj} 的\cnglo{refcnt}增一。執行成功後會返回 \cenum{CL_SUCCESS}。
否則,返回下列錯誤的一種:
\startigBase
\item \cenum{CL_INVALID_MEM_OBJECT},如果 \carg{memobj} 無效。

\item \cenum{CL_OUT_OF_RESOURCES}——如果\scdevfailres。

\item \cenum{CL_OUT_OF_HOST_MEMORY}——如果\schostfailres。
\stopigBase

\capi{clCreateBuffer}、 \capi{clCreateSubBuffer} 以及 \capi{clCreateImage} 會實施隱式的\cnglo{retain}。

函式
\startCLFUNC
cl_int clReleaseMemObject (cl_mem memobj)
\stopCLFUNC
會使 \carg{memobj} 的\cnglo{refcnt}減一。執行成功後會返回 \cenum{CL_SUCCESS}。
否則,返回下列錯誤的一種:
\startigBase
\item \cenum{CL_INVALID_MEM_OBJECT},如果 \carg{memobj} 無效。

\item \cenum{CL_OUT_OF_RESOURCES}——如果\scdevfailres。

\item \cenum{CL_OUT_OF_HOST_MEMORY}——如果\schostfailres。
\stopigBase

一旦 \carg{memobj} 的\cnglo{refcnt}變成零,並且所有使用他的\cnglo{cmd}都執行完畢,此\cnglo{memobj}就會被刪除。
如果 \carg{memobj} 是一個\cnglo{bufobj},只有等到其所有子\cnglo{bufobj}都被刪除後,才能將其刪除。

函式
\startCLFUNC
cl_int clSetMemObjectDestructorCallback (cl_mem memobj,
		void (CL_CALLBACK *pfn_notify)(cl_mem memobj,
			void *user_data),
		void *user_data)
\stopCLFUNC
會向 \carg{memobj} 註冊一個回調函式。
每次調用 \capi{clSetMemObjectDestructorCallback} 都會將回調函式註冊到 \carg{memobj} 的回調棧上。
回調函式的調用順序與其註冊順序相反。
對於一個\cnglo{memobj}而言,這些回調函式是在其資源被釋放以及將其刪除之前調用的。
這樣就提供了一種機制,
當 \carg{host_ptr} (創建 \carg{memobj} 時指定)所引用的內存(用來存儲\cnglo{memobj}的內容)可以重新使用或是被釋放時,
正在使用 \carg{memobj} 的\cnglo{app}(以及庫)可以收到通知。

\carg{memobj} 是一個\cnglo{memobj}。

\carg{pfn_notify} 即\cnglo{app}所註冊的回調函式。此函式可能會被 OpenCL 實作異步調用。
\cnglo{app}需要保證此函式是線程安全的。
此函式的參數有:
\startigBase
\item \carg{memobj} 就是要被刪除的\cnglo{memobj}。當此函式被調用時,這個\cnglo{memobj}就已經無效了。此處提供他只是出於參考的目的。

\item \carg{user_data} 指向用戶提供的數據。
\stopigBase

\carg{user_data} 將在調用 \carg{pfn_notify} 時作為參數 \carg{user_data} 傳入。
 \carg{user_data} 可以是 \cmacro{NULL}。

如果執行成功, \capi{clSetMemObjectDestructorCallback} 會返回 \cenum{CL_SUCCESS}。
否則返回下列錯誤碼之一:
\startigBase
\item \cenum{CL_INVALID_MEM_OBJECT},如果 \carg{memobj} 無效。

\item \cenum{CL_INVALID_VALUE},如果 \carg{pfn_notify} 是 \cmacro{NULL}。

\item \cenum{CL_OUT_OF_RESOURCES}——如果\scdevfailres。

\item \cenum{CL_OUT_OF_HOST_MEMORY}——如果\schostfailres。
\stopigBase

注意:

當 OpenCL 實作調用回調函式時, \carg{host_ptr} (如果是用 \cenum{CL_MEM_USE_HOST_PTR} 創建的\cnglo{memobj})所指向的內容是未定義的。
此回調函式的一種典型應用就是用來釋放或重新使用 \carg{host_ptr} 所指向的內存。

如果在回調函式中調用了下列 OpenCL API,其行為是未定義的:
\startigBase
\item \capi{clFinish}
\item \capi{clWaitForEvents}
\item 對 \capi{clEnqueueReadBuffer}、 \capi{clEnqueueReadBufferRect}、 \capi{clEnqueueWriteBuffer}、 \capi{clEnqueueWriteBufferRect} 的阻塞式調用
\item 對 \capi{clEnqueueReadImage}、 \capi{clEnqueueWriteImage} 的阻塞式調用
\item 對 \capi{clEnqueueMapBuffer}、 \capi{clEnqueueMapImage} 的阻塞式調用
\item 對 \capi{clBuildProgram}、 \capi{clCompileProgram}、 \capi{clLinkProgram} 的阻塞式調用
\stopigBase

如果\cnglo{app}需要在回調函式中等待上面某一例程執行完畢,請使用相應例程的非阻塞方式,並設置回調來完成剩下的工作。
需要注意的是,如果回調(或其他代碼)在\cnglo{cmdq}中插入了\cnglo{cmd},在隊列刷新(flush)前不要求這些\cnglo{cmd}已經開始執行。
在標準用法中,阻塞式的入隊調用通過顯式的刷新隊列來達到目的。
由於回調中不允許阻塞式的調用,對於那些會在隊列中插入\cnglo{cmd}的回調而言,需要在返回前調用 \capi{clFlush} 或者將其安排在其他線程中隨後調用。

回調函式不能用參數 \carg{memobj} 調用其他 OpenCL API,否則其行為未定義。

