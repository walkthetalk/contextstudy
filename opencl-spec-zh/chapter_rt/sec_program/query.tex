\subsection{程式對象相關的查詢}

\topclfunc{clGetProgramInfo}

\startCLFUNC
cl_int clGetProgramInfo (cl_program program,
			cl_program_info param_name,
			size_t param_value_size,
			void *param_value,
			size_t *param_value_size_ret)
\stopCLFUNC

此函式會返回\cnglo{programobj}的相關資訊。

\carg{program} 即所要查詢的\cnglo{programobj}。

\carg{param_name} 指定要查詢什麼資訊。
對於所支持的資訊類型以及 \carg{param_value} 中返回的資訊,請參見\reftab{clGetProgramInfo}。

\carg{param_value} 指向的內存用來存儲查詢結果。
如果 \carg{param_value} 是 \cmacro{NULL},則忽略。

\carg{param_value_size} 即 \carg{param_value} 所指內存塊的大小。
其值必須 >= \reftab{clGetProgramInfo}中返回型別的大小。

\carg{param_value_size_ret} 返回查詢結果的實際大小。
如果 \carg{param_value_size_ret} 是 \cmacro{NULL},則忽略。

\placetable[here,split][tab:clGetProgramInfo]
{\capi{clGetProgramInfo} 所支持的 \carg{param_names}}
{\startETD[cl_program_info][返回型別]

\clETD{CL_PROGRAM_REFERENCE_COUNT}{cl_uint}{
返回 \carg{program} 的\cnglo{refcnt}。
\footnote{在返回的那一刻,此引用計數就已過時。應用中一般不太適用。提供此特性主要是為了檢測內存泄漏。}
}

\clETD{CL_PROGRAM_CONTEXT}{cl_context}{
返回創建\cnglo{programobj}時所指定的\cnglo{context}。
}

\clETD{CL_PROGRAM_NUM_DEVICES}{cl_uint}{
返回 \carg{program} 所關聯\cnglo{device}的數目。
}

\clETD{CL_PROGRAM_DEVICES}{cl_device_id[]}{
返回 \carg{program} 所關聯的\cnglo{device}。
可以是創建\cnglo{programobj}時所用\cnglo{context}中的\cnglo{device};
也可以是用 \capi{clCreateProgramWithBinary} 創建\cnglo{programobj}時所指定的\cnglo{device}中的一個子集。
}

\clETD{CL_PROGRAM_SOURCE}{char[]}{
返回傳給 \capi{clCreateProgramWithSource} 的\cnglo{program}源碼。
返回的字符串中將所有源碼都串在了一起,並以 null 終止(原有的 null 會被剝離)。

如果 \carg{program} 是用 \capi{clCreateProgramWithBinary} 或 \capi{clCreateProgramWithBuiltinKernels} 創建的,
可能返回空字符串,或者相應的\cnglo{program}源碼,這取決於二元碼中是否包含\cnglo{program}源碼。

\carg{param_value_size_ret} 中會返回源碼中字符的實際數目,包含 null 終止符。
}

\clETD{CL_PROGRAM_BINARY_SIZES}{size_t[]}{
返回一個陣列,內含 \carg{program} 所關聯的所有\cnglo{device}對應的\cnglo{program}二元碼(可以是可執行二元碼、編譯過的二元碼或者庫的二元碼)的大小。
陣列的大小等於與 \carg{program} 所關聯的\cnglo{device}的個數。
如果任一\cnglo{device}沒有對應的二元碼,則返回零。

如果 \carg{program} 是用 \capi{clCreateProgramWithBuiltinKernels} 創建的,
可能陣列中的所有元素都是零。
}

\clETD{CL_PROGRAM_BINARIES}{unsigned char *[]}{
返回一個陣列,內含 \carg{program} 所關聯的所有\cnglo{device}對應的\cnglo{program}二元碼(可以是可執行二元碼、編譯過的二元碼或者庫的二元碼)。
對於 \carg{program} 所關聯的每個\cnglo{device}而言,
所返回的二元碼可能是用 \capi{clCreateProgramWithBinary} 創建 \carg{program} 時所指定的二元碼,
或者是用 \capi{clBuildProgram} 或 \capi{clLinkProgram} 所生成的可執行二元碼。
如果是用 \capi{clCreateProgramWithSource} 生成的 \carg{program},
則返回的是 \capi{clBuildProgram}、 \capi{clCompileProgram} 或 \capi{clLinkProgram} 所生成的二元碼。
所返回的可能是特定實作的中間表示(又叫做 IR),或特定設備的可執行二元碼,也可能二者兼有。
至於二元碼中會返回哪種信息由 OpenCL 實作來決定。

\carg{param_value} 指向一個包含\math{n}個指針的陣列,所有指針都由調用者分配,
其中\math{n}就是 \carg{program} 所關聯\cnglo{device}的數目。
對於每個指針需要分配多少內存,可以通過此表中的 \cenum{CL_PROGRAM_BINARY_SIZES} 進行查詢。

實作可以使用陣列中的元素來存儲特定\cnglo{device}所對應的\cnglo{program}二元碼,如果有的話。
至於陣列中的元素都對應於哪個\cnglo{device},可以使用 \cenum{CL_PROGRAM_DEVICES} 進行查詢。
 \cenum{CL_PROGRAM_BINARIES} 和 \cenum{CL_PROGRAM_DEVICES} 所返回的陣列具有一對一的關係。
}

\clETD{CL_PROGRAM_NUM_KERNELS}{size_t}{
返回 \carg{program} 中聲明的\cnglo{kernel}總數。
對於 \carg{program} 所關聯的\cnglo{device}而言,至少要為其中之一成功構建了可執行\cnglo{program},
然後才能使用此信息。
}

\clETD{CL_PROGRAM_KERNEL_NAMES}{char[]}{
返回 \carg{program} 中\cnglo{kernel}的名字,以分號間隔。
對於 \carg{program} 所關聯的\cnglo{device}而言,至少要為其中之一成功構建了可執行\cnglo{program},
然後才能使用此信息。
}

\stopETD

}

如果執行成功,\capi{clGetProgramInfo} 會返回 \cenum{CL_SUCCESS}。
否則,返回下列錯誤碼之一:
\startigBase
\item \cenum{CL_INVALID_VALUE},如果 \carg{param_name} 無效,
或者 \carg{param_value_size} < \reftab{clGetProgramInfo}中返回型別的大小,
且 \carg{param_value} 不是 \cmacro{NULL}。

\item \cenum{CL_INVALID_PROGRAM},如果 \carg{program} 無效。

\startitem \cenum{CL_INVALID_PROGRAM_EXECUTABLE},
如果 \carg{param_name} 是:
\startigBase[indentnext=no]
\item \cenum{CL_PROGRAM_NUM_KERNELS}
\item 或 \cenum{CL_PROGRAM_KERNEL_NAMES},
\stopigBase
並且在 \carg{program} 所關聯的\cnglo{device}中至少有一個\cnglo{device}
所對應的\cnglo{program}執行體沒有成功構建。
\stopitem

\item \cenum{CL_OUT_OF_RESOURCES},如果\scdevfailres。

\item \cenum{CL_OUT_OF_HOST_MEMORY},如果\schostfailres。
\stopigBase

\topclfunc{clGetProgramBuildInfo}

\startCLFUNC
cl_int clGetProgramBuildInfo (cl_program program,
			cl_device_id device,
			cl_program_build_info param_name,
			size_t param_value_size,
			void *param_value,
			size_t *param_value_size_ret)
\stopCLFUNC

此函式會返回\cnglo{programobj}中某個\cnglo{device}所對應的構建資訊。

\carg{program} 即所要查詢的\cnglo{programobj}。

\carg{device} 指定要查詢哪個\cnglo{device}的構建資訊。
\carg{device} 必須是 \carg{program} 所關聯的\cnglo{device}。

\carg{param_name} 指定要查詢什麼資訊。
對於所支持的資訊類型以及 \carg{param_value} 中返回的資訊,
請參見\reftab{clGetProgramBuildInfo}。

\carg{param_value} 指向的內存用來存儲查詢結果。
如果 \carg{param_value} 是 \cmacro{NULL},則忽略。

\carg{param_value_size} 即 \carg{param_value} 所指內存塊的大小。
其值必須 >= \reftab{clGetProgramBuildInfo}中返回型別的大小。

\carg{param_value_size_ret} 返回查詢結果的實際大小。
如果 \carg{param_value_size_ret} 是 \cmacro{NULL},則忽略。

\placetable[here][tab:clGetProgramBuildInfo]
{\capi{clGetProgramBuildInfo} 所支持的 \carg{param_names}}
{\startETD[cl_program_buid_info][返回型別]

\clETD{CL_PROGRAM_BUILD_STATUS}{cl_build_status}{
返回構建、編譯或鏈接的狀態,在 \carg{program} 上為 \carg{device} 最後實施的那個
(\capi{clBuildProgram}、 \capi{clCompileProgram} 或 \capi{clLinkProgram})。

可以是下列之一:
\startigBase
\item \cenum{CL_BUILD_NONE}。如果三個都沒有實施過。

\item \cenum{CL_BUILD_ERROR}。如果最後實施的那個產生了錯誤。

\item \cenum{CL_BUILD_SUCCESS}。如果最後實施的那個成功了。

\item \cenum{CL_BUILD_IN_PROGRESS}。如果最後實施的那個還未完成。
\stopigBase
}

\clETD{CL_PROGRAM_BUILD_OPTIONS}{char[]}{
返回構建、編譯或鏈接的選項,即最後實施的那個的參數 \carg{options}。
如果狀態是 \cenum{CL_BUILD_NONE},則返回空字串。
}

\clETD{CL_PROGRAM_BUILD_LOG}{char[]}{
返回最後實施的那個的日誌。
如果狀態是 \cenum{CL_BUILD_NONE},則返回空字串。
}

\clETD{CL_PROGRAM_BINARY_TYPE}{cl_program_binary_type}{
返回 \carg{device} 所對應的二元碼的類型。
可以是下列之一:
\startigBase
\item \cenum{CL_PROGRAM_BINARY_TYPE_NONE},沒有對應的二元碼。

\item \cenum{CL_PROGRAM_BINARY_TYPE_COMPILED_OBJECT},有編譯過的二元碼。
如果 \carg{program} 是使用 \capi{clCreateProgramWithSource} 創建
並使用 \capi{clCompileProgram} 編譯的,
或者是用 \capi{clCreateProgramWithBinary} 裝載的編譯過的二元碼,都屬於這種情況。

\item \cenum{CL_PROGRAM_BINARY_TYPE_LIBRARY},有庫的二元碼。
如果用 \capi{clLinkProgram} 創建 \carg{program} 時指定了鏈接選項 \ccmm{-create-library},
或者是用 \capi{clCreateProgramWithBinary} 裝載的庫的二元碼,都屬於這種情況。

\item \cenum{CL_PROGRAM_BINARY_TYPE_EXECUTABLE},有可執行的二元碼。
如果用 \capi{clLinkProgram} 創建 \carg{program} 時沒有指定鏈接選項 \ccmm{-create-library},
或者是用 \capi{clCreateProgramWithBinary} 裝載的可執行的二元碼,都屬於這種情況。
\stopigBase
}

\stopETD
}

如果執行成功,\capi{clGetProgramBuildInfo} 會返回 \cenum{CL_SUCCESS}。
否則,返回下列錯誤碼之一:
\startigBase
\item \cenum{CL_INVALID_DEVICE},
如果 \carg{device} 不是與 \carg{program} 相關聯的\cnglo{device}。

\item \cenum{CL_INVALID_VALUE},如果 \carg{param_name} 無效,
或者 \carg{param_value_size} < \reftab{clGetProgramBuildInfo}中返回型別的大小,
且 \carg{param_value} 不是 \cmacro{NULL}。

\item \cenum{CL_INVALID_PROGRAM},如果 \carg{program} 無效。

\item \cenum{CL_OUT_OF_RESOURCES},如果\scdevfailres。

\item \cenum{CL_OUT_OF_HOST_MEMORY},如果\schostfailres。
\stopigBase

\startnotepar
為\cnglo{pardev}構建的\cnglo{program}二元碼(編譯過的二元碼、庫的二元碼或者可執行的二元碼)
可以被他的所有\cnglo{subdev}所使用。
如果沒有為某個\cnglo{subdev}構建相應的\cnglo{program}二元碼,則會使用其\cnglo{pardev}的。

一個\cnglo{device}的\cnglo{program}二元碼,
無論是為 \capi{clCreateProgramWithBinary} 所指定的,
還是用 \capi{clGetProgramInfo} 查詢得到的,
都可以用到相應\cnglo{rootdev}、及其任意級別的\cnglo{subdev}上。
\stopnotepar
