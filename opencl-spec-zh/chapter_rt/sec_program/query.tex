\subsection{程序對象相關的查詢}

% clGetProgramInfo
函數
\startclc
cl_int clGetProgramInfo (cl_program program,
			cl_program_info param_name,
			size_t param_value_size,
			void *param_value,
			size_t *param_value_size_ret)
\stopclc
會返回\cnglo{programobj}的相關信息。

\carg{program} 即所要查詢的\cnglo{programobj}。

\carg{param_name} 指定要查詢什麼信息。
對於所支持的信息類型以及 \carg{param_value} 中返回的信息,請參見\reftab{clGetProgramInfo}。

\carg{param_value} 指向的內存用來存儲查詢結果。
如果 \carg{param_value} 是 \cenum{NULL},則忽略。

\carg{param_value_size} 即 \carg{param_value} 所指內存塊的大小。
其值必須 >= \reftab{clGetProgramInfo}中返回類型的大小。

\carg{param_value_size_ret} 返回查詢結果的實際大小。
如果 \carg{param_value_size_ret} 是 \cenum{NULL},則忽略。

\startbuffer[tblclgetprginfo]
\capi{clGetProgramInfo} 所支持的 \carg{param_names}
\stopbuffer
\splitfloat{
\placetable[here,force][tab:clGetProgramInfo]{\getbuffer[tblclgetprginfo]}
}{
{\input{chapter_rt/tbl/tbl_clgetprginfo.tex}}
}

如果執行成功,\capi{clGetProgramInfo} 會返回 \cenum{CL_SUCCESS}。
否則,返回下列錯誤碼之一:
\startigBase
\item \cenum{CL_INVALID_VALUE},如果 \carg{param_name} 無效,
或者 \carg{param_value_size} < \reftab{clGetProgramInfo}中返回類型的大小,
且 \carg{param_value} 不是 \cenum{NULL}。

\item \cenum{CL_INVALID_PROGRAM},如果 \carg{program} 無效。

\item \cenum{CL_INVALID_PROGRAM_EXECUTABLE},
如果 \carg{param_name} 是 \cenum{CL_PROGRAM_NUM_KERNELS} 或 \cenum{CL_PROGRAM_KERNEL_NAMES},
且在 \carg{program} 所關聯的\cnglo{device}中至少有一個\cnglo{device}所對應的可執行\cnglo{program}沒有成功構建。

\item \cenum{CL_OUT_OF_RESOURCES}——如果\scdevfailres。

\item \cenum{CL_OUT_OF_HOST_MEMORY}——如果\schostfailres。
\stopigBase

% clGetProgramBuildInfo
函數
\startclc
cl_int clGetProgramBuildInfo (cl_program program,
			cl_device_id device,
			cl_program_build_info param_name,
			size_t param_value_size,
			void *param_value,
			size_t *param_value_size_ret)
\stopclc
會返回\cnglo{programobj}中某個\cnglo{device}所對應的構建信息。

\carg{program} 即所要查詢的\cnglo{programobj}。

\carg{device} 指定要查詢哪個\cnglo{device}的構建信息。
 \carg{device} 必須是 \carg{program} 所關聯的\cnglo{device}。

\carg{param_name} 指定要查詢什麼信息。
對於所支持的信息類型以及 \carg{param_value} 中返回的信息,請參見\reftab{clGetProgramBuildInfo}。

\carg{param_value} 指向的內存用來存儲查詢結果。
如果 \carg{param_value} 是 \cenum{NULL},則忽略。

\carg{param_value_size} 即 \carg{param_value} 所指內存塊的大小。
其值必須 >= \reftab{clGetProgramBuildInfo}中返回類型的大小。

\carg{param_value_size_ret} 返回查詢結果的實際大小。
如果 \carg{param_value_size_ret} 是 \cenum{NULL},則忽略。

\startbuffer[tblclgetprgbuildinfo]
\capi{clGetProgramBuildInfo} 所支持的 \carg{param_names}
\stopbuffer
\splitfloat{
\placetable[here,force][tab:clGetProgramBuildInfo]{\getbuffer[tblclgetprgbuildinfo]}
}{
{\input{chapter_rt/tbl/tbl_clgetprgbuildinfo.tex}}
}

如果執行成功,\capi{clGetProgramBuildInfo} 會返回 \cenum{CL_SUCCESS}。
否則,返回下列錯誤碼之一:
\startigBase
\item \cenum{CL_INVALID_DEVICE},如果 \carg{device} 不是與 \carg{program} 相關聯的\cnglo{device}。

\item \cenum{CL_INVALID_VALUE},如果 \carg{param_name} 無效,
或者 \carg{param_value_size} < \reftab{clGetProgramBuildInfo}中返回類型的大小,
且 \carg{param_value} 不是 \cenum{NULL}。

\item \cenum{CL_INVALID_PROGRAM},如果 \carg{program} 無效。

\item \cenum{CL_OUT_OF_RESOURCES}——如果\scdevfailres。

\item \cenum{CL_OUT_OF_HOST_MEMORY}——如果\schostfailres。
\stopigBase
