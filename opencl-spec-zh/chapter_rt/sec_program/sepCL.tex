\subsection{編譯和鏈接的分離}

OpenCL 1.2 擴展了\cnglo{program}的編譯和鏈接,從而支持:
\startigBase
\item 編譯階段、鏈接階段的分離。編譯階段可以將\cnglo{program}源碼編譯程二進制目標碼,而鏈接階段可以將其與其它目標碼鏈接程可執行\cnglo{program}。

\item 內嵌頭文件。在 OpenCL 1.0 和 1.1 中,對於\cnglo{program}源碼所包含的頭文件而言,構建選項 -I 可以用來指定其搜索路徑。
而 OpenCL 1.2 對其進行了擴展,允許目標碼中內嵌頭文件的源碼。

\item 庫。可以使用鏈接器將目標碼和庫鏈接程可執行\cnglo{program},或者創建一個新庫。
\stopigBase

\input{chapter_rt/sec_program/subsec_sepCL/compile.tex}
\input{chapter_rt/sec_program/subsec_sepCL/link.tex}
