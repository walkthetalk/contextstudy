\subsection{構建可執行程序}

函數
\startclc
cl_int clBuildProgram (cl_program program,
			cl_uint num_devices,
			const cl_device_id *device_list,
			const char *options,
			void (CL_CALLBACK *pfn_notify)(cl_program program,
						void *user_data),
			void *user_data)
\stopclc
會由\cnglo{program}源碼或二進制\cnglo{program}構建(編譯 & 鏈接)可執行\cnglo{program},
它可以在 \carg{program} 所在\cnglo{context}中的所有或部分\cnglo{device}上執行。
對於用 \capi{clCreateProgramWithSource} 或 \capi{clCreateProgramWithBinary} 創建的 \carg{program},
必須調用 \capi{clBuildProgram} 來構建可執行程序。
如果是用 \capi{clCreateProgramWithBinary} 創建的 \carg{program},
二進制\cnglo{program}必須是可執行的(而不是經過編譯的二進制數據或庫)。

對於可執行的二進制\cnglo{program}而言,
可以用 \capi{clGetProgramInfo}(\carg{program}, \cenum{CL_PROGRAM_BINARIES}, ...) 進行查詢,
也可以由 \capi{clCreateProgramWithBinary} 用來創建新的\cnglo{programobj}。

\carg{program} 就是\cnglo{programobj}。

\carg{device_list} 指向 \carg{program} 所關聯的\cnglo{device}。
如果 \carg{device_list} 是 \cenum{NULL},則會為 \carg{program} 所關聯的所有\cnglo{device}構建可執行\cnglo{program}。
否則,僅為此清單中的\cnglo{device}構建可執行\cnglo{program}。

\carg{num_devices} 即 \carg{device_list} 中\cnglo{device}的數目。

\carg{options} 指向一個以 null 結尾的字符串,用來描述構建選項。所支持的選項請參閱\todo{5.6.4}。

\carg{pfn_notify} 是\cnglo{app}所註冊的一個回調函數。在構建完可執行\cnglo{program}後(無論成功還是失敗)會被調用。
如果 \carg{pfn_notify} 不是 \cenum{NULL},一旦可以開始構建, \capi{clBuildProgram} 就會立刻返回,而不必等待構建完成。
如果\cnglo{context}、所要編譯鏈接的\cnglo{program}、\cnglo{device}清單以及構建選項都是有效的,
以及實施構建所需的\cnglo{host}和\cnglo{device}資源都可用,則就可以開始進行構建了。
如果 \carg{pfn_notify} 是 \cenum{NULL},直到構建完畢, \capi{clBuildProgram} 才會返回。
對此函數的調用可能是異步的。
\cnglo{app}需要保證此函數是線程安全的。

\carg{user_data} 在調用 \carg{pfn_notify} 時作為參數傳入,可以是 \cenum{NULL}。

如果執行成功,則 \capi{clBuildProgram} 會返回 \cenum{CL_SUCCESS}。
否則,返回下列錯誤碼之一:
\startigBase
\item \cenum{CL_INVALID_PROGRAM},如果 \carg{program} 無效。

\item \cenum{CL_INVALID_VALUE},如果 \carg{device_list} 是 \cenum{NULL} 而 \carg{num_devices} 大於零,
或者 \carg{device_list} 不是 \cenum{NULL} 而 \carg{num_devices} 等於零。

\item \cenum{CL_INVALID_VALUE},如果 \carg{pfn_notify} 是 \cenum{NULL},而 \carg{user_data} 不是 \cenum{NULL}。

\item \cenum{CL_INVALID_DEVICE},如果 \carg{device_list} 中的 OpenCL \cnglo{device} 不是 \carg{program} 所關聯\cnglo{device}。

\item \cenum{CL_INVALID_BINARY},如果 \carg{program} 是用 \capi{clCreateProgramWithBinary} 創建的,
但沒有為 \carg{device_list} 中的\cnglo{device}加載相應的二進制\cnglo{program}。

\item \cenum{CL_INVALID_BUILD_OPTIONS},如果 \carg{options} 所指定的構建選項無效。

\item \cenum{CL_INVALID_OPERATION},如果對於 \carg{device_list} 中任一\cnglo{device}而言,
之前調用 \capi{clBuildProgram} 在其上為 \carg{program} 構建可執行\cnglo{program}的動作還未完成。

\item \cenum{CL_COMPILER_NOT_AVAILABLE},如果 \carg{program} 是用 \capi{clCreateProgramWithSource} 創建的,
但是沒有可用的編譯器,即 \cenum{CL_DEVICE_COMPILER_AVAILABLE} 是 \cenum{CL_FALSE},參見\reftab{cldevquery}。

\item \cenum{CL_BUILD_PROGRAM_FAILURE},如果構建可執行\cnglo{program}失敗。
如果 \capi{clBuildProgram} 沒有將此錯誤返回,則在構建完成時會將其返回。

\item \cenum{CL_INVALID_OPERATION},如果有附着到 \carg{program} 上的\cnglo{kernelobj}。

\item \cenum{CL_INVALID_OPERATION},如果 \carg{program} 不是由 \capi{clCreateProgramWith{Source | Binary}} 創建的。

\item \cenum{CL_OUT_OF_RESOURCES},如果\scdevfailres。

\item \cenum{CL_OUT_OF_HOST_MEMORY},如果\schostfailres。
\stopigBase
