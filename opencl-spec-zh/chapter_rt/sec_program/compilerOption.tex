\subsection{編譯器選項}

編譯器選項分為以下幾種:預處理器選項、數學本徵選項、優化控制選項以及其它選項。
本規範定義了一套標準選項,用於在線或離線構建可執行\cnglo{program},所有 OpenCL C 編譯器都必須支持。
也可以通過增加一些針對特定供應商或特定平台的選項對其進行擴展。

\subsubsection{預處理器選項}
這些選項用於控制預處理器,在真正編譯前對源碼進行預處理。

\startclOption{-D \carg{name}}
預定義一個名為 \carg{name} 的宏,其值為 1。
\stopclOption

\startclOption{-D \carg{name}=\carg{definition}}
就像處理指令 \ccmm{#define} 那樣,在翻譯的第三個階段中會將 \carg{definition} 的內容符號化並處理。
特別是遇到換行符會進行截斷。
\stopclOption

對於選項 -D,會按照在 \capi{clBuildProgram} 或 \capi{clCompileProgram} 的參數 \carg{options} 中出現的順序進行處理。

\startclOption{-I \carg{dir}}
將 \carg{dir} 加入到頭文件的搜索路徑中。
\stopclOption

\subsubsection{數學本徵選項}
這些選項會控制編譯器中與浮點算術有關的行為。
它們會影響速度與正確性之間的權衡。

\startclOption{-cl-single-precision-constant}
將雙精度浮點常數視為單精度浮點常數。
\stopclOption

\startclOption{-cl-denorms-are-zero}
此選項用來控制如何處理單、雙精度浮點去規格化數( denormalized number )。
如果作為構建選項,單精度去規格化數會被改成零;
此時如果支持雙精度,則雙精度去規格化數也可能會被改成零。
此選項僅作為性能建議,如果\cnglo{device}支持單精度(或雙精度)去規格化數,
 OpenCL 編譯器可以選擇不將其改成零。

如果\cnglo{device}不支持單精度(或雙精度)去規格化數,
即 \cenum{CL_DEVICE_SINGLE_FP_CONFIG} 中沒有設置 \cenum{CL_FP_DENORM},
則對於單精度數此選項會被忽略。

如果\cnglo{device}不支持雙精度,或者支持雙精度但不支持雙精度(或雙精度)去規格化數,
即 \cenum{CL_DEVICE_DOUBLE_FP_CONFIG} 中沒有設置 \cenum{CL_FP_DENORM},
則對於雙精度數此選項會被忽略。

此選項僅對\cnglo{program}中的標量或矢量浮點變量以及其上的運算起作用,
對於讀寫\cnglo{imgobj}無效。
\stopclOption

\startclOption{-cl-fp32-correctly-rounded-divide-sqrt}
對於\cnglo{program}源碼中的單精度浮點除法( \ccmm{x/y} 和 \ccmm{1/x} )和 \ccmm{sqrt} 而言,
\cnglo{app}可以使用此選項來保證對它們進行正確的舍入;
而如果沒有使用此選項,則它們的最小精度在\todo{節7.4}中定義。

只有在對應\cnglo{device}的 \cenum{CL_DEVICE_SINGLE_FP_CONFIG} (參見\reftab{cldevquery})
中設置了 \cenum{CL_FP_CORRECTLY_ROUNDED_DIVIDE_SQRT} 時,才能使用此選項;
否則編譯會失敗。
\stopclOption
