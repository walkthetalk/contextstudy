\subsection{創建程序對象}

函式
\startCLFUNC
cl_program clCreateProgramWithSource (cl_context context,
			cl_uint count,
			const char **strings,
			const size_t *lengths,
			cl_int *errcode_ret)
\stopCLFUNC
可以創建一個\cnglo{programobj},並將 \carg{strings} 中的源碼加載到這個對象中。
此對象所對應的\cnglo{device}與 \carg{context} 所對應的\cnglo{device}是一樣的。
 \carg{strings} 中的源碼是 OpenCL C 程序源碼、頭檔,或者自定義的代碼(對應與自定義\cnglo{device},並且需要在線編譯器的支持)。

\carg{context} 是一個 OpenCL \cnglo{context}。

\carg{strings} 是一個字串陣列,有 \carg{count} 個元素,字串可以 null 結尾,這些字串就構成了源代碼。

\carg{lengths} 也是一個陣列,其中的元素代表字符個數(字串長度)。如果某個元素是零,則相應的字串以 null 結尾。
如果 \carg{lengths} 是 \cmacro{NULL},則 \carg{strings} 中所有字串都以 null 結尾。
所有大於零的值都不包括 null 終止符。

\carg{errcode_ret} 會返回相應的錯誤碼。如果 \carg{errcode_ret} 是 \cmacro{NULL},則不會返回錯誤碼。

如果成功創建了\cnglo{programobj},則 \capi{clCreateProgramWidthSource} 會將其返回,
並將 \carg{errcode_ret} 置為 \cenum{CL_SUCCESS}。
否則,返回 \cmacro{NULL},並將 \carg{errcode_ret} 置為下列錯誤碼之一:
\startigBase
\item \cenum{CL_INVALID_CONTEXT},如果 \carg{context} 無效。

\item \cenum{CL_INVALID_VALUE},如果 \carg{count} 是零,或者 \carg{strings} 或其中任一項是 \cmacro{NULL}。

\item \cenum{CL_OUT_OF_RESOURCES},如果\scdevfailres。

\item \cenum{CL_OUT_OF_HOST_MEMORY},如果\schostfailres。
\stopigBase

函式
\startCLFUNC
cl_program clCreateProgramWithBinary (cl_context context,
			cl_uint num_devices,
			const cl_device_id *device_list,
			const size_t *lengths,
			const unsigned char **binaries,
			cl_int *binary_status,
			cl_int *errcode_ret)
\stopCLFUNC
會創建一個\cnglo{programobj},並將 \carg{binary} 中的內容加載到此對象中。

\carg{context} 是一個 OpenCL \cnglo{context}。

\carg{device_list} 指向 \carg{context} 中的\cnglo{device}。
\carg{device_list} 不能是 \cmacro{NULL}。
所加載的內容會在此清單中的\cnglo{device}上執行。

\carg{num_devices} 即 \carg{device_list} 中\cnglo{device}的數目。

此\cnglo{programobj}所對應的\cnglo{device}就是 \carg{device_list} 中的\cnglo{device}。
而 \carg{device_list} 中的\cnglo{device}又必須是 \carg{context} 中的\cnglo{device}。

\carg{lengths} 是一個陣列,其元素是所加載的\cnglo{program}的字節數。

\carg{binaries} 也是一個陣列,每個元素都指向所要加載的\cnglo{program}。
$binaries[i]$ 對應於 $device\_list[i]$,其大小為 $lengths[i]$。
$lengths[i]$ 不能是零,並且 $binaries[i]$ 不能是 \cmacro{NULL}。

\carg{binaries} 中的內容為下列之一:
\startigBase
\item 可執行\cnglo{program},可以運行在 \carg{context} 中的\cnglo{device}上;

\item 為 \carg{context} 中的\cnglo{device}編譯過的\cnglo{program};或

\item 為 \carg{context} 中的\cnglo{device}編譯過的\cnglo{program}庫。
\stopigBase

二進制\cnglo{program}可能包含下列之一,或二者兼有:
\startigBase
\item 特定\cnglo{device}( device-specific )的代碼,和/或
\item 特定實作( implementation-specific )的中間表示( intermediate representation,简称 IR )(將被轉換程特定\cnglo{device}的代碼)。
\stopigBase

\carg{binary_status} 將返回對應 \carg{device_list} 中\cnglo{device}的二進制\cnglo{program}是否加載成功。
他是一個有 \carg{num_devices} 個元素的陣列。
如果 $device\_list[i]$ 所對應的二進制\cnglo{program}加載成功,則 $binary\_status[i]$ 為 \cenum{CL_SUCCESS};
如果 $lengths[i]$ 是零或者 $binaries[i]$ 是 \cmacro{NULL},則 $binary\_status[i]$ 為 \cenum{CL_INVALID_VALUE};
如果二進制\cnglo{program}無效,則 $binary\_status[i]$ 為 \cenum{CL_INVALID_BINARY}。

\carg{errcode_ret} 會返回相應的錯誤碼。如果 \carg{errcode_ret} 是 \cmacro{NULL},則不會返回錯誤碼。

如果成功創建了\cnglo{programobj},則 \capi{clCreateProgramWithBinary} 會將其返回,
並將 \carg{errcode_ret} 置為 \cenum{CL_SUCCESS}。
否則,返回 \cmacro{NULL},並將 \carg{errcode_ret} 置為下列錯誤碼之一:
\startigBase
\item \cenum{CL_INVALID_CONTEXT},如果 \carg{context} 無效。

\item \cenum{CL_INVALID_VALUE},如果 \carg{device_list} 是 \cmacro{NULL},或者 \carg{num_devices} 是零。

\item \cenum{CL_INVALID_DEVICE},如果 \carg{device_list} 中的 OpenCL \cnglo{device} 不在 \carg{context} 中。

\item \cenum{CL_INVALID_VALUE},如果 \carg{lengths} 或者 \carg{binaries} 是 \cmacro{NULL},
或者任一項 $lengths[i]$ 是零,或者 $binaries[i]$ 是 \cmacro{NULL}。

\item \cenum{CL_INVALID_BINARY},如果任一二進制\cnglo{program}無效。 \carg{binary_status} 會返回對應於每個\cnglo{device}的狀態。

\item \cenum{CL_OUT_OF_RESOURCES},如果\scdevfailres。

\item \cenum{CL_OUT_OF_HOST_MEMORY},如果\schostfailres。
\stopigBase

OpenCL 中,\cnglo{app}可以使用\cnglo{program}源碼或二進制版本創建\cnglo{programobj},並構建相應的可執行\cnglo{program}。
這非常有用,當\cnglo{program}在系統中相應\cnglo{device}上第一次使用時,\cnglo{app}可以加載\cnglo{program}源碼並編譯、鏈接,在線生成可執行\cnglo{program}。
然後\cnglo{app}就可以查詢這些執行體並將其緩存起來,後續再使用此\cnglo{program}時就不必編譯、鏈接了。
\cnglo{app}可以讀取並加載緩存起來的執行體,從而很大程度上減少\cnglo{app}花在初始化上的時間。

函式
\startCLFUNC
cl_program clCreateProgramWithBuiltInKernels(
			cl_context context,
			cl_uint num_devices,
			const cl_device_id *device_list,
			const char *kernel_names,
			cl_int *errcode_ret)
\stopCLFUNC
會創建一個\cnglo{programobj},並將內建\cnglo{kernel}的相關資訊加載到此對象中。

\carg{context} 是一個 OpenCL \cnglo{context}。

\carg{device_list} 指向 \carg{context} 中的\cnglo{device}。
\carg{device_list} 不能是 \cmacro{NULL}。
所加載的內建\cnglo{kernel}會在此清單中的\cnglo{device}上執行。

\carg{num_devices} 即 \carg{device_list} 中\cnglo{device}的數目。

此\cnglo{programobj}所對應的\cnglo{device}就是 \carg{device_list} 中的\cnglo{device}。
而 \carg{device_list} 中的\cnglo{device}又必須是 \carg{context} 中的\cnglo{device}。

\carg{kernel_names} 是一組內建\cnglo{kernel}的名字,以分號分隔。

\carg{errcode_ret} 會返回相應的錯誤碼。如果 \carg{errcode_ret} 是 \cmacro{NULL},則不會返回錯誤碼。

如果成功創建了\cnglo{programobj},則 \capi{clCreateProgramWithBuiltInKernels} 會將其返回,
並將 \carg{errcode_ret} 置為 \cenum{CL_SUCCESS}。
否則,返回 \cmacro{NULL},並將 \carg{errcode_ret} 置為下列錯誤碼之一:
\startigBase
\item \cenum{CL_INVALID_CONTEXT},如果 \carg{context} 無效。

\item \cenum{CL_INVALID_VALUE},如果 \carg{device_list} 是 \cmacro{NULL},或者 \carg{num_devices} 是零。

\item \cenum{CL_INVALID_VALUE},如果 \carg{kernel_names} 是 \cmacro{NULL},
或者對於 \carg{kernel_names} 中的任一\cnglo{kernel}, \carg{device_list} 中的所有\cnglo{device}都不支持。

\item \cenum{CL_INVALID_DEVICE},如果 \carg{device_list} 中的 OpenCL \cnglo{device} 不在 \carg{context} 中。

\item \cenum{CL_OUT_OF_RESOURCES},如果\scdevfailres。

\item \cenum{CL_OUT_OF_HOST_MEMORY},如果\schostfailres。
\stopigBase

%% retain
函式
\startCLFUNC
cl_int clRetainProgram (cl_program program)
\stopCLFUNC
會使 \carg{program} 的\cnglo{refcnt}增一。
\capi{clRetainProgram} 會實施隱式的\cnglo{retain}。
執行成功後會返回 \cenum{CL_SUCCESS}。
否則,返回下列錯誤的一種:
\startigBase
\item \cenum{CL_INVALID_PROGRAM},如果 \carg{program} 無效。

\item \cenum{CL_OUT_OF_RESOURCES}——如果\scdevfailres。

\item \cenum{CL_OUT_OF_HOST_MEMORY}——如果\schostfailres。
\stopigBase

函式
\startCLFUNC
cl_int clReleaseProgram (cl_program program)
\stopCLFUNC
會使 \carg{program} 的\cnglo{refcnt}減一。
當與 \carg{program} 相關的所有\cnglo{kernelobj}都被刪除,並且 \carg{program} 的\cnglo{refcnt}變成零,此\cnglo{programobj}就會被刪除。
執行成功後會返回 \cenum{CL_SUCCESS}。
否則,返回下列錯誤的一種:
\startigBase
\item \cenum{CL_INVALID_PROGRAM},如果 \carg{program} 無效。

\item \cenum{CL_OUT_OF_RESOURCES},如果\scdevfailres。

\item \cenum{CL_OUT_OF_HOST_MEMORY},如果\schostfailres。
\stopigBase

