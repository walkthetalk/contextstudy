函式
\startclc
cl_int clCompileProgram (cl_program program,
			cl_uint num_devices,
			const cl_device_id *device_list,
			const char *options,
			cl_uint num_input_headers,
			const cl_program *input_headers,
			const char **header_include_names,
			void (CL_CALLBACK *pfn_notify)(cl_program program,
					void *user_data),
			void *user_data)
\stopclc
可以編譯\cnglo{program}源碼。
編譯前會先運行預處理器。
編譯工作的目標\cnglo{device}是 \carg{program} 所關聯的所有\cnglo{device}或指定的那些\cnglo{device}。
對於編譯過的二進制目標碼,可以使用 \capi{clGetProgramInfo}(\carg{program}, \cenum{CL_PROGRAM_BINARIES}, ...) 對其進行查詢,
也可以由 \capi{clCreateProgramWithBinary} 用來創建新的\cnglo{programobj}。

\carg{program} 是一個\cnglo{programobj},即編譯的目標。

\carg{device_list} 指向 \carg{program} 所關聯的\cnglo{device}。
如果 \carg{device_list} 是 \cenum{NULL},則編譯動作的目標\cnglo{device}為 \carg{program} 所關聯的所有\cnglo{device}。
否則,僅為此清單中的\cnglo{device}實施編譯。

\carg{num_devices} 即 \carg{device_list} 中\cnglo{device}的數目。

\carg{options} 指向一個以 null 結尾的字符串,用來描述編譯選項。所支持的選項請參閱\todo{5.6.4}。

\carg{num_input_headers} 即數組 \carg{input_headers} 中\cnglo{program}的數目。

\carg{input_headers} 是一個數組,元素為內嵌頭文件內容的\cnglo{program},
這些\cnglo{program}由 \capi{clCreateProgramWithSource} 創建。

\carg{header_include_names} 也是一個數組,並與 \carg{input_headers} 一一對應。
每一項就是相應\cnglo{program}源碼中一個內嵌頭文件的名字。
 \carg{input_headers} 中對應項即為包含此頭文件源碼的\cnglo{programobj}。
搜索頭文件時,內嵌頭文件比編譯選項 -I 所列目錄(\todo{5.6.4.1})有更高優先級。
如果 \carg{header_include_names} 中有多項名字相同的頭文件,則使用最先搜索到的。

例如,假設有如下\cnglo{program}源碼:
\startclc
#include <foo.h>
#include <mydir/myinc.h>

__kernel void image_filter (int n, int m,
			__constant float *filter_weights,
			__read_only image2d_t src_image,
			__write_only image2d_t dst_image)
{
	...
}
\stopclc
這個\cnglo{kernel}包含了兩個頭文件 foo.h 和 mydir/myinc.h。
下面來看怎樣將其傳遞並內嵌到\cnglo{programobj}中:
\startclc
cl_program foo_pg = clCreateProgramWithSource(context,
				1, &foo_header_src, NULL, &err);
cl_program myinc_pg = clCreateProgramWithSource(context,
				1, &myinc_header_src, NULL, &err);

// let’s assume the program source described above is given
// by program_A and is loaded via clCreateProgramWithSource

cl_program input_headers[2] = { foo_pg, myinc_pg };
char * input_header_names[2] = { “foo.h”, “mydir/myinc.h” };
clCompileProgram(program_A,
		0, NULL,	// num_devices & device_list
		NULL,		// compile_options
		2,		// num_input_headers
		input_headers,
		input_header_names,
		NULL, NULL);	// pfn_notify & user_data
\stopclc

\carg{pfn_notify} 是\cnglo{app}所註冊的一個回調函式。在編譯完成後(無論成功還是失敗)會被調用。
如果 \carg{pfn_notify} 不是 \cenum{NULL},一旦可以開始編譯, \capi{clCompileProgram} 就會立刻返回,而不必等待編譯完成。
如果\cnglo{context}、所要編譯的\cnglo{program}源碼、\cnglo{device}清單、所輸入的頭文件以及相應的\cnglo{program},還有構建選項都是有效的,
並且實施編譯所需的\cnglo{host}和\cnglo{device}資源都可用,則就可以開始進行編譯了。
如果 \carg{pfn_notify} 是 \cenum{NULL},直到編譯完畢, \capi{clCompileProgram} 才會返回。
對此函式的調用可能是異步的。
\cnglo{app}需要保證此函式是線程安全的。

\carg{user_data} 在調用 \carg{pfn_notify} 時作為參數傳入,可以是 \cenum{NULL}。

如果執行成功,則 \capi{clCompileProgram} 會返回 \cenum{CL_SUCCESS}。
否則,返回下列錯誤碼之一:
\startigBase
\item \cenum{CL_INVALID_PROGRAM},如果 \carg{program} 無效。

\item \cenum{CL_INVALID_VALUE},如果 \carg{device_list} 是 \cenum{NULL} 而 \carg{num_devices} 大於零,
或者 \carg{device_list} 不是 \cenum{NULL} 而 \carg{num_devices} 等於零。

\item \cenum{CL_INVALID_VALUE},如果 \carg{num_input_headers} 是零,而 \carg{header_include_names} 或 \carg{input_headers} 不是 \cenum{NULL};
或者 \carg{num_input_headers} 不是零,而 \carg{header_include_names} 或 \carg{input_headers} 是 \cenum{NULL}。

\item \cenum{CL_INVALID_VALUE},如果 \carg{pfn_notify} 是 \cenum{NULL},而 \carg{user_data} 不是 \cenum{NULL}。

\item \cenum{CL_INVALID_DEVICE},如果 \carg{device_list} 中的 OpenCL \cnglo{device} 不是 \carg{program} 所關聯\cnglo{device}。

\item \cenum{CL_INVALID_COMPILER_OPTIONS},如果 \carg{options} 所指定的編譯選項無效。

\item \cenum{CL_INVALID_OPERATION},如果對於 \carg{device_list} 中任一\cnglo{device}而言,
之前調用 \capi{clCompileProgram} 或 \capi{clBuildProgram} 在其上為 \carg{program} 編譯或構建可執行\cnglo{program}的動作還未完成。

\item \cenum{CL_COMPILER_NOT_AVAILABLE},如果沒有可用的編譯器,
即 \cenum{CL_DEVICE_COMPILER_AVAILABLE} 是 \cenum{CL_FALSE},參見\reftab{cldevquery}。

\item \cenum{CL_COMPILE_PROGRAM_FAILURE},如果編譯\cnglo{program}源碼失敗。
如果 \capi{clCompileProgram} 沒有將此錯誤返回,則在編譯完成時會將其返回。

\item \cenum{CL_INVALID_OPERATION},如果有附着到 \carg{program} 上的\cnglo{kernelobj}。

\item \cenum{CL_INVALID_OPERATION},如果沒有 \carg{program} 的源碼,
即 \carg{program} 不是由 \capi{clCreateProgramWithSource} 創建的。

\item \cenum{CL_OUT_OF_RESOURCES},如果\scdevfailres。

\item \cenum{CL_OUT_OF_HOST_MEMORY},如果\schostfailres。
\stopigBase
