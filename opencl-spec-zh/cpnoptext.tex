\startcomponent cpnoptext
\product opencl-spec-zh

\chapter[chapter:optExt]{可選擴展}

%OpenCL 1.2 所支持的可選特性在《OpenCL 1.2 擴展規範》中有所描述。
本章列出了 OpenCL 1.2 所支持的可選特性。
一些 OpenCL \cnglo{device} 可能支持這些可選擴展。
對於符合 OpenCL 的實作而言,不要求支持這些可選擴展,
但仍然希望這些可選擴展能夠獲取廣泛的支持。
將來修訂 OpenCL 規範時,可能將這些可選擴展中所定義的功能移入必需的特性集中。
下面簡單描述了這些擴展的定義。

OpenCL 工作組所批准的 OpenCL 擴展均遵守寫列命名約定:
\startigBase
\item 每個擴展都有一個唯一的{\ftRef{名字字串}},形如“\clext{cl_khr_<名字>}”。
如果實作支持此擴展,
則此字串會出現在\reftab{cldevquery}中的 \cenum{CL_PLATFORM_EXTENSIONS} 或 \cenum{CL_DEVICE_EXTENSIONS} 中。

\item 對於擴展中所定義的所有 API 函式,其名字均形如:\clapi{cl<函式名>KHR}。

\item 對於擴展中所定義的所有枚舉,其名字均形如:\clapi{CL_<枚舉名>_KHR}。
\stopigBase

後續修訂 OpenCL 時, OpenCL 工作組所批准的 OpenCL 擴展都可能被{\ftRef{提升}}為必需的核心特性。
一旦如此,就會將相應的擴展規範合併到核心規範中。
而對於其中的函式和枚舉,則會移除其後綴 {\ftEmp{KHR}}。
對於相應的 OpenCL 實作而言,
仍然要在 \cenum{CL_PLATFORM_EXTENSIONS} 或 \cenum{CL_DEVICE_EXTENSIONS} 中導出此擴展名,
而且也要支持帶有後綴 {\ftEmp{KHR}} 的函式和枚舉,
以方便在不同的 OpenCL 版本間進行遷移。

對於供應商自定義的擴展,則要遵循下列命名約定:
\startigBase
\item 每個擴展都有一個唯一的{\ftRef{名字字串}},形如“\clext{cl_<供應商的名字>_<名字>}”。
如果實作支持此擴展,
則此字串會出現在\reftab{cldevquery}中的 \cenum{CL_PLATFORM_EXTENSIONS} 或 \cenum{CL_DEVICE_EXTENSIONS} 中。

\item 對於擴展中所定義的所有 API 函式,其名字均形如:\clapi{cl<函式名><供應商的名字>}。

\item 對於擴展中所定義的所有枚舉,其名字均形如:\clapi{CL_<枚舉名>_<供應商的名字>}。
\stopigBase

\section{可選擴展的編譯指示}

編譯指示 \cdrtemp{#pragma OPENCL EXTENSION} 控制着 OpenCL 編譯器關於擴展的行為。
此指示定義如下:
\startclc[indentnext=no]
#pragma OPENCL EXTENSION /BTEX\cref{extension_name}/ETEX : /BTEX\cref{behavior}/ETEX
#pragma OPENCL EXTENSION all : /BTEX\cref{behavior}/ETEX
\stopclc
其中 \cref{extension_name} 就是擴展的名字,
形如 \clext{cl_khr_<名字>}(OpenCL 工作組所批准的擴展)或者 \clext{cl_<供應商的名字>_<名字>}(供應商的擴展)。
而符記 \cemp{all} 則意味着在編譯器所支持的所有擴展上應用此行為。
可以將 \cref{behavior} 設置為\reftab{behaviorDsc}中的值。

\placetable[here][tab:behaviorDsc]
{可選擴展的行為}
{\startbuffer[enExtDsc]
其行為遵守擴展 \cref{extension_name} 的規定。

如果指定了 \cemp{all},或者不支持 \cref{extension_name},
則會在 \cdrtemp{#pragma OPENCL EXTENSION} 上報告錯誤。
\stopbuffer

\startbuffer[disExtDsc]
其行為(包括發出錯誤、警告)如同語言定義中沒有擴展 \cref{extension_name} 那樣。

如果指定了 \cemp{all},則其行為必須回退到所編譯語言的無擴展核心版本。

如果不支持 \cref{extension_name},
則會在 \cdrtemp{#pragma OPENCL EXTENSION} 上發出警告。
\stopbuffer

\startCLOD[behavior][描述]
\clOD{enable}{\getbuffer[enExtDsc]}
\clOD{disable}{\getbuffer[disExtDsc]}
\stopCLOD
}

就設定每個擴展的行為而言,
編譯指示 \cdrtemp{#pragma OPENCL EXTENSION} 是一種簡單、底層的機制。
但是他並沒有定義任何策略,比如哪種組合比較恰當;
這些必須在其他地方定義。
在設定每個擴展的行為時,指示的順序非常重要。
後出現的指示會覆蓋早出現的。
而變體 \cemp{all} 會設定所有擴展的行為,將覆蓋前面出現的所有擴展指示,
但 \cref{behavior} 只能設置為 \cemp{disable}。

編譯器的初始狀態相當於如下指示:
\startclc[indentnext=no]
#pragma OPENCL EXTENSION all : disable
\stopclc
告訴編譯器必須按照此規格來報告所有錯誤和警告,忽略所有擴展。

每個擴展,只要會影響 OpenCL 語言的語義、文法或者為語言增加了內建函式,
都必須創建一個預處理 \ccmm{#define} 來匹配擴展名。
當且僅當實作支持此擴展時,這個 \ckey{#define} 才可用。

例:

如果一個擴展增加了擴展字串“\clext{cl_khr_3d_image_writes}”,
那麼同時應當增加相應的預處理 \cemp{#define}。
現在\cnglo{kernel}就可以像這樣來使用這個預處理 \ccmm{#define}:
\startclc
#ifdef cl_khr_3d_image_writes
	// do something using the extension
#else
	// do something else or #error!
#endif
\stopclc

\section[sec:getFuncPtr]{獲取 OpenCL API 擴展函式指位器}

\topclfunc{clGetExtensionFunctionAddressForPlatform}

\startCLFUNC
void* clGetExtensionFunctionAddressForPlatform (
			cl_platform_id platform,
			const char *funcname)
\stopCLFUNC

\startnotepar
由於沒有任何方式來限定對\cnglo{device}的查詢,
對於此擴展在這個\cnglo{platform}中不同\cnglo{device}上的所有實作而言,
所返回的函式指位器必須都能正常運作。
如果\cnglo{device}不支持此擴展,則調用擴展中的函式時,其行為\cnglo{undef}。
\stopnotepar

此函式會返回給定 \carg{platform} 的擴展函式 \carg{funcname} 的位址。
需要將所返回的指位器轉型成對應的函式指位器型別,
此擴展函式在對應的擴展規範和頭檔中定義。
如果返回的是 \cmacro{NULL},則表明實作中沒有所指定的函式,或者 \carg{platform} 無效。
即使返回的不是 \cmacro{NULL},也不保證 \carg{platform} 真正支持此擴展函式。
要想確定 OpenCL 實作是否支持某個擴展,
\cnglo{app}必須用下列兩種方式之一進行查詢:
\startclc
clGetPlatformInfo(platform, CL_PLATFORM_EXTENSIONS, ... )
clGetDeviceInfo(device, CL_DEVICE_EXTENSIONS, ... )
\stopclc

對於 OpenCL 的核心函式(非擴展函式),不能使用此函式進行查詢。
而對於那些能用此函式進行查詢的函式,
實作也可以選擇由實現那些函式的目標庫靜態導入那些函式。
然而,\cnglo{app}要想可移植,就不能依賴這種行為。

對於所有會增加 API 引入點的擴展,都必須聲明 \ccmm{typedef} 的函式指位器型別。
這些 \ccmm{typedef} 是擴展接口所必需的一部分,在相應頭檔中提供
(如果是 OpenCL 擴展,則為 \ccmm{cl_ext.h};
而如果是 OpenCL / OpenGL 共享擴展,則為 \ccmm{cl_gl_ext.h})。

所有會影響\cnglo{host} API 的擴展都必須遵循下列約定:
\startclc[indentnext=no]
#ifndef extension_name
#define extension_name		1

// all data typedefs, token #defines, prototypes, and
// function pointer typedefs for this extension

// function pointer typedefs must use the
// following naming convention
typedef CL_API_ENTRY /BTEX{\ftRef{return type}}/ETEX
		(CL_API_CALL */BTEX{\ftRef{clextension_func_nameTAG_fn}}/ETEX)(...);

#endif // extension_name
\stopclc
其中 \ccmm{TAG} 可以是 \ccmm{KHR}、 \ccmm{EXT} 或 \ccmm{vendor-specific}。

例如,擴展 \clext{cl_khr_gl_sharing} 在 \ccmm{cl_gl_ext.h} 中增加了下列代碼:
\startclc
#ifndef cl_khr_gl_sharing
#define cl_khr_gl_sharing	1

// all data typedefs, token #defines, prototypes, and
// function pointer typedefs for this extension
#define CL_INVALID_GL_SHAREGROUP_REFERENCE_KHR	-1000
#define CL_CURRENT_DEVICE_FOR_GL_CONTEXT_KHR	0x2006
#define CL_DEVICES_FOR_GL_CONTEXT_KHR		0x2007
#define CL_GL_CONTEXT_KHR			0x2008
#define CL_EGL_DISPLAY_KHR			0x2009
#define CL_GLX_DISPLAY_KHR			0x200A
#define CL_WGL_HDC_KHR				0x200B
#define CL_CGL_SHAREGROUP_KHR			0x200C

// function pointer typedefs must use the
// following naming convention
typedef CL_API_ENTRY cl_int
	(CL_API_CALL *clGetGLContextInfoKHR_fn)(
		const cl_context_properties * /* properties */,
		cl_gl_context_info /* param_name */,
		size_t /* param_value_size */,
		void * /* param_value */,
		size_t * /*param_value_size_ret*/);

#endif // cl_khr_gl_sharing
\stopclc

\section{64 位原子函式}

下列兩個可選擴展實現了 \cqlf{__global} 和 \cqlf{__local} 內存中的 64 位
帶符號和無符號整數上的原子運算:
\startigBase
\item \clext{cl_khr_int64_base_atomics}
\item \clext{cl_khr_int64_extended_atomics}
\stopigBase

\cnglo{app}中要想使用這些擴展,
必須在 OpenCL \cnglo{program}源碼中包含下列編譯指示之一:
\startclc
#pragma OPENCL EXTENSION cl_khr_int64_base_atomics : enable
#pragma OPENCL EXTENSION cl_khr_int64_extended_atomics : enable
\stopclc

\reftab{atomic64_base}中列出了擴展 \clext{cl_khr_int64_base_atomics} 所支持的原子函式,
其中所有函式都在一個原子事務內實施。

\placetable[here,split][tab:atomic64_base]
{擴展 \clext{cl_khr_int64_base_atomics} 的內建原子函式}
{% atomic_add
\startbuffer[funcproto:atomic64_add]
long atomic_add (
	volatile __global long *p,
	long val)
long atomic_add (
	volatile __local long *p,
	long val)

ulong atomic_add (
	volatile __global ulong *p,
	ulong val)
ulong atomic_add (
	volatile __local ulong *p,
	ulong val)
\stopbuffer
\startbuffer[funcdesc:atomic64_add]
讀取 \carg{p} 所指向的 64 位值(記為 \math{old})。
計算 \math{(old + \marg{val})} 並將結果存儲到 \carg{p} 所指位置中。
此函式返回 \math{old}。
\stopbuffer

% atomic_sub
\startbuffer[funcproto:atomic64_sub]
long atomic_sub (
	volatile __global long *p,
	long val)
long atomic_sub (
	volatile __local long *p,
	long val)

ulong atomic_sub (
	volatile __global ulong *p,
	ulong val)
ulong atomic_sub (
	volatile __local ulong *p,
	ulong val)
\stopbuffer
\startbuffer[funcdesc:atomic64_sub]
讀取 \carg{p} 所指向的 64 位值(記為 \math{old})。
計算 \math{(old - \marg{val})} 並將結果存儲到 \carg{p} 所指位置中。
此函式返回 \math{old}。
\stopbuffer

% atomic_xchg
\startbuffer[funcproto:atomic64_xchg]
long atomic_xchg (
	volatile __global long *p,
	long val)
long atomic_xchg (
	volatile __local long *p,
	long val)

ulong atomic_xchg (
	volatile __global ulong *p,
	ulong val)
ulong atomic_xchg (
	volatile __local ulong *p,
	ulong val)
\stopbuffer
\startbuffer[funcdesc:atomic64_xchg]
將位置 \carg{p} 中所存儲的值 \math{old} 和 \carg{val} 中的新值相互交換。
返回 \math{old}。
\stopbuffer

% atomic_inc
\startbuffer[funcproto:atomic64_inc]
long atomic_inc (volatile __global long *p)
long atomic_inc (volatile __local long *p)

ulong atomic_inc (
	volatile __global ulong *p)
ulong atomic_inc (
	volatile __local ulong *p)
\stopbuffer
\startbuffer[funcdesc:atomic64_inc]
讀取 \carg{p} 所指向的 64 位值(記為 \math{old})。
計算 \math{(old+1)} 並將結果存儲到 \carg{p} 所指位置中。
此函式返回 \math{old}。
\stopbuffer

% atomic_dec
\startbuffer[funcproto:atomic64_dec]
long atomic_dec (volatile __global long *p)
long atomic_dec (volatile __local long *p)

ulong atomic_dec (
	volatile __global ulong *p)
ulong atomic_dec (
	volatile __local ulong *p)
\stopbuffer
\startbuffer[funcdesc:atomic64_dec]
讀取 \carg{p} 所指向的 64 位值(記為 \math{old})。
計算 \math{(old-1)} 並將結果存儲到 \carg{p} 所指位置中。
此函式返回 \math{old}。
\stopbuffer

% atomic_cmpchg
\startbuffer[funcproto:atomic64_cmpxchg]
long atomic_cmpxchg (
	volatile __global long *p,
	long cmp, long val)
long atomic_cmpxchg (
	volatile __local long *p,
	long cmp,
	long val)

ulong atomic_cmpxchg (
	volatile __global ulong *p,
	ulong cmp,
	ulong val)
ulong atomic_cmpxchg (
	volatile __local ulong *p,
	ulong cmp,
	ulong val)
\stopbuffer
\startbuffer[funcdesc:atomic64_cmpxchg]
讀取 \carg{p} 所指向的 64 位值(記為 \math{old})。
計算 \math{(old == cmp) ? val : old} 並將結果存儲到 \carg{p} 所指位置中。
此函式返回 \math{old}。
\stopbuffer


% begin table
\startCLFD
\clFD{atomic64_add}
\clFD{atomic64_sub}
\clFD{atomic64_xchg}
\clFD{atomic64_inc}
\clFD{atomic64_dec}
\clFD{atomic64_cmpxchg}
\stopCLFD
}

\reftab{atomic64_ext}中列出了擴展 \clext{cl_khr_int64_extended_atomics} 所支持的原子函式,
其中所有函式都在一個原子事務內實施。

\placetable[here,split][tab:atomic64_ext]
{擴展 \clext{cl_khr_int64_extended_atomics} 的內建原子函式}
{% atomic_min
\startbuffer[funcproto:atomic64_min]
long atomic_min (
	volatile __global long *p,
	long val)
long atomic_min (
	volatile __local long *p,
	long val)

ulong atomic_min (
	volatile __global ulong *p,
	ulong val)
ulong atomic_min (
	volatile __local ulong *p,
	ulong val)
\stopbuffer
\startbuffer[funcdesc:atomic64_min]
讀取 \carg{p} 所指向的 64 位值(記為 \math{old})。
計算 \math{\mapiemp{min}(old, \marg{val})} 並將結果存儲到 \carg{p} 所指位置中。
此函式返回 \math{old}。
\stopbuffer

% atomic_max
\startbuffer[funcproto:atomic64_max]
long atomic_max (
	volatile __global long *p,
	long val)
long atomic_max (
	volatile __local long *p,
	long val)

ulong atomic_max (
	volatile __global ulong *p,
	ulong val)
ulong atomic_max (
	volatile __local ulong *p,
	ulong val)
\stopbuffer
\startbuffer[funcdesc:atomic64_max]
讀取 \carg{p} 所指向的 64 位值(記為 \math{old})。
計算 \math{\mapiemp{max}(old, \marg{val})} 並將結果存儲到 \carg{p} 所指位置中。
此函式返回 \math{old}。
\stopbuffer

% atomic_and
\startbuffer[funcproto:atomic64_and]
long atomic_and (
	volatile __global long *p,
	long val)
long atomic_and (
	volatile __local long *p,
	long val)

ulong atomic_and (
	volatile __global ulong *p,
	ulong val)
ulong atomic_and (
	volatile __local ulong *p,
	ulong val)
\stopbuffer
\startbuffer[funcdesc:atomic64_and]
讀取 \carg{p} 所指向的 64 位值(記為 \math{old})。
計算 \math{(old \mcmm{&} \marg{val})} 並將結果存儲到 \carg{p} 所指位置中。
此函式返回 \math{old}。
\stopbuffer

% atomic_or
\startbuffer[funcproto:atomic64_or]
long atomic_or (
	volatile __global long *p,
	long val)
long atomic_or (
	volatile __local long *p,
	long val)

ulong atomic_or (
	volatile __global ulong *p,
	ulong val)
ulong atomic_or (
	volatile __local ulong *p,
	ulong val)
\stopbuffer
\startbuffer[funcdesc:atomic64_or]
讀取 \carg{p} 所指向的 64 位值(記為 \math{old})。
計算 \math{(old \mcmm{|} \marg{val})} 並將結果存儲到 \carg{p} 所指位置中。
此函式返回 \math{old}。
\stopbuffer

% atomic_xor
\startbuffer[funcproto:atomic64_xor]
long atomic_xor (
	volatile __global long *p,
	long val)
long atomic_xor (
	volatile __local long *p,
	long val)

ulong atomic_xor (
	volatile __global ulong *p,
	ulong val)
ulong atomic_xor (
	volatile __local ulong *p,
	ulong val)
\stopbuffer
\startbuffer[funcdesc:atomic64_xor]
讀取 \carg{p} 所指向的 64 位值(記為 \math{old})。
計算 \math{(old \mcmm{^} \marg{val})} 並將結果存儲到 \carg{p} 所指位置中。
此函式返回 \math{old}。
\stopbuffer


% begin table
\startCLFD
\clFD{atomic64_min}
\clFD{atomic64_max}
\clFD{atomic64_and}
\clFD{atomic64_or}
\clFD{atomic64_xor}
\stopCLFD
}

對於執行這些原子函式的\cnglo{device}而言,這些事務都是原子的。
而對於在多個\cnglo{device}上執行的\cnglo{kernel}而言,
如果這些原子運算是在同一內存位置上實施的,則不保證其原子性。

\startnotepar
64 位整數和 32 位整數(包括 \ctype{float})上的原子運算相互之間也是原子的。
\stopnotepar

\section{寫入 3D 圖像對象}

OpenCL 支持\cnglo{kernel}讀寫 2D \cnglo{imgobj}。
同一\cnglo{kernel}不能對同一 2D \cnglo{imgobj}既讀又寫。
OpenCL 也支持\cnglo{kernel}讀取 3D \cnglo{imgobj},
但是不支持寫入 3D \cnglo{imgobj},除非實現了擴展 \clext{cl_khr_3d_image_writes}。
同一\cnglo{kernel}不能對同一 3D \cnglo{imgobj}既讀又寫。

\cnglo{app}要想使用此擴展寫入 3D \cnglo{imgobj},
需要在 OpenCL \cnglo{program}源碼中包含下列編譯指示:
\startclc
#pragma OPENCL EXTENSION cl_khr_3d_image_writes : enable
\stopclc

\reftab{write_3d_image}中列出了擴展 \clext{cl_khr_3d_image_writes} 所實現的內建函式。

\placetable[here][tab:write_3d_image]
{擴展 \clext{cl_khr_3d_image_writes} 所實現的內建函式}
{% atomic_add
\startbuffer[funcproto:write_image_3d]
void write_imagef (image3d_t image,
		int4 coord,
		float4 color)

void write_imagei (image3d_t image,
		int4 coord,
		int4 color)

void write_imageui (image3d_t image,
		int4 coord,
		uint4 color)
\stopbuffer
\startbuffer[funcdesc:write_image_3d]
將 \carg{color} 的值寫入 3D \cnglo{imgobj} \carg{image}中坐標 \math{(x,y,z)} 處。
寫入前會對顏色值進行恰當的數據格式轉換。
會將 \carg{coord.x}、 \carg{coord.y} 和 \carg{coord.z} 視為非歸一化坐標,
且其值必須分別位於區間 \math{0\cdots\mvar{圖像寬度}-1}、 \math{0\cdots\mvar{圖像高度}-1} 和 \math{0\cdots\mvar{圖像深度}-1} 內。

對於 \capi{write_imagef},
創建\cnglo{imgobj}時所用的 \carg{image_channel_data_type} 必須是預定義壓縮過的格式
或者 \cenum{CL_SNORM_INT8}、 \cenum{CL_UNORM_INT8}、 \cenum{CL_SNORM_INT16}、
 \cenum{CL_UNORM_INT16}、 \cenum{CL_HALF_FLOAT} 或 \cenum{CL_FLOAT}。
會將通道數據由浮點值轉換成存儲數據所用的實際數據格式。

對於 \capi{write_imagei} 而言,
創建\cnglo{imgobj}時所用的 \carg{image_channel_data_type} 必須是下列值之一:
\startigBase
\item \cenum{CL_SIGNED_INT8}
\item \cenum{CL_SIGNED_INT16}
\item \cenum{CL_SIGNED_INT32}
\stopigBase

對於 \capi{write_imageui} 而言,
創建\cnglo{imgobj}時所用的 \carg{image_channel_data_type} 必須是下列值之一:
\startigBase
\item \cenum{CL_UNSIGNED_INT8}
\item \cenum{CL_UNSIGNED_INT16}
\item \cenum{CL_UNSIGNED_INT32}
\stopigBase

如果創建\cnglo{imgobj}時所用的 \carg{image_channel_data_type} 不再上述所列範圍內,
或者坐標 \math{(x, y, z)} 不在 \math{(0 \cdots \mvar{圖像寬度}-1, 0 \cdots \mvar{圖像高度}-1, 0 \cdots \mvar{圖像深度}-1)} 範圍內,
則 \capi{write_imagef}、 \capi{write_imagei} 和 \capi{write_imageui} 的行為\cnglo{undef}。
\stopbuffer

% begin table
\startCLFD
\clFD{write_image_3d}
\stopCLFD
}

\section{半精度浮點數}

此擴展增加了對 \cldt{half} 標量和矢量型別的支持,
可以 \cldt{half} 作為內建型別進行算術運算、轉換等。
\cnglo{app}要想使用型別 \cldt{half} 和 \cldt[n]{half},
必須包含編譯指示 \cemp{#pragma OPENCL EXTENSION cl_khr_fp16 : enable}。

\reftab{builtInScalarDataTypes}和\reftab{builtInVectorDataTypes}中所列內建標量、矢量數據型別又做了如下擴充:

\placetable[here][tab:half_type_dsc]
{\cldt{half} 相關數據型別}
{\startCLOD[型別][描述]

\clOD{\cldt{half2}}{2 組件半精度浮點矢量。}

\clOD{\cldt{half3}}{3 組件半精度浮點矢量。}

\clOD{\cldt{half4}}{4 組件半精度浮點矢量。}

\clOD{\cldt{half8}}{8 組件半精度浮點矢量。}

\clOD{\cldt{half16}}{16 組件半精度浮點矢量。}

\stopCLOD
}

在 OpenCL API(以及頭檔)中,內建矢量數據型別 \cldt[n]{half} 被聲明為其他型別,
以更好的為\cnglo{app}所用。
\reftab{bihalf2appdt}中列出了 OpenCL C 編程語言中
所定義的內建矢量數據型別 \cldt[n]{half} 與\cnglo{app}所用型別間的對應關係。

\placetable[here][tab:bihalf2appdt]
{內建矢量數據型別與應用程式所用型別的對應關係}
{\startCLOO[OpenCL 語言中的型別][\cnglo{app}所用 API 中的型別]

\clOO{\cldt{half2}}{\cldt{cl_half2}}
\clOO{\cldt{half3}}{\cldt{cl_half3}}
\clOO{\cldt{half4}}{\cldt{cl_half4}}
\clOO{\cldt{half8}}{\cldt{cl_half8}}
\clOO{\cldt{half16}}{\cldt{cl_half16}}

\stopCLOO


}

\refsec{operator}中所描述的關係、相等、邏輯以及邏輯單元算子
均可用於 \cldt{half} 標量和 \cldt[n]{half} 矢量型別,
所產生的結果分別為標量 \cldt{int} 和矢量 \cldt[n]{short}。

可以為浮點常值添加後綴 \ccmm{h} 或 \ccmm{H},
以表明此常值型別為 \cldt{half}。

\subsection{轉換}

現在,\refsec{implicityConversion}中的隱式轉換規則也適用於 \cldt{half} 標量和 \cldt[n]{half} 矢量數據型別。

\refsec{explicitCast}中的顯式轉型也做了擴充,
適用於 \cldt{half} 標量數據型別和 \cldt[n]{half} 矢量數據型別。

\refsec{explicitConversion}中所描述的顯式轉換函式也做了擴充,
適用於 \cldt{half} 標量數據型別和 \cldt[n]{half} 矢量數據型別。

\refsec{as_typen}中所描述的用於重釋型別的函式 \clapi[n]{as_type} 也做了擴充,
允許在 \cldt[n]{short}、 \cldt[n]{ushort} 和 \cldt[n]{half} 標量、矢量數據型別間進行無需轉換的轉型。

\subsection{數學函式}

對\reftab{svMathFunc}中所列內建數學函式作了擴充,
函式引數和返回值也可以是 \cldt{half} 和 \cldt[n]{half},
參見\reftab{svMathFuncHalf}。
現在, \cldt{gentype} 也包含 \cldt{half} 和 \cldt[n]{half},
其中 \ccmmsuffix{n} 可以是 2、 3、 4、 8、 16。

對於函式的任一特定用法,所有引數以及返回值的實際型別必須相同。

\placetable[here,split][tab:svMathFuncHalf]
{標量和矢量引數內建數學函式表}
{\startCLFD

\clFD{acos}
\clFD{acosh}
\clFD{acospi}
\clFD{asin}
\clFD{asinh}
\clFD{asinpi}
\clFD{atan}
\clFD{atan2}
\clFD{atanh}
\clFD{atanpi}
\clFD{atan2pi}
\clFD{cbrt}
\clFD{ceil}
\clFD{copysign}
\clFD{cos}
\clFD{cosh}
\clFD{cospi}
\clFD{erfc}
\clFD{erf}
\clFD{exp}
\clFD{exp2}
\clFD{exp10}
\clFD{expm1}
\clFD{fabs}
\clFD{fdim}
\clFD{floor}
\clFD{fma}
\clFD{fmaxh}
\clFD{fminh}
\clFD{fmod}
\clFD{fract}
\clFD{frexph}
\clFD{hypot}
\clFD{ilogbh}
\clFD{ldexph}
\clFD{lgammah}
\clFD{log}
\clFD{log2}
\clFD{log10}
\clFD{log1p}
\clFD{logb}
\clFD{mad}
\clFD{maxmag}
\clFD{minmag}
\clFD{modf}
\clFD{nanh}
\clFD{nextafter}
\clFD{pow}
\clFD{pownh}
\clFD{powr}
\clFD{remainder}
\clFD{remquoh}
\clFD{rint}
\clFD{rootnh}
\clFD{rsqrt}
\clFD{sin}
\clFD{sincos}
\clFD{sinh}
\clFD{sinpi}
\clFD{sqrt}
\clFD{tan}
\clFD{tanh}
\clFD{tanpi}
\clFD{tgamma}
\clFD{trunc}

\stopCLFD
}

巨集 \cmacroemp{FP_FAST_FMAF} 用來指明對於半精度浮點數,
 \capi{fma} 函式族是否比直接編碼更快。
如果定義了此巨集,則表明對算元為 \ctype{float} 的乘、加運算,
函式 \capi{fma} 一般跟直接編碼一樣快,或者更快。

下列巨集必須使用指定的值。
可以在預處理指示 \ccmm{#if} 中使用這些常量算式。
\startclc
#define HALF_DIG		3
#define HALF_MANT_DIG		11
#define HALF_MAX_10_EXP		+4
#define HALF_MAX_EXP		+16
#define HALF_MIN_10_EXP		-4
#define HALF_MIN_EXP		-13
#define HALF_RADIX		2
#define HALF_MAX		0x1.ffcp15h
#define HALF_MIN		0x1.0p-14h
#define HALF_EPSILON		0x1.0p-10f
\stopclc

\reftab{tblHalfMacroAndApp}中給出了上面所列巨集與\cnglo{app}所用的巨集名字之間的對應關係。

\placetable[here][tab:tblHalfMacroAndApp]
{半精度浮點巨集與應用程式所用巨集的對應關係}
{\startCLOO[OpenCL 語言中的巨集][\cnglo{app}所用的巨集]

\clMMH{DIG}
\clMMH{MANT_DIG}
\clMMH{MAX_10_EXP}
\clMMH{MAX_EXP}
\clMMH{MIN_10_EXP}
\clMMH{MIN_EXP}
\clMMH{RADIX}
\clMMH{MAX}
\clMMH{MIN}
\clMMH{EPSILSON}

\stopCLOO
}

除此之外,還有一些常量可用,如\reftab{tblHalfMacroConst}所示。
他們的型別都是 \ctype{float},在 \ctype{float} 型別的精度內是準確的。

\placetable[here][tab:tblHalfMacroConst]
{半精度浮點常量}
{\startCLOO[常量][描述]

\clCM{M_E_H}{e}
\clCM{M_LOG2E_H}{log_{2}e}
\clCM{M_LOG10E_H}{log_{10}e}
\clCM{M_LN2_H}{log_{e}2}
\clCM{M_LN10_H}{log_{e}10}
\clCM{M_PI_H}{\pi}
\clCM{M_PI_2_H}{\pi/2}
\clCM{M_PI_4_H}{\pi/4}
\clCM{M_1_PI_H}{1/\pi}
\clCM{M_2_PI_H}{2/\pi}
\clCM{M_2_SQRTPI_H}{2/\sqrt{\pi}}
\clCM{M_SQRT2_H}{\sqrt{2}}
\clCM{M_SQRT1_2_H}{1/\sqrt{2}}

\stopCLOO
}

\subsection{公共函式}

對\reftab{svCommonFunc}中所列的內建公共函式作了擴充,
函式引數和返回值也可以是 \cldt{half} 和 \cldt[n]{half},
參見\reftab{svCommonFuncHalf}。
現在, \cldt{gentype} 也包含 \cldt{half} 和 \cldt[n]{half},
其中 \ccmmsuffix{n} 可以是 2、 3、 4、 8、 16。

\startnotepar
可以使用化簡(如 \capi{mad} 或 \capi{fma})來實現 \capi{mix} 和 \capi{smoothstep}。
\stopnotepar

\placetable[here][tab:svCommonFuncHalf]
{內建公共函式}
{\startCLFD

\clFD{clamp_half}
\clFD{degrees}
\clFD{max_half}
\clFD{min_half}
\clFD{mix_half}
\clFD{radians}
\clFD{step_half}
\clFD{smoothstep_half}
\clFD{sign}

\stopCLFD
}

\subsection[sec:geomtricFunc]{幾何函式}

對\reftab{svGeometricFunc}中所列的內建幾何函式作了擴充,
函式引數和返回值也可以是 \cldt{half} 和 \cldt[n]{half},
參見\reftab{svGeometricFuncHalf}。
現在, \cldt{gentype} 也包含 \cldt{half} 和 \cldt[n]{half},
其中 \ccmmsuffix{n} 可以是 2、 3、 4。

\startnotepar
可以使用化簡(如 \capi{mad} 或 \capi{fma})來實現幾何函式。
\stopnotepar

\placetable[here][tab:svGeometricFuncHalf]
{內建幾何函式}
{\startCLFD

\clFD{cross_half}
\clFD{dot_half}
\clFD{distance_half}
\clFD{length_half}
\clFD{normalize_half}

\stopCLFD
}

\subsection[sec:relationFunc]{關係函式}

對\reftab{svRelationalFunc}中所列內建關係函式作了擴充,
引數可以為 \cldt{half} 和 \cldt[n]{half},
其中 \ccmmsuffix{n} 可以是 2、 3、 4、 8、 16,
參見\reftab{relationalFuncHalf}。

關係算子和相等算子(<、 <=、 >、 >=、 !=、 ==)也可用於矢量型別 \cldt[n]{half},
所產生的結果為 \cldt[n]{short},參見\refsec{operator}。

對於標量型別的引數,如果所指定的關係為 {\ftRef{false}},則下列函式(參見\reftab{svRelationalFunc})會返回 0,否則返回 1:
\startigBase[indentnext=no]
\item \capi{isequal}、 \capi{isnotequal}、
\item \capi{isgreater}、 \capi{isgreaterequal}、
\item \capi{isless}、 \capi{islessequal}、
\item \capi{islessgreater}、
\item \capi{isfinite}、 \capi{isinf}、
\item \capi{isnan}、 \capi{isnormal}、
\item \capi{isordered}、 \capi{isunordered} 和
\item \capi{signbit}。
\stopigBase
而對於矢量型別的引數,如果所指定的關係為 {\ftRef{false}},則返回 0,
否則返回 -1 (即所有位都是 1)。

如果任一引數為 NaN,則下列關係函式返回 0:
\startigBase[indentnext=no]
\item \capi{isequal}、
\item \capi{isgreater}、 \capi{isgreaterequal}、
\item \capi{isless}、 \capi{islessequal} 和
\item \capi{islessgreater}。
\stopigBase
如果引數為標量,則當任一引數為 NaN 時, \capi{isnotequal} 返回 1;
而如果引數為矢量,則當任一引數為 NaN 時, \capi{isnotequal} 返回 -1。

\placetable[here,split][tab:relationalFuncHalf]
{內建關係函式}
{\startCLFD

\clFD{isequal_half}
\clFD{isnotequal_half}
\clFD{isgreater_half}
\clFD{isgreaterequal_half}
\clFD{isless_half}
\clFD{islessequal_half}
\clFD{islessgreater_half}
\clFD{isfinite_half}
\clFD{isinf_half}
\clFD{isnan_half}
\clFD{isnormal_half}
\clFD{isordered_half}
\clFD{isunordered_half}
\clFD{signbit_half}
\clFD{bitselect_half}
\clFD{select_half}

\stopCLFD
}

\subsection[sec:vectorLsFuncHalf]{矢量數據裝載和存儲函式}

對\reftab{vectorLsFunc}中所列的矢量數據裝載(\clapi[n]{vload})和存儲(\clapi[n]{vstore})函式作了擴充,
可以讀寫 \cldt{half} 標量和矢量值,
參見\reftab{vectorLsFuncHalf}

對泛型 \cldt{gentype} 也作了擴充,包含 \cldt{half}。
而對泛型 \cldt[n]{gentype} 也作了擴充,包含了 \cldt[n]{half},
其中 \ccmmsuffix{n} 為 2、 3、 4、 8 或 16。

\startnotepar
\capi{vload3} 和 \capi{vstore3}
均由位址 \math{(\marg{p} + (\marg{offset}\times 3))} 讀寫矢量組件 \ccmm{x}、 \ccmm{y}、 \ccmm{z}。
\stopnotepar

\placetable[here,split][tab:vectorLsFuncHalf]
{矢量數據裝載、存儲函式表}
{\startCLFD
\clFD{vloadn}
\clFD{vstoren}
\stopCLFD
}

\subsection[sec:asyncCopyPrefetch]{在全局內存和局部內存間的異步拷貝以及預取}

OpenCL C 編程語言實現了\reftab{asyncCopyPrefetch}中所列函式,
可在\cnglo{glbmem}和\cnglo{locmem}間進行異步拷貝,
以及從\cnglo{glbmem}中預取(prefetch)。

對泛型 \ctype{gentype} 作了擴充,
包含 \cldt{half} 和 \cldt[n]{half},
其中 \ccmmsuffix{n} 可以是 2、 3、 4、 8、 16。

%\placetable[here][tab:asyncCopyPrefetch]
%{內建異步拷貝和預取函式}
%{% async_work_group_copy
\startbuffer[funcproto:async_work_group_copy]
event_t async_work_group_copy ( 
	__local gentype *dst, 
	const __global gentype *src, 
	size_t num_gentypes, 
	event_t event) 
event_t async_work_group_copy (
	__global gentype *dst,
	const __local gentype *src,
	size_t num_gentypes,
	event_t event)
\stopbuffer
\startbuffer[funcdesc:async_work_group_copy]
從 \carg{src} 異步拷貝 \carg{num_gentypes} 個 \ctype{gentype} 元素
到 \carg{dst} 中。
\cnglo{workgrp}中的所有\cnglo{workitem}都會實施此異步拷貝,
因此在\cnglo{workgrp}中,
使用相同引數值執行\cnglo{kernel}的所有\cnglo{workitem}必須都能執行到此函式,
否則結果未定義。

返回的\cnglo{evtobj}可由 \capi{wait_group_events} 用來等待異步拷貝完畢。
可以使用引數 \carg{event} 將 \capi{async_work_group_copy} 與之前的異步拷貝關聯在一起,
從而使得多個異步拷貝間可共享同一事件;否則 \carg{event} 必須是零。

如果引數 \carg{event} 非零,則會將其中的\cnglo{evtobj}返回。

此函式不會對源數據實施隱式同步,如在拷貝前執行 \capi{barrier}。
\stopbuffer

% async_work_group_strided_copy
\startbuffer[funcproto:async_work_group_strided_copy]
event_t async_work_group_strided_copy (
	__local gentype *dst,
	const __global gentype *src,
	size_t num_gentypes,
	size_t src_stride,
	event_t event)
event_t async_work_group_strided_copy (
	__global gentype *dst,
	const __local gentype *src,
	size_t num_gentypes,
	size_t dst_stride,
	event_t event)
\stopbuffer
\startbuffer[funcdesc:async_work_group_strided_copy]
從 \carg{src} 異步採集 \carg{num_gentypes} 個 \ctype{gentype} 元素
到 \carg{dst} 中。
參數 \carg{src_stride} 為從 \carg{src} 中讀取元素時所用跨距。
參數 \carg{dst_stride} 為將元素寫入 \carg{dst} 中時所用跨距。
\cnglo{workgrp}中的所有\cnglo{workitem}都會實施此異步採集,
因此在\cnglo{workgrp}中,
使用相同引數值執行\cnglo{kernel}的所有\cnglo{workitem}必須都能執行到此函式,
否則結果未定義。

返回的\cnglo{evtobj}可由 \capi{wait_group_events} 用來等待異步拷貝完畢。
可以使用引數 \carg{event} 將 \capi{async_work_group_strided_copy} 與之前的異步拷貝
關聯在一起,
從而使得多個異步拷貝間可共享同一事件;
否則 \carg{event} 必須是零。

如果引數 \carg{event} 非零,則會將其中的\cnglo{evtobj}返回。

此函式不會對源數據實施隱式同步,如在拷貝前執行 \capi{barrier}。

如果 \carg{src_stride} 或 \carg{dst_stride} 是 0,
或者拷貝時, \carg{src_stride} 或 \carg{dst_stride} 使得
 \carg{src} 或 \carg{dst} 指針超過了位址空間的上界,
則 \capi{async_work_group_strided_copy} 的行為未定義。
\stopbuffer

% wait_group_events
\startbuffer[funcproto:wait_group_events]
void wait_group_events (
	int num_events,
	event_t *event_list)
\stopbuffer
\startbuffer[funcdesc:wait_group_events]
等待用來表示 \capi{async_work_group_copy} 操作完成的事件。
實施等待後會釋放 \carg{event_list} 中的\cnglo{evtobj}。
對於某個\cnglo{workgrp}中的\cnglo{workitem}而言,
如果執行\cnglo{kernel}時
用的 \carg{num_events} 以及 \carg{event_list} 中的\cnglo{evtobj}一樣,
則它們必須都能執行到此函式;
否則結果未定義。
\stopbuffer

% prefetch
\startbuffer[funcproto:prefetch]
void prefetch (
	const __global gentype *p,
	size_t num_gentypes)
\stopbuffer
\startbuffer[funcdesc:prefetch]
預取 \math{\text{\carg{num_gentypes}}
 \times \text{\capi{sizeof}}(\text{\ctype{gentype}})} 字節到全局緩存中。
預取指令會作用到\cnglo{workgrp}中的\cnglo{workitem}上,
不會影響\cnglo{kernel}的功能性行為。
\stopbuffer


% begin table
\startCLFD
\clFD{async_work_group_copy}
\clFD{async_work_group_strided_copy}
\clFD{wait_group_events}
\clFD{prefetch}
\stopCLFD
}


\section{由 GL 上下文或共享組創建 CL 上下文}

% Overview
\subsection{概覽}

\refsec{clShareGl}中定義了如何與 OpenGL 實作中的材質和緩衝對象共享數據,
但對於如何在 OpenCL \cnglo{context}和 OpenGL \cnglo{context}或共享組間建立聯繫,
則沒有定義。
此擴展為創建 OpenCL \cnglo{context}的例程定義了一個可選特性,
可以將 GL \cnglo{context}或共享組對象與新建的 OpenCL \cnglo{context}關聯起來。
如果實作支持此擴展,則\reftab{cldevquery}中所描述的 \cenum{CL_PLATFORM_EXTENSIONS} 或
\cenum{CL_DEVICE_EXTENSIONS} 中將包含字串 \clext{cl_khr_gl_sharing}。

此擴展要求 OpenGL 實作支持\cnglo{bufobj},以及與 OpenCL 共享材質和\cnglo{bufobj}圖像。

% New Procedures and Functions
\subsection{新的程序和函式}

\topclfunc{clGetGLContextInfoKHR}

\startclc
cl_intclGetGLContextInfoKHR (
		const cl_context_properties *properties,
		cl_gl_context_info param_name,
		size_t param_value_size,
		void *param_value,
		size_t *param_value_size_ret);
\stopclc

% New Tokens
\subsection{新的符記}

如果 \carg{properties} 中所指定的 OpenGL \cnglo{context}或共享組對象句柄無效,
則 \clapi{clCreateContext}、 \clapi{clCreateContextFromType} 和 \clapi{clGetGLContextInfoKHR} 會返回:
\startclc
CL_INVALID_GL_SHAREGROUP_REFERENCE_KHR		-1000
\stopclc

\clapi{clGetGLContextInfoKHR} 的引數 \carg{param_name} 接受下列值:
\startclc
CL_CURRENT_DEVICE_FOR_GL_CONTEXT_KHR		0x2006
CL_DEVICES_FOR_GL_CONTEXT_KHR			0x2007
\stopclc

\clapi{clCreateContext} 和 \clapi{clCreateContextFromType} 的引數 \carg{properties} 接受下列特性名:
\startclc
CL_GL_CONTEXT_KHR	0x2008
CL_EGL_DISPLAY_KHR	0x2009
CL_GLX_DISPLAY_KHR	0x200A
CL_WGL_HDC_KHR		0x200B
CL_CGL_SHAREGROUP_KHR	0x200C
\stopclc

%  Additions to Chapter 4 of the OpenCL 1.2 Specification
\subsection{對第 4 章的補充}

\refsec{contexts}中,用下列內容取代 \clapi{clCreateContext} 後面對 \carg{properties} 的描述:

\carg{properties} 指向一個特性列,即 \ccmm{<特性名, 值>} 的陣列,此陣列已排好序,以零終止。
如果此陣列中沒有某個特性,則使用其缺省值,參見\reftab{prptForclCreateContext}。
如果 \carg{properties} 是 \cenum{NULL} 或者是空的(第一個值就是零),所有特性都使用缺省值。

\refsec{clShareGl}中定義了一些特性,
用來控制如何與 OpenGL 緩衝、材質和渲染緩衝對象共享 OpenCL \cnglo{memobj}。

可以設置下列特性來識別 OpenGL \cnglo{context},當然,
這取決於一些特定\cnglo{platform}的 API (用來將 OpenGL \cnglo{context}綁定到視窗系統上):
\startigBase
\item 如果支持 CGL\footnote{CGL 是 Mac OS X 的 OpenGL 接口。} 綁定 API,
應當將特性 \cenum{CL_CGL_SHAREGROUP_KHR} 設置為一個 CGLShareGroup 句柄,
指向一個 CGL 共享組對象。

\item 如果支持 EGL\footnote{%
EGL 是 Khronos 渲染 API (如 OpenGL ES 或 OpenVG)和底層原生平台視窗系統之間的接口。%
} 綁定 API,
應當將特性 \cenum{CL_GL_CONTEXT_KHR} 設置為一個 EGLContext 句柄,
指向一個 OpenGL ES 或 OpenGL \cnglo{context},
而將特性 \cenum{CL_EGL_DISPLAY_KHR} 設置為一個 EGLDisplay 句柄,
指向用於創建這個 OpenGL ES 或 OpenGL \cnglo{context}的顯示屏。

\item 如果支持 GLX\footnote{GLX 是 X11 的 OpenGL 接口。} 綁定 API,
應當將特性 \cenum{CL_GL_CONTEXT_KHR} 設置為一個 GLXContext 句柄,
指向一個 OpenGL \cnglo{context},
而將特性 \cenum{CL_GLX_DISPLAY_KHR} 設置為一個 Display 句柄,
指向用於創建這個 OpenGL \cnglo{context}的 X 視窗系統顯示屏。

\item 如果支持 WGL\footnote{WGL 是 Microsoft Windows 的 OpenGL 接口。} 綁定 API,
應當將特性 \cenum{CL_GL_CONTEXT_KHR} 設置為一個 HGLRC 句柄,
指向一個 OpenGL \cnglo{context},
而將特性 \cenum{CL_WGL_HDC_KHR} 設置為一個 HDC 句柄,
指向用於創建這個 OpenGL \cnglo{context}的顯示屏。
\stopigBase

如果是在這樣的\cnglo{context}中創建的\cnglo{memobj},
那麼他可以被指定的 OpenGL 或 OpenGL ES \cnglo{context}
(也包括此\cnglo{context}的共享列中其他 OpenGL \cnglo{context},參見 GLX 1.4 和 EGL 1.4 規範,以及 Microsoft Windows 上 OpenGL 實作 WGL 的文檔)、
或者 CGL 共享組所共享。

如果特性列中沒有指定 OpenGL 或 OpenGL ES \cnglo{context}或者共享組,
那麼就不能共享\cnglo{memobj},
並且調用\refsec{clShareGl}中的\cnglo{cmd}時
會導致錯誤 \cenum{CL_INVALID_GL_SHAREGROUP_REFERENCE_KHR}。

OpenCL / OpenGL 間的共享不支持屬性 \cenum{CL_CONTEXT_INTEROP_USER_SYNC}
 (參見\reftab{prptForclCreateContext})。
如果創建帶有 OpenCL / OpenGL 共享的\cnglo{context}時指定了此屬性,則會返回錯誤。

\reftab{prptForclCreateContextPF}是對\reftab{prptForclCreateContext}的補充。

\placetable[here][tab:prptForclCreateContextPF]
{創建上下文時所用特性}
{\startETD[cl_context_properties][屬性值]

\clETD{CL_GL_CONTEXT_KHR}{OpenGL \cnglo{context}句柄}{
OpenCL \cnglo{context}所關聯的 OpenGL \cnglo{context}。
缺省值為 \cenumemp{0}。
}

\clETD{CL_CGL_SHAREGROUP_KHR}{OpenGL 共享組句柄}{
OpenCL \cnglo{context}所關聯的 CGL 共享組。
缺省值為 \cenumemp{0}。
}

\clETD{CL_EGL_DISPLAY_KHR}{EGLDisplay 句柄}{
OpenGL \cnglo{context}所對應的 EGLDisplay。
缺省值為 \cenumemp{EGL_NO_DISPLAY}。
}

\clETD{CL_GLX_DISPLAY_KHR}{X 句柄}{
OpenGL \cnglo{context}所對應的 X Display。
缺省值為 \cenumemp{None}。
}

\clETD{CL_WGL_HDC_KHR}{HDC 句柄}{
OpenGL \cnglo{context}所對應的 HDC。
缺省值為 \cenumemp{0}。
}

\stopETD

}

下列內容取代 \clapi{clCreateContext} 所返回的錯誤列中的第一個:
\startreplacepar
如果\cnglo{context}由下列任一方式指定:
\startigBase[indentnext=no]
\item 通過設置特性 \cenum{CL_GL_CONTEXT_KHR} 和 \cenum{CL_EGL_DISPLAY_KHR} 為
基於 EGL 的 OpenGL ES 或 OpenGL 實作指定了一個\cnglo{context}。

\item 通過設置特性 \cenum{CL_GL_CONTEXT_KHR} 和 \cenum{CL_GLX_DISPLAY_KHR} 為
基於 GLX 的 OpenGL 實作指定了一個\cnglo{context}。

\item 通過設置特性 \cenum{CL_GL_CONTEXT_KHR} 和 \cenum{CL_WGL_HDC_KHR} 為
基於 WGL 的 OpenGL 實作指定了一個\cnglo{context}。
\stopigBase
並且滿足下列任一條件:
\startigBase[indentnext=no]
\item 所指定的 display 和\cnglo{context}特性不能標識一個有效的 OpenGL 或 OpenGL ES \cnglo{context}。

\item 所指定的\cnglo{context}不支持\cnglo{bufobj}和渲染緩衝對象。

\item 所指定的\cnglo{context}與所創建的 OpenCL \cnglo{context}不兼容。
例如,位於物理上不同的位址空間內,如另一個硬件設備上;或者由於實作的局限不支持與 OpenCL 共享數據。
\stopigBase
則 \carg{errcode_ret} 會返回 \cenum{CL_INVALID_GL_SHAREGROUP_REFERENCE_KHR}。

如果通過設置特性 \cenum{CL_CGL_SHAREGROUP_KHR} 為基於 CGL 的 OpenGL 實作指定了一個共享組,
但是所指定的共享組不能標識一個有效的 CGL 共享組對象,
那麼 \carg{errcode_ret} 會返回 \cenum{CL_INVALID_GL_SHAREGROUP_REFERENCE_KHR}。

如果按上面所描述的那樣指定了一個\cnglo{context},並且滿足下列任一條件:
\startigBase[indentnext=no]
\item 為 CGL、 EGL、 GLX 或 WGL 其中之一指定了一個\cnglo{context}或共享組對象,
但是 OpenGL 實作不支持視窗系統綁定 API。

\item 為 \cenum{CL_CGL_SHAREGROUP_KHR}、 \cenum{CL_EGL_DISPLAY_KHR}、
 \cenum{CL_GLX_DISPLAY_KHR} 以及 \cenum{CL_WGL_HDC_KHR} 中一個以上的特性設置了非缺省值。

\item 為特性 \cenum{CL_CGL_SHAREGROUP_KHR} 和 \cenum{CL_GL_CONTEXT_KHR} 都設置了非缺省值。

\item 引數 \carg{devices} 中任一\cnglo{device}不支持 OpenCL 對象與 OpenGL 對象共享數據存儲,
參見\refsec{clShareGl}。
\stopigBase
則 \carg{errcode_ret} 會返回 \cenum{CL_INVALID_OPERATION}。

如果 \carg{properties} 中任一特性名無效
或者 \carg{properties} 中有特性 \cenum{CL_CONTEXT_INTEROP_USER_SYNC},
則 \carg{errcode_ret} 會返回 \cenum{CL_INVALID_PROPERTY}。
\stopreplacepar

下列內容取代 \clapi{clCreateContextFromType} 中對 \carg{properties} 的描述:
\startreplacepar
\carg{properties} 指向一個特性列,
其格式以及有效內容與 \clapi{clCreateContext} 的引數 \carg{properties} 相同。
\stopreplacepar

用上面所描述的兩個新錯誤取代 \clapi{clCreateContextFromType} 的錯誤列中的第一個。

% Additions to section 9.7 of the OpenCL 1.2 Extension Specification
\subsection{對節 9.7 的補充}

下列內容作為{\ftRef{節 9.7.7}}:
\startreplacepar
可以查詢 OpenGL \cnglo{context}對應的 OpenCL \cnglo{device}。
不一定有這樣的\cnglo{device}
(例如, OpenGL \cnglo{context}所在 GPU 可能不支持 OpenCL \cnglo{cmdq},
但卻支持共享的 CL / GL 對象),即使有,也可能隨時間發生變化。
如果存在這樣的\cnglo{device},
在此\cnglo{device}所對應的\cnglo{cmdq}上兼并和釋放共享的 CL / GL 對象
可能會比在 OpenCL \cnglo{context}可用的其他\cnglo{device}所對應的\cnglo{cmdq}上要快一些。
用以下函式查詢當前所對應的\cnglo{device}:

\topclfunc{clGetGLContextInfoKHR}

\startCLFUNC
cl_int clGetGLContextInfoKHR (
		const cl_context_properties *properties,
		cl_gl_context_info param_name,
		size_t param_value_size,
		void *param_value,
		size_t *param_value_size_ret)
\stopCLFUNC

\carg{properties} 指向一個特性列,
其格式和有效內容與 \clapi{clCreateContext} 的引數 \carg{properties} 相同。
\carg{properties} 必須能夠識別唯一一個有效的 GL \cnglo{context}或 \cnglo{glsharegrp}對象。

\carg{param_name} 是一個常量,指定所要查詢的 GL \cnglo{context}資訊,
有效值參見\reftab{ctxprop}。

\placetable[here][tab:ctxprop]
{可以用 \clapi{clGetGLContextInfoKHR} 查詢的 GL 上下文資訊}
{\startETD[cl_gl_context_info][返回型別]

\clETD{CL_CURRENT_DEVICE_FOR_GL_CONTEXT_KHR}{cl_device_id}{
返回與指定 OpenGL \cnglo{context}目前所關聯的 CL \cnglo{device}。
}

\clETD{CL_DEVICES_FOR_GL_CONTEXT_KHR}{cl_device_id[]}{
返回與指定 OpenGL \cnglo{context}所關聯的所有 CL \cnglo{device}。
}

\stopETD

}

\carg{param_value} 所指內存用來存儲查詢結果,參見\reftab{ctxprop}。
如果其值為 \cenum{NULL},則忽略。

\carg{param_value_size} 為 \carg{param_value} 所指內存的字節數。
其值必須大於等於\reftab{ctxprop}中返回型別的大小。

\carg{param_value_size_ret} 返回查詢結果的實際字節數。
如果其值為 \cenum{NULL},則忽略。

如果執行成功, \clapi{clGetGLContextInfoKHR} 會返回 \cenum{CL_SUCCESS}。
如果 \carg{param_name} 沒有對應的\cnglo{device},
調用不會失敗,但是 \carg{param_value_size_ret} 的值將會是零。

如果\cnglo{context}由下列任一方式指定:
\startigBase[indentnext=no]
\item 通過設置特性 \cenum{CL_GL_CONTEXT_KHR} 和 \cenum{CL_EGL_DISPLAY_KHR} 為
基於 EGL 的 OpenGL ES 或 OpenGL 實作指定了一個\cnglo{context}。

\item 通過設置特性 \cenum{CL_GL_CONTEXT_KHR} 和 \cenum{CL_GLX_DISPLAY_KHR} 為
基於 GLX 的 OpenGL 實作指定了一個\cnglo{context}。

\item 通過設置特性 \cenum{CL_GL_CONTEXT_KHR} 和 \cenum{CL_WGL_HDC_KHR} 為
基於 WGL 的 OpenGL 實作指定了一個\cnglo{context}。
\stopigBase
並且滿足下列任一條件:
\startigBase[indentnext=no]
\item 所指定的 display 和\cnglo{context}特性不能標識一個有效的 OpenGL 或 OpenGL ES \cnglo{context}。

\item 所指定的\cnglo{context}不支持\cnglo{bufobj}和渲染緩衝對象。

\item 所指定的\cnglo{context}與所創建的 OpenCL \cnglo{context}不兼容。
例如,位於物理上不同的位址空間內,如另一個硬件設備上;或者由於實作的局限不支持與 OpenCL 共享數據。
\stopigBase
則 \clapi{clGetGLContextInfoKHR} 會返回 \cenum{CL_INVALID_GL_SHAREGROUP_REFERENCE_KHR}。

如果通過設置特性 \cenum{CL_CGL_SHAREGROUP_KHR} 為基於 CGL 的 OpenGL 實作指定了一個共享組,
但是所指定的共享組不能標識一個有效的 CGL 共享組對象,
那麼 \clapi{clGetGLContextInfoKHR} 會返回 \cenum{CL_INVALID_GL_SHAREGROUP_REFERENCE_KHR}。

如果按上面所描述的那樣指定了一個\cnglo{context},並且滿足下列任一條件:
\startigBase[indentnext=no]
\item 為 CGL、 EGL、 GLX 或 WGL 其中之一指定了一個\cnglo{context}或共享組對象,
但是 OpenGL 實作不支持視窗系統綁定 API。

\item 為 \cenum{CL_CGL_SHAREGROUP_KHR}、 \cenum{CL_EGL_DISPLAY_KHR}、
 \cenum{CL_GLX_DISPLAY_KHR} 以及 \cenum{CL_WGL_HDC_KHR} 中一個以上的特性設置了非缺省值。

\item 為特性 \cenum{CL_CGL_SHAREGROUP_KHR} 和 \cenum{CL_GL_CONTEXT_KHR} 都設置了非缺省值。

\item 引數 \carg{devices} 中任一\cnglo{device}不支持 OpenCL 對象與 OpenGL 對象共享數據存儲,
參見\refsec{clShareGl}。
\stopigBase
則 \clapi{clGetGLContextInfoKHR} 會返回 \cenum{CL_INVALID_OPERATION}。

如果 \carg{properties} 中任一特性名無效,
則 \carg{errcode_ret} 會返回 \cenum{CL_INVALID_VALUE}。
\problem{maybe should CL_INVALID_PROPERTIES}

另外,如果 \carg{param_name} 無效(參見\reftab{ctxprop}),
或者 \carg{param_value_size} 的值小於\reftab{ctxprop}中返回型別的大小,
並且 \carg{param_value} 不是 \cenum{NULL},
則 \clapi{clGetGLContextInfoKHR} 會返回 \cenum{CL_INVALID_VALUE}。
如果\schostfailres,則返回 \cenum{CL_OUT_OF_RESOURCES}。
如果\scdevfailres,則返回 \cenum{CL_OUT_OF_HOST_MEMORY}。

\stopreplacepar

% Issues
\subsection{問題}

\startQUESTION
創建所關聯的 OpenCL \cnglo{context}時,如何識別 OpenGL \cnglo{context}?
\stopQUESTION
\startANSWER
已解決:使用特性對 \ccmm{(display, context handle)} 來
識別任一 OpenGL 或 OpenGL ES \cnglo{context}
(此\cnglo{context}與某一視窗系統綁定層,即 EGL、 GLX 或 WGL 相關);
用共享組句柄來識別 CGL 共享組。
如果指定了一個\cnglo{context},對於調用 \clapi{clCreateContext} 的線程而言,
\cnglo{context}不必是最近的。
\stopANSWER


\section[sec:clShareGl]{與 OpenGL/OpenGL ES 緩衝、材質和渲染緩衝對象共享內存對象}

本節中的 OpenCL 函式允許\cnglo{app}將 OpenGL 緩衝、材質和渲染緩衝對象用作 OpenCL \cnglo{memobj}。
這樣可以在 OpenCL 和 OpenGL 間高效的共享數據。
 OpenCL API 所執行的\cnglo{kernel}即可讀寫\cnglo{memobj},也可讀寫 OpenGL 對象。

OpenCL \cnglo{imgobj}可能是由 OpenGL 材質或渲染緩衝對象創建的。
OopenCL \cnglo{bufobj}可能是由 OpenGL \cnglo{bufobj}創建的。

當且僅當 OpenCL \cnglo{context}是由 OpenGL 共享組對象或\cnglo{context}創建的,
才能由 OpenGL 對象創建 OpenCL \cnglo{memobj}。
OpenGL 共享組和\cnglo{context}都是由特定平台 API 創建的,
如 EGL、 CGL、 WGL 以及 GLX。
在 MacOS X 上, OpenCL \cnglo{context}可能是
用 OpenCL \cnglo{platform}擴展 \clext{cl_apple_gl_sharing} 由 OpenGL 共享組對象創建的。
在其他\cnglo{platform}上,包括 Microsoft Windows、 Linux/Unix 等等,
 OpenCL \cnglo{context}可能是
用 Khronos \cnglo{platform}擴展 \clext{cl_khr_gl_sharing} 由 OpenGL \cnglo{context}創建的。
請參考對應的 OpenCL 實作的\cnglo{platform}文檔,
或者訪問 \from[khronosRegistryCL] 以獲取更多資訊。

對於 GL 共享組對象、或 GL \cnglo{context}
(由其創建的 CL \cnglo{context})所關聯的共享組中
所定義的任何 OpenGL 對象,只要支持,就可以被共享,
但是有個例外,就是缺省的 OpenGL 對象(即命名為零的對象)不能被共享。

\subsection{共享對象的生命周期}

只要對應的 GL 對象沒有被刪除,
則由其創建的 OpenCL \cnglo{memobj}(下文中稱為“共享的 CL/GL 對象”)就會一直有效。
如果通過 GL API (如 \capi{glDeleteBuffers}、 \capi{glDeleteTextures} 或 \capi{glDeleteRenderbuffers})刪除了 GL 對象,
則後續使用 CL \cnglo{bufobj}或\cnglo{imgobj}時會導致\cnglo{undef}的行為,
包括但不限於 CL 錯誤和數據訛誤,但不會使\cnglo{program}終止。

CL \cnglo{context}和對應的\cnglo{cmdq}依賴於 GL 共享組對象、
或 GL \cnglo{context}(由其創建的 CL \cnglo{context})所關聯的共享組的存在。
如果銷毀了 GL 共享組對象或者共享組中的所有 GL \cnglo{context},
則使用 CL \cnglo{context}或\cnglo{cmdq}會導致\cnglo{undef}的行為,
包括\cnglo{program}終止。
在銷毀對應的 GL 共享組或\cnglo{context}之前,
\cnglo{app}應當先銷毀 CL \cnglo{cmdq}和 CL \cnglo{context}。




\section[sec:clEvtObjFromGlSync]{由 GL 同步對象創建 CL 事件對象}

% Overview
\subsection{簡介}

本擴展允許由 OpenGL 隔柵同步對象創建 OpenCL \cnglo{evtobj},
從而潛在地提高在兩種 API 間共享圖像和緩衝的效率。
對應的擴展 \clext{GL_ARB_cl_event} 可以由 OpenCL \cnglo{evtboj}創建 OpenGL 同步對象,
他可以與本擴展互補。

另外,本擴展修改了 \clapi{clEnqueueAcquireGLObjects} 和 \clapi{clEnqueueReleaseGLObjects} 的行為,
如果 OpenGL \cnglo{context} 與 OpenCL \cnglo{context} 綁定到了同一個線程中,
則他可以隱式保證其同步。

如果實作支持本擴展,
則 \cenum{CL_PLATFORM_EXTENSIONS} 或 \cenum{CL_DEVICE_EXTENSIONS} 中
應該有字串 \clext{cl_khr_gl_event},
參見\reftab{cldevquery}。

% New Procedures and Functions
\subsection{新例程和新函式}

\startCLFUNC
cl_event clCreateEventFromGLsyncKHR (
			cl_context context,
			GLsync sync,
			cl_int *errcode_ret);
\stopCLFUNC

% New Tokens
\subsection{新的符記}

調用 \clapi{clGetEventInfo} 時
如果 \carg{param_name} 是 \cenum{CL_EVENT_COMMAND_TYPE},則會返回:
\startclc
CL_COMMAND_GL_FENCE_SYNC_OBJECT_KHR	0x200D
\stopclc

% Additions to Chapter 5 of the OpenCL 1.2 Specification
\subsection{對第五章的補充}

將下面內容添加到\refsec{evtObj}的第四段中(\clapi{clCreateUserEvent} 的描述之前):
\startreplacepar
\cnglo{evtobj}也可用來反映 OpenGL 同步對象的狀態。
同步對象指的是 OpenGL 命令流中所執行的隔柵(fence)命令。
這為在 OpenGL 和 OpenCL 間共享緩衝和圖像提供了一種新方法(參見\refsec{syncCLGL})。
\stopreplacepar

在\reftab{clGetEventInfo}關於 \clapi{clGetEventInfo} 的描述中,
\carg{param_name} 為 \cenum{CL_EVENT_COMMAND_TYPE} 時,
所返回的 \carg{param_value} 的值增加一項: \cenum{CL_COMMAND_GL_FENCE_SYNC_OBJECT_KHR}。

新添{\ftRef{節 5.9.1}} {\ftEmp{將\cnglo{evtobj}鏈接到 OpenGL 同步對象上}}:
\startreplacepar
可以通過鏈接 OpenGL {\ftEmp{同步對象}}來創建\cnglo{evtobj}。
這種\cnglo{evtobj}的完成就相當於等待與所鏈接 GL 同步對象相關聯隔柵命令的完成。

\topclfunc{clCreateEventFromGLsyncKHR}

\startCLFUNC
cl_event clCreateEventFromGLsyncKHR (
			cl_context context,
			GLsync sync,
			cl_int *errcode_ret)
\stopCLFUNC

此函式會創建一個帶鏈接的\cnglo{evtobj}。

\carg{context} 是一個利用擴展 \clext{cl_khr_gl_sharing},
由 OpenGL \cnglo{context}或共享組創建的的 OpenCL \cnglo{context}。

\carg{sync} 是 \carg{context} 所關聯 GL 共享組中同步對象的名字。

如果成功創建了\cnglo{evtobj},則 \clapi{clCreateEventFromGLsyncKHR} 會將其返回,
並將 \carg{errcode_ret} 置為 \cenum{CL_SUCCESS}。
否則,返回 \cmacro{NULL},並將 \carg{errcode_ret} 置為下列錯誤碼之一:
\startigBase
\item \cenum{CL_INVALID_CONTEXT},
如果 \carg{context} 無效,或者不是由 GL \cnglo{context}創建的。

\item \cenum{CL_INVALID_GL_OBJECT},
如果 \carg{sync} 不是 \carg{context} 所關聯 GL 共享組中同步對象的名字。
\stopigBase

對於這種\cnglo{evtobj}調用 \clapi{clGetEventInfo} 時會返回下列值:
\startigBase
\item 如果查詢的是 \cenum{CL_EVENT_COMMAND_QUEUE},則結果為 \cmacro{NULL},
因為此事件沒有與任何 OpenCL \cnglo{cmdq}關聯。

\item 如果查詢的是 \cenum{CL_EVENT_COMMAND_TYPE},
則結果是 \cenum{CL_COMMAND_GL_FENCE_SYNC_OBJECT_KHR},
表明此事件關聯的是 GL 同步對象,而不是 OpenCL \cnglo{cmd}。

\item 如果查詢的是 \cenum{CL_EVENT_COMMAND_EXECUTION_STATUS},
則結果要麼是 \cenum{CL_SUBMITTED},表明同步對象所關聯的隔柵\cnglo{cmd}還未完成,
要麼是 \cenum{CL_COMPLETE},表明隔柵\cnglo{cmd}完成了。
\stopigBase

\clapi{clCreateEventFromGLsyncKHR} 會在返回的\cnglo{evtobj}上實施隱式的 \clapi{clRetainEvent}。
創建這種帶鏈接的\cnglo{evtobj}時也會在所鏈接的 GL 同步對象上放置一個引用。
當這種\cnglo{evtobj}被刪除時,這個引用也會被移除。

\clapi{clCreateEventFromGLsyncKHR} 所返回的事件可能
只能由 \clapi{clEnqueueAcquireGLObjects} 使用。
將這種事件傳遞給其他 CL API 會生成錯誤 \cenum{CL_INVALID_EVENT}。
\stopreplacepar % stop replace par

% Additions to Chapter 9 of the OpenCL 1.2 Specification
\subsection{對第九章的補充}

對 \clapi{clEnqueueAcquireGLObjects} 的參數 \carg{event} 增加以下描述:
\startreplacepar
如果一個 OpenGL \cnglo{context}綁定到了當前線程上,
那麼同時滿足下列兩個條件的 OpenGL \cnglo{cmd}會在執行
緊跟 \clapi{clEnqueueAcquireGLObjects} 的所有 OpenCL \cnglo{cmd}
(這些 OpenCL \cnglo{cmd}會影響或存取 \carg{mem_objects} 中的\cnglo{memobj})前完成:
\startigNum[indentnext=no]
\item 會影響或存取 \carg{mem_objects} 中\cnglo{memobj}的內容;
\item 是在調用 \clapi{clEnqueueAcquireGLObjects} 之前在此 OpenGL \cnglo{context}上發起的。
\stopigNum
如果所返回的\cnglo{evtobj}不是 \cmacro{NULL},
則只有當這樣的 OpenGL \cnglo{cmd}完成後,他才會報告完成。
\stopreplacepar

對 \clapi{clEnqueueReleaseGLObjects} 的參數 \carg{event} 增加以下描述:
\startreplacepar
如果一個 OpenGL \cnglo{context}綁定到了當前線程上,
那麼只有當 \clapi{clEnqueueReleaseGLObjects} 之前的所有 OpenCL \cnglo{cmd}
(這些 OpenCL \cnglo{cmd}會影響或存取 \carg{mem_objects} 中的\cnglo{memobj})
執行完畢後,
同時滿足下列兩個條件的 OpenGL \cnglo{cmd}才會執行:
\startigNum[indentnext=no]
\item 會影響或存取 \carg{mem_objects} 中\cnglo{memobj}的內容;
\item 是在調用 \clapi{clEnqueueReleaseGLObjects} 之後在此 OpenGL \cnglo{context}上發起的。
\stopigNum
如果所返回的\cnglo{evtobj}不是 \cmacro{NULL},
則只有當他報告完成後,才會執行這些 OpenGL \cnglo{cmd}。
\stopreplacepar

用下列內容取代\refsec{syncCLGL}的第二段:
\startreplacepar
調用 \clapi{clEnqueueAcquireGLObjects} 之前,
\cnglo{app}必須確保會存取 \carg{mem_objects} 中對象並且處於擱置狀態的 OpenGL 操作全部完成。

如果支持擴展 \clext{cl_khr_gl_event},
並且 OpenGL \cnglo{context}與 OpenCL \cnglo{context}綁定到了同一線程上,
則 OpenCL 實作會確保(針對此 OpenGL \cnglo{context})這種擱置的 OpenGL 操作全部完成。
這也叫{\ftRef{隱式同步}}。

如果支持擴展 \clext{cl_khr_gl_event},
並且所談及的 OpenGL \cnglo{context}支持隔柵同步對象,
要想確定 OpenGL \cnglo{cmd}完成了,
可以用 \capi{glFenceSync} 在那些\cnglo{cmd}後面放置一個 GL 隔柵\cnglo{cmd},
然後用 \clapi{clCreateEventFromGLsyncKHR} 由所產生的 GL 同步對象創建一個事件,
並通過 \clapi{clEnqueueAcquireGLObjects} 來確定這個\cnglo{evtobj}完成了。
這種方法可能比 \clapi{glFinish} 更加高效,
稱作{\ftRef{顯式同步}}。
當 OpenGL \cnglo{context}綁定的是存取\cnglo{memobj}的另一個線程時,
顯式同步最有用。

如果支持擴展 \clext{cl_khr_gl_event},
要想確定 OpenGL \cnglo{cmd}完成了,
可以在所有帶有對這些對象的擱置引用的 OpenGL \cnglo{context}上
發起並等待\cnglo{cmd} \clapi{glFinish} 的完成。
一些實作可能提供其他有效的同步方法。
如果存在這樣的方法,會在特定\cnglo{platform}的文檔中對其進行描述。

注意,對於所有 OpenGL 實作以及所有 OpenCL 實作,
只有 \clapi{glFinish} 才是可移植的,其他方法都不是。
鑒於 \clapi{glFinish} 是一種代價高昂的操作,
如果在某個\cnglo{platform}上支持擴展 \clext{cl_khr_gl_event},
應盡量避免使用 \clapi{glFinish}。
\stopreplacepar

% Issues
\subsection{問題}

\startQUESTION
如何處理 CL 事件和 GL 同步的相互引用?
\stopQUESTION
\startANSWER
已有提案:帶鏈接的 CL 事件會在 GL 同步對象上放置一個引用。
刪除 CL 事件時會移除這個引用。
還有一些代價更高的方案可以通過 GL 同步來反映 CL 事件\cnglo{refcnt}的變化。
\stopANSWER

\startQUESTION
在其他 API 中如何處理到同步基元的鏈接?
\stopQUESTION
\startANSWER
還未解決。
我們想至少要有一種方式可以將事件鏈接到 EGL 同步對象上。
可能沒有模擬 DX 的概念。
對於所鏈接的每種同步基元應當都有一個入口點,
如 \clapi{clCreateEventFromEGLSyncKHR}。

另一種方案就是通用的 \clapi{clCreateEventFromExternalEvent},可以接受一個特性列。
這個特性列中可以包含外部基元的類型以及其附屬資訊
(GL 同步對象句柄、 EGL display 以及同步對象句柄,等等)。
這樣就可以重用同一個入口點。

這些可能會作為一個獨立的擴展。
\stopANSWER

\startQUESTION
\cenum{CL_EVENT_COMMAND_TYPE} 對應的是\cnglo{cmd}(隔柵)的類型還是
所鏈接同步對象的類型?
\stopQUESTION
\startANSWER
已有提案:所鏈接同步對象的類型。
\stopANSWER

\startQUESTION
是否要同時支持顯式同步和隱式同步?
\stopQUESTION
\startANSWER
已有提案:是的。
隱式同步適用於 GL 和 CL 在同一\cnglo{app}線程中執行的情況。
而顯式同步則適用於在不同線程中執行、但 \capi{glFinish} 代價又太高的情況。
\stopANSWER

\startQUESTION
本擴展是\cnglo{platform}擴展還是\cnglo{device}擴展?
\stopQUESTION
\startANSWER
已有提案:\cnglo{platform}擴展。
這樣的話,要想僅僅使用公開的 GL API 來實現 sync->event 語義需要做大量的工作;
但是,同一運行時中可能會有多個對 GL 支持層級不同的驅動和\cnglo{device}同時存在。
\stopANSWER

\startQUESTION
什麼地方才能使用由 GL 同步對象生成的事件?
\stopQUESTION
\startANSWER
已有提案:僅當調用 \clapi{clEnqueueAcquireGLObjects} 時才能使用,
其他任何地方使用這種事件都會產生錯誤。
其他地方也沒有明確的用例,而且要想支持他還有一個成本問題,
在所有其他將其作為參數的\cnglo{cmd}中檢查事件源都要有相應的成本,
經過權衡,採用了目前的方案。
\stopANSWER



\stopcomponent

