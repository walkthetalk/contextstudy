\startcomponent cpnoptext
\product opencl-spec-zh

\chapter[chapter:optExt]{可選擴展}

%OpenCL 1.2 所支持的可選特性在《OpenCL 1.2 擴展規範》中有所描述。
本章列出了 OpenCL 1.2 所支持的可選特性。
一些 OpenCL \cnglo{device} 可能支持這些可選擴展。
對於符合 OpenCL 的實作而言,不要求支持這些可選擴展,
但仍然希望這些可選擴展能夠獲取廣泛的支持。
將來修訂 OpenCL 規範時,可能將這些可選擴展中所定義的功能移入必需的特性集中。
下面簡單描述了這些擴展的定義。

OpenCL 工作組所批准的 OpenCL 擴展均遵守寫列命名約定:
\startigBase
\item 每個擴展都有一個唯一的{\ftRef{名字字串}},形如“\clext{cl_khr_<名字>}”。
如果實作支持此擴展,
則此字串會出現在\reftab{cldevquery}中的 \cenum{CL_PLATFORM_EXTENSIONS} 或 \cenum{CL_DEVICE_EXTENSIONS} 中。

\item 對於擴展中所定義的所有 API 函式,其名字均形如:\clapi{cl<函式名>KHR}。

\item 對於擴展中所定義的所有枚舉,其名字均形如:\clapi{CL_<枚舉名>_KHR}。
\stopigBase

後續修訂 OpenCL 時, OpenCL 工作組所批准的 OpenCL 擴展都可能被{\ftRef{提升}}為必需的核心特性。
一旦如此,就會將相應的擴展規範合併到核心規範中。
而對於其中的函式和枚舉,則會移除其後綴 {\ftEmp{KHR}}。
對於相應的 OpenCL 實作而言,
仍然要在 \cenum{CL_PLATFORM_EXTENSIONS} 或 \cenum{CL_DEVICE_EXTENSIONS} 中導出此擴展名,
而且也要支持帶有後綴 {\ftEmp{KHR}} 的函式和枚舉,
以方便在不同的 OpenCL 版本間進行遷移。

對於供應商自定義的擴展,則要遵循下列命名約定:
\startigBase
\item 每個擴展都有一個唯一的{\ftRef{名字字串}},形如“\clext{cl_<供應商的名字>_<名字>}”。
如果實作支持此擴展,
則此字串會出現在\reftab{cldevquery}中的 \cenum{CL_PLATFORM_EXTENSIONS} 或 \cenum{CL_DEVICE_EXTENSIONS} 中。

\item 對於擴展中所定義的所有 API 函式,其名字均形如:\clapi{cl<函式名><供應商的名字>}。

\item 對於擴展中所定義的所有枚舉,其名字均形如:\clapi{CL_<枚舉名>_<供應商的名字>}。
\stopigBase

\section{可選擴展的編譯指示}

編譯指示 \cdrtemp{#pragma OPENCL EXTENSION} 控制着 OpenCL 編譯器關於擴展的行為。
此指示定義如下:
\startclc[indentnext=no]
#pragma OPENCL EXTENSION /BTEX\cref{extension_name}/ETEX : /BTEX\cref{behavior}/ETEX
#pragma OPENCL EXTENSION all : /BTEX\cref{behavior}/ETEX
\stopclc
其中 \cref{extension_name} 就是擴展的名字,
形如 \clext{cl_khr_<名字>}(OpenCL 工作組所批准的擴展)或者 \clext{cl_<供應商的名字>_<名字>}(供應商的擴展)。
而符記 \cemp{all} 則意味着在編譯器所支持的所有擴展上應用此行為。
可以將 \cref{behavior} 設置為\reftab{behaviorDsc}中的值。

\placetable[here][tab:behaviorDsc]
{可選擴展的行為}
{\startbuffer[enExtDsc]
其行為遵守擴展 \cref{extension_name} 的規定。

如果指定了 \cemp{all},或者不支持 \cref{extension_name},
則會在 \cdrtemp{#pragma OPENCL EXTENSION} 上報告錯誤。
\stopbuffer

\startbuffer[disExtDsc]
其行為(包括發出錯誤、警告)如同語言定義中沒有擴展 \cref{extension_name} 那樣。

如果指定了 \cemp{all},則其行為必須回退到所編譯語言的無擴展核心版本。

如果不支持 \cref{extension_name},
則會在 \cdrtemp{#pragma OPENCL EXTENSION} 上發出警告。
\stopbuffer

\startCLOD[behavior][描述]
\clOD{enable}{\getbuffer[enExtDsc]}
\clOD{disable}{\getbuffer[disExtDsc]}
\stopCLOD
}

就設定每個擴展的行為而言,
編譯指示 \cdrtemp{#pragma OPENCL EXTENSION} 是一種簡單、底層的機制。
但是他並沒有定義任何策略,比如哪種組合比較恰當;
這些必須在其他地方定義。
在設定每個擴展的行為時,指示的順序非常重要。
後出現的指示會覆蓋早出現的。
而變體 \cemp{all} 會設定所有擴展的行為,將覆蓋前面出現的所有擴展指示,
但 \cref{behavior} 只能設置為 \cemp{disable}。

編譯器的初始狀態相當於如下指示:
\startclc[indentnext=no]
#pragma OPENCL EXTENSION all : disable
\stopclc
告訴編譯器必須按照此規格來報告所有錯誤和警告,忽略所有擴展。

每個擴展,只要會影響 OpenCL 語言的語義、文法或者為語言增加了內建函式,
都必須創建一個預處理 \ccmm{#define} 來匹配擴展名。
當且僅當實作支持此擴展時,這個 \ckey{#define} 才可用。

例:

如果一個擴展增加了擴展字串“\clext{cl_khr_3d_image_writes}”,
那麼同時應當增加相應的預處理 \cemp{#define}。
現在\cnglo{kernel}就可以像這樣來使用這個預處理 \ccmm{#define}:
\startclc
#ifdef cl_khr_3d_image_writes
	// do something using the extension
#else
	// do something else or #error!
#endif
\stopclc

\section[sec:getFuncPtr]{獲取 OpenCL API 擴展函式指位器}

\topclfunc{clGetExtensionFunctionAddressForPlatform}

\startCLFUNC
void* clGetExtensionFunctionAddressForPlatform (
			cl_platform_id platform,
			const char *funcname)
\stopCLFUNC

\startnotepar
由於沒有任何方式來限定對\cnglo{device}的查詢,
對於此擴展在這個\cnglo{platform}中不同\cnglo{device}上的所有實作而言,
所返回的函式指位器必須都能正常運作。
如果\cnglo{device}不支持此擴展,則調用擴展中的函式時,其行為\cnglo{undef}。
\stopnotepar

此函式會返回給定 \carg{platform} 的擴展函式 \carg{funcname} 的位址。
需要將所返回的指位器轉型成對應的函式指位器型別,
此擴展函式在對應的擴展規範和頭檔中定義。
如果返回的是 \cmacro{NULL},則表明實作中沒有所指定的函式,或者 \carg{platform} 無效。
即使返回的不是 \cmacro{NULL},也不保證 \carg{platform} 真正支持此擴展函式。
要想確定 OpenCL 實作是否支持某個擴展,
\cnglo{app}必須用下列兩種方式之一進行查詢:
\startclc
clGetPlatformInfo(platform, CL_PLATFORM_EXTENSIONS, ... )
clGetDeviceInfo(device, CL_DEVICE_EXTENSIONS, ... )
\stopclc

對於 OpenCL 的核心函式(非擴展函式),不能使用此函式進行查詢。
而對於那些能用此函式進行查詢的函式,
實作也可以選擇由實現那些函式的目標庫靜態導入那些函式。
然而,\cnglo{app}要想可移植,就不能依賴這種行為。

對於所有會增加 API 引入點的擴展,都必須聲明 \ccmm{typedef} 的函式指位器型別。
這些 \ccmm{typedef} 是擴展接口所必需的一部分,在相應頭檔中提供
(如果是 OpenCL 擴展,則為 \ccmm{cl_ext.h};
而如果是 OpenCL / OpenGL 共享擴展,則為 \ccmm{cl_gl_ext.h})。

所有會影響\cnglo{host} API 的擴展都必須遵循下列約定:
\startclc[indentnext=no]
#ifndef extension_name
#define extension_name		1

// all data typedefs, token #defines, prototypes, and
// function pointer typedefs for this extension

// function pointer typedefs must use the
// following naming convention
typedef CL_API_ENTRY /BTEX{\ftRef{return type}}/ETEX
		(CL_API_CALL */BTEX{\ftRef{clextension_func_nameTAG_fn}}/ETEX)(...);

#endif // extension_name
\stopclc
其中 \ccmm{TAG} 可以是 \ccmm{KHR}、 \ccmm{EXT} 或 \ccmm{vendor-specific}。

例如,擴展 \clext{cl_khr_gl_sharing} 在 \ccmm{cl_gl_ext.h} 中增加了下列代碼:
\startclc
#ifndef cl_khr_gl_sharing
#define cl_khr_gl_sharing	1

// all data typedefs, token #defines, prototypes, and
// function pointer typedefs for this extension
#define CL_INVALID_GL_SHAREGROUP_REFERENCE_KHR	-1000
#define CL_CURRENT_DEVICE_FOR_GL_CONTEXT_KHR	0x2006
#define CL_DEVICES_FOR_GL_CONTEXT_KHR		0x2007
#define CL_GL_CONTEXT_KHR			0x2008
#define CL_EGL_DISPLAY_KHR			0x2009
#define CL_GLX_DISPLAY_KHR			0x200A
#define CL_WGL_HDC_KHR				0x200B
#define CL_CGL_SHAREGROUP_KHR			0x200C

// function pointer typedefs must use the
// following naming convention
typedef CL_API_ENTRY cl_int
	(CL_API_CALL *clGetGLContextInfoKHR_fn)(
		const cl_context_properties * /* properties */,
		cl_gl_context_info /* param_name */,
		size_t /* param_value_size */,
		void * /* param_value */,
		size_t * /*param_value_size_ret*/);

#endif // cl_khr_gl_sharing
\stopclc

\section{64 位原子函式}

下列兩個可選擴展實現了 \cqlf{__global} 和 \cqlf{__local} 內存中的 64 位
帶符號和無符號整數上的原子運算:
\startigBase
\item \clext{cl_khr_int64_base_atomics}
\item \clext{cl_khr_int64_extended_atomics}
\stopigBase

\cnglo{app}中要想使用這些擴展,
必須在 OpenCL \cnglo{program}源碼中包含下列編譯指示之一:
\startclc
#pragma OPENCL EXTENSION cl_khr_int64_base_atomics : enable
#pragma OPENCL EXTENSION cl_khr_int64_extended_atomics : enable
\stopclc

\reftab{atomic64_base}中列出了擴展 \clext{cl_khr_int64_base_atomics} 所支持的原子函式,
其中所有函式都在一個原子事務內實施。

\placetable[here,split][tab:atomic64_base]
{擴展 \clext{cl_khr_int64_base_atomics} 的內建原子函式}
{% atomic_add
\startbuffer[funcproto:atomic64_add]
long atomic_add (
	volatile __global long *p,
	long val)
long atomic_add (
	volatile __local long *p,
	long val)

ulong atomic_add (
	volatile __global ulong *p,
	ulong val)
ulong atomic_add (
	volatile __local ulong *p,
	ulong val)
\stopbuffer
\startbuffer[funcdesc:atomic64_add]
讀取 \carg{p} 所指向的 64 位值(記為 \math{old})。
計算 \math{(old + \marg{val})} 並將結果存儲到 \carg{p} 所指位置中。
此函式返回 \math{old}。
\stopbuffer

% atomic_sub
\startbuffer[funcproto:atomic64_sub]
long atomic_sub (
	volatile __global long *p,
	long val)
long atomic_sub (
	volatile __local long *p,
	long val)

ulong atomic_sub (
	volatile __global ulong *p,
	ulong val)
ulong atomic_sub (
	volatile __local ulong *p,
	ulong val)
\stopbuffer
\startbuffer[funcdesc:atomic64_sub]
讀取 \carg{p} 所指向的 64 位值(記為 \math{old})。
計算 \math{(old - \marg{val})} 並將結果存儲到 \carg{p} 所指位置中。
此函式返回 \math{old}。
\stopbuffer

% atomic_xchg
\startbuffer[funcproto:atomic64_xchg]
long atomic_xchg (
	volatile __global long *p,
	long val)
long atomic_xchg (
	volatile __local long *p,
	long val)

ulong atomic_xchg (
	volatile __global ulong *p,
	ulong val)
ulong atomic_xchg (
	volatile __local ulong *p,
	ulong val)
\stopbuffer
\startbuffer[funcdesc:atomic64_xchg]
將位置 \carg{p} 中所存儲的值 \math{old} 和 \carg{val} 中的新值相互交換。
返回 \math{old}。
\stopbuffer

% atomic_inc
\startbuffer[funcproto:atomic64_inc]
long atomic_inc (volatile __global long *p)
long atomic_inc (volatile __local long *p)

ulong atomic_inc (
	volatile __global ulong *p)
ulong atomic_inc (
	volatile __local ulong *p)
\stopbuffer
\startbuffer[funcdesc:atomic64_inc]
讀取 \carg{p} 所指向的 64 位值(記為 \math{old})。
計算 \math{(old+1)} 並將結果存儲到 \carg{p} 所指位置中。
此函式返回 \math{old}。
\stopbuffer

% atomic_dec
\startbuffer[funcproto:atomic64_dec]
long atomic_dec (volatile __global long *p)
long atomic_dec (volatile __local long *p)

ulong atomic_dec (
	volatile __global ulong *p)
ulong atomic_dec (
	volatile __local ulong *p)
\stopbuffer
\startbuffer[funcdesc:atomic64_dec]
讀取 \carg{p} 所指向的 64 位值(記為 \math{old})。
計算 \math{(old-1)} 並將結果存儲到 \carg{p} 所指位置中。
此函式返回 \math{old}。
\stopbuffer

% atomic_cmpchg
\startbuffer[funcproto:atomic64_cmpxchg]
long atomic_cmpxchg (
	volatile __global long *p,
	long cmp, long val)
long atomic_cmpxchg (
	volatile __local long *p,
	long cmp,
	long val)

ulong atomic_cmpxchg (
	volatile __global ulong *p,
	ulong cmp,
	ulong val)
ulong atomic_cmpxchg (
	volatile __local ulong *p,
	ulong cmp,
	ulong val)
\stopbuffer
\startbuffer[funcdesc:atomic64_cmpxchg]
讀取 \carg{p} 所指向的 64 位值(記為 \math{old})。
計算 \math{(old == cmp) ? val : old} 並將結果存儲到 \carg{p} 所指位置中。
此函式返回 \math{old}。
\stopbuffer


% begin table
\startCLFD
\clFD{atomic64_add}
\clFD{atomic64_sub}
\clFD{atomic64_xchg}
\clFD{atomic64_inc}
\clFD{atomic64_dec}
\clFD{atomic64_cmpxchg}
\stopCLFD
}

\reftab{atomic64_ext}中列出了擴展 \clext{cl_khr_int64_extended_atomics} 所支持的原子函式,
其中所有函式都在一個原子事務內實施。

\placetable[here,split][tab:atomic64_ext]
{擴展 \clext{cl_khr_int64_extended_atomics} 的內建原子函式}
{% atomic_min
\startbuffer[funcproto:atomic64_min]
long atomic_min (
	volatile __global long *p,
	long val)
long atomic_min (
	volatile __local long *p,
	long val)

ulong atomic_min (
	volatile __global ulong *p,
	ulong val)
ulong atomic_min (
	volatile __local ulong *p,
	ulong val)
\stopbuffer
\startbuffer[funcdesc:atomic64_min]
讀取 \carg{p} 所指向的 64 位值(記為 \math{old})。
計算 \math{\mapiemp{min}(old, \marg{val})} 並將結果存儲到 \carg{p} 所指位置中。
此函式返回 \math{old}。
\stopbuffer

% atomic_max
\startbuffer[funcproto:atomic64_max]
long atomic_max (
	volatile __global long *p,
	long val)
long atomic_max (
	volatile __local long *p,
	long val)

ulong atomic_max (
	volatile __global ulong *p,
	ulong val)
ulong atomic_max (
	volatile __local ulong *p,
	ulong val)
\stopbuffer
\startbuffer[funcdesc:atomic64_max]
讀取 \carg{p} 所指向的 64 位值(記為 \math{old})。
計算 \math{\mapiemp{max}(old, \marg{val})} 並將結果存儲到 \carg{p} 所指位置中。
此函式返回 \math{old}。
\stopbuffer

% atomic_and
\startbuffer[funcproto:atomic64_and]
long atomic_and (
	volatile __global long *p,
	long val)
long atomic_and (
	volatile __local long *p,
	long val)

ulong atomic_and (
	volatile __global ulong *p,
	ulong val)
ulong atomic_and (
	volatile __local ulong *p,
	ulong val)
\stopbuffer
\startbuffer[funcdesc:atomic64_and]
讀取 \carg{p} 所指向的 64 位值(記為 \math{old})。
計算 \math{(old \mcmm{&} \marg{val})} 並將結果存儲到 \carg{p} 所指位置中。
此函式返回 \math{old}。
\stopbuffer

% atomic_or
\startbuffer[funcproto:atomic64_or]
long atomic_or (
	volatile __global long *p,
	long val)
long atomic_or (
	volatile __local long *p,
	long val)

ulong atomic_or (
	volatile __global ulong *p,
	ulong val)
ulong atomic_or (
	volatile __local ulong *p,
	ulong val)
\stopbuffer
\startbuffer[funcdesc:atomic64_or]
讀取 \carg{p} 所指向的 64 位值(記為 \math{old})。
計算 \math{(old \mcmm{|} \marg{val})} 並將結果存儲到 \carg{p} 所指位置中。
此函式返回 \math{old}。
\stopbuffer

% atomic_xor
\startbuffer[funcproto:atomic64_xor]
long atomic_xor (
	volatile __global long *p,
	long val)
long atomic_xor (
	volatile __local long *p,
	long val)

ulong atomic_xor (
	volatile __global ulong *p,
	ulong val)
ulong atomic_xor (
	volatile __local ulong *p,
	ulong val)
\stopbuffer
\startbuffer[funcdesc:atomic64_xor]
讀取 \carg{p} 所指向的 64 位值(記為 \math{old})。
計算 \math{(old \mcmm{^} \marg{val})} 並將結果存儲到 \carg{p} 所指位置中。
此函式返回 \math{old}。
\stopbuffer


% begin table
\startCLFD
\clFD{atomic64_min}
\clFD{atomic64_max}
\clFD{atomic64_and}
\clFD{atomic64_or}
\clFD{atomic64_xor}
\stopCLFD
}

對於執行這些原子函式的\cnglo{device}而言,這些事務都是原子的。
而對於在多個\cnglo{device}上執行的\cnglo{kernel}而言,
如果這些原子運算是在同一內存位置上實施的,則不保證其原子性。

\startnotepar
64 位整數和 32 位整數(包括 \ctype{float})上的原子運算相互之間也是原子的。
\stopnotepar

\section{寫入 3D 圖像對象}

OpenCL 支持\cnglo{kernel}讀寫 2D \cnglo{imgobj}。
同一\cnglo{kernel}不能對同一 2D \cnglo{imgobj}既讀又寫。
OpenCL 也支持\cnglo{kernel}讀取 3D \cnglo{imgobj},
但是不支持寫入 3D \cnglo{imgobj},除非實現了擴展 \clext{cl_khr_3d_image_writes}。
同一\cnglo{kernel}不能對同一 3D \cnglo{imgobj}既讀又寫。

\cnglo{app}要想使用此擴展寫入 3D \cnglo{imgobj},
需要在 OpenCL \cnglo{program}源碼中包含下列編譯指示:
\startclc
#pragma OPENCL EXTENSION cl_khr_3d_image_writes : enable
\stopclc

\reftab{write_3d_image}中列出了擴展 \clext{cl_khr_3d_image_writes} 所實現的內建函式。

\placetable[here][tab:write_3d_image]
{擴展 \clext{cl_khr_3d_image_writes} 所實現的內建函式}
{% atomic_add
\startbuffer[funcproto:write_image_3d]
void write_imagef (image3d_t image,
		int4 coord,
		float4 color)

void write_imagei (image3d_t image,
		int4 coord,
		int4 color)

void write_imageui (image3d_t image,
		int4 coord,
		uint4 color)
\stopbuffer
\startbuffer[funcdesc:write_image_3d]
將 \carg{color} 的值寫入 3D \cnglo{imgobj} \carg{image}中坐標 \math{(x,y,z)} 處。
寫入前會對顏色值進行恰當的數據格式轉換。
會將 \carg{coord.x}、 \carg{coord.y} 和 \carg{coord.z} 視為非歸一化坐標,
且其值必須分別位於區間 \math{0\cdots\mvar{圖像寬度}-1}、 \math{0\cdots\mvar{圖像高度}-1} 和 \math{0\cdots\mvar{圖像深度}-1} 內。

對於 \capi{write_imagef},
創建\cnglo{imgobj}時所用的 \carg{image_channel_data_type} 必須是預定義壓縮過的格式
或者 \cenum{CL_SNORM_INT8}、 \cenum{CL_UNORM_INT8}、 \cenum{CL_SNORM_INT16}、
 \cenum{CL_UNORM_INT16}、 \cenum{CL_HALF_FLOAT} 或 \cenum{CL_FLOAT}。
會將通道數據由浮點值轉換成存儲數據所用的實際數據格式。

對於 \capi{write_imagei} 而言,
創建\cnglo{imgobj}時所用的 \carg{image_channel_data_type} 必須是下列值之一:
\startigBase
\item \cenum{CL_SIGNED_INT8}
\item \cenum{CL_SIGNED_INT16}
\item \cenum{CL_SIGNED_INT32}
\stopigBase

對於 \capi{write_imageui} 而言,
創建\cnglo{imgobj}時所用的 \carg{image_channel_data_type} 必須是下列值之一:
\startigBase
\item \cenum{CL_UNSIGNED_INT8}
\item \cenum{CL_UNSIGNED_INT16}
\item \cenum{CL_UNSIGNED_INT32}
\stopigBase

如果創建\cnglo{imgobj}時所用的 \carg{image_channel_data_type} 不再上述所列範圍內,
或者坐標 \math{(x, y, z)} 不在 \math{(0 \cdots \mvar{圖像寬度}-1, 0 \cdots \mvar{圖像高度}-1, 0 \cdots \mvar{圖像深度}-1)} 範圍內,
則 \capi{write_imagef}、 \capi{write_imagei} 和 \capi{write_imageui} 的行為\cnglo{undef}。
\stopbuffer

% begin table
\startCLFD
\clFD{write_image_3d}
\stopCLFD
}


\stopcomponent

