\startcomponent cpnembprofile
\product opencl-spec-zh

\chapter[chapter:openclEmbProfile]{OpenCL 嵌入式規格}

前面章節描述了對桌面平台的特性要求。
本節則描述針對手持或嵌入式平台的 OpenCL 1.2 嵌入式規格,
此規格是完整規格的一個子集。
《OpenCL 1.2 擴展規範》中定義的可選擴展對兩種規格都有效。

OpenCL 1.2 嵌入式規格有如下局限:
\startigNum
\startitem
64 位整數,即 \ctype{long}、 \ctype{ulong},包括相應的矢量數據型別,
以及相關的運算都是可選的。
如果嵌入式規格的實作支持 64 位整數,
則報告中有擴展字串 \clext{cles_khr_int64}\footnote{%
在不同的嵌入式設備上, 64 位整數算術的性能可能有重大差異。}。
\stopitem

\startitem
對 3D 圖像的支持是可選的。是否支持 3D 圖像主要取決於下列值:
\startigBase
\item \cenum{CL_DEVICE_IMAGE3D_MAX_WIDTH}、
\item \cenum{CL_DEVICE_IMAGE3D_MAX_HEIGHT} 和
\item \cenum{CL_DEVICE_IMAGE3D_MAX_DEPTH}。
\stopigBase

如果這些值為零,則嵌入式規格中調用 \capi{clCreateImage} 創建 3D 圖像將失敗,
並在引數 \carg{errcode_ret} 中返回 \cenum{CL_INVALID_OPERATION};
而且在\cnglo{kernel}中聲明型別為 \ctype{image3d_t} 的引數將會導致編譯錯誤。

如果這些值 大於零,則表明嵌入式規格的實作是支持 3D 圖像的。
\capi{clCreateImage} 會和完整規格中定義的那樣正常工作。
\cnglo{kernel}中可以使用數據型別 \ctype{image3d_t}。
\stopitem

\startitem
对写入 2D 图像阵列的支持是可选的。
如果嵌入式规格支持扩展 \capi{ cles_khr_2d_image_array_writes},
则表明支持写入 2D 图像阵列。
\stopitem

\startitem
如果創建圖像和圖像陣列是所用的
\cvar{image_channel_data_type} 為 \cenum{CL_FLOAT} 或 \cenum{CL_HALF_FLOAT},
則他們只能使用濾波模式為 \cenum{CL_FILTER_NEAREST} 的\cnglo{sampler}。
如果 2D 和 3D 圖像的
\cvar{image_channel_data_type} 為 \cenum{CL_FLOAT} 或 \cenum{CL_HALF_FLOAT},
而\cnglo{sampler}的 \cvar{filter_mode} 為 \cenum{CL_FILTER_LINEAR},
則 \capi{read_imagef} 和 \capi{read_imageh}\footnote{%
如果支持擴展 \clext{cl_khr_fp16}。} 的返回值未定義。
\stopitem

\startitem
對於圖像和圖像陣列,支持下列\cnglo{sampler}尋址模式:
\startigBase
\item \cenum{CLK_ADDRESS_NONE}、
\item \cenum{CLK_ADDRESS_MIRRORED_REPEAT}、
\item \cenum{CLK_ADDRESS_REPEAT}、
\item \cenum{CLK_ADDRESS_CLAMP_TO_EDGE}、
\item \cenum{CLK_ADDRESS_CLAMP}。
\stopigBase
\stopitem

\startitem
規定單精度浮點能力(由 \cenum{CL_DEVICE_SINGLE_FP_CONFIG} 給出)至少要為
\cenum{CL_FP_ROUND_TO_ZERO} 或 \cenum{CL_FP_ROUND_TO_NEAREST}。
如果支持 \cenum{CL_FP_ROUND_TO_NEAREST},
則缺省的捨入模式為捨入為最近偶數,否則缺省的捨入模式為向零捨入。
\stopitem

\startitem
單精度浮點運算(加、減和乘)必須要能正確捨入。
如果結果是零,則可能始終是 0.0。
除法和開方的精確度由\reftab{EP_ulpMathFunc}給出。

如果 \cenum{CL_DEVICE_SINGLE_FP_CONFIG} 中沒有設置 \cenum{CL_FP_INF_NAN},
但是某個算元或加、減、乘、除的結果引發了上溢或無效異常(參見 IEEE 754 規範),
則結果的值\cnglo{impdef}。
同樣,如果某個算元是 NaN,
則單精度比較算子
(\ccmm{<}、 \ccmm{>}、 \ccmm{<=}、 \ccmm{>=}、 \ccmm{==}、 \ccmm{!=})
所返回的值也\cnglo{impdef}。

所有情況下,轉換(\refsec{conversionCast}和\refsec{vectorLsFunc})
都要像 \cenum{FULL_PROFILE} 一樣正確捨入,包括那些會消耗或產生 INF 或 NaN 的轉換。
內建數學函式(\refsec{mathFunc})的表現也與 \cenum{FULL_PROFILE} 中描述的一樣,
包括\refsec{addReqBeyondC99}中所描述的邊界條件下的行為,
但是精確度則以\reftab{EP_ulpMathFunc}為准。

\startnotepar
如果減法和乘法缺省為向零捨入,
\capi{fract}、 \capi{fma} 和 \capi{fdim} 所產生的結果應當已經按此模式正確捨入了。
\stopnotepar

對於基本的浮點運算,上述內容實際是放寬了 IEEE 754 中的要求,儘管極度不情願,
但是這樣可以為那些硬件預算有嚴格限制的嵌入式設備提供更大的靈活性。
\stopitem

\startitem
對於由 \cenum{CL_UNORM_INT8}、 \cenum{CL_SNORM_INT8}、 \cenum{CL_UNORM_INT16}
和 \cenum{CL_SNORM_INT16} 到 \ctype{float} 的轉換,
在嵌入式規格中要求其精度 \math{\leq 2 ulp},
這取代了\refsec{normIntToFloat}中的 \math{\leq 1.5 ulp}。
在嵌入式規格中,\refsec{normIntToFloat}中的異常情況以及下列異常情況都有效。

\startclCmmDesc{對於 \cenum{CL_UNORM_INT8}}
\ccmm{0} 必須轉換為 \ccmm{0.0f}

\ccmm{255} 必須轉換為 \ccmm{1.0f}
\stopclCmmDesc

\startclCmmDesc{對於 \cenum{CL_UNORM_INT16}}
\ccmm{0} 必須轉換為 \ccmm{0.0f}

\ccmm{65535} 必須轉換為 \ccmm{1.0f}
\stopclCmmDesc

\startclCmmDesc{對於 \cenum{CL_SNORM_INT8}}
\ccmm{-128} 和 \ccmm{-127} 必須轉換為 \ccmm{-1.0f}

\ccmm{0} 必須轉換為 \ccmm{0.0f}

\ccmm{127} 必須轉換為 \ccmm{1.0f}
\stopclCmmDesc

\startclCmmDesc{對於 \cenum{CL_SNORM_INT16}}
\ccmm{-32768} 和 \ccmm{-32767} 必須轉換為 \ccmm{-1.0f}

\ccmm{0} 必須轉換為 \ccmm{0.0f}

\ccmm{32767} 必須轉換為 \ccmm{1.0f}
\stopclCmmDesc

\startclCmmDesc{對於 \cenum{CL_UNORM_INT_101010}}
\ccmm{0} 必須轉換為 \ccmm{0.0f}

\ccmm{1023} 必須轉換為 \ccmm{1.0f}
\stopclCmmDesc
\stopitem

\startitem
\refsec{atomicFunc}中所定義的原子函式是可選的。
\stopitem
\stopigNum

《OpenCL 1.2 擴展規範》中所定義的下列可選擴展對嵌入式規格同樣可用:
\startigBase
\item \clext{cl_khr_int64_base_atomics}
\item \clext{cl_khr_int64_extended_atomics}
\item \clext{cl_khr_fp16}
\item \clext{cles_khr_int64}。
如果支持雙精度浮點數,即 \cenum{CL_DEVICE_DOUBLE_FP_CONFIG} 不是零,
則必須支持 \clext{cles_khr_int64}。
\item \clext{cles_khr_2d_image_array_writes}。
此扩展指明此嵌入式规格\cnglo{device}支持写入 2D 图像阵列。

\stopigBase

OpenCL 1.0 和 OpenCL 1.1 規範中所定義的下列擴展({\ftRef{第 9 章}})對嵌入式規格同樣可用:
\startigBase
\item \clext{cl_khr_global_int32_base_atomics}
\item \clext{cl_khr_global_int32_extended_atomics}
\item \clext{cl_khr_local_int32_base_atomics}
\item \clext{cl_khr_local_int32_extended_atomics}
\stopigBase

\reftab{EP_ulpMathFunc}中描述了嵌入式規格中,
單精度浮點算術運算的最小精確度,以 ULP 值的形式給出。
計算 ULP 值時參考的是無限精確的結果。
其中 0 ulp 表示相應函式無需捨入。

\placetable[force,here,split][tab:EP_ulpMathFunc]
{內建數學函式的 ULP 值}
{\startCLFA[函式][最小精度—— ULP 值]

\clFAM{x+y}{正確捨入}
\clFAM{x-y}{正確捨入}
\clFAM{x*y}{正確捨入}
\clFAM{1.0/x}{<= 3 ulp}
\clFAM{x/y}{<= 3 ulp}

\clFAA{acos}{<= 4 ulp}
\clFAA{acospi}{<= 5 ulp}
\clFAA{asin}{<= 4 ulp}
\clFAA{asinpi}{<= 5 ulp}
\clFAA{atan}{<= 5 ulp}
\clFAA{atan2}{<= 6 ulp}
\clFAA{atanpi}{<= 5 ulp}
\clFAA{atan2pi}{<= 6 ulp}
\clFAA{acosh}{<= 4 ulp}
\clFAA{asinh}{<= 4 ulp}
\clFAA{atanh}{<= 5 ulp}
\clFAA{cbrt}{<= 4 ulp}%
\clFAA{ceil}{正確捨入}
\clFAA{copysign}{0 ulp}
\clFAA{cos}{<= 4 ulp}
\clFAA{cosh}{<= 4 ulp}
\clFAA{cospi}{<= 4 ulp}
\clFAA{erfc}{<= 16 ulp}
\clFAA{erf}{<= 16 ulp}
\clFAA{exp}{<= 4 ulp}%
\clFAA{exp2}{<= 4 ulp}%
\clFAA{exp10}{<= 4 ulp}%
\clFAA{expm1}{<= 4 ulp}%
\clFAA{fabs}{0 ulp}
\clFAA{fdim}{正確捨入}
\clFAA{floor}{正確捨入}
\clFAA{fma}{正確捨入}
\clFAA{fmax}{0 ulp}
\clFAA{fmin}{0 ulp}
\clFAA{fmod}{0 ulp}
\clFAA{fract}{正確捨入}
\clFAA{frexp}{0 ulp}
\clFAA{hypot}{<= 4 ulp}
\clFAA{ilogb}{0 ulp}
\clFAA{ldexp}{正確捨入}
\clFAA{log}{<= 4 ulp}%
\clFAA{log2}{<= 4 ulp}%
\clFAA{log10}{<= 4 ulp}%
\clFAA{log1p}{<= 4 ulp}%
\clFAA{logb}{0 ulp}
\clFAA{mad}{所允許的任何值(無窮 ulp)}
\clFAA{maxmag}{0 ulp}
\clFAA{minmag}{0 ulp}
\clFAA{modf}{0 ulp}
\clFAA{nan}{0 ulp}
\clFAA{nextafter}{0 ulp}
\clFAA{pow(x, y)}{<= 16 ulp}
\clFAA{pown(x, y)}{<= 16 ulp}
\clFAA{powr(x, y)}{<= 16 ulp}
\clFAA{remainder}{0 ulp}
\clFAA{remquo}{0 ulp}
\clFAA{rint}{正確捨入}
\clFAA{rootn}{<= 16 ulp}
\clFAA{round}{正確捨入}
\clFAA{rsqrt}{<= 4 ulp}%
\clFAA{sin}{<= 4 ulp}
\clFAA{sincos}{正弦值和餘弦值都是 <= 4 ulp}
\clFAA{sinh}{<= 4 ulp}
\clFAA{sinpi}{<= 4 ulp}
\clFAA{sqrt}{<= 4 ulp}%
\clFAA{tan}{<= 5 ulp}
\clFAA{tanh}{<= 5 ulp}
\clFAA{tanpi}{<= 6 ulp}
\clFAA{tgamma}{<= 16 ulp}
\clFAA{trunc}{正確捨入}

\clFAA{half_cos}{<= 8192 ulp}
\clFAA{half_divide}{<= 8192 ulp}
\clFAA{half_exp}{<= 8192 ulp}
\clFAA{half_exp2}{<= 8192 ulp}
\clFAA{half_exp10}{<= 8192 ulp}
\clFAA{half_log}{<= 8192 ulp}
\clFAA{half_log2}{<= 8192 ulp}
\clFAA{half_log10}{<= 8192 ulp}
\clFAA{half_powr}{<= 8192 ulp}
\clFAA{half_recip}{<= 8192 ulp}
\clFAA{half_rsqrt}{<= 8192 ulp}
\clFAA{half_sin}{<= 8192 ulp}
\clFAA{half_sqrt}{<= 8192 ulp}
\clFAA{half_tan}{<= 8192 ulp}

\clFAA{native_cos}{\cnglo{impdef}}
\clFAA{native_divide}{\cnglo{impdef}}
\clFAA{native_exp}{\cnglo{impdef}}
\clFAA{native_exp2}{\cnglo{impdef}}
\clFAA{native_exp10}{\cnglo{impdef}}
\clFAA{native_log}{\cnglo{impdef}}
\clFAA{native_log2}{\cnglo{impdef}}
\clFAA{native_log10}{\cnglo{impdef}}
\clFAA{native_powr}{\cnglo{impdef}}
\clFAA{native_recip}{\cnglo{impdef}}
\clFAA{native_rsqrt}{\cnglo{impdef}}
\clFAA{native_sin}{\cnglo{impdef}}
\clFAA{native_sqrt}{\cnglo{impdef}}
\clFAA{native_tan}{\cnglo{impdef}}

\stopCLFA
}

語言中加入了巨集 \cmacro{__EMBEDDED_PROFILE__} (參見\refsec{pp_macro})。
對於那些實現了嵌入式規格的 OpenCL \cnglo{device},此巨集為整數常量 1,
否則未定義此巨集。

如果 OpenCL 實作僅支持嵌入式規格,
則\reftab{plfquery}中的 \cenum{CL_PLATFORM_PROFILE}
會返回字串 \cenum{EMBEDDED_PROFILE}。

嵌入式規格中,\reftab{cldevquery}所指定的最小值和最大值變化如下:

\startETD[cl_device_info][返回型別]

\clETD{CL_DEVICE_TYPE}{cl_device_type}{
OpenCL \cnglo{device}類型,當前支持:
\startigBase
\item \cenum{CL_DEVICE_TYPE_CPU},
\item \cenum{CL_DEVICE_TYPE_GPU},
\item \cenum{CL_DEVICE_TYPE_ACCELERATOR},
\item \cenum{CL_DEVICE_TYPE_DEFAULT},或者
\item 以上值的组合,或者
\item \cenum{CL_DEVICE_TYPE_CUSTOM}。
\stopigBase
}

\clETD{CL_DEVICE_VENDOR_ID}{cl_uint}{
唯一的\cnglo{device}供應商標識符。例如可以是 PCIe ID。
}

\clETD{CL_DEVICE_MAX_COMPUTE_UNITS}{cl_uint}{
OpenCL \cnglo{device}上的並行\cnglo{computeunit}的數目。最小值是1。
}

\clETD{CL_DEVICE_MAX_WORK_ITEM_DIMENSIONS}{cl_uint}{
\cnglo{dppm}中所用全局和局部\cnglo{workitem} ID 的最大維數
(參見 \capi{clEnqueueNDRangeKernel})。
對於類型不是 \cenum{CL_DEVICE_TYPE_CUSTOM} 的\cnglo{device},其最小值是 3。
}

\clETD{CL_DEVICE_MAX_WORK_ITEM_SIZES}{size_t[]}{
對\capi{clEnqueueNDRangeKernel}而言,
\cnglo{workgrp}中每個維度上可以指派\cnglo{workitem}的最大數目。

返回 \carg{n} 個型別為 \ctype{size_t} 的表項。
其中 \carg{n} 是查詢 \cenum{CL_DEVICE_MAX_WORK_ITEM_DIMENSIONS} 時所返回的值。

對於不是 \cenum{CL_DEVICE_TYPE_CUSTOM} 的\cnglo{device},最小值是\math{(1, 1, 1)}。
}

\clETD{CL_DEVICE_MAX_WORK_GROUP_SIZE}{size_t}{
用\cnglo{dppm}執行\cnglo{kernel}時,
\cnglo{workgrp}中所能容納\cnglo{workitem}的最大數目
(参见 \capi{clEnqueueNDRangeKernel})。
最小值是1。
}

\clETD{%
CL_DEVICE_PREFERRED_VECTOR_WIDTH_CHAR
CL_DEVICE_PREFERRED_VECTOR_WIDTH_SHORT
CL_DEVICE_PREFERRED_VECTOR_WIDTH_INT
CL_DEVICE_PREFERRED_VECTOR_WIDTH_LONG
CL_DEVICE_PREFERRED_VECTOR_WIDTH_FLOAT
CL_DEVICE_PREFERRED_VECTOR_WIDTH_DOUBLE
CL_DEVICE_PREFERRED_VECTOR_WIDTH_HALF
}{cl_uint}{
可以放入矢量中的內建標量型別所期望的原生矢量的寬度。
矢量寬度定義為可以可以容納標量元素的數目。

如果不支持雙精度浮點數,\cenum{CL_DEVICE_PREFERRED_VECTOR_WIDTH_DOUBLE} 必須返回0。

如果不支持擴展 \clext{cl_khr_fp16},
\cenum{CL_DEVICE_PREFERRED_VECTOR_WIDTH_HALF} 必須返回0。
}

\clETD{
CL_DEVICE_NATIVE_VECTOR_WIDTH_CHAR
CL_DEVICE_NATIVE_VECTOR_WIDTH_SHORT
CL_DEVICE_NATIVE_VECTOR_WIDTH_INT
CL_DEVICE_NATIVE_VECTOR_WIDTH_LONG
CL_DEVICE_NATIVE_VECTOR_WIDTH_FLOAT
CL_DEVICE_NATIVE_VECTOR_WIDTH_DOUBLE
CL_DEVICE_NATIVE_VECTOR_WIDTH_HALF
}{cl_uint}{
返回原生 ISA 矢量寬度。
此矢量寬度定義為所能容納標量元素的數目。

如果不支持雙精度浮點數,\cenum{CL_DEVICE_NATIVE_VECTOR_WIDTH_DOUBLE} 必須返回0。

如果不支持擴展 \clext{cl_khr_fp16},
\cenum{CL_DEVICE_NATIVE_VECTOR_WIDTH_HALF} 必須返回0。
}

\clETD{CL_DEVICE_MAX_CLOCK_FREQUENCY}{cl_uint}{
\cnglo{device}的時鐘頻率可以配置成的最大值,單位:MHz。
}

\clETD{CL_DEVICE_ADDRESS_BITS}{cl_uint}{
計算設備的位址空間缺省大小,無符號整數,單位:bit。
當前支持 32 位或 64 位。
如果嵌入式規格報告的值是 64,則必須支持擴展 \clext{cles_khr_int64}。%
}

\problem{多了一個 CL_DEVICE_MAX_WORK_GROUP_SIZE}

\clETD{CL_DEVICE_MAX_MEM_ALLOC_SIZE}{cl_ulong}{\problem{unsigned long long}
所能分配的\cnglo{memobj}大小的最大值,單位:字節。
對於類型不是 \cenum{CL_DEVICE_TYPE_CUSTOM} 的\cnglo{device},此值最小為:

\math{max(\menum{CL_DEVICE_GLOBAL_MEM_SIZE} * 1/4, 1 * 1024 * 1024)}%
}

\clETD{}{}{}

\clETD{CL_DEVICE_IMAGE_SUPPORT}{cl_bool}{
如果 OpenCL \cnglo{device}支持圖像,則為 \cenum{CL_TRUE},否則為 \cenum{CL_FALSE}。
}

\clETD{CL_DEVICE_MAX_READ_IMAGE_ARGS}{cl_uint}{
\cnglo{kernel}可以同時讀取多少\cnglo{imgobj}。
如果 \cenum{CL_DEVICE_IMAGE_SUPPORT} 是 \cenum{CL_TRUE},則此值至少是 8。%
}

\clETD{CL_DEVICE_MAX_WRITE_IMAGE_ARGS}{cl_uint}{
\cnglo{kernel}可以同時寫入多少\cnglo{imgobj}。
如果 \cenum{CL_DEVICE_IMAGE_SUPPORT} 是 \cenum{CL_TRUE},則此值至少是 1。%
}

\clETD{CL_DEVICE_IMAGE2D_MAX_WIDTH}{size_t}{
2D 图像的最大寬度,單位:像素。%
如果 \cenum{CL_DEVICE_IMAGE_SUPPORT} 是 \cenum{CL_TRUE},則此值至少是 2048。%
}

\clETD{CL_DEVICE_IMAGE2D_MAX_HEIGHT}{size_t}{
2D 圖像的最大高度,單位:像素。
如果 \cenum{CL_DEVICE_IMAGE_SUPPORT} 是 \cenum{CL_TRUE},則此值至少是 2048。%
}

\clETD{CL_DEVICE_IMAGE3D_MAX_WIDTH}{size_t}{
3D 圖像的最大寬度,單位:像素。
如果 \cenum{CL_DEVICE_IMAGE_SUPPORT} 是 \cenum{CL_TRUE},則此值至少是 0。%
}

\clETD{CL_DEVICE_IMAGE3D_MAX_HEIGHT}{size_t}{
3D 圖像的最大高度,單位:像素。
如果 \cenum{CL_DEVICE_IMAGE_SUPPORT} 是 \cenum{CL_TRUE},則此值至少是 0。%
}

\clETD{CL_DEVICE_IMAGE3D_MAX_DEPTH}{size_t}{
3D 圖像的最大深度,單位:像素。
如果 \cenum{CL_DEVICE_IMAGE_SUPPORT} 是 \cenum{CL_TRUE},則此值至少是 0。%
}

\clETD{CL_DEVICE_IMAGE_MAX_BUFFER_SIZE}{size_t}{
由\cnglo{bufobj}所創建的 1D 圖像的最大像素數。
如果 \cenum{CL_DEVICE_IMAGE_SUPPORT} 是 \cenum{CL_TRUE},則此值至少是 2048。%
}

\clETD{CL_DEVICE_IMAGE_MAX_ARRAY_SIZE}{size_t}{
1D 或 2D 圖像陣列中圖像的最大數目。
如果 \cenum{CL_DEVICE_IMAGE_SUPPORT} 是 \cenum{CL_TRUE},則此值至少是 256。%
}

\clETD{CL_DEVICE_MAX_SAMPLERS}{cl_uint}{
一個\cnglo{kernel}內最多可以使用多少個\cnglo{sampler}。
關於\cnglo{sampler}的細節請參考\refsec{imgRwFunc}。

如果 \cenum{CL_DEVICE_IMAGE_SUPPORT} 是 \cenum{CL_TRUE},則此值至少是 8。%
}

\clETD{}{}{}

\clETD{CL_DEVICE_MAX_PARAMETER_SIZE}{size_t}{
\cnglo{kernel}引數的最大字節數。

如果設備類型不是 \cenum{CL_DEVICE_TYPE_CUSTOM},則此值至少要是 256。%
}

\clETD{CL_DEVICE_MEM_BASE_ADDR_ALIGN}{cl_uint}{
如果設備類型不是 \cenum{CL_DEVICE_TYPE_CUSTOM},
至少要是\cnglo{device}所支持的 OpenCL 內建數據型別中最大的那種的大小,單位:bit。
( FULL 規格中是 \ctype{long16},EMBEDDED 規格中是 \ctype{long16} 或 \ctype{int16} )
}

\clETD{}{}{}

\clETD{CL_DEVICE_SINGLE_FP_CONFIG}{cl_device_fp_config}{
描述\cnglo{device}的單精度浮點能力。此位欄支持下列值:
\startigBase
\item \cenum{CL_FP_DENORM}——支持去規格化數( denorm )。
\item \cenum{CL_FP_INF_NAN}——支持 INF 和 qNaN。
\item \cenum{CL_FP_ROUND_TO_NEAREST}——支持捨入為最近偶數。
\item \cenum{CL_FP_ROUND_TO_ZERO}——支持向零捨入。
\item \cenum{CL_FP_ROUND_TO_INF}——支持向正無窮和負無窮捨入。
\item \cenum{CL_FP_FMA}——支持 IEEE754-2008 中的積和熔加运算(fused multiply-add, FMA)。
\item \cenum{CL_FP_CORRECTLY_ROUNDED_DIVIDE_SQRT}——除法和開方可以按 IEEE754 規範進行正確的捨入。
\item \cenum{CL_FP_SOFT_FLOAT}——軟件中實現了基本的浮點運算(加、減、乘)。
\stopigBase

如果\cnglo{device}類型不是 \cenum{CL_DEVICE_TYPE_CUSTOM},其浮點能力至少要是:
\cenum{CL_FP_ROUND_TO_ZERO} 或 \cenum{CL_FP_ROUND_TO_NEAREST}。%
}

\clETD{CL_DEVICE_DOUBLE_FP_CONFIG}{cl_device_fp_config}{
描述\cnglo{device}的雙精度浮點能力。此位欄支持下列值:
\startigBase
\item \cenum{CL_FP_DENORM}——支持去規格化數。
\item \cenum{CL_FP_INF_NAN}——支持 INF 和 qNaN。
\item \cenum{CL_FP_ROUND_TO_NEAREST}——支持捨入為最近偶數。
\item \cenum{CL_FP_ROUND_TO_ZERO}——支持向零捨入。
\item \cenum{CL_FP_ROUND_TO_INF}——支持向正無窮和負無窮捨入。
\item \cenum{CL_FP_FMA}——支持 IEEE75-2008 中的積和熔加运算(fused multiply-add, FMA)。
\item \cenum{CL_FP_SOFT_FLOAT}——軟件中實現了基本的浮點運算(加、減、乘)。
\stopigBase

由於雙精度浮點是一個可選特性,所以最小的雙精度浮點能力可以是0。

而如果\cnglo{device}支持雙精度浮點,則其能力至少要是:

\cenum{CL_FP_FMA} \textbar

\cenum{CL_FP_ROUND_TO_NEAREST} \textbar

\cenum{CL_FP_ROUND_TO_ZERO} \textbar

\cenum{CL_FP_ROUND_TO_INF} \textbar

\cenum{CL_FP_INF_NAN} \textbar

\cenum{CL_FP_DENORM}。
}

\clETD{}{}{}

\clETD{CL_DEVICE_GLOBAL_MEM_CACHE_TYPE}{cl_device_mem_cache_type}{
所支持的全局內存緩存的類型。其值可以是:
\startigBase
\item \cenum{CL_NONE},
\item \cenum{CL_READ_ONLY_CACHE} 和
\item \cenum{CL_READ_WRITE_CACHE}。
\stopigBase
}

\clETD{CL_DEVICE_GLOBAL_MEM_CACHELINE_SIZE}{cl_uint}{
\cnglo{glbmem}緩存列(cache line)的字節數。
}

\clETD{CL_DEVICE_GLOBAL_MEM_CACHE_SIZE}{cl_ulong}{
\cnglo{glbmem}緩存的字節數。
}

\clETD{CL_DEVICE_GLOBAL_MEM_SIZE}{cl_ulong}{
全局\cnglo{device}內存的字節數。
}

\clETD{}{}{}

\clETD{CL_DEVICE_MAX_CONSTANT_BUFFER_SIZE}{cl_ulong}{
一次所能分配的常量緩衝區的最大字節數。
對於類型不是 \cenum{CL_DEVICE_TYPE_CUSTOM} 的\cnglo{device},最小值是 1KB。%
}

\clETD{CL_DEVICE_MAX_CONSTANT_ARGS}{cl_uint}{
單個\cnglo{kernel}中,最多能有多少個參數在聲明時帶有限定符\cqlf{__constant}。
對於類型不是 \cenum{CL_DEVICE_TYPE_CUSTOM} 的\cnglo{device},最小值是 4。
}

\clETD{}{}{}

\clETD{CL_DEVICE_LOCAL_MEM_TYPE}{cl_device_local_mem_type}{
所支持的\cnglo{locmem}的類型。
可以是 \cenum{CL_LOCAL}(意指專用的\cnglo{locmem},如 SRAM )或 \cenum{CL_GLOBAL}。

對於\cnglo{customdev},如果不支持\cnglo{locmem},可以返回 \cenum{CL_NONE}。
}

\clETD{CL_DEVICE_LOCAL_MEM_SIZE}{cl_ulong}{
\cnglo{locmem}區的字節數。
對於類型不是 \cenum{CL_DEVICE_TYPE_CUSTOM} 的\cnglo{device},最小值是 1KB。%
}

\clETD{CL_DEVICE_ERROR_CORRECTION_SUPPORT}{cl_bool}{
如果所有對\cnglo{computedevmem}(包括\cnglo{glbmem}和\cnglo{constmem})的訪問,
都可以由\cnglo{device}進行糾錯,則為 \cenum{CL_TRUE},否則為 \cenum{CL_FALSE}。
}

\clETD{}{}{}

\clETD{CL_DEVICE_HOST_UNIFIED_MEMORY}{cl_bool}{
如果\cnglo{device}和\cnglo{host}共有一個統一的內存子系統,
則為 \cenum{CL_TRUE},否則為 \cenum{CL_FALSE}。
}

\clETD{}{}{}

\clETD{CL_DEVICE_PROFILING_TIMER_RESOLUTION}{size_t}{
\cnglo{device}定時器的精度。單位是納秒。詳情參見\refsec{profileMoKernel}。
}

\clETD{}{}{}

\clETD{CL_DEVICE_ENDIAN_LITTLE}{cl_bool}{
如果 OpenCL \cnglo{device}是小端(little-endian)的,
則為 \cenum{CL_TRUE},否則為\cenum{CL_FALSE}。
}

\clETD{CL_DEVICE_AVAILABLE}{cl_bool}{
如果\cnglo{device}可用,則為 \cenum{CL_TRUE},否則為\cenum{CL_FALSE}。
}

\clETD{}{}{}

\clETD{CL_DEVICE_COMPILER_AVAILABLE}{cl_bool}{
如果沒有可用的編譯器來編譯程式源碼,則為 \cenum{CL_FALSE},否則為 \cenum{CL_TRUE}。

只有嵌入式平台的規格才可以是 \cenum{CL_FALSE}。
}

\clETD{CL_DEVICE_LINKER_AVAILABLE}{cl_bool}{
如果沒有可用的鏈接器,則為 \cenum{CL_FALSE},否則為 \cenum{CL_TRUE}。

只有嵌入式平台的規格才可以是 \cenum{CL_FALSE}。

如果 \cenum{CL_DEVICE_COMPILER_AVAILABLE} 是 \cenum{CL_TRUE},則他必須是 \cenum{CL_TRUE}。
}

\clETD{}{}{}

\clETD{CL_DEVICE_EXECUTION_CAPABILITIES}{cl_device_exec_capabilities}{
描述\cnglo{device}的執行能力。此位欄包含以下值:
\startigBase
\item \cenum{CL_EXEC_KERNEL}——這個 OpenCL \cnglo{device}可以執行 OpenCL \cnglo{kernel}。
\item \cenum{CL_EXEC_NATIVE_KERNEL}——這個 OpenCL \cnglo{device}可以執行原生\cnglo{kernel}。
\stopigBase

其中 \cenum{CL_EXEC_KERNEL} 是必需的。
}

\clETD{}{}{}

\clETD{CL_DEVICE_QUEUE_PROPERTIES}{cl_command_queue_properties}{
\cnglo{cmdq}的屬性。此位欄包含以下值:
\startigBase
\item \cenum{CL_QUEUE_OUT_OF_ORDER_EXEC_MODE_ENABLE}
\item \cenum{CL_QUEUE_PROFILING_ENABLE}
\stopigBase

參見\reftab{clcmdprop}。

其中 \cenum{CL_QUEUE_PROFILING_ENABLE} 是必需的。
}

\clETD{CL_DEVICE_BUILT_IN_KERNELS}{char[]}{
\cnglo{device}所支持的內建\cnglo{kernel}的清單,以分號分隔。
如果不支持內建\cnglo{kernel},則返回空字串。
}

\if 0
\clETD{}{}{}

\clETD{CL_DEVICE_PLATFORM}{cl_platform_id}{
此\cnglo{device}所關聯的\cnglo{platform}。
}

\clETD{}{}{}

\clETD{CL_DEVICE_NAME}{char[]}{
\cnglo{device}的名字。
}

\clETD{CL_DEVICE_VENDOR}{char[]}{
供應商的名字。
}

\clETD{CL_DEVICE_VERSION}{char[]}{
OpenCL 軟件驅動的版本,格式為:

\cfmt{major_number.minor_number}。
}

\clETD{CL_DEVICE_PROFILE}{char[]}{
OpenCL 規格字串。
返回\cnglo{device}所支持的規格名稱。
可以是下列字串之一:
\startigBase
\item \cenum{FULL_PROFILE}——如果\cnglo{device}支持 OpenCL 規範
(核心規格所定義的功能,不要求支持任何擴展)。

\item \cenum{EMBEDDED_PROFILE}——如果\cnglo{device}支持 OpenCL 嵌入式規格。
\stopigBase

返回的是 OpenCL \cnglo{framework}所實現了的規格。
如果返回的是 \cenum{FULL_PROFILE},
則 OpenCL \cnglo{framework}支持符合 \cenum{FULL_PROFILE} 的\cnglo{device},
可能也支持符合 \cenum{EMBEDDED_PROFILE} 的 \cnglo{device}。
所有 \cnglo{device} 都得有可用的編譯器,
即 \cenum{CL_DEVICE_COMPILER_AVAILABLE} 必須是 \cenum{CL_TRUE}。
而如果返回的是 \cenum{EMBEDDED_PROFILE},
則僅支持符合 \cenum{EMBEDDED_PROFILE} 的\cnglo{device}。
}

\clETD{CL_DEVICE_VERSION}{char[]}{
OpenCL 版本字串。返回\cnglo{device}所支持的 OpenCL 版本。
格式如下:

\cfmt{OpenCL<space><major_version.minor_version><space><vendor-specific information>}

所返回的 \cfmt{major_version.minor_version} 的值將是\scver。
}

\clETD{CL_DEVICE_OPENCL_C_VERSION}{char[]}{
OpenCL C 版本字串。
對於類型不是 \cenum{CL_DEVICE_TYPE_CUSTOM} 的 \cnglo{device},
返回編譯器在其上所支持的 OpenCL C 的最高版本。
格式如下:

\cfmt{OpenCL<space>C<space><major_version.minor_version><space><vendor-specific information>}

如果 \cenum{CL_DEVICE_VERSION} 是 OpenCL \scver,
則 \cfmt{major_version.minor_version} 必須是 \scver。
如果 \cenum{CL_DEVICE_VERSION} 是 OpenCL 1.1,
則 \cfmt{major_version.minor_version} 必須是 1.1。
如果 \cenum{CL_DEVICE_VERSION} 是 OpenCL 1.0,
則 \cfmt{major_version.minor_version} 可以是 1.0 或 1.1。
}

\clETD{CL_DEVICE_EXTENSIONS}{char[]}{
返回\cnglo{device}所支持的擴展名清單,以空格分隔(擴展名本身不包含空格)。
所返回的清單可能包含供應商支持的擴展名,也可能是下列已獲 Khronos 批准的擴展名:
\startigBase
\item \clext{cl_khr_int64_base_atomics}
\item \clext{cl_khr_int64_extended_atomics}
\item \clext{cl_khr_fp16}
\item \clext{cl_khr_gl_sharing}
\item \clext{cl_khr_gl_event}
\item \clext{cl_khr_d3d10_sharing}
\item \clext{cl_khr_media_sharing}
\item \clext{cl_khr_d3d11_sharing}
\stopigBase

對於支持 OpenCL C \scver 的\cnglo{device},
所返回的清單中必須包含下列已獲 Khronos 批准的擴展名:
\startigBase
\item \clext{cl_khr_global_int32_base_atomics}
\item \clext{cl_khr_global_int32_extended_atomics}
\item \clext{cl_khr_local_int32_base_atomics}
\item \clext{cl_khr_local_int32_extended_atomics}
\item \clext{cl_khr_byte_addressable_store}
\item \clext{cl_khr_fp64}(如果支持雙精度浮點,為向後兼容必須支持此擴展)
\stopigBase

詳情請參考《OpenCL \scver 擴展規範》。
}

\clETD{}{}{}
\fi

\clETD{CL_DEVICE_PRINTF_BUFFER_SIZE}{size_t}{
\cnglo{kernel}調用 printf 時,由一個內部緩衝區存儲其輸出,此區域大小的最大值。
對於嵌入式規格,最小為 1KB。
}

\clETD{}{}{}

\clETD{CL_DEVICE_PREFERRED_INTEROP_USER_SYNC}{cl_bool}{
OpenCL 和其他 API (如 DirectX )間共享\cnglo{memobj}時,
如果\cnglo{device}的偏好是讓用戶自己負責同步,則其值為 \cenum{CL_TRUE};
而如果\cnglo{device}或實作已經具備有效的方式來進行同步,則其值為 \cenum{CL_FALSE}。
}

\clETD{CL_DEVICE_PARENT_DEVICE}{cl_device_id}{
返回此\cnglo{subdev}所屬\cnglo{pardev}的 \ctype{cl_device_id}。
如果 \carg{device} 是\cnglo{rootdev},則返回 \cmacro{NULL}。
}

\clETD{CL_DEVICE_PARTITION_MAX_SUB_DEVICES}{cl_uint}{
劃分\cnglo{device}時,所能創建的\cnglo{subdev}的最大數目。

所返回的值不能超過 \cenum{CL_DEVICE_MAX_COMPUTE_UNITS}。
}

\clETD{CL_DEVICE_PARTITION_PROPERTIES}{cl_device_partition_property[]}{
返回 \carg{device} 所支持的劃分方式。
這是一個陣列,元素型別為 \ctype{cl_device_partition_property},其值可以是:
\startigBase
\item \cenum{CL_DEVICE_PARTITION_EQUALLY}
\item \cenum{CL_DEVICE_PARTITION_BY_COUNTS}
\item \cenum{CL_DEVICE_PARTITION_BY_AFFINITY_DOMAIN}
\stopigBase

如果此\cnglo{device}不支持任何劃分方式,則返回 0。
}

\clETD{CL_DEVICE_PARTITION_AFFINITY_DOMAIN}{cl_device_affinity_domain}{
用 \cenum{CL_DEVICE_PARTITION_BY_AFFINITY_DOMAIN} 劃分 \carg{device} 時,
所支持的相似域( affinity domain )。
此位欄的值如下所示:
\startigBase
\item \cenum{CL_DEVICE_AFFINITY_DOMAIN_NUMA}
\item \cenum{CL_DEVICE_AFFINITY_DOMAIN_L4_CACHE}
\item \cenum{CL_DEVICE_AFFINITY_DOMAIN_L3_CACHE}
\item \cenum{CL_DEVICE_AFFINITY_DOMAIN_L2_CACHE}
\item \cenum{CL_DEVICE_AFFINITY_DOMAIN_L1_CACHE}
\item \cenum{CL_DEVICE_AFFINITY_DOMAIN_NEXT_PARTITIONABLE}
\stopigBase

如果以上都不支持,就返回 0。
}

\clETD{CL_DEVICE_PARTITION_TYPE}{cl_device_partition_property[]}{
如果 \carg{device} 是\cnglo{subdev},
則會返回調用 \capi{clCreateSubDevices} 時所指定的引數 \carg{properties}。
否則返回的 \carg{param_value_size_ret} 可能是 0 即不存在任何劃分方式;
或者 \carg{param_value} 所指內存中的屬性值是 0 (0 用來終止屬性清單)。
}

\clETD{CL_DEVICE_REFERENCE_COUNT}{cl_uint}{
返回 \carg{device} 的\cnglo{refcnt}。
如果是\cnglo{rootdev},則返回 1。
}

\stopETD



如果\reftab{cldevquery}中的 \cenum{CL_DEVICE_IMAGE_SUPPORT} 是 \cenum{CL_TRUE},
則實作指派給下列項的值必須大於或等於上表中給出的最小值:
\startigBase[indentnext=no]
\item \cenum{CL_DEVICE_MAX_READ_IMAGE_ARGS}、
\item \cenum{CL_DEVICE_MAX_WRITE_IMAGE_ARGS}、
\item \cenum{CL_DEVICE_IMAGE2D_MAX_WIDTH}、
\item \cenum{CL_DEVICE_IMAGE2D_MAX_HEIGHT}、
\item \cenum{CL_DEVICE_IMAGE3D_MAX_WIDTH}、
\item \cenum{CL_DEVICE_IMAGE3D_MAX_HEIGHT}、
\item \cenum{CL_DEVICE_IMAGE3D_MAX_DEPTH}、
\item \cenum{CL_DEVICE_MAX_SAMPLERS}。
\stopigBase
另外, OpenCL 嵌入式規格的實作必須支持下列圖像格式。

對於 2D 和可選的 3D 圖像,至少要支持下列圖像格式(讀寫都要支持):

\bTABLE[option=stretch]
\bTABLEhead
\bTR[background=color,backgroundcolor=gray]
\bTH image_num_channels \eTH
\bTH image_channel_order \eTH
\bTH image_channel_data_type \eTH
\eTR
\eTABLEhead
\bTABLEbody

\bTR
  \bTD 4 \eTD
  \bTD \cenum{CL_RGBA} \eTD
  \bTD
\cenum{CL_UNORM_INT8}\par
\cenum{CL_UNORM_INT16}\par

\cenum{CL_SIGNED_INT8}\par
\cenum{CL_SIGNED_INT16}\par
\cenum{CL_SIGNED_INT32}\par

\cenum{CL_UNSIGNED_INT8}\par
\cenum{CL_UNSIGNED_INT16}\par
\cenum{CL_UNSIGNED_INT32}\par

\cenum{CL_HALF_FLOAT}\par
\cenum{CL_FLOAT}
  \eTD
\eTR

\eTABLEbody
\eTABLE

\stopcomponent

