\startcomponent cpn-appdatetype
\product opencl-spec-zh

\chapter{應用程式數據型別}

本節中會列出為\cnglo{host}\cnglo{app}提供的數據結構、型別和常量定義。
這些定義是所有\cnglo{platform}和架構所共有的。
這些額外細節可以證實我們要維護一個可移植編程環境的承諾,
潛在地也會阻止對所供應頭檔的修改。

\section{共享的應用程式標量數據型別}

為了讓\cnglo{app}用起來更加方便,提供了下列\cnglo{app}標量型別:
\startclc
/BTEX\cldt{cl_char}/ETEX
/BTEX\cldt{cl_uchar}/ETEX
/BTEX\cldt{cl_short}/ETEX
/BTEX\cldt{cl_ushort}/ETEX
/BTEX\cldt{cl_int}/ETEX
/BTEX\cldt{cl_uint}/ETEX
/BTEX\cldt{cl_long}/ETEX
/BTEX\cldt{cl_ulong}/ETEX
/BTEX\cldt{cl_half}/ETEX
/BTEX\cldt{cl_float}/ETEX
/BTEX\cldt{cl_double}/ETEX
\stopclc

\section{所支持的應用程式矢量數據型別}

\cnglo{app}矢量型別是聯合體,用來創建上一節中標量型別的矢量。
為了讓\cnglo{app}用起來更加方便,提供了下列\cnglo{app}矢量型別:
\startclc[indentnext=no]
/BTEX\cldt[n]{cl_char}/ETEX
/BTEX\cldt[n]{cl_uchar}/ETEX
/BTEX\cldt[n]{cl_short}/ETEX
/BTEX\cldt[n]{cl_ushort}/ETEX
/BTEX\cldt[n]{cl_int}/ETEX
/BTEX\cldt[n]{cl_uint}/ETEX
/BTEX\cldt[n]{cl_long}/ETEX
/BTEX\cldt[n]{cl_ulong}/ETEX
/BTEX\cldt[n]{cl_half}/ETEX
/BTEX\cldt[n]{cl_float}/ETEX
/BTEX\cldt[n]{cl_double}/ETEX
\stopclc
其中 \ccmmsuffix{n} 可以是 2、 3、 4、 8 或 16。

\cnglo{app}標量和矢量數據型別的定義在頭檔 \cemp{cl_platform.h} 中。

\section[appendix:appDataTypeAlignment]{應用程式數據型別的齊位}

用戶要負責確保傳入和傳出 OpenCL 緩衝的數據相對於緩衝的起始位址原生對齊,
遵守\refsec{alignmentOfTypes}中的要求。
其中隱含着,
用 \cenum{CL_MEM_USE_HOST_PTR} 創建 OpenCL 緩衝時所提供的\cnglo{host}內存指位器
必須按\cnglo{kernel}中存取這些緩衝時所用的數據型別進行對齊。
此外,用戶也要負責傳入和傳出 OpenCL 圖像的數據按用來表示單個像素的數據粒度進行對齊
(如:\ccmm{image_num_channels * sizeof(image_channel_data_type)},
不過對於 \cenum{CL_RGB} 和 \cenum{CL_RGBx} 圖像,只需按像素的單個通道的粒度進行對齊即可
(即 \ccmm{sizeof(image_channel_data_type)})。

對於那些由\cnglo{app}所定義的、既不是緩衝也不是圖像的數據型別, OpenCL 不對其齊位作任何要求,
但是如果包含矢量(如 \ccmm{__cl_float4}),
必須要能用 \ccmm{cl_type} 聯合體中欄位的名字直接存取才行(\refappendixsec{appendixVecCom})。
儘管如此,還是建議頭檔 \cemp{cl_platform.h} 將 OpenCL 所定義的\cnglo{app}數據型別
(如 \ccmm{cl_float4})按其型別自然地對齊。

\section{常值矢量}

\cnglo{app}常值矢量可用來對矢量進行組件級賦值。
使用常值矢量時編譯器可能會對其進行轉換。
\startclc
cl_float2 foo = { .s[1] = 2.0f };
cl_int8 bar = {{ 2, 4, 6, 8, 10, 12, 14, 16 }};
\stopclc

\section[sec:appendixVecCom]{矢量組件}

可以使用符號 \ccmm{<vector_name>.s[<index>]} 對\cnglo{app}矢量型別的組件進行尋址。

例如:
\startclc
foo.s[0] = 1.0f; // Sets the 1st vector component of foo
pos.s[6] = 2; // Sets the 7th vector component of bar
\stopclc

一些情況下,也可以使用下列符號存取矢量組件。
不保證所有實作都會支持這些符號,
所以使用他們時要檢查一下相應的預處理器記號。

\subsection{命名矢量組件符號}

可以使用 \ccmm{.sN}、 \ccmm{.sn} 或 \ccmm{.xyzw} 來存取矢量數據型別的組件,
跟 OpenCL 語言中所用類似。
\ccmm{.xyzw} 僅可存取前四個組件。
可以通過預處理器記號 \cmacro{CL_HAS_NAMED_VECTOR_FIELDS} 來識別是否支持這些符號。
例如:
\startclc
#ifdef CL_HAS_NAMED_VECTOR_FIELDS
	cl_float4 foo;
	cl_int16 bar;
	foo.x = 1.0f;	// Set first component
	foo.s0 = 1.0f;	// Same as above
	bar.z = 3;	// Set third component
	bar.se = 11;	// Same as bar.s[0xe]
	bar.sD = 12;	// Same as bar.s[0xd]
#endif
\stopclc

與 OpenCL 語言不同的是,一次只能存取一個組件。
這個局限使得不能像 OpenCL 語言型別那樣對組件進行重排或複製。
如果存取時組件編號越界會導致失敗。

\startclc
foo.xy		// illegal - illegal field name combination
bar.s1234	// illegal - illegal field name combination
foo.s7		// illegal - no component s7
\stopclc

\subsection{High/Low 矢量組件符號}

可以使用 \ccmm{.hi} 和 \ccmm{.lo} 來存取矢量數據型別的組件,
跟所支持的語言型別類似。
可以通過預處理器記號 \cmacro{CL_HAS_HI_LO_VECTOR_FIELDS} 來識別是否支持這些符號。
例如:
\startclc
#ifdef CL_HAS_HI_LO_VECTOR_FIELDS
	cl_float4 foo;
	cl_float2 new_hi = 2.0f, new_lo = 4.0f;
	foo.hi = new_hi;
	foo.lo = new_lo;
#endif
\stopclc

\subsection{原生矢量型別符號}

為了將矢量型別映射到內建矢量型別上,還定義了原生矢量型別。
不同於上述的\cnglo{app}矢量型別,
這些原生型別僅僅是在有限的基礎上才支持,取決於所支持的架構和編譯器。

這些型別不是聯合體,但可以更方便地映射到底層架構中的內建矢量型別上。
原生型別的名字與對應的\cnglo{app}型別一樣,只不過帶有前綴雙下劃線“\ccmm{__}”。

例如, \ccmm{__cl_float4} 就是一個原生矢量型別,
對應於\cnglo{app}矢量型別 \ccmm{cl_float4}。
用 \ccmm{__cl_float4} 可能可以直接存取內建的 \ccmm{__m128} 或矢量浮點型別,
而 \ccmm{cl_float4} 則會被視為聯合體。

另外,為方便存取,上述\cnglo{app}數據型別可能含有原生矢量數據型別的成員。
用子矢量符號 \ccmm{.vN} 存取原生組件,其中 \ccmm{N} 為子矢量中的元素編號。
如果原生型別是一個更大型別(具有更多組件)的子集,
則此符號就成為子矢量型別的一個基於索引的陣列。

通過與原生型別名相匹配的預處理器記號 \ccmm{__CL_TYPEN__} 來識別是否支持原生矢量型別。
例如:
\startclc
#ifdef __CL_FLOAT4__	// Check for native cl_float4 type
	cl_float8 foo;
	__cl_float4 bar; // Use of native type
	bar = foo.v4[1]; // Access the second native float4 vector
#endif
\stopclc

\section{隱式轉換}

不支持\cnglo{app}矢量型別間的隱式轉換。

\section{顯式轉型}

不支持對\cnglo{app}矢量型別 \cldt[n]{cl_type} 的顯式轉型。
對原生矢量型別 \cldt[n]{__cl_type} 的顯式轉型則由外部編譯器定義。

\section{其他算子和函式}

對於\cnglo{app}矢量型別 \cldt[n]{cl_type} 和原生矢量型別 \cldt[n]{__cl_type} 而言,
標準算子和函式的行為由外部編譯器定義。

\section{應用程式常量的定義}

除了以上\cnglo{app}型別的定義,還有下列常值定義。

\startCLOD[常量][描述]

\clMD{CL_CHAR_BIT}{字符位寬}
\clMD{CL_SCHAR_MAX}{型別 \ctype{cl_char} 的最大值}
\clMD{CL_SCHAR_MIN}{型別 \ctype{cl_char} 的最小值}
\clMD{CL_CHAR_MAX}{型別 \ctype{cl_char} 的最大值}
\clMD{CL_CHAR_MIN}{型別 \ctype{cl_char} 的最小值}
\clMD{CL_UCHAR_MAX}{型別 \ctype{cl_uchar} 的最大值}
\clMD{CL_SHORT_MAX}{型別 \ctype{cl_short} 的最大值}
\clMD{CL_SHORT_MIN}{型別 \ctype{cl_short} 的最小值}
\clMD{CL_USHORT_MAX}{型別 \ctype{cl_ushort} 的最大值}
\clMD{CL_INT_MAX}{型別 \ctype{cl_int} 的最大值}
\clMD{CL_INT_MIN}{型別 \ctype{cl_int} 的最小值}
\clMD{CL_UINT_MAX}{型別 \ctype{cl_uint} 的最大值}
\clMD{CL_LONG_MAX}{型別 \ctype{cl_long} 的最大值}
\clMD{CL_LONG_MIN}{型別 \ctype{cl_long} 的最小值}
\clMD{CL_ULONG_MAX}{型別 \ctype{cl_ulong} 的最大值}

\clMD{CL_FLT_DIAG}{型別 \ctype{cl_float} 的精度,十進制小數的位數}
\clMD{CL_FLT_MANT_DIG}{型別 \ctype{cl_float} 的尾數的數字位數}
\clMD{CL_FLT_MAX_10_EXP}{型別 \ctype{cl_float} 所能表示的規格化浮點數中能以 \math{10^n - 1} 表示的數中 \math{n} 的最大值,其中 \math{n} 為正整數。}
\clMD{CL_FLT_MAX_EXP}{型別 \ctype{cl_float} 的最大指數值}
\clMD{CL_FLT_MIN_10_EXP}{型別 \ctype{cl_float} 所能表示的規格化浮點數中能以 \math{10^n - 1} 表示的數中 \math{n} 的最小值,其中 \math{n} 為負整數。}
\clMD{CL_FLT_MIN_EXP}{型別 \ctype{cl_float} 的最小指數值}
\clMD{CL_FLT_RADIX}{型別 \ctype{cl_float} 的底數}
\clMD{CL_FLT_MAX}{型別 \ctype{cl_float} 的最大值}
\clMD{CL_FLT_MIN}{型別 \ctype{cl_float} 的最小值}
\clMD{CL_FLT_EPSILON}{型別為 \ctype{cl_float} 且能使 \ccmm{1.0 + CL_FLT_EPSILON != 1} 為真的最小正浮點數}

\clMD{CL_DBL_DIAG}{型別 \ctype{cl_double} 的精度,十進制小數的位數}
\clMD{CL_DBL_MANT_DIG}{型別 \ctype{cl_double} 的尾數的數字位數}
\clMD{CL_DBL_MAX_10_EXP}{型別 \ctype{cl_double} 所能表示的規格化浮點數中能以 \math{10^n - 1} 表示的數中 \math{n} 的最大值,其中 \math{n} 為正整數。}
\clMD{CL_DBL_MAX_EXP}{型別 \ctype{cl_double} 的最大指數值}
\clMD{CL_DBL_MIN_10_EXP}{型別 \ctype{cl_double} 所能表示的規格化浮點數中能以 \math{10^n - 1} 表示的數中 \math{n} 的最小值,其中 \math{n} 為負整數。}
\clMD{CL_DBL_MIN_EXP}{型別 \ctype{cl_double} 的最小指數值}
\clMD{CL_DBL_RADIX}{型別 \ctype{cl_double} 的底數}
\clMD{CL_DBL_MAX}{型別 \ctype{cl_double} 的最大值}
\clMD{CL_DBL_MIN}{型別 \ctype{cl_double} 的最小值}
\clMD{CL_DBL_EPSILON}{型別為 \ctype{cl_double} 且能使 \ccmm{1.0 + CL_DBL_EPSILON != 1} 為真的最小正浮點數}

\clMD{CL_NAN}{此巨集展開後的值表示 NaN}
\clMD{CL_HUGE_VALF}{型別 \ctype{cl_float} 可表示的最大值}
\clMD{CL_HUGE_VAL}{型別 \ctype{cl_double} 可表示的最大值}
\clMD{CL_MAXFLOAT}{型別 \ctype{cl_float} 的最大值}
\clMD{CL_INFINITY}{此巨集展開後的值表示 INF}

\stopCLOD


這些常值定義位於頭檔 \cemp{cl_platform.h} 中。

\stopcomponent

