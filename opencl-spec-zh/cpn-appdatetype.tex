\startcomponent cpn-appdatetype
\product opencl-spec-zh

\chapter{應用數據型別}

本節中會列出為\cnglo{host}\cnglo{app}提供的數據結構、型別和常量定義。
這些定義是所有\cnglo{platform}和架構所共有的。
這些額外細節可以證實我們要維護一個可移植編程環境的承諾,
潛在地也會阻止對所供應頭檔的修改。

\section{共享的應用標量數據型別}

為了讓\cnglo{app}用起來更加方便,提供了下列\cnglo{app}標量型別:
\startclc
/BTEX\cldt{cl_char}/ETEX
/BTEX\cldt{cl_uchar}/ETEX
/BTEX\cldt{cl_short}/ETEX
/BTEX\cldt{cl_ushort}/ETEX
/BTEX\cldt{cl_int}/ETEX
/BTEX\cldt{cl_uint}/ETEX
/BTEX\cldt{cl_long}/ETEX
/BTEX\cldt{cl_ulong}/ETEX
/BTEX\cldt{cl_half}/ETEX
/BTEX\cldt{cl_float}/ETEX
/BTEX\cldt{cl_double}/ETEX
\stopclc

\section{所支持的應用矢量數據型別}

\cnglo{app}矢量型別是聯合體,用來創建上一節中標量型別的矢量。
為了讓\cnglo{app}用起來更加方便,提供了下列\cnglo{app}矢量型別:
\startclc[indentnext=no]
/BTEX\cldt[n]{cl_char}/ETEX
/BTEX\cldt[n]{cl_uchar}/ETEX
/BTEX\cldt[n]{cl_short}/ETEX
/BTEX\cldt[n]{cl_ushort}/ETEX
/BTEX\cldt[n]{cl_int}/ETEX
/BTEX\cldt[n]{cl_uint}/ETEX
/BTEX\cldt[n]{cl_long}/ETEX
/BTEX\cldt[n]{cl_ulong}/ETEX
/BTEX\cldt[n]{cl_half}/ETEX
/BTEX\cldt[n]{cl_float}/ETEX
/BTEX\cldt[n]{cl_double}/ETEX
\stopclc
其中 \ccmmsuffix{n} 可以是 2、 3、 4、 8 或 16。

\cnglo{app}標量和矢量數據型別的定義在頭檔 \cemp{cl_platform.h} 中。

\section{應用數據型別的齊位}

用戶要負責確保傳入和傳出 OpenCL 緩衝的數據相對於緩衝的起始位址原生對齊,
遵守\refsec{alignmentOfTypes}中的要求。
其中隱含着,
用 \cenum{CL_MEM_USE_HOST_PTR} 創建 OpenCL 緩衝時所提供的\cngo{host}內存指針
必須按\cnglo{kernel}中訪問這些緩衝時所用的數據型別進行對齊。
此外,用戶也要負責傳入和傳出 OpenCL 圖像的數據按用來表示單個像素的數據粒度進行對齊
(如:\ccmm{image_num_channels * sizeof(image_channel_data_type)},
不過對於 \cenum{CL_RGB} 和 \cenum{CL_RGBx} 圖像,只需按像素的單個通道的粒度進行對齊即可
(即 \ccmm{sizeof(image_channel_data_type)})。

對於那些由\cnglo{app}所定義的、既不是緩衝也不是圖像的數據型別, OpenCL 不對其齊位作任何要求,
但是如果包含矢量(如 \ccmm{__cl_float4}),
必須要能用 \ccmm{cl_type} 聯合體中欄位的名字直接訪問才行(\refsec{appendixVecCom})。
儘管如此,還是建議頭檔 \cemp{cl_platform.h} 將 OpenCL 所定義的\cnglo{app}數據型別
(如 \ccmm{cl_float4})按其型別自然地對齊。

\section{常值矢量}

\cnglo{app}常值矢量可用來對矢量進行組件級賦值。
使用常值矢量時編譯器可能會對齊進行轉換。
\startclc
cl_float2 foo = { .s[1] = 2.0f };
cl_int8 bar = {{ 2, 4, 6, 8, 10, 12, 14, 16 }};
\stopclc

\section[sec:appendixVecCom]{矢量組件}

可以使用符號 \ccmm{<vector_name>.s[<index>]} 對\cnglo{app}矢量型別的組件進行尋址。

例如:
\startclc
foo.s[0] = 1.0f; // Sets the 1st vector component of foo
pos.s[6] = 2; // Sets the 7th vector component of bar
\stopclc

一些情況下,也可以使用下列符號訪問矢量組件。
不保證所有實作都會支持這些符號,
所以使用他們時要檢查一下相應的預處理器記號。

\subsection{命名矢量組件符號}

\stopcomponent

