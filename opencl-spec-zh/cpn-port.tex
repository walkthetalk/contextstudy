\startcomponent cpn-port
\product opencl-spec-zh

\chapter{可移植性}

OpenCL 的設計目標之一就是要能移植到其他架構和硬件上。
OpenCL 的核心是基於 C99 的編程語言。
浮點算術基於 {\ftEmp{IEEE-754}} 和 {\ftEmp{IEEE-754-2008}} 標準。
設計\cnglo{memobj}、指針限定符以及弱序內存(weakly ordered memory)的目的就是
要為 OpenCL \cnglo{device}所實現的分離式內存架構提供最大的兼容性。
\cnglo{cmdq}和\cnglo{barrier}允許在\cnglo{host}和 OpenCL \cnglo{device}間進行同步。
OpenCL 的設計、能力以及限制很好的反映了下層硬件的能力。

不幸的是,有很多事情某種硬件能夠做到,而對於其他硬件來說則無能為力。
拜常駐 CPU 的操作系統所賜,在一些實作中,
\cnglo{kernel}在 CPU 上執行時可以調出系統服務,
而在 GPU 上同樣的調用則可能會失敗,至少目前如此。(參見\refchapter{optExt})。
出於調試的目的,這些服務還是非常有用的,實作可以用 OpenCL 的擴展機制來實現這些服務。

同樣,計算架構的不同可能意味着,一個在 CPU 上執行速度還可以接受的循環結構,
在 GPU 上執行時性能可能會很差。
CPU 一般都是為單線程任務、時延敏感的算法設計的,因此這種任務執行的很好,
而 GPU 則可能遭遇相當長的時延,甚至是指數級的惡化。
志於編寫可移植代碼的開發人員可能會發現,
要想確保關鍵算法在多種設備上都工作的很好,在每種設備上都要進行測試。
我們建議增加\cnglo{workitem}的數目。
希望在未來的歲月中逐漸積累經驗,能夠產生一些統一的最佳實踐,
讓我們能夠在多種計算設備上都能夠使用。

我們可能更關心的是端序。
鑒於最開始 OpenCL 的一些實作都是針對小端設備的,
開發人員需要確保其\cnglo{kernel}在大端、小端設備上都做過測試,
從而保證源碼對現在以及將來的 OpenCL 設備的兼容性。
OpenCL 編程語言還支持端序特性限定符,
允許開發人員指定數據使用\cnglo{host}的端序還是 OpenCL \cnglo{device}的端序。
這樣使得 OpenCL 編譯器在裝載、存儲數據時可以進行適當的端序轉換。

下面我們來描述端序是怎樣導致意料之外的結果的:

當大端矢量機器(如 AltiVec, CELL SPE)裝載矢量時,數據的次序保持不變,
即每個矢量元素內部的字節序以及元素間的次序都與內存中的一樣。
當小端矢量機器(如 SSE)裝載矢量時,會將寄存器中的數據反序(所有工作都是在寄存器中完成的),
每個矢量元素內部的字節序以及元素間的次序{\ftEmp{都}}會反序。
\define[5]\endianExample{%
\bTABLE[frame=on,style={\tt\tf},color=blue,]
\bTR[align={middle,middle}]
\bTD[frame=off,width=.2\textwidth]#1 \eTD
\bTD[width=.2\textwidth]#2\eTD
\bTD[width=.2\textwidth]#3\eTD
\bTD[width=.2\textwidth]#4\eTD
\bTD[width=.2\textwidth]#5\eTD
\eTR
\eTABLE
}

內存:

\endianExample{uint4 a = }{\ftEmp 0x00010203}{0x04050607}{0x08090A0B}{0x0C0D0E0F}

寄存器(大端):

\endianExample{uint4 a = }{\ftEmp 0x00010203}{0x04050607}{0x08090A0B}{0x0C0D0E0F}

寄存器(小端):

\endianExample{uint4 a = }{0x0F0E0D0C}{0x0B0A0908}{0x07060504}{\ftEmp 0x03020100}

無論矢量中每塊數據有多大,小端機器一次就能將其全部裝載。
也就是說,矢量為 \ctype{uchar16} 還是 \ctype{ulong2},所進行的變換都一樣。
當然,眾所周知,對於陣列元素,小端機器\inmargin{%
注意,此處我們談論的是編程模型。
事實上,小端系統可能選擇其他方式,如只是簡單地從“右側”尋址或將字節中的位反序。
這兩種方式都意味着硬件上無需大的交換空間。
}實際上按反向字節序存儲數據,以作為小端存儲格式的補償:

內存(大端):

\endianExample{uint4 a = }{\ftEmp 0x00010203}{0x04050607}{0x08090A0B}{0x0C0D0E0F}

內存(小端):

\endianExample{uint4 a = }{\ftEmp 0x03020100}{0x07060504}{0x0B0A0908}{0x0F0E0D0C}

一旦數據裝載到矢量中,以此結束:

寄存器(大端):

\endianExample{uint4 a = }{\ftEmp 0x00010203}{0x04050607}{0x08090A0B}{0x0C0D0E0F}

寄存器(小端):

\endianExample{uint4 a = }{0x0C0D0E0F}{0x08090A0B}{0x04050607}{\ftEmp 0x00010203}

在糾正每個元素內的字節序過程中,同時也會反轉元素間的次序。
\ccmm{0x00010203} 位於大端矢量的左側,但位於小端矢量的右側。

如果\cnglo{host}和\cnglo{device}具有不同的端序,
則開發人員需要保證\cnglo{kernel}引數的值得到了正確處理。
實作不一定會自動轉換\cnglo{kernel}引數的端序。
至於這些情況下要如何處置\cnglo{kernel}引數,開發人員需要查閱供應商的文檔。

通過按元素在內存中的次序進行編號, OpenCL 為不同架構提供了一致的編程模型。
\ccmm{even/odd} 和 \ccmm{high/low} 的概念也是這樣的。
一旦將數據裝載到寄存器中,我們會發現元素 \ccmm{0} 是大端矢量的最左側元素,
卻是小端矢量的最右側元素。

\startclc
float x[4];
float4 v = vload4( 0, x );
\stopclc

大端:
\startclc
v contains { x[0], x[1], x[2], x[3] }
\stopclc

小端:
\startclc
v contains { x[3], x[2], x[1], x[0] }
\stopclc

編譯器知道所發生的交換並據此引用元素。
只要我們是通過數值索引(像 \ccmm{.s0123456789abcdef})或
描述符(像 \ccmm{.xyzw}、 \ccmm{.hi}、 \ccmm{.lo}、 \ccmm{.even} 和 \ccmm{.odd})
來訪問這些元素,那麼這種交換就是透明的。
當將數據存儲回內存中時,所有次序反轉都會被撤銷。
對於大多數問題而言,開發人員都應當按大端編程模型工作,並忽略矢量元素的次序問題。
此機制依賴於以下事實:我們可以信賴始終如一的元素編號。
一旦我們改變的編號系統,例如通過無需轉換的轉型(使用 \cldt[n]{as_type})
將一個矢量轉型成另外一個大小相同但元素個數不同的矢量,
則在不同的實作中可能得到不同的結果,這取決於系統是大端的、小端的,還是根本沒有矢量單元。
(因此,對於元素數目不同的矢量間的直轉(bitcast),
其行為\cnglo{impdef},參見\refsec{reinterpret}。)

看下面這個例子:
\startclc[indentnext=no]
float x[4] = { 0.0f, 1.0f, 2.0f, 3.0f };
float4 v = vload4( 0, x );
uint4 y = (uint4) v;		// legal, portable
ushort8 z = (ushort8) v;	// legal, not portable
				// element size changed

Big-endian:
	v contains { 0.0f, 1.0f, 2.0f, 3.0f }
	y contains { 0x00000000, 0x3f800000,
			0x40000000, 0x40400000 }
	z contains { 0x0000, 0x0000, /BTEX\ftEmp{0x3f80}/ETEX, 0x0000,
			0x4000, 0x0000, 0x4040, 0x0000 }
	z.z is 0x3f80

Little-endian:
	v contains { 3.0f, 2.0f, 1.0f, 0.0f }
	y contains { 0x40400000, 0x40000000,
			0x3f800000, 0x00000000 }
	z contains { 0x4040, 0x0000, 0x4000,
			0x0000, 0x3f80, /BTEX\ftEmp{0x0000}/ETEX, 0x0000, 0x0000 }
	z.z is 0
\stopclc
其中 \ccmm{z.z} 中的值在大端機器和小端機器上是不同的。

OpenCL 以移植性的名義規定,如果無需轉換的轉型改變了元素的數目,則\cnglo{illegal}。
然而,雖然 OpenCL 提供了一組公共的矢量算子(取自矢量機器),
還是不能以一種始終如一、統一且可移植的方式來訪問每個 ISA 都會提供的所有東西。
許多矢量 ISA 提供了一些特定目的的指令,可以極大地加速一些特定運算,
像 DCT、 SAD、或 3D 幾何。
我們不打算讓 OpenCL 如此笨重以致難以上手,
如果那樣的話,在編寫對事件有嚴格要求、性能敏感的算法時,
即使是經驗豐富的開發人員也難以使其接近頂峰性能。
開發人員更傾向於拋開移植性,在代碼中使用特定平台的指令。
有鑒於此, OpenCL 允許傳統的矢量 C 語言編程擴展作為 OpenCL 的擴展直接使用 OpenCL 數據型別,
像 AltiVec C 編程接口,或 Intel C 編程接口(在 \ccmm{emmintrin.h} 中可以找到)。
由於這些接口要想正常運作,依賴於無需轉換、但會改變矢量元素數目的轉型,
所以 OpenCL 也允許這種轉型。

作為通用規則,任何矢量型別的運算,只要元素數目與實際矢量型別的元素數目不同,
那麼他在其他不同端序或不同矢量架構的硬件上運行時都可能終止。

這樣的例子有:
\startigBase
\item 使用算子 \ccmm{.even} 和 \ccmm{.odd} 將兩個 \ccmm{uchar8} 分別
作為 \ccmm{ushort} 的高字節和低字節組合成一個 \ccmm{ushort8}
(請使用 \capi{upsample},參見\refsec{integerFunc})。

\item 任何會改變矢量元素數目的直轉。(新型別上的運算是不可移植的。)

\item 所用塊大小與元素大小不同,且會改變數據次序的重排運算。
\stopigBase

下列運算則則是可移植的:
\startigBase
\item 使用算子 \ccmm{.even} 和 \ccmm{.odd} 將兩個 \ccmm{uint8} 組合成一個 \ccmm{uchar16}。
例如將左右兩個聲道的音頻流間插到一起。

\item 任何不會改變矢量元素數目的直轉(如 \ccmm{(float4)uint4},為浮點型別定義存儲格式)。

\item 單位為矢量元素的重排運算。
\stopigBase

OpenCL 為 C 增加了一些東西,時\cnglo{app}的行為更可靠。
最主要的就是 OpenCL 中定義了下列運算的行為,而這些在 C99 中並沒有定義:
\startigBase
\item OpenCL 為所有型別間的轉換都提供了 \ccmm{convert_} 算子。
在將浮點型別轉換成整數型別時,在捨入後浮點值可能仍然超出了整數可表示的範圍,
這時會發生什麼在 C99 中沒有定義。
如果使用了 \ccmm{_sat} 轉換,則浮點值會被轉換成最近的可表示的整數。
類似地,對於 NaN 會發生什麼, OpenCL 也給出了建議。
對於提供有硬件實現的飽和轉換的硬件製造商,
OpenCL \ccmm{convert_} 算子的飽和版本和非飽和版本可能都會使用此硬件。
而對於非飽和版本,如果浮點算元在捨入後仍然超出了整數可表示的範圍,
這時會發生什麼 OpenCL 則沒有定義。

\item 在 IEEE-754 標準草案中,型別 \ctype{half}、 \ctype{float} 和 \ctype{double}
的格式定義為 binary16、 binary32 和 binary64。
(後兩個在現有的 IEEE-754 標準中是相同的。)
你可能要依賴於此定義中各個位的位置和意義。

\item OpenCL 中定義了移位越界時的行為。
如果移位運算的移位數大於或等於第一個算子的位數,則會對齊取模。
例如,如果要將 \ccmm{int4} 左移 \ccmm{33} 位,則 OpenCL 將按左移 \ccmm{33%32 = 1} 對待。

\item 對於數學庫函式中的大量邊界條件,比 C99 定義的更加嚴格。參見\refsec{edgeCaseBehavior}。
\stopigBase

\stopcomponent
