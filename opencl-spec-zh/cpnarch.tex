%
% author:	Ni Qingliang
% date:		2011-02-11
%
\startcomponent cpnarch
\product opencl-spec-zh
%\starttext

\chapter{OpenCL 架構}

{\ftEmp{OpenCL}}是一個開放的工業標準,可以為 CPU、 GPU 以及其他分散的計算設備
(這些設備被組織到同一\cnglo{platform}中)所組成的異構群進行編程。
他不只是一種語言。
OpenCL 是一個並行編程\cnglo{framework},
包括一種語言、 API、庫以及一個運行時系統來支持軟件開發。
例如,使用 OpenCL,程式員可以寫出一個能在 GPU 上執行的通用程式,
而不必將其算法映射到 3D 圖形 API (如 OpenGL 或 DirectX)上。

OpenCL 的目標是讓程式員可以寫出可移植並且高效的代碼。
這包括庫作者、中間件供應商,以及以性能為導向的應用程式員。
因此 OpenCL 提供了底層的硬件抽象和一個\cnglo{framework}來支持編程,
同時也暴露了底層硬件的許多細節。

我們將使用如下的模型體系來描述 OpenCL 背後的核心理念:
\startigBase
\item 平台模型
\item 內存模型
\item 執行模型
\item 編程模型
\stopigBase

\section{平台模型}

OpenCL 平台模型的定義可以查看\reffig{plfmodel}。
此模型中有一個\empglo{host},
以及與其相連的一個或多個 {\ftEmp{OpenCL}} \cnglo{device}。
OpenCL \cnglo{device}被劃分成一個或多個\cnglo{computeunit}(CU),
每個\cnglo{computeunit}又被劃分成一個或多個\cnglo{prcele}(PE)。
\cnglo{device}上的計算是在\cnglo{prcele}中進行的。

\startbuffer[buffigplfmodelcaption]
\cnglo{platform}模型 …… 一個\cnglo{host}加上一個或多個計算\cnglo{device},
每個\cnglo{device}具有一個或多個\cnglo{computeunit},
每個\cnglo{computeunit}具有一個或多個\cnglo{prcele}。
\stopbuffer

\placefigure
[here,force][fig:plfmodel]
{\getbuffer[buffigplfmodelcaption]}
{\useMPgraphic{box}}

OpenCL \cnglo{app}會按照\cnglo{host}\cnglo{platform}的原生模型
在這個\cnglo{host}上運行。
\cnglo{host}上的 OpenCL \cnglo{app}提交\empglo{cmd}
給\cnglo{device}中的\cnglo{prcele}以執行計算任務。
\cnglo{computeunit}中的\cnglo{prcele}會作為 SIMD 單元(執行指令流的步伐一致)
或 SPMD 單元(每個 PE 維護自己的程序計數器)執行指令流。

\subsection{平台對多版本的支持}
OpenCL 的設計目標就是要支持同一\cnglo{platform}中多種具有不同能力的設備。
這些設備可以符合不同版本的 OpenCL 規範。
對於一個 OpenCL 系統,有三個重要的版本 ID 要考慮:
\cnglo{platform}版本、\cnglo{device}版本、
\cnglo{device}所支持的 OpenCL C 語言的版本。

\cnglo{platform}版本表明了所支持的 OpenCL 運行時的版本。
這包括可由\cnglo{host}用來與 OpenCL 運行時交互的所有 API,
像\cnglo{context}、\cnglo{memobj}、\cnglo{device}、\cnglo{cmdq}。

\cnglo{device}版本表明了\cnglo{device}的能力,
獨立於運行時和編譯器的版本,由 \capi{clGetDeviceInfo} 所返回的\cnglo{device}信息來描述。
有很多特性都與\cnglo{device}版本有關,如資源限制和擴展功能。
所返回的版本號即為此\cnglo{device}所符合的 OpenCL 規範的最高版本號,
但不會高於\cnglo{platform}版本。

語言版本,可以讓開發人員據此知道此\cnglo{device}所支持的 OpenCL 編程語言具備哪些特性。
此版本會是所支持語言的最高版本。

OpenCL C 被設計為向後兼容的,
因此對於一個\cnglo{device}而言,只要支持語言的某一個版本,就可以說它符合標準。
如果某個\cnglo{device}支持多個語言版本,編譯器缺省使用最高的那個版本。
語言的版本不會高於\cnglo{platform}的版本,
但可能高於\cnglo{device}的版本(參見\refsec{ctrlcveroption})。



\section{執行模型}
OpenCL \cnglo{program}的執行分為兩種情況:
在一個或多個 {\ftEmp{OpenCL}} \cngloemp{device}上執行\cngloemp{kernel};
在\cnglo{host}上執行\cngloemp{host}\cngloemp{program}。
\cnglo{host}\cnglo{program}為\cnglo{kernel}定義了\cnglo{context}
並管理\cnglo{kernel}的執行。

OpenCL 執行模型的核心就是\cnglo{kernel}是怎麼執行的。
\cnglo{host}提交\cnglo{kernel}時會定義一個索引空間。
\cnglo{kernel}的實例會在此空間中的所有點上執行。
\cnglo{kernel}的實例稱為\cngloemp{workitem},
通過在索引空間中的坐標來標識,這個坐標就是\cnglo{workitem}的\cnglo{glbid}。
所有\cnglo{workitem}都會執行相同的代碼,但是代碼的執行路徑和參與運算的數據可能會不同。

\cnglo{workitem}被組織到\cngloemp{workgrp}中。
\cnglo{workgrp}以更粗粒度對索引空間進行了分解。
\cnglo{workgrp}帶有一個唯一的 ID,它與\cnglo{workitem}所使用的索引空間具有同樣的維數。
\cnglo{workitem}具有一個\cnglo{locid},此 ID 在其所隸屬的\cnglo{workgrp}中是唯一的;
因此任一\cnglo{workitem}都可以通過其\cnglo{glbid}
或其\cnglo{locid}加\cnglo{workgrp} ID 來唯一標識。
同一\cnglo{workgrp}中的\cnglo{workitem}
會在同一\cnglo{computeunit}中的多個\cnglo{prcele}上並發執行。

在 OpenCL 中,索引空間又叫做 NDRange。
 NDRange 是一個 N 維的索引空間,其中 N 可以是一、二或者三。
 NDRange 由一個長度為 N 的整數數組來定義,
指定了索引空間各個維度上的寬度(起自偏移索引 F, F 缺省為 0)。
每個\cnglo{workitem}的\cnglo{glbid}和\cnglo{locid}都是 N 維元組。
\cnglo{glbid}的取值範圍從 F 開始,直到 F 加相應維度上的元素個數減一。

\cnglo{workgrp}的 ID 跟\cnglo{workitem}的\cnglo{glbid}差不多。
一個長度為 N 的數組定義了每個維度上\cnglo{workgrp}的數目。
\cnglo{workitem}在所隸屬的\cnglo{workgrp}中有一個\cnglo{locid},
此 ID 中各維度的取值範圍為0到\cnglo{workgrp}在相應維度上的大小減一。
因此,\cnglo{workgrp}的 ID 加上其中一個\cnglo{locid}可以唯一確定一個\cnglo{workitem}。
有兩種途徑來標識一個\cnglo{workitem}:
根據全局索引,或根據\cnglo{workgrp}索引加一個局部索引。

接下來請看\reffig{indexspace}中的二維索引空間。
\cnglo{workitem}的索引空間為 \math{(G_x, G_y)},
每個\cnglo{workgrp}的大小是 \math{(S_x, S_y)},
\cnglo{glbid}的偏移量是 \math{(F_x, F_y)}。
全局索引定義了一個 \math{G_x} 乘 \math{G_y} 的索引空間,
所能容納的\cnglo{workitem}總數是 \math{G_x} 和 \math{G_y} 的乘積。
局部索引定義了一個 \math{S_x} 乘 \math{S_y} 的索引空間,
一個\cnglo{workgrp}中所能容納\cnglo{workitem}的數目是 \math{S_x} 和 \math{S_y} 的乘積。
如果知道每個\cnglo{workgrp}的大小和\cnglo{workitem}的總數,
就能算出有多少\cnglo{workgrp}。
\cnglo{workgrp}是由一個二維的索引空間來唯一標識的。
\cnglo{workitem}可以用它的\cnglo{glbid} \math{(g_x, g_y)} 標識,
或用\cnglo{workgrp} ID \math{(w_x, w_y)}、
\cnglo{workgrp}的大小 \math{(S_x, S_y)} 和
在\cnglo{workgrp}中的\cnglo{locid} \math{(s_x, s_y)} 三項組合起來標識:
\startformula
(g_x, g_y) = (w_x * S_x + s_x + F_x, w_y * S_y + s_y + F_y)
\stopformula

\cnglo{workgrp}的數目可以這樣計算:
\startformula
(W_x, W_y) = (G_x / S_x, G_y / S_y)
\stopformula

給定\cnglo{glbid}和\cnglo{workgrp}大小,\cnglo{workitem}所屬\cnglo{workgrp}的 ID 為:
\startformula
(w_x, w_y) = ((g_x - s_x - F_x) / S_x, (g_y - s_y - F_y) / S_y)
\stopformula

\startbuffer[buffigindexspacecaption]
NDRange 索引空间的示例,包括\cnglo{workitem}、其\cnglo{glbid}以及相應的 ID 元組:
\cnglo{workgrp} ID 和\cnglo{locid}。
\stopbuffer
\placefigure
[here,force][fig:indexspace]
{\getbuffer[buffigindexspacecaption]}
{\useMPgraphic{box}}

很多編程模型都可以映射到這個執行模型上。
 OpenCL 明確支持的有兩種:\cngloemp{dppm}和\cngloemp{tppm}。

% Execution Model: Context and Command Queues
\subsection[sec:exemodel:contextandcmdq]{執行模型:上下文和命令隊列}

\cnglo{host}為執行\cnglo{kernel}定義了一個\cnglo{context}。
\cnglo{context}包括以下\cnglo{res}:
\startigNum
\item \cngloemp{device}:\cnglo{host}可以使用的 OpenCL \cnglo{device}集。
\item \cngloemp{kernel}:運行在 OpenCL \cnglo{device}上的 OpenCL 函數。
\item \cngloemp{programobj}:實現\cnglo{kernel}的\cnglo{program}源碼和執行體。
\item \cngloemp{memobj}:一組\cnglo{memobj},
對\cnglo{host}和 OpenCL \cnglo{device}可見。
\cnglo{memobj}包含一些值,\cnglo{kernel}實例可以在其上進行運算。
\stopigBase

\cnglo{host}使用 OpenCL API 中的函數來創建並操控\cnglo{context}。
\cnglo{host}會創建一個稱為\cnglo{cmdq}的數據結構
來協調\cnglo{device}上\cnglo{kernel}的執行。
\cnglo{host}還會將\cnglo{cmd}插入\cnglo{cmdq},
這些\cnglo{cmd}將在\cnglo{context}中的\cnglo{device}上被調度。
這些\cnglo{cmd}包括:
\startigBase
\item {\ftEmp{\cnglo{kernel}執行命令}}:
在\cnglo{device}的\cnglo{prcele}上執行\cnglo{kernel}。

\item {\ftEmp{内存命令}}:讀寫\cnglo{memobj}或者在\cnglo{memobj}間傳輸數據,
或者從\cnglo{host}的地址空間中映射、解映射\cnglo{memobj}。

\item {\ftEmp{同步命令}}:限制命令的執行順序。
\stopigBase

\cnglo{cmdq}負責\cnglo{cmd}的調度,使其可以在\cnglo{device}上執行。
在\cnglo{host}和\cnglo{device}上,\cnglo{cmd}的執行是異步的。
\cnglo{cmd}的執行有兩種模式:
\startigBase
\item \cngloemp{inordexec}:
\cnglo{cmd}嚴格按照在\cnglo{cmdq}中出現的順序開始和結束執行。
換言之,前面的\cnglo{cmd}結束後,才能執行後面的\cnglo{cmd}。
這使隊列中\cnglo{cmd}的執行順序串行化。

\item \cngloemp{outordexec}:
按順序執行\cnglo{cmd},但後續\cnglo{cmd}執行前不必等待前面\cnglo{cmd}結束。
任何順序上的限制都是由程序員通過顯式的同步\cnglo{cmd}强加的。
\stopigBase

提交給隊列的\cnglo{kernel}執行命令和內存命令都會生成\cnglo{evtobj}。
這些\cnglo{evtobj}可以用來控制\cnglo{cmd}的執行順序、
協調\cnglo{cmd}在\cnglo{host}和\cnglo{device}間的運行。

一個\cnglo{context}中可以有多個隊列。
這些隊列並發運行、相互獨立,OpenCL 中沒有顯式的機制來對它們進行同步。

% Execution Model: Categories of Kernels
\subsection{執行模型:內核的種類}

OpenCL 執行模型支持兩種\cnglo{kernel}:
\startigBase
\item {\ftEmp{OpenCL \cnglo{kernel}}},
用 OpenCL C 編程語言寫就,並用 OpenCL C 編譯器編譯而成。
所有 OpenCL 的實現都支持 OpenCL \cnglo{kernel}。
實現也可能提供其它機制創建 OpenCL \cnglo{kernel}。

\item {\ftEmp{原生\cnglo{kernel}}},
通過\cnglo{host}函數指針訪問。
原生\cnglo{kernel}與 OpenCL \cnglo{kernel}一起入隊在\cnglo{device}上執行,
並共享\cnglo{memobj}。
例如,這些原生\cnglo{kernel}可以是\cnglo{app}代碼中定義的函數,也可以是從庫中導出的函數。
注意,在 OpenCL 中,
執行原生\cnglo{kernel}的能力是一個可選功能,原生\cnglo{kernel}的語義\cnglo{impdef}。
可以使用 OpenCL API 中的一些函數來查詢\cnglo{device}的能力
並確定\cnglo{device}是否具有某種能力。
\stopigBase



\section{內存模型}
\cnglo{workitem}在執行\cnglo{kernel}時可以存取四塊不同的\cnglo{memregion}:

\startigBase
\item \cngloemp{glbmem}:
所有\cnglo{workgrp}中的所有\cnglo{workitem}都可以對其進行讀寫。
\cnglo{workitem}可以讀寫\cnglo{memobj}中的任意元素。
對\cnglo{glbmem}的讀寫可能會被緩存起來,這取決於\cnglo{device}的能力。

\item \cngloemp{constmem}:
\cnglo{glbmem}中的一塊區域,在\cnglo{kernel}的執行過程中保持不變。
\cnglo{host}負責對此中\cnglo{memobj}的分配和初始化。

\item \cngloemp{locmem}:
隸屬於某個\cnglo{workgrp}。
可以用來分配一些變量,這些變量由此\cnglo{workgrp}中的所有\cnglo{workitem}共享。
在 OpenCL \cnglo{device}上,可能會將其實現成一塊專用的\cnglo{memregion},
也可能將其映射到\cnglo{glbmem}中。

\item \cngloemp{prvmem}:
隸屬於某個\cnglo{workitem}。
一個\cnglo{workitem}的\cnglo{prvmem}中所定義的變量
對另外一個\cnglo{workitem}而言是不可見的。
\stopigBase

\reftab{memregion}列出了這些資訊:
\cnglo{kernel}或\cnglo{host}是否可以從某個\cnglo{memregion}中分配內存、
怎樣分配(靜態編譯時 vs. 動態運行時)
以及允許如何存取(即\cnglo{kernel}或\cnglo{host}是否可以對其進行讀寫)。

\placetable[here,force][tab:memregion]{內存區域——分配以及存取}{
\bTABLE

\bTABLEhead
\bTR
\bTD[nc=2] \eTD \bTD\cnglo{glbmem}\eTD \bTD\cnglo{constmem}\eTD \bTD\cnglo{locmem}\eTD \bTD\cnglo{prvmem}\eTD
\eTR
\eTABLEhead

\bTABLEbody
\bTR
\bTD[nr=2]\cnglo{host}\eTD \bTD分配\eTD \bTD動態\eTD \bTD動態\eTD \bTD動態\eTD \bTD NO\eTD
\eTR

\bTR
\bTD 訪問 \eTD \bTD 讀寫 \eTD \bTD 讀寫 \eTD \bTD NO \eTD \bTD NO \eTD
\eTR

\bTR
\bTD[nr=2]\cnglo{kernel}\eTD \bTD 分配 \eTD \bTD NO \eTD \bTD 靜態 \eTD \bTD 靜態 \eTD \bTD 靜態 \eTD
\eTR

\bTR
\bTD 訪問 \eTD \bTD 讀寫 \eTD \bTD 只讀 \eTD \bTD 讀寫 \eTD \bTD 讀寫 \eTD
\eTR
\eTABLEbody

\eTABLE


}

\reffig{openclarch}描述了\cnglo{memregion}以及與\cnglo{platform}模型的關係。
圖中含有\cnglo{prcele}( PE )、\cnglo{computeunit}和\cnglo{device},
但是沒有畫出\cnglo{host}。

\placefigure
[here,force][fig:openclarch]
{OpenCL 設備架構的概念模型}
{\useMPgraphic{box}}

\cnglo{app}在\cnglo{host}上運行時,
使用 OpenCL API 在\cnglo{glbmem}中創建\cnglo{memobj},
並將內存命令(\refsec{exemodel:contextandcmdq}中有所描述)入隊以操作他們。

多數情況下,\cnglo{host}和 OpenCL \cnglo{device}的內存模型是相互獨立的。
其必然性主要在於 OpenCL 沒有囊括\cnglo{host}的定義。
然而,有時他們確實需要交互。
有兩種交互方式:顯式拷貝數據、將\cnglo{memobj}的部分区域映射和解映射。

為了顯式拷貝數據,\cnglo{host}會將一些\cnglo{cmd}插入隊列,
用來在\cnglo{memobj}和\cnglo{host}內存之間傳輸數據。
這些用於傳輸內存的命令可以是阻塞式的,也可以是非阻塞式的。
對於前者,一旦\cnglo{host}上相關內存資源可以被安全的重用,OpenCL 函式調用就會立刻返回。
而對於後者,一旦命令入隊,OpenCL 函式調用就會返回,而不管\cnglo{host}內存是否可以安全使用。

用映射、解映射的方法處理\cnglo{host}和 OpenCL \cnglo{memobj}的交互時,
\cnglo{host}可以將\cnglo{memobj}的某個區域映射到自己的位址空間中。
內存映射命令可能是阻塞的,也可能是非阻塞的。
一旦映射了\cnglo{memobj}的某個區域,\cnglo{host}就可以讀寫這塊區域。
當\cnglo{host}對這塊區域的存取(讀和/或寫)結束後,就會將其解映射。

% Memory Consistency
\subsection{內存一致性}
OpenCL 所使用的一致性內存模型比較寬鬆;
即,不保證不同\cnglo{workitem}所看到的內存狀態始終一致。

在\cnglo{workitem}內部,內存具有裝載、存儲的一致性。
在隸屬於同一\cnglo{workgrp}的\cnglo{workitem}之間,
\cnglo{locmem}在\cnglo{workgrpbarrier}上是一致的。
\cnglo{glbmem}亦是如此,
但對於執行同一\cnglo{kernel}的不同\cnglo{workgrp},則不保證其內存一致性。

對於已經入隊的\cnglo{cmd}所共享的\cnglo{memobj},其內存一致性由同步點來強制實施。


\section{編程模型}
OpenCL 執行模型支持\cnglo{dppm}和\cnglo{tppm},同時也支持這兩種模型的混合體。
對於 OpenCL 而言,驅動其設計的首要模型是數據並行。

% Data Parallel Programming Model
\subsection{數據並行編程模型}

\cnglo{dppm}依據同時應用到\cnglo{memobj}的多個元素上的指令序列來定義計算( computation )。
OpenCL 執行模型所關聯的索引空間定義了\cnglo{workitem},
以及數據怎樣映射到\cnglo{workitem}上。
在嚴格的數據並行模型中,\cnglo{workitem}和\cnglo{memobj}的元素間的映射關係為一對一,
\cnglo{kernel}可在這些元素上並行執行。
對於\cnglo{dppm},OpenCL 所實現的版本則比較寬鬆,不要求嚴格的一對一的映射。

OpenCL 所提供的\cnglo{dppm}是分級的。
有兩種方式來進行分級。
一種是顯式方式,程序員定義可以並行執行的\cnglo{workitem}的總數、
以及怎樣將這些\cnglo{workitem}劃分到\cnglo{workgrp}中。
另一種是隱式方式,程序員僅指定前者,後者由 OpenCL 實作來管理。

% Task Parallel Programming Model
\subsection{任務並行編程模型}
在 OpenCL 的\cnglo{tppm}中,\cnglo{kernel}的實體在執行時獨立於任何索引空間。
這在邏輯上等同於:在\cnglo{computeunit}上執行\cnglo{kernel}時,
相應\cnglo{workgrp}中只有一個\cnglo{workitem}。
這種模型下,用戶以如下方式表示並行:
\startigBase
\item 使用\cnglo{device}所實現的矢量數據型別;
\item 將多個任務入隊,和/或
\item 將多個原生\cnglo{kernel}入隊(他們是使用一個與 OpenCL 正交的編程模型開發的)。
\stopigBase

% Synchronization
\subsection{同步}
在 OpenCL 中,有兩方面的同步:
\startigBase
\item 隸屬同一\cnglo{workgrp}的\cnglo{workitem}之間;
\item 隸屬同一\cnglo{context}的\cnglo{cmdq}中的\cnglo{cmd}之間。
\stopigBase

前者是通過\cnglo{workgrpbarrier}實現的。
對於同一\cnglo{workgrp}中的所有\cnglo{workitem}來說,
任意一個要想越過\cnglo{barrier}繼續執行,
所有\cnglo{workitem}都必須先執行這個\cnglo{barrier}。
注意,在同一\cnglo{workgrp}中,
所有正在執行\cnglo{kernel}的\cnglo{workitem}必須都能執行到這個\cnglo{workgrpbarrier},
或者都不會去執行。
\cnglo{workgrp}之間沒有同步機制。

\cnglo{cmdq}中\cnglo{cmd}間的同步點是:
\startigBase
\item \cnglo{cmdqbarrier}。
他保證:所有之前排隊的\cnglo{cmd}都執行完畢,
並且他們對\cnglo{memobj}的所有更新,在後續\cnglo{cmd}開始執行前都是可見的。
他只能在隸屬同一\cnglo{cmdq}的\cnglo{cmd}間進行同步。

\item 等在一個事件上。
所有會將\cnglo{cmd}入隊的 OpenCL API 函式都會返回一個事件
(用來標識這個\cnglo{cmd}以及他所更新的\cnglo{memobj})。
如果某個後續\cnglo{cmd}正在等待那個事件,可以保證在其開始執行前,
可以見到對那些\cnglo{memobj}的所有更新。
\stopigBase


\section{內存對象}

\cnglo{memobj}分成兩類:\refglo{bufobj}和\refglo{imgobj}。
\refglo{bufobj}中所存儲的元素是一維的,
而\refglo{imgobj}則用來存儲二維或三維的材質、幀緩衝(frame-buffer)或圖像。

\refglo{bufobj}中的元素可以是標量數據型別(如 \ctype{int}、 \ctype{float})、
矢量數據型別或用戶自定義的結構體。
\refglo{imgobj}用來表示材質、幀緩衝等緩衝。
\cnglo{imgobj}中元素的格式必須從預定義格式中選取。
\cnglo{memobj}中至少要有一個元素。

\refglo{bufobj}和\refglo{imgobj}的根本區別是:
\startigBase
\item \refglo{bufobj}中的元素是順序存儲的,
\cnglo{kernel}在\cnglo{device}上運行時可以用指位器存取這些元素。
而\refglo{imgobj}中元素的存儲格式對用戶是透明的,不能通過指位器直接存取。
可以使用 OpenCL C 編程語言提供的內建函式來讀寫\cnglo{imgobj}。

\item 對於\refglo{bufobj},\cnglo{kernel}按其存儲格式存取其中的數據。
而對於\refglo{imgobj},其元素的存儲格式可能與\cnglo{kernel}中使用的數據格式不一樣。
\cnglo{kernel}中的圖像元素始終是四元矢量(每一元都可以是浮點型別或者帶符號/無符號整形)。
內建函式在讀寫圖像元素時會進行相應的格式轉換。
\stopigBase

\cnglo{memobj}是用 \ctype{cl_mem} 來表示的。
\cnglo{kernel}的輸入和輸出都是\cnglo{memobj}。


\section{OpenCL 框架}
OpenCL 框架中,一個\cnglo{host}、加上不少於一個的 OpenCL \cnglo{device}就可以構成一個異構並行計算機系統,由\cnglo{app}來使用。
這個框架包含以下組件:
\startigBase
\item {\ftEmp{OpenCL 平台層:}}\cnglo{host}\cnglo{program}可以發現 OpenCL \cnglo{device}及其能力,也可以創建\cnglo{context}。

\item {\ftEmp{OpenCL runtime:}}一旦創建了\cnglo{context},\cnglo{host}\cnglo{program}就可以操控它。

\item {\ftEmp{OpenCL 編譯器:}}OpenCL 編譯器可以創建包含 OpenCL \cnglo{kernel}的程序執行體。
它所實現的 OpenCL C 編程語言支持 ISO C99 的一個子集,並帶有並行擴展。
\stopigBase



\stopcomponent

