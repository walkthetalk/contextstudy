\startETD[cl_device_info][返回型別]

\clETD{CL_DEVICE_TYPE}{cl_device_type}{
OpenCL \cnglo{device}類型,當前支持:
\startigBase
\item \cenum{CL_DEVICE_TYPE_CPU},
\item \cenum{CL_DEVICE_TYPE_GPU},
\item \cenum{CL_DEVICE_TYPE_ACCELERATOR},
\item \cenum{CL_DEVICE_TYPE_DEFAULT},或者
\item 以上值的组合,或者
\item \cenum{CL_DEVICE_TYPE_CUSTOM}。
\stopigBase
}

\clETD{CL_DEVICE_VENDOR_ID}{cl_uint}{
唯一的\cnglo{device}供應商標識符。例如可以是 PCIe ID。
}

\clETD{CL_DEVICE_MAX_COMPUTE_UNITS}{cl_uint}{
OpenCL \cnglo{device}上的並行\cnglo{computeunit}的數目。最小值是 1。
}

\clETD{CL_DEVICE_MAX_WORK_ITEM_DIMENSIONS}{cl_uint}{
\cnglo{dppm}中所用全局和局部\cnglo{workitem} ID 的最大維數
(參見 \capi{clEnqueueNDRangeKernel})。
對於類型不是 \cenum{CL_DEVICE_TYPE_CUSTOM} 的\cnglo{device},其最小值是 3。
}

\clETD{CL_DEVICE_MAX_WORK_ITEM_SIZES}{size_t[]}{
對\capi{clEnqueueNDRangeKernel}而言,
\cnglo{workgrp}中每個維度上可以指派\cnglo{workitem}的最大數目。

返回 \carg{n} 個型別為 \ctype{size_t} 的表項。
其中 \carg{n} 是查詢 \cenum{CL_DEVICE_MAX_WORK_ITEM_DIMENSIONS} 時所返回的值。

對於不是 \cenum{CL_DEVICE_TYPE_CUSTOM} 的\cnglo{device},最小值是\math{(1, 1, 1)}。
}

\clETD{CL_DEVICE_MAX_WORK_GROUP_SIZE}{size_t}{
用\cnglo{dppm}執行\cnglo{kernel}時,
\cnglo{workgrp}中所能容納\cnglo{workitem}的最大數目
(参见 \capi{clEnqueueNDRangeKernel})。
最小值是 1。
}

\clETDMulti{%
\cenum{CL_DEVICE_PREFERRED_VECTOR_WIDTH_CHAR}\par
\cenum{CL_DEVICE_PREFERRED_VECTOR_WIDTH_SHORT}\par
\cenum{CL_DEVICE_PREFERRED_VECTOR_WIDTH_INT}\par
\cenum{CL_DEVICE_PREFERRED_VECTOR_WIDTH_LONG}\par
\cenum{CL_DEVICE_PREFERRED_VECTOR_WIDTH_FLOAT}\par
\cenum{CL_DEVICE_PREFERRED_VECTOR_WIDTH_DOUBLE}\par
\cenum{CL_DEVICE_PREFERRED_VECTOR_WIDTH_HALF}
}{cl_uint}{
可以放入矢量中的內建標量型別所期望的原生矢量的寬度。
矢量寬度定義為可以可以容納標量元素的數目。

如果不支持雙精度浮點數,\cenum{CL_DEVICE_PREFERRED_VECTOR_WIDTH_DOUBLE} 必須返回0。

如果不支持擴展 \clext{cl_khr_fp16},
\cenum{CL_DEVICE_PREFERRED_VECTOR_WIDTH_HALF} 必須返回0。
}

\clETDMulti{
\cenum{CL_DEVICE_NATIVE_VECTOR_WIDTH_CHAR}\par
\cenum{CL_DEVICE_NATIVE_VECTOR_WIDTH_SHORT}\par
\cenum{CL_DEVICE_NATIVE_VECTOR_WIDTH_INT}\par
\cenum{CL_DEVICE_NATIVE_VECTOR_WIDTH_LONG}\par
\cenum{CL_DEVICE_NATIVE_VECTOR_WIDTH_FLOAT}\par
\cenum{CL_DEVICE_NATIVE_VECTOR_WIDTH_DOUBLE}\par
\cenum{CL_DEVICE_NATIVE_VECTOR_WIDTH_HALF}
}{cl_uint}{
返回原生 ISA 矢量寬度。
此矢量寬度定義為所能容納標量元素的數目。

如果不支持雙精度浮點數,\cenum{CL_DEVICE_NATIVE_VECTOR_WIDTH_DOUBLE} 必須返回0。

如果不支持擴展 \clext{cl_khr_fp16},
\cenum{CL_DEVICE_NATIVE_VECTOR_WIDTH_HALF} 必須返回0。
}

\clETD{CL_DEVICE_MAX_CLOCK_FREQUENCY}{cl_uint}{
\cnglo{device}的時鐘頻率可以配置成的最大值,單位:MHz。
}

\clETD{CL_DEVICE_ADDRESS_BITS}{cl_uint}{
計算設備的位址空間缺省大小,無符號整數,單位:bit。
當前支持 32 位或 64 位。
如果嵌入式規格報告的值是 64,則必須支持擴展 \clext{cles_khr_int64}。%
}

\problem{多了一個 CL_DEVICE_MAX_WORK_GROUP_SIZE}

\clETD{CL_DEVICE_MAX_MEM_ALLOC_SIZE}{cl_ulong}{\problem{unsigned long long}
所能分配的\cnglo{memobj}大小的最大值,單位:字節。
對於類型不是 \cenum{CL_DEVICE_TYPE_CUSTOM} 的\cnglo{device},此值最小為:

\math{max(\menum{CL_DEVICE_GLOBAL_MEM_SIZE} * 1/4, 1 * 1024 * 1024)}%
}

\clETD{}{}{}

\clETD{CL_DEVICE_IMAGE_SUPPORT}{cl_bool}{
如果 OpenCL \cnglo{device}支持圖像,則為 \cenum{CL_TRUE},否則為 \cenum{CL_FALSE}。
}

\clETD{CL_DEVICE_MAX_READ_IMAGE_ARGS}{cl_uint}{
\cnglo{kernel}可以同時讀取多少\cnglo{imgobj}。
如果 \cenum{CL_DEVICE_IMAGE_SUPPORT} 是 \cenum{CL_TRUE},則此值至少是 8。%
}

\clETD{CL_DEVICE_MAX_WRITE_IMAGE_ARGS}{cl_uint}{
\cnglo{kernel}可以同時寫入多少\cnglo{imgobj}。
如果 \cenum{CL_DEVICE_IMAGE_SUPPORT} 是 \cenum{CL_TRUE},則此值至少是 1。%
}

\clETD{CL_DEVICE_IMAGE2D_MAX_WIDTH}{size_t}{
2D 图像的最大寬度,單位:像素。%
如果 \cenum{CL_DEVICE_IMAGE_SUPPORT} 是 \cenum{CL_TRUE},則此值至少是 2048。%
}

\clETD{CL_DEVICE_IMAGE2D_MAX_HEIGHT}{size_t}{
2D 圖像的最大高度,單位:像素。
如果 \cenum{CL_DEVICE_IMAGE_SUPPORT} 是 \cenum{CL_TRUE},則此值至少是 2048。%
}

\clETD{CL_DEVICE_IMAGE3D_MAX_WIDTH}{size_t}{
3D 圖像的最大寬度,單位:像素。
如果 \cenum{CL_DEVICE_IMAGE_SUPPORT} 是 \cenum{CL_TRUE},則此值至少是 0。%
}

\clETD{CL_DEVICE_IMAGE3D_MAX_HEIGHT}{size_t}{
3D 圖像的最大高度,單位:像素。
如果 \cenum{CL_DEVICE_IMAGE_SUPPORT} 是 \cenum{CL_TRUE},則此值至少是 0。%
}

\clETD{CL_DEVICE_IMAGE3D_MAX_DEPTH}{size_t}{
3D 圖像的最大深度,單位:像素。
如果 \cenum{CL_DEVICE_IMAGE_SUPPORT} 是 \cenum{CL_TRUE},則此值至少是 0。%
}

\clETD{CL_DEVICE_IMAGE_MAX_BUFFER_SIZE}{size_t}{
由\cnglo{bufobj}所創建的 1D 圖像的最大像素數。
如果 \cenum{CL_DEVICE_IMAGE_SUPPORT} 是 \cenum{CL_TRUE},則此值至少是 2048。%
}

\clETD{CL_DEVICE_IMAGE_MAX_ARRAY_SIZE}{size_t}{
1D 或 2D 圖像陣列中圖像的最大數目。
如果 \cenum{CL_DEVICE_IMAGE_SUPPORT} 是 \cenum{CL_TRUE},則此值至少是 256。%
}

\clETD{CL_DEVICE_MAX_SAMPLERS}{cl_uint}{
一個\cnglo{kernel}內最多可以使用多少個\cnglo{sampler}。
關於\cnglo{sampler}的細節請參考\refsec{imgRwFunc}。

如果 \cenum{CL_DEVICE_IMAGE_SUPPORT} 是 \cenum{CL_TRUE},則此值至少是 8。%
}

\clETD{}{}{}

\clETD{CL_DEVICE_MAX_PARAMETER_SIZE}{size_t}{
\cnglo{kernel}引數的最大字節數。

如果設備類型不是 \cenum{CL_DEVICE_TYPE_CUSTOM},則此值至少要是 256。%
}

\clETD{CL_DEVICE_MEM_BASE_ADDR_ALIGN}{cl_uint}{
如果設備類型不是 \cenum{CL_DEVICE_TYPE_CUSTOM},
至少要是\cnglo{device}所支持的 OpenCL 內建數據型別中最大的那種的大小,單位:bit。
( FULL 規格中是 \ctype{long16},EMBEDDED 規格中是 \ctype{long16} 或 \ctype{int16} )
}

\clETD{}{}{}

\clETD{CL_DEVICE_SINGLE_FP_CONFIG}{cl_device_fp_config}{
描述\cnglo{device}的單精度浮點能力。此位欄支持下列值:
\startigBase
\item \cenum{CL_FP_DENORM}——支持去規格化數( denorm )。
\item \cenum{CL_FP_INF_NAN}——支持 INF 和 qNaN。
\item \cenum{CL_FP_ROUND_TO_NEAREST}——支持捨入為最近偶數。
\item \cenum{CL_FP_ROUND_TO_ZERO}——支持向零捨入。
\item \cenum{CL_FP_ROUND_TO_INF}——支持向正無窮和負無窮捨入。
\item \cenum{CL_FP_FMA}——支持 IEEE754-2008 中的積和熔加运算(fused multiply-add, FMA)。
\item \cenum{CL_FP_CORRECTLY_ROUNDED_DIVIDE_SQRT}——除法和開方可以按 IEEE754 規範進行正確的捨入。
\item \cenum{CL_FP_SOFT_FLOAT}——軟件中實現了基本的浮點運算(加、減、乘)。
\stopigBase

如果\cnglo{device}類型不是 \cenum{CL_DEVICE_TYPE_CUSTOM},其浮點能力至少要是:
\cenum{CL_FP_ROUND_TO_ZERO} 或 \cenum{CL_FP_ROUND_TO_NEAREST}。%
}

\clETD{CL_DEVICE_DOUBLE_FP_CONFIG}{cl_device_fp_config}{
描述\cnglo{device}的雙精度浮點能力。此位欄支持下列值:
\startigBase
\item \cenum{CL_FP_DENORM}——支持去規格化數。
\item \cenum{CL_FP_INF_NAN}——支持 INF 和 qNaN。
\item \cenum{CL_FP_ROUND_TO_NEAREST}——支持捨入為最近偶數。
\item \cenum{CL_FP_ROUND_TO_ZERO}——支持向零捨入。
\item \cenum{CL_FP_ROUND_TO_INF}——支持向正無窮和負無窮捨入。
\item \cenum{CL_FP_FMA}——支持 IEEE75-2008 中的積和熔加运算(fused multiply-add, FMA)。
\item \cenum{CL_FP_SOFT_FLOAT}——軟件中實現了基本的浮點運算(加、減、乘)。
\stopigBase

由於雙精度浮點是一個可選特性,所以最小的雙精度浮點能力可以是0。

而如果\cnglo{device}支持雙精度浮點,則其能力至少要是:

\cenum{CL_FP_FMA} \textbar

\cenum{CL_FP_ROUND_TO_NEAREST} \textbar

\cenum{CL_FP_ROUND_TO_ZERO} \textbar

\cenum{CL_FP_ROUND_TO_INF} \textbar

\cenum{CL_FP_INF_NAN} \textbar

\cenum{CL_FP_DENORM}。
}

\clETD{}{}{}

\clETD{CL_DEVICE_GLOBAL_MEM_CACHE_TYPE}{cl_device_mem_cache_type}{
所支持的全局內存緩存的類型。其值可以是:
\startigBase
\item \cenum{CL_NONE},
\item \cenum{CL_READ_ONLY_CACHE} 和
\item \cenum{CL_READ_WRITE_CACHE}。
\stopigBase
}

\clETD{CL_DEVICE_GLOBAL_MEM_CACHELINE_SIZE}{cl_uint}{
\cnglo{glbmem}緩存列(cache line)的字節數。
}

\clETD{CL_DEVICE_GLOBAL_MEM_CACHE_SIZE}{cl_ulong}{
\cnglo{glbmem}緩存的字節數。
}

\clETD{CL_DEVICE_GLOBAL_MEM_SIZE}{cl_ulong}{
全局\cnglo{device}內存的字節數。
}

\clETD{}{}{}

\clETD{CL_DEVICE_MAX_CONSTANT_BUFFER_SIZE}{cl_ulong}{
一次所能分配的常量緩衝區的最大字節數。
對於類型不是 \cenum{CL_DEVICE_TYPE_CUSTOM} 的\cnglo{device},最小值是 1KB。%
}

\clETD{CL_DEVICE_MAX_CONSTANT_ARGS}{cl_uint}{
單個\cnglo{kernel}中,最多能有多少個參數在聲明時帶有限定符\cqlf{__constant}。
對於類型不是 \cenum{CL_DEVICE_TYPE_CUSTOM} 的\cnglo{device},最小值是 4。
}

\clETD{}{}{}

\clETD{CL_DEVICE_LOCAL_MEM_TYPE}{cl_device_local_mem_type}{
所支持的\cnglo{locmem}的類型。
可以是 \cenum{CL_LOCAL}(意指專用的\cnglo{locmem},如 SRAM )或 \cenum{CL_GLOBAL}。

對於\cnglo{customdev},如果不支持\cnglo{locmem},可以返回 \cenum{CL_NONE}。
}

\clETD{CL_DEVICE_LOCAL_MEM_SIZE}{cl_ulong}{
\cnglo{locmem}區的字節數。
對於類型不是 \cenum{CL_DEVICE_TYPE_CUSTOM} 的\cnglo{device},最小值是 1KB。%
}

\clETD{CL_DEVICE_ERROR_CORRECTION_SUPPORT}{cl_bool}{
如果所有對\cnglo{computedevmem}(包括\cnglo{glbmem}和\cnglo{constmem})的存取,
都可以由\cnglo{device}進行糾錯,則為 \cenum{CL_TRUE},否則為 \cenum{CL_FALSE}。
}

\clETD{}{}{}

\clETD{CL_DEVICE_HOST_UNIFIED_MEMORY}{cl_bool}{
如果\cnglo{device}和\cnglo{host}共有一個統一的內存子系統,
則為 \cenum{CL_TRUE},否則為 \cenum{CL_FALSE}。
}

\clETD{}{}{}

\clETD{CL_DEVICE_PROFILING_TIMER_RESOLUTION}{size_t}{
\cnglo{device}定時器的精度。單位是納秒。詳情參見\refsec{profileMoKernel}。
}

\clETD{}{}{}

\clETD{CL_DEVICE_ENDIAN_LITTLE}{cl_bool}{
如果 OpenCL \cnglo{device}是小端(little-endian)的,
則為 \cenum{CL_TRUE},否則為\cenum{CL_FALSE}。
}

\clETD{CL_DEVICE_AVAILABLE}{cl_bool}{
如果\cnglo{device}可用,則為 \cenum{CL_TRUE},否則為\cenum{CL_FALSE}。
}

\clETD{}{}{}

\clETD{CL_DEVICE_COMPILER_AVAILABLE}{cl_bool}{
如果沒有可用的編譯器來編譯程式源碼,則為 \cenum{CL_FALSE},否則為 \cenum{CL_TRUE}。

只有嵌入式平台的規格才可以是 \cenum{CL_FALSE}。
}

\clETD{CL_DEVICE_LINKER_AVAILABLE}{cl_bool}{
如果沒有可用的鏈接器,則為 \cenum{CL_FALSE},否則為 \cenum{CL_TRUE}。

只有嵌入式平台的規格才可以是 \cenum{CL_FALSE}。

如果 \cenum{CL_DEVICE_COMPILER_AVAILABLE} 是 \cenum{CL_TRUE},則他必須是 \cenum{CL_TRUE}。
}

\clETD{}{}{}

\clETD{CL_DEVICE_EXECUTION_CAPABILITIES}{cl_device_exec_capabilities}{
描述\cnglo{device}的執行能力。此位欄包含以下值:
\startigBase
\item \cenum{CL_EXEC_KERNEL}——這個 OpenCL \cnglo{device}可以執行 OpenCL \cnglo{kernel}。
\item \cenum{CL_EXEC_NATIVE_KERNEL}——這個 OpenCL \cnglo{device}可以執行原生\cnglo{kernel}。
\stopigBase

其中 \cenum{CL_EXEC_KERNEL} 是必需的。
}

\clETD{}{}{}

\clETD{CL_DEVICE_QUEUE_PROPERTIES}{cl_command_queue_properties}{
\cnglo{cmdq}的屬性。此位欄包含以下值:
\startigBase
\item \cenum{CL_QUEUE_OUT_OF_ORDER_EXEC_MODE_ENABLE}
\item \cenum{CL_QUEUE_PROFILING_ENABLE}
\stopigBase

參見\reftab{clcmdprop}。

其中 \cenum{CL_QUEUE_PROFILING_ENABLE} 是必需的。
}

\clETD{CL_DEVICE_BUILT_IN_KERNELS}{char[]}{
\cnglo{device}所支持的內建\cnglo{kernel}的清單,以分號分隔。
如果不支持內建\cnglo{kernel},則返回空字串。
}

\if 0
\clETD{}{}{}

\clETD{CL_DEVICE_PLATFORM}{cl_platform_id}{
此\cnglo{device}所關聯的\cnglo{platform}。
}

\clETD{}{}{}

\clETD{CL_DEVICE_NAME}{char[]}{
\cnglo{device}的名字。
}

\clETD{CL_DEVICE_VENDOR}{char[]}{
供應商的名字。
}

\clETD{CL_DEVICE_VERSION}{char[]}{
OpenCL 軟件驅動的版本,格式為:

\cfmt{major_number.minor_number}。
}

\clETD{CL_DEVICE_PROFILE}{char[]}{
OpenCL 規格字串。
返回\cnglo{device}所支持的規格名稱。
可以是下列字串之一:
\startigBase
\item \cenum{FULL_PROFILE}——如果\cnglo{device}支持 OpenCL 規範
(核心規格所定義的功能,不要求支持任何擴展)。

\item \cenum{EMBEDDED_PROFILE}——如果\cnglo{device}支持 OpenCL 嵌入式規格。
\stopigBase

返回的是 OpenCL \cnglo{framework}所實現了的規格。
如果返回的是 \cenum{FULL_PROFILE},
則 OpenCL \cnglo{framework}支持符合 \cenum{FULL_PROFILE} 的\cnglo{device},
可能也支持符合 \cenum{EMBEDDED_PROFILE} 的 \cnglo{device}。
所有 \cnglo{device} 都得有可用的編譯器,
即 \cenum{CL_DEVICE_COMPILER_AVAILABLE} 必須是 \cenum{CL_TRUE}。
而如果返回的是 \cenum{EMBEDDED_PROFILE},
則僅支持符合 \cenum{EMBEDDED_PROFILE} 的\cnglo{device}。
}

\clETD{CL_DEVICE_VERSION}{char[]}{
OpenCL 版本字串。返回\cnglo{device}所支持的 OpenCL 版本。
格式如下:

\cfmt{OpenCL<space><major_version.minor_version><space><vendor-specific information>}

所返回的 \cfmt{major_version.minor_version} 的值將是\scver。
}

\clETD{CL_DEVICE_OPENCL_C_VERSION}{char[]}{
OpenCL C 版本字串。
對於類型不是 \cenum{CL_DEVICE_TYPE_CUSTOM} 的 \cnglo{device},
返回編譯器在其上所支持的 OpenCL C 的最高版本。
格式如下:

\cfmt{OpenCL<space>C<space><major_version.minor_version><space><vendor-specific information>}

如果 \cenum{CL_DEVICE_VERSION} 是 OpenCL \scver,
則 \cfmt{major_version.minor_version} 必須是 \scver。
如果 \cenum{CL_DEVICE_VERSION} 是 OpenCL 1.1,
則 \cfmt{major_version.minor_version} 必須是 1.1。
如果 \cenum{CL_DEVICE_VERSION} 是 OpenCL 1.0,
則 \cfmt{major_version.minor_version} 可以是 1.0 或 1.1。
}

\clETD{CL_DEVICE_EXTENSIONS}{char[]}{
返回\cnglo{device}所支持的擴展名清單,以空格分隔(擴展名本身不包含空格)。
所返回的清單可能包含供應商支持的擴展名,也可能是下列已獲 Khronos 批准的擴展名:
\startigBase
\item \clext{cl_khr_int64_base_atomics}
\item \clext{cl_khr_int64_extended_atomics}
\item \clext{cl_khr_fp16}
\item \clext{cl_khr_gl_sharing}
\item \clext{cl_khr_gl_event}
\item \clext{cl_khr_d3d10_sharing}
\item \clext{cl_khr_media_sharing}
\item \clext{cl_khr_d3d11_sharing}
\stopigBase

對於支持 OpenCL C \scver 的\cnglo{device},
所返回的清單中必須包含下列已獲 Khronos 批准的擴展名:
\startigBase
\item \clext{cl_khr_global_int32_base_atomics}
\item \clext{cl_khr_global_int32_extended_atomics}
\item \clext{cl_khr_local_int32_base_atomics}
\item \clext{cl_khr_local_int32_extended_atomics}
\item \clext{cl_khr_byte_addressable_store}
\item \clext{cl_khr_fp64}(如果支持雙精度浮點,為向後兼容必須支持此擴展)
\stopigBase

詳情請參考《OpenCL \scver 擴展規範》。
}

\clETD{}{}{}
\fi

\clETD{CL_DEVICE_PRINTF_BUFFER_SIZE}{size_t}{
\cnglo{kernel}調用 printf 時,由一個內部緩衝區存儲其輸出,此區域大小的最大值。
對於嵌入式規格,最小為 1KB。
}

\clETD{}{}{}

\clETD{CL_DEVICE_PREFERRED_INTEROP_USER_SYNC}{cl_bool}{
OpenCL 和其他 API (如 DirectX )間共享\cnglo{memobj}時,
如果\cnglo{device}的偏好是讓用戶自己負責同步,則其值為 \cenum{CL_TRUE};
而如果\cnglo{device}或實作已經具備有效的方式來進行同步,則其值為 \cenum{CL_FALSE}。
}

\clETD{CL_DEVICE_PARENT_DEVICE}{cl_device_id}{
返回此\cnglo{subdev}所屬\cnglo{pardev}的 \ctype{cl_device_id}。
如果 \carg{device} 是\cnglo{rootdev},則返回 \cmacro{NULL}。
}

\clETD{CL_DEVICE_PARTITION_MAX_SUB_DEVICES}{cl_uint}{
劃分\cnglo{device}時,所能創建的\cnglo{subdev}的最大數目。

所返回的值不能超過 \cenum{CL_DEVICE_MAX_COMPUTE_UNITS}。
}

\clETD{CL_DEVICE_PARTITION_PROPERTIES}{cl_device_partition_property[]}{
返回 \carg{device} 所支持的劃分方式。
這是一個陣列,元素型別為 \ctype{cl_device_partition_property},其值可以是:
\startigBase
\item \cenum{CL_DEVICE_PARTITION_EQUALLY}
\item \cenum{CL_DEVICE_PARTITION_BY_COUNTS}
\item \cenum{CL_DEVICE_PARTITION_BY_AFFINITY_DOMAIN}
\stopigBase

如果此\cnglo{device}不支持任何劃分方式,則返回 0。
}

\clETD{CL_DEVICE_PARTITION_AFFINITY_DOMAIN}{cl_device_affinity_domain}{
用 \cenum{CL_DEVICE_PARTITION_BY_AFFINITY_DOMAIN} 劃分 \carg{device} 時,
所支持的相似域( affinity domain )。
此位欄的值如下所示:
\startigBase
\item \cenum{CL_DEVICE_AFFINITY_DOMAIN_NUMA}
\item \cenum{CL_DEVICE_AFFINITY_DOMAIN_L4_CACHE}
\item \cenum{CL_DEVICE_AFFINITY_DOMAIN_L3_CACHE}
\item \cenum{CL_DEVICE_AFFINITY_DOMAIN_L2_CACHE}
\item \cenum{CL_DEVICE_AFFINITY_DOMAIN_L1_CACHE}
\item \cenum{CL_DEVICE_AFFINITY_DOMAIN_NEXT_PARTITIONABLE}
\stopigBase

如果以上都不支持,就返回 0。
}

\clETD{CL_DEVICE_PARTITION_TYPE}{cl_device_partition_property[]}{
如果 \carg{device} 是\cnglo{subdev},
則會返回調用 \capi{clCreateSubDevices} 時所指定的引數 \carg{properties}。
否則返回的 \carg{param_value_size_ret} 可能是 0 即不存在任何劃分方式;
或者 \carg{param_value} 所指內存中的屬性值是 0 (0 用來終止屬性清單)。
}

\clETD{CL_DEVICE_REFERENCE_COUNT}{cl_uint}{
返回 \carg{device} 的\cnglo{refcnt}。
如果是\cnglo{rootdev},則返回 1。
}

\stopETD

