%
% author:	Ni Qingliang
% date:		2011-02-11
%
\startcomponent cpnglo
\environment envcmm
\product opencl

\chapter{术语}

\myglo{app}{应用}{Application}
运行在宿主机和\scopencl设备上的程序。

\myglo{beac}{阻塞的入队API调用}{Blocking Enqueue API calls}
\myglo{nbeac}{非阻塞的入队API调用}{Non-Blocking Enqueue API calls}
一个\refglo{nbeac}将会在\refglo{cmdq}中放置一个\refglo{cmd}后立刻返回。而对于\refglo{beac},直到\refglo{cmd}完成后才会返回。

\startmyglo{barrier}{隔层}{Barrier}
有两种类型的\refglo{barrier}——\refglo{cmdqbarrier}和\refglo{workgrpbarrier}。
\startigBase
\item \refglo{cmdqbarrier}本身是一个\refglo{cmd},\scopenclapi提供一个函数用来将其入队。这个\refglo{barrier}命令可以保证只有之前入队的所有命令都执行完毕后,后续命令才能开始执行。
\item \scopenclc语言提供了一个内建的\refglo{workgrpbarrier}函数。运行在\refglo{device}上的\refglo{kernel}可以利用这个函数在(正在执行此\refglo{kernel}的)\refglo{workgrp}中的\refglo{workitem}间执行同步。\refglo{workgrp}中的任意一个\refglo{workitem}要想在\refglo{barrier}外继续执行,所有\refglo{workitem}都必须先执行这个\refglo{barrier}。
\stopigBase
\stopmyglo

\startmyglo{bufobj}{缓冲对象}{Buffer Object}
存储一个线性字节序列的\refglo{memobj}。在(正在\refglo{device}上运行的)\refglo{kernel}中,可以通过一个指针来访问\refglo{bufobj}。\refglo{host}可以用\scopenclapi调用来操控\refglo{bufobj}。一个\refglo{bufobj}包含以下信息:
\startigBase
\item 字节数
\item 一些属性,用来描述使用情况和分配自哪个区域。
\item 缓冲数据
\stopigBase
\stopmyglo

\myglo{cmd}{命令}{Command}
\scopencl运算,可以提交给\refglo{cmdq}来执行。例如,一些\scopencl\refglo{cmd}可以在计算设备上执行\refglo{kernel}、操控\refglo{memobj},等等。

\myglo{cmdq}{命令队列}{Command-queue}
一个持有\refglo{cmd}的对象,这些\cnglo{cmd}将在某个\refglo{device}上执行。\refglo{cmdq}是在一个\refglo{context}中的某个\refglo{device}上创建的。\refglo{cmd}按序入队,但可能顺序执行也可能乱序执行。参见\refglo{inordexec}和\refglo{outordexec}。

\myglo{cmdqbarrier}{命令队列隔层}{Command-queue Barrier}
请看\refglo{barrier}。

\myglo{computedevmem}{计算设备内存}{Compute Device Memory}
计算设备上所附着的一块或多块内存。

\myglo{computeunit}{计算单元}{Compute Unit}
一个\scopencl\refglo{device}会包含一个或多个\refglo{computeunit}。一个\refglo{workgrp}只能在一个\refglo{computeunit}上执行。一个\refglo{computeunit}由一个或多个\refglo{prcele}组成。一个\refglo{computeunit}也可能包含专用的材质过滤单元,此单元可以被它的\refglo{prcele}访问。

\myglo{concurrency}{并发}{Concurrency}
系统的一个属性,而此系统中会有一系列任务,这些任务可以同时保持活跃并取得进展。为了在运行程序时利用并发执行,程序员必须找出问题中可以并发的部分,在源码中标示出来,用一个支持并发的标记进行开发。

\myglo{constmem}{不变内存}{Constant Memory}
\refglo{glbmem}中的一个区域,在\refglo{kernel}执行过程中保持不变。\refglo{host}分配并初始化\cnglo{memobj},然后放到\refglo{constmem}中。

\myglo{context}{上下文}{Context}
\refglo{kernel}在执行时所处的环境,同步和内存管理就是在此范围内定义的。\refglo{context}包含一些\refglo{device}、这些\refglo{device}可以访问的内存、相应的内存属性、一个或多个\refglo{cmdq}(用来对\refglo{kernel}的执行和\refglo{memobj}相关运算进行调度)。

\myglo{dppm}{数据并行编程模型}{Data Parallel Programming Model}
传统意义上,这个术语涉及的编程模型中,并发由单个程序中应用到数据结构中多个元素上的指令来表示。在\scopencl中,对这个术语进行了推广,此模型中,单个程序的一套指令同时应用到索引的抽象域中的每一点上。

\myglo{device}{设备}{Device}
\refglo{device}是\refglo{computeunit}的集合。一个\refglo{cmdq}被用来对操作\refglo{device}的\refglo{cmd}进行排队。这样的\refglo{cmd}包括执行\refglo{kernel}或读写\refglo{memobj}。典型的\scopencl\cnglo{device}有GPU、多核CPU和其它处理器(如DSP和Cell/B.E.处理器)。

\myglo{evtobj}{事件对象}{Event Object}
\refglo{evtobj}封装了运算(如一个\refglo{cmd})的状态。它可以用来对某个\refglo{context}中的运算进行同步。

\myglo{evtwaitlist}{事件等待列表}{Event Wait List}
\refglo{evtwaitlist}是\refglo{evtobj}的列表,可用来控制何时执行\refglo{cmd}。

\myglo{framework}{框架}{Framework}
一个软件系统,包含一套组件用以支持软件开发和执行。一个典型的\refglo{framework}包括库、API、\scruntime系统、编译器等等。

\myglo{glbid}{全局 ID}{Global ID}
一个\refglo{glbid}用来唯一标识一个\refglo{workitem},源自执行\refglo{kernel}时所指定的全局\refglo{workitem}数目。\refglo{glbid}是一个N维的值,起自(0, 0, … 0)。请参看\refglo{locid}。

\myglo{glbmem}{全局内存}{Global Memory}
一个\refglo{context}中的所有\refglo{workitem}均可访问的内存区域。\refglo{host}可以用一些\refglo{cmd}(如read、write和map)来访问它。

\myglo{glsharegrp}{GL共享组}{GL share group}
\refglo{glsharegrp}对象用来管理共享的OpenGL或OpenGL ES资源(如材质、缓存、帧缓存或渲染缓存),并与一个或多个GL上下文对象相关联。典型的\refglo{glsharegrp}是一个不透明的对象,不能直接访问。

\myglo{handle}{句柄}{Handle}
一个不透明的类型,用来引用\scopencl所分配的\refglo{obj}。对\refglo{obj}的任何运算都是通过引用其句柄来进行的。

\myglo{host}{宿主机}{host}
\refglo{host}使用\scopenclapi与\refglo{context}进行交互。

\myglo{hostptr}{宿主机指针}{Host pointer}
指针,指向\refglo{host}上的虚拟地址空间中的内存。

\myglo{illegal}{违规}{Illegal}
明显不允许的行为,\scopencl会将其视为错误进行报告。

\startmyglo{imgobj}{图像对象}{Image Object}
一种\refglo{memobj},用来存储二维或三维的结构化数组。只有通过读写函数才能访问图像数据。读函数需要使用\refglo{sampler}。

\refglo{imgobj}封装了以下信息:
\startigBase
\item 图像的维数。
\item 图像中每个元素的描述。
\item 一些属性,用来描述使用情况和分配自哪个区域。
\item 图像数据
\stopigBase
图像中的所有元素均选自一个预定义图像格式的列表。
\stopmyglo

\myglo{impdef}{依赖于具体实现}{Implementation Defined}
在遵循\scopencl标准的实现之间,明确被允许可以不同的行为。要求\scopencl的实现者为其提供相关文档。

\myglo{inordexec}{顺序执行}{In-order Execution}
\scopencl中的一种执行模型,\refglo{cmdq}中的\refglo{cmd}按提交的顺序执行,只有当正在运行的\refglo{cmd}完成后,下一个才能开始运行。参看\refglo{outordexec}。

\myglo{kernel}{内核}{Kernel}
\refglo{kernel}是\refglo{program}中声明的一个函数,可以在\scopencl\refglo{device}上执行。对于\refglo{program}中定义的任何函数,都可以通过加上限定符\sckernel标识成\refglo{kernel}。

\myglo{kernelobj}{内核对象}{Kernel Object}
一个\refglo{kernelobj}封装了\refglo{program}中声明的一个\sckernel函数和执行此函数所用的参数。

\myglo{locid}{局部ID}{Local ID}
\refglo{locid}即一个\refglo{workitem}ID,在一个正在执行\refglo{kernel}的\refglo{workgrp}中是唯一的。\refglo{locid}是一个N维的值,起于(0, 0, … 0)。参见\refglo{glbid}。

\myglo{locmem}{局部内存}{Local Memory}
属于某个\refglo{workgrp}内存区域,只有那个\refglo{workgrp}中的\refglo{workitem}才可以访问它。

\myglo{marker}{记号}{Marker}
排在\refglo{cmdq}中的一个\refglo{cmd},可以用来给之前入队的所有\refglo{cmd}做标记。此命令会返回一个\scevent,\refglo{app}可以等在这个\scevent上,例如等待处在此\refglo{marker}之前的所有命令全部完成。

\myglo{memobj}{内存对象}{Memory Object}
\refglo{glbmem}中某区域的句柄,此区域带有引用计数。参见\refglo{bufobj}和\refglo{imgobj}。

\myglo{memregion}{内存区域}{Memory Region}
或
\myglo{mempool}{内存池}{Memory Pool}
\scopencl中的一个明确的地址空间。不同的\refglo{memregion}在物理上可能重叠,但在逻辑上\scopencl会将其视为不同。可以将其指定为私有的({\ftCQlf private})、局部的({\ftCQlf local})、不变的({\ftCQlf constant})和全局的({\ftCQlf global})。

\myglo{obj}{对象}{Object}
对\scopenclapi可以操控的资源的一种抽象表示。包含\refglo{programobj}、\refglo{kernelobj}、\refglo{memobj}。

\myglo{outordexec}{乱序执行}{Out-of-Order Execution}
一种执行模型,\refglo{cmdq}中\refglo{cmd}的开始和完成执行可以以任意顺序进行,只要与\refglo{evtwaitlist}和\refglo{cmdqbarrier}相容即可。参见\refglo{inordexec}。

\myglo{platform}{平台}{platform}
\refglo{host}加一些可以被\scopencl\refglo{framework}所管理的设备。\refglo{app}可以在\refglo{platform}的\refglo{device}上共享\refglo{res}和执行\refglo{kernel}。

\myglo{prvmem}{私有内存}{Private Memory}
专属于某个\refglo{workitem}的一块内存。一个\refglo{workitem}的\refglo{prvmem}中定义的变量对另一个\refglo{workitem}是不可见的。

\myglo{prcele}{处理元件}{Processing Element}
一个虚拟的标量处理器。一个\refglo{workitem}可以在一个或多个\refglo{prcele}上执行。

\myglo{program}{程序}{Program}
一个\scopencl\refglo{program}由一套\refglo{kernel}组成,还可能包含一些辅助函数(由\sckernel函数调用)和常量数据。

\startmyglo{programobj}{程序对象}{Program Object}
一个\refglo{programobj}封装有以下信息:
\startigBase
\item 对所关联\refglo{context}的引用。
\item \refglo{program}源码或\scbinary。
\item 最近成功构建的可执行程序,\refglo{device}列表(用来运行此程序),所用的构建选项和构建日志。
\item 当前所附\refglo{kernelobj}的数目。
\stopigBase

\myglo{refcnt}{引用计数}{Reference Count}
一个\scopencl对象的生命周期由其\refglo{refcnt}来决定,这个内部计数会记录对此对象的引用的数目。当你在\scopencl中创建一个对象时,其\refglo{refcnt}被设成1。后续对\refglo{retain}API(如 {\ftCKey clRetainContext}、{\ftCKey clRetainCommandQueue})的调用会增大\refglo{refcnt}。对\refglo{release}API(如{\ftCKey clReleaseContext}、{\ftCKey clReleaseCommandQueue})的调用会减小\refglo{refcnt}。当\refglo{refcnt}降为0后,\scopencl会回收此对象的资源。

\myglo{relaxconsist}{放宽的一致性}{Relaxed Consistency}
在内存一致性模型中,不同的\refglo{workitem}或\refglo{cmd}所看到的内容也可能不同,当然,\refglo{barrier}和其他显式同步点除外。

\myglo{res}{资源}{Resource}
\scopencl所定义的一类\refglo{obj}。一个\refglo{res}的实例就是一个\refglo{obj}。最常用的\refglo{res}是\refglo{context}、\refglo{cmdq}、\refglo{programobj}、\refglo{kernelobj}和\refglo{memobj}。计算用资源主要指那些参与推进程序计数器动作的硬件元件,包括\refglo{host}、\refglo{device}、\refglo{computeunit}和\refglo{prcele}。

\myglo{retain}{保留}{Retain}
\myglo{release}{释放}{Release}
动作,会增大(retain)和减小(release)一个\scopencl\refglo{obj}的\refglo{refcnt}。这是一个记账功能,它可以保证在使用这个\refglo{obj}的所有实例都完成之前,系统不会移除这个\refglo{obj}。参见\refglo{refcnt}。

\myglo{sampler}{采样器}{Sampler}
此\refglo{obj}用来描述在\refglo{kernel}读取图像时怎样对其采样。读取图像的函数将\refglo{sampler}作为一个参数。\refglo{sampler}指定图像的寻址方式,如图像的坐标越限时怎样处理、滤波模式、输入的图像坐标是否已规范化。

\myglo{simd}{单指令多数据}{SIMD}
这是一个编程模型,一个\refglo{kernel}在多个\refglo{prcele}上并发执行,每个\refglo{prcele}上都有自己的数据,还有一个共享的程序计数器。所有\refglo{prcele}执行严格一致的一套指令。

\myglo{spmd}{单程序多数据}{SPMD}
此编程模型中,一个\refglo{kernel}在多个\refglo{prcele}上并发执行,每个\refglo{prcele}上都有自己的数据和自己的程序计数器。因此,运行同一个\refglo{kernel}的所有计算资源都会维护自己的指令计数器;同时由于\refglo{kernel}中的不同分支,这些\refglo{prcele}中的实际指令序列可能会有很大不同。

\myglo{tppm}{任务并行编程模型}{Task Parallel Programming Model}
这个编程模型中,计算表示为多个并发的任务,其中,一个任务就是单个工作组(大小是 1)中执行的那个\refglo{kernel}。这些并发的任务可以运行不同的\refglo{kernel}。

\myglo{thsafe}{线程安全}{Thread-safe}
对于一个\scopenclapi调用,只有在被多个\refglo{host}线程同时调用时,\scopencl所管理的内部状态保持一致,才认为它是\refglo{thsafe}的。如果一个\scopenclapi调用是\refglo{thsafe}的,那么就允许\refglo{app}在多个\refglo{host}线程中同时调用它,而不必在这些线程间实施互斥,同时,它们也是重入安全的。

\myglo{undef}{未定义}{Undefined}
指\scopencl没有显式定义那些API调用、内建函数(由\refglo{kernel}使用或用来执行\refglo{kernel})的行为。至于当\scopencl碰到一个未定义的构造时会发生什么,不要求实现指名。

\myglo{workgrp}{工作组}{Work-group}
相关\refglo{workitem}的集合,这些\refglo{workitem}在同一个\refglo{computeunit}上执行。一个\refglo{workgrp}中的所有\refglo{workitem}执行同一个\refglo{kernel},共享\refglo{locmem}和\refglo{workgrpbarrier}。

\myglo{workgrpbarrier}{工作组隔层}{Work-group Barrier}
参见\refglo{barrier}。

\myglo{workitem}{工作项}{Work-item}
\refglo{kernel}的并行执行体中的一个。作为一个\refglo{computeunit}中所执行的一个\refglo{workgrp}中的一部分,一个\refglo{workitem}可以由一个或多个\refglo{prcele}执行。用其\refglo{glbid}和\refglo{locid}来区分一个\refglo{workitem}和所在\refglo{workgrp}中的其他\refglo{workitem}。

\section{\scopencl类图}
\reffig{classdiag}使用统一建模语言{\footnote[uml]{统一建模语言(http://www.uml.org/)是Object Management Group(OMG)的一个商标。}}(UML)以类图的方式对\scopencl规范进行了描述。此图中,节点代表类,边代表类之间的关系。简化起见,类中没有特性和操作,抽象类带有注释“\{abstract\}”。对于类之间的关系,有聚集(aggregation,带有实心方块)、关联(association,没有注释)和继承(inheritance,带有开放的箭头)。边的两端带有关系的基数。基数“*”代表“多个”,基数“1”代表“一个而且只有一个”,基数“0..1”代表“可选的一个”,而基数“1..”代表“一个或多个”。用普通箭头表示关系的方向性。

\if 0
%自定义metauml字体会导致对齐错误,如果要使用默认字体,必须先加载8r-base.map
\pdfmapfile{8r-base.map}
\startMPinclusions
%string metauml_defaultFont;
%metauml_defaultFont := "\truefontname{Serif}";
%string metauml_defaultFontOblique;
%metauml_defaultFontOblique := "\truefontname{SerifItalic}";
%string metauml_defaultFontBold;
%metauml_defaultFontBold := "\truefontname{SerifBold}";
%string metauml_defaultFontBoldOblique;
%metauml_defaultFontBoldOblique := "\truefontname{SerifBoldItalic}";
input metauml;
\stopMPinclusions

\startreusableMPgraphic{xx}
%save B;
%save A;

Class.A("Point")
       ("+x: int",
        "+y: int") ();
%A.info.iName.iFont.name := metauml_defaultFontBold;
%A.info.iName.iFont.scale := 1.2;
%A.info.iAttributeStack.iPict.iFont.scale := 0.8;
%A.info.iAttributeStack.top := 10;
%A.info.iAttributeStack.spacing := 11;
%A.info.iMethodStack.iPict.iFont.scale := 2;
%A.info.iMethodStack.spacing := 17;
%A.info.iMethodStack.bottom := 10;

Class.B("Circle")
       ("radius: int")
       ("+getRadius(): int",
        "+setRadius(r: int):void");

topToBottom(45)(A, B);

drawObjects(A, B);

clink(aggregationUni)(A, B);
\stopreusableMPgraphic

\placefigure
[here,force][fig:classdiag]
{嵌入 \METAFUN\ 图形}{\reuseMPgraphic{xx}}
\fi

\placefigure
[here,force][fig:classdiag]
{\scopencl类图}{\externalfigure[figclassdiag-1.pdf]}

\stopcomponent

