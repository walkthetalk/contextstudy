%
% author:	Ni Qingliang
% date:		2011-02-11
%
\startcomponent cpnglo
\environment envcmm
\product opencl

\chapter{術語}

\myglo{app}{應用}{Application}
運行在\refglo{host}和 OpenCL \refglo{device}上的程序。

\myglo{beac}{阻塞的入隊 API 調用}{Blocking Enqueue API calls}
\myglo{nbeac}{非阻塞的入隊 API 調用}{Non-Blocking Enqueue API calls}
\refglo{nbeac}會在\refglo{cmdq}中放置一個\refglo{cmd}後立刻返回。而對於\refglo{beac},在\refglo{cmd}結束後才會返回。

\startmyglo{barrier}{屏障}{Barrier}
有兩種類型的\refglo{barrier}——\refglo{cmdqbarrier}和\refglo{workgrpbarrier}。
\startigBase
\item \refglo{cmdqbarrier}本身是一個\refglo{cmd}, OpenCL API 提供一個函數可以將其插入隊列。此命令可以保證之前入隊的所有命令都執行完畢後,後續命令才能開始執行。
\item OpenCL C 語言內建有一個\refglo{workgrpbarrier}函數。運行在\refglo{device}上的\refglo{kernel}可以利用此函數在(正在執行此\refglo{kernel}的)同一\refglo{workgrp}中的\refglo{workitem}間進行同步。\refglo{workgrp}中的任意一個\refglo{workitem}要想在\refglo{barrier}外繼續執行,所有\refglo{workitem}都必須先執行此\refglo{barrier}。
\stopigBase
\stopmyglo

\startmyglo{bufobj}{緩衝對象}{Buffer Object}
存儲線性字節序列的\refglo{memobj}。在(正在\refglo{device}上運行的)\refglo{kernel}中,可以通過一個指針來訪問它。
\refglo{host}可以用 OpenCL API 調用來操控\refglo{bufobj}。\refglo{bufobj}包含以下信息:
\startigBase
\item 字節數;
\item 一些屬性,用來描述使用情況以及分配自哪個區域;
\item 緩衝數據。
\stopigBase
\stopmyglo

\myglo{cmd}{命令}{Command}
即 OpenCL 運算,可以提交給\refglo{cmdq}執行。
例如,一些 OpenCL \refglo{cmd}可以在計算設備上執行\refglo{kernel}、操控\refglo{memobj}等等。

\myglo{cmdq}{命令隊列}{Command-queue}
這種對象中持有\refglo{cmd},而這些\cnglo{cmd}將在某個\refglo{device}上執行。
\refglo{cmdq}是在一個\refglo{context}中的某個\refglo{device}上創建的。
\refglo{cmd}按序入隊,但可能順序執行也可能亂序執行。
參見\refglo{inordexec}和\refglo{outordexec}。

\myglo{cmdqbarrier}{命令隊列屏障}{Command-queue Barrier}
請看\refglo{barrier}。

\myglo{computedevmem}{計算設備內存}{Compute Device Memory}
計算設備上的一塊或多塊內存。

\myglo{computeunit}{計算器件}{Compute Unit}
一個 OpenCL \refglo{device}中包含一個或多個\refglo{computeunit}。
一個\refglo{workgrp}只能在一個\refglo{computeunit}上執行。
一個\refglo{computeunit}由一個或多個\refglo{prcele}組成。
\refglo{computeunit}也可能包含專用的材質過濾器件,此器件可以被它的\refglo{prcele}訪問。

\myglo{concurrency}{並發}{Concurrency}
系統的一個屬性,這樣的系統中有一系列任務,這些任務可以同時保持活躍並取得進展。
運行程序時要想並發執行,程序員必須找出問題中可以並發的部分,並在源碼中標示出來,用一個支持並發的標記進行開發。

\myglo{constmem}{不變內存}{Constant Memory}
\refglo{glbmem}中的一個區域,在\refglo{kernel}執行過程中保持不變。\refglo{constmem}中的\cnglo{memobj}是由\refglo{host}分配並初始化的。

\myglo{context}{上下文}{Context}
\refglo{kernel}在執行時所處的環境,同步和內存管理就是在此概念上定義的。
\refglo{context}包含一系列\refglo{device}、這些\refglo{device}可以訪問的內存、相應的內存屬性、\refglo{cmdq}(用來對\refglo{kernel}的執行和\refglo{memobj}相關運算進行調度)。

\myglo{customdev}{自定義設備}{Custom Device}
一種 OpenCL \refglo{device},完整實現了 OpenCL runtime,但不支持用 OpenCL C 寫的\refglo{program}。
\refglo{customdev}可能是不可編程的專用硬件,高效節能,僅可執行定向任務;也可能具有有限的編程能力,如專用的 DSP。

\myglo{dppm}{數據並行編程模型}{Data Parallel Programming Model}
傳統意義上,此種編程模型中,並發由程序中的指令表示,這些指令會應用到數據結構中的多個元素上。
在 OpenCL 中,對這個術語進行了推廣,此模型中,程序中的一些指令同時應用到索引的抽象域中的所有點上。

\myglo{device}{设备}{Device}
\refglo{device}是\refglo{computeunit}的集合。\refglo{cmdq}用來將\refglo{cmd}在\refglo{device}上進行排隊。
這樣的\refglo{cmd}包括執行\refglo{kernel}或讀寫\refglo{memobj}。
典型的 OpenCL \cnglo{device}指 GPU、多核 CPU 和其它處理器(如 DSP 和 Cell/B.E. 處理器)。

\myglo{evtobj}{事件對象}{Event Object}
\refglo{evtobj}封裝了運算(如一個\refglo{cmd})的狀態。
它可以用來對\refglo{context}中的運算進行同步。

\myglo{evtwaitlist}{事件等待序列}{Event Wait List}
\refglo{evtwaitlist}是\refglo{evtobj}的序列,可用來控制何時執行\refglo{cmd}。

\myglo{framework}{框架}{Framework}
一個軟件系統,包含一套組件用以支持軟件開發和執行。典型的\refglo{framework}包括庫、 API、 runtime 系統、編譯器等等。

\myglo{glbid}{全局 ID}{Global ID}
\refglo{glbid}用來唯一標識一個\refglo{workitem},源自執行\refglo{kernel}時所指定的全局\refglo{workitem}數目。
\refglo{glbid}是一個 N 維的值,起自\math{(0, 0, … 0)}。請參考\refglo{locid}。

\myglo{glbmem}{全局內存}{Global Memory}
一塊內存,\refglo{context}中的所有\refglo{workitem}都可訪問。\refglo{host}可以使用\refglo{cmd}(如 read、 write 和 map)來訪問它。

\myglo{glsharegrp}{GL 共享組}{GL share group}
此對象用來管理共享的 OpenGL 或 OpenGL ES 资源(如材質、緩存、幀緩存或渲染緩存),並與一個或多個 GL 上下文對象關聯在一起。
典型的\refglo{glsharegrp}是一個不透明的對象,不能直接訪問。

\myglo{handle}{句柄}{Handle}
一個不透明的類型,用來引用 OpenCL 所分配的\refglo{obj}。對\refglo{obj}的任何運算都是通過引用其句柄來進行的。

\myglo{host}{宿主機}{host}
\refglo{host}使用 OpenCL API 與\refglo{context}進行交互。

\myglo{hostptr}{宿主機指針}{Host pointer}
一種指針,,指向\refglo{host}上虛擬地址空間中的內存。

\myglo{illegal}{違規}{Illegal}
明確規定不允許的行為,遇到這種問題時, OpenCL 會將其視為錯誤進行報告。

\startmyglo{imgobj}{圖像對象}{Image Object}
一種\refglo{memobj},用來存儲二維或三維的結構化數組。
只能通過讀寫函數訪問其中的圖像數據。讀函數需要使用\refglo{sampler}。

\refglo{imgobj}封裝了以下信息:
\startigBase
\item 圖像的維數;
\item 圖像中每個元素的描述;
\item 一些屬性,用來描述是喲功能情況以及分配自哪個區域;
\item 圖像數據。
\stopigBase
圖像中的所有元素均選自預定義的一系列圖像格式。
\stopmyglo

\myglo{impdef}{依賴於具體實現}{Implementation Defined}
對於兼容 OpenCL 標準的實現,明確規定其行為可以不同。對於這種行為,要求 OpenCL 的實現者提供相關文檔。

\myglo{inordexec}{順序執行}{In-order Execution}
OpenCL 中的一種執行模型,\refglo{cmdq}中的\refglo{cmd}按提交的順序執行,只有當正在運行的\refglo{cmd}完成後,下一個才能開始運行。参看\refglo{outordexec}。

\myglo{kernel}{內核}{Kernel}
\refglo{kernel}是\refglo{program}中聲明的一種函數,可以在 OpenCL \refglo{device}上執行。
對於\refglo{program}中定義的任何函數,都可以通過加上限定符 {\ftCKey \_\_kernel} 或 {\ftCKey kernel} 標識成\refglo{kernel}。

\myglo{kernelobj}{內核對象}{Kernel Object}
\refglo{kernelobj}封裝了\refglo{program}中聲明的一個 {\ftCKey \_\_kernel} 函數以及執行此函數所需的參數。

\myglo{locid}{局部 ID}{Local ID}
即\refglo{workitem} ID,在執行\refglo{kernel}的\refglo{workgrp}中是唯一的。
\refglo{locid}是一個 N 維的值,起自\math{(0, 0, … 0)}。參見\refglo{glbid}。

\myglo{locmem}{本地內存}{Local Memory}
隸屬於某個\refglo{workgrp},只有此\refglo{workgrp}中的\refglo{workitem}才能訪問。

\myglo{marker}{記號}{Marker}
一種\refglo{cmd},可用來給之前入隊的所有\refglo{cmd}作標記。
此命令會返回一個事件,\refglo{app}可以等在這個事件上,例如等待處在此\refglo{marker}之前的所有命令全部完成。

\myglo{memobj}{內存對象}{Memory Object}
某塊帶有引用計數的\refglo{glbmem}的句柄。參見\refglo{bufobj}和\refglo{imgobj}。

\myglo{memregion}{內存區域}{Memory Region}
OpenCL 中的一個確切的地址空間。不同的\refglo{memregion}在物理上可能重疊,但在邏輯上 OpenCL 會將其視為不同。可以將其指定為私有的({\ftCQlf private})、局部的({\ftCQlf local})、不變的({\ftCQlf constant})和全局的({\ftCQlf global})。

\myglo{mempool}{內存池}{Memory Pool}
同\refglo{memregion}。

\myglo{obj}{對象}{Object}
這是對 OpenCL API 可以操控的資源的一種抽象表示。
包含\refglo{programobj}、\refglo{kernelobj}、\refglo{memobj}。

\myglo{outordexec}{亂序執行}{Out-of-Order Execution}
一種執行模型,\refglo{cmdq}中的\refglo{cmd},其執行的開始和結束的順序沒有要求,只要符合\refglo{evtwaitlist}和\refglo{cmdqbarrier}即可。
參見\refglo{inordexec}。

\myglo{pardev}{父設備}{Parent Device}
一種 OpenCL \refglo{device},可以將其分割來創建\refglo{subdev}。
並不是所有的\refglo{pardev}都是\refglo{rootdev}。
\refglo{rootdev}可以被拆分,而\refglo{subdev}可以繼續被拆分。
這種情況下,第一組\refglo{subdev}可能是第二組的\refglo{pardev},但不是\refglo{rootdev}。
請參考\refglo{device}、\refglo{subdev}和\refglo{rootdev}。


\myglo{platform}{平台}{platform}
\refglo{host}加上一些被 OpenCL \refglo{framework}所管理的\refglo{device}。
\refglo{app}可以在這些\refglo{device}上共享\refglo{res}、執行\refglo{kernel}。

\myglo{prvmem}{私有內存}{Private Memory}
專屬某個\refglo{workitem}的一塊內存。在一個\refglo{workitem}的\refglo{prvmem}中定義的變量對另一個\refglo{workitem}是不可見的。

\myglo{prcele}{處理元件}{Processing Element}
一個虛擬的標量處理器。一個\refglo{workitem}可以在一個或多個\refglo{prcele}上執行。

\myglo{program}{程序}{Program}
由一組\refglo{kernel}組成,可能還包含一些輔助函數(由{\ftCKey \_\_kernel}函數調用)和常量数据。

\startmyglo{programobj}{程序對象}{Program Object}
\refglo{programobj}封裝有以下信息:
\startigBase
\item 對所關聯\refglo{context}的引用。
\item \refglo{program}源碼或二進制。
\item 最新成功構建的程序執行體,用來運行此程序的\refglo{device}清單,所用的構建選項以及構建日誌。
\item 當前所附\refglo{kernelobj}的數目。
\stopigBase

\myglo{refcnt}{引用計數}{Reference Count}
OpenCL 對象的生命周期由其\refglo{refcnt}來確定,此內部計數會記錄對此對象的引用數目。
在 OpenCL 中創建一個對象時,其\refglo{refcnt}被置為0。
后续对\refglo{retain} API (如 {\ftCKey clRetainContext}、 {\ftCKey clRetainCommandQueue})的調用會使\refglo{refcnt}增一。
對\refglo{release} API (如 {\ftCKey clReleaseContext}、 {\ftCKey clReleaseCommandQueue})的調用會使\refglo{refcnt}減一。
當\refglo{refcnt}降為0後, OpenCL 會回收此對象的資源。

\myglo{relaxconsist}{放寬的一致性}{Relaxed Consistency}
是一種內存一致性模型,不同的\refglo{workitem}或\refglo{cmd}所看到的內容可能不同,當然,\refglo{barrier}和其它顯式同步點除外。

\myglo{res}{資源}{Resource}
OpenCL 所定義的一類\refglo{obj}。\refglo{res}的一個實例就是一個\refglo{obj}。
最常用的\refglo{res}是\refglo{context}、\refglo{cmdq}、\refglo{programobj}、\refglo{kernelobj}和\refglo{memobj}。
計算用資源主要指那些參與推進程序計數器動作的硬件元件。
例如\refglo{host}、\refglo{device}、\refglo{computeunit}和\refglo{prcele}。

\myglo{retain}{保留}{Retain}
\myglo{release}{釋放}{Release}
會使 OpenCL \refglo{obj}的\refglo{refcnt}增一( retain)或減一( release)。
這是一個記賬功能,可以保證在使用此\refglo{obj}的所有實例都完成之前,系統不會移除此\refglo{obj}。
參見\refglo{refcnt}。

\myglo{rootdev}{根設備}{Root Device}
一種 OpenCL \refglo{device},還未分塊。
參見\refglo{device}、\refglo{pardev}和\refglo{subdev}。

\myglo{sampler}{採樣器}{Sampler}
一種\refglo{obj},用來控制\refglo{kernel}讀取圖像時怎樣對其採樣。
讀取圖像的函數將\refglo{sampler}作為一個參數。
\refglo{sampler}指定圖像的尋址方式,如圖像的坐標越限時怎樣處理、濾波模式、輸入的圖像坐標是否已規格化。

\myglo{simd}{單指令多數據}{SIMD}
一種編程模型,一個\refglo{kernel}在多個\refglo{prcele}上並發執行,每個\refglo{prcele}上都有自己的數據,還有一個共享的程序計數器。
所有\refglo{prcele}執行嚴格一致的一組指令。

\myglo{spmd}{單程序多數據}{SPMD}
一種编程模型,一個\refglo{kernel}在多個\refglo{prcele}上並發執行,每個\refglo{prcele}上都有自己的數據和自己的程序計數器。
因此,運行同一個\refglo{kernel}的所有計算資源都會維護自己的指令計數器;
同時由於\refglo{kernel}中的不同分支,這些\refglo{prcele}中的實際指令序列可能會有很大不同。

\myglo{subdev}{子設備}{Sub Device}
一個 OpenCL \refglo{device}可以劃分成多個\refglo{subdev}。
按照劃分方案,新的\refglo{subdev}是\refglo{pardev}中一組\refglo{comuteunit}的別名。
任何情況下,只要可以使用其\refglo{pardev},就可以使用這些\refglo{subdev}。
劃分\refglo{device}不會摧毀\refglo{pardev},可以繼續和\refglo{subdev}混合使用。
請參考\refglo{device}、\refglo{pardev}和\refglo{rootdev}。


\myglo{tppm}{任務並行編程模型}{Task Parallel Programming Model}
一種編程模型,其中計算表示為多個並發的任務,一個任務就是在\refglo{workgrp}(大小是1)中執行的\refglo{kernel}。
這些並發的任務可以運行不同的\refglo{kernel}。

\myglo{thsafe}{線程安全}{Thread-safe}
對於一個 OpenCL API 調用,只有在被多個\refglo{host}線程同時調用時, OpenCL 所管理的內部狀態保持一致,才認為它是\refglo{thsafe}的。
如果一个 OpenCL API 調用是\refglo{thsafe}的,那麼就允許\refglo{app}在多個\refglo{host}線程中同時調用它,而不必在這些線程間實施互斥,即它們也是重入安全的。

\myglo{undef}{未定義}{Undefined}
對於 OpenCL API 調用、\refglo{kernel}中使用內建函數或者執行\refglo{kernel},如果 OpenCL 沒有對其行為顯式定義,那麼這種行為就是\refglo{undef}。
至於在 OpenCL 中碰到這種情況時會發生什麼,不要求實現明確指出。

\myglo{workgrp}{作業組}{Work-group}
一組相關的\refglo{workitem},在同一個\refglo{computeunit}上執行。其中所有\refglo{workitem}執行同一個\refglo{kernel},共享\refglo{locmem}和\refglo{workgrpbarrier}。

\myglo{workgrpbarrier}{工作组屏障}{Work-group Barrier}
參見\refglo{barrier}。

\myglo{workitem}{作業項}{Work-item}
\refglo{kernel}的並行執行體中的一個。
作為在\refglo{computeunit}中執行的\refglo{workgrp}中的一部分,一個\refglo{workitem}可以由一個或多個\refglo{prcele}執行。
用其\refglo{glbid}和\refglo{locid}來區分\refglo{workitem}。

\section{\scopencl类图}
\reffig{classdiag}使用統一建模語言
{\footnote[uml]{統一建模語言(http://www.uml.org/)是 Object Management Group (OMG)的一個商標。}}
(UML)以類圖的方式對 OpenCL 規範進行了描述。
圖中的節點和邊分別代表類和類之間的關係。
簡化起見,類中沒有特性和操作。
抽象類帶有注釋“\{abstract\}”。
對於類之間的關係,有聚集(aggregation,帶有實心方塊)、關聯(association,没有註解)和繼承(inheritance,帶有開放的箭頭)。
邊兩端帶有關係的基數。
基數“*”代表“多個”,基數“1”代表“一個且只有一個”,基數“0..1”代表“可選的一個”,而基數“1..”代表“一個或多個”。
普通箭頭代表關係的方向。

\placefigure
[here,force][fig:classdiag]
{OpenCL 類圖}{\externalfigure[figclassdiag-1.pdf]}

\stopcomponent

