\section{與 Direct3D 10 共享內存對象}

\subsection{簡介}

本擴展是為了使得在 OpenCL 和 Direct3D 10 間可以進行互操作。
其運作方式與\refsec{clShareGl}和\refsec{clEvtObjFromGlSync}中的 OpenGL 類似。
如果實作支持本擴展,
則 \cenum{CL_PLATFORM_EXTENSIONS} 或 \cenum{CL_DEVICE_EXTENSIONS} 中
應包含字串 \clext{cl_khr_d3d10_sharing},參見\reftab{cldevquery}。

\subsection{頭檔}

目前本擴展中的所有接口都由 \ccmm{cl_d3d10.h} 提供。

\subsection{新例程和新函式}

\startCLFUNC
cl_int clGetDeviceIDsFromD3D10KHR (
		cl_platform_id platform,
		cl_d3d10_device_source_khr d3d_device_source,
		void *d3d_object,
		cl_d3d10_device_set_khr d3d_device_set,
		cl_uint num_entries,
		cl_device_id *devices,
		cl_uint *num_devices)
cl_mem clCreateFromD3D10BufferKHR (
		cl_context context,
		cl_mem_flags flags,
		ID3D10Buffer *resource,
		cl_int *errcode_ret)
cl_mem clCreateFromD3D10Texture2DKHR (
		cl_context context,
		cl_mem_flags flags,
		ID3D10Texture2D *resource,
		UINT subresource,
		cl_int *errcode_ret)
cl_mem clCreateFromD3D10Texture3DKHR (
		cl_context context,
		cl_mem_flags flags,
		ID3D10Texture3D *resource,
		UINT subresource,
		cl_int *errcode_ret)
cl_int clEnqueueAcquireD3D10ObjectsKHR (
		cl_command_queue command_queue,
		cl_uint num_objects,
		const cl_mem *mem_objects,
		cl_uint num_events_in_wait_list,
		const cl_event *event_wait_list,
		cl_event *event)
cl_int clEnqueueReleaseD3D10ObjectsKHR (
		cl_command_queue command_queue,
		cl_uint num_objects,
		const cl_mem *mem_objects,
		cl_uint num_events_in_wait_list,
		const cl_event *event_wait_list,
		cl_event *event)
\stopCLFUNC

\subsection{新符記}

\clapi{clGetDeviceIDsFromD3D10KHR} 的參數 \carg{d3d_device_source} 的值可以是:
\startclc
CL_D3D10_DEVICE_KHR		0x4010
CL_D3D10_DXGI_ADAPTER_KHR	0x4011
\stopclc

\clapi{clGetDeviceIDsFromD3D10KHR} 的參數 \carg{d3d_device_set} 的值可以是:
\startclc
CL_PREFERRED_DEVICES_FOR_D3D10_KHR	0x4012
CL_ALL_DEVICES_FOR_D3D10_KHR		0x4013
\stopclc

\clapi{clCreateContext} 和 \clapi{clCreateContextFromType} 的
參數 \carg{properties} 可以是:
\startclc
CL_CONTEXT_D3D10_DEVICE_KHR	0x4014
\stopclc

\clapi{clGetContextInfo} 的參數 \carg{param_name} 可以是:
\startclc
CL_CONTEXT_D3D10_PREFER_SHARED_RESOURCES_KHR	0x402C
\stopclc

\clapi{clGetMemObjectInfo} 的參數 \carg{param_name} 可以是:
\startclc
CL_MEM_D3D10_RESOURCE_KHR	0x4015
\stopclc

\clapi{clGetImageInfo} 的參數 \carg{param_name} 可以是:
\startclc
CL_IMAGE_D3D10_SUBRESOURCE_KHR	0x4016
\stopclc

當 \carg{param_name} 是 \cenum{CL_EVENT_COMMAND_TYPE} 時,
 \clapi{clGetEentInfo} 的參數 \carg{param_alue} 可以是:
\startclc
CL_COMMAND_ACQUIRE_D3D10_OBJECTS_KHR	0x4017
CL_COMMAND_RELEASE_D3D10_OBJECTS_KHR	0x4018
\stopclc

如果指定的 Direct3D 10 \cnglo{device}與\cnglo{context}中的\cnglo{device}不兼容,
則 \clapi{clCreateContext} 和 \clapi{clCreateContextFromType} 會返回:
\startclc
CL_INVALID_D3D10_DEVICE_KHR		-1002
\stopclc

\clapi{clCreateFromD3D10BufferKHR} 的參數 \carg{resource} 必須是 Direct3D 10 \cnglo{bufobj},
而 \clapi{clCreateFromD3D10Texture2DKHR} 和 \clapi{clCreateFromD3D10Texture3DKHR} 的
參數 \carg{resource} 必須是 Direct3D 10 材質對象,
否則這些函式會返回:
\startclc
CL_INVALID_D3D10_RESOURCE_KHR		-1003
\stopclc

如果參數 \carg{mem_objects} 中的任一對象已經由 OpenCL 獲取了,
那麼 \clapi{clEnqueueAcquireD3D10ObjectsKHR} 會返回:
\startclc
CL_D3D10_RESOURCE_ALREADY_ACQUIRED_KHR	-1004
\stopclc

如果參數 \carg{mem_objects} 中的任一對象已經由 OpenCL 獲取了,
那麼 \clapi{clEnqueueReleaseD3D10ObjectsKHR} 會返回:
\startclc
CL_D3D10_RESOURCE_NOT_ACQUIRED_KHR	-1005
\stopclc

% Additions to Chapter 4 of the OpenCL 1.2 Specification
\subsection{對第 4 章的補充}

\refsec{contexts}中,用下列內容取代 \clapi{clCreateContext} 後面對 \carg{properties} 的描述:
\startreplacepar
\carg{properties} 指向一個特性列,即 \ccmm{<特性名, 值>} 的陣列,以零終止。
如果此陣列中沒有某個特性,則使用其缺省值,參見\reftab{prptForclCreateContext}。
如果 \carg{properties} 是 \cmacro{NULL} 或者是空的(第一個值就是零),所有特性都使用缺省值。
\stopreplacepar

\reftab{prptCtx_D3D10}是對\reftab{prptForclCreateContext}的補充。

\placetable[here][tab:prptCtx_D3D10]
{創建上下文時所用特性}
{\startETD[cl_context_properties][屬性值]

\clETD{CL_CONTEXT_D3D10_DEVICE_KHR}{ID3D10Device *}{
使用 \cldt{ID3D10Device *} 來與 Direct3D 10 進行互操作。
缺省值為 \cmacro{NULL}。
}

\stopETD

}

下列內容作為對 \clapi{clCreateContext} 所返回的錯誤列的補充:
\startreplacepar
\startigBase
\item \cenum{CL_INVALID_D3D10_DEVICE_KHR},
如果屬性 \cenum{CL_CONTEXT_D3D10_DEVICE_KHR} 的值不是 \cmacro{NULL},
並且沒有指定一個有效的 Direct3D 10 \cnglo{device}(\carg{cl_device_ids})。

\item \cenum{CL_INVALID_OPERATION},
如果屬性 \cenum{CL_INVALID_D3D10_DEVICE_KHR} 的值不是 \cmacro{NULL},
以表明要與 Direct3D 10 進行互操作,但同時又指定了要與其他圖形 API 進行互操作。
\stopigBase
\stopreplacepar

\clapi{clCreateContextFromType} 所返回的錯誤列中也要添加上面所描述的兩個新錯誤。

\reftab{paramNameclGetContextInfo}要增加\reftab{prpt_clGetContextInfo_d3d10}中所列內容:

\placetable[here][tab:prpt_clGetContextInfo_d3d10]
{\clapi{clGetContextInfo} 所支持的屬性名}
{\startETD[cl_context_info][返回型別]

\clETD{CL_CONTEXT_D3D10_PREFER_SHARED_RESOURCES_KHR}{cl_bool}{
在與 OpenCL 共享 Direct3D 10 資源時,
如果 \carg{MiscFlags} 中設置了 \cenum{D3D10_RESOURCE_MISC_SHARED} (創建時指定)的資源
比沒有設置此標誌的更快,則返回 \cenum{CL_TRUE},
否則返回 \cenum{CL_FALSE}。
}

\stopETD

}

% Additions to Chapter 5 of the OpenCL 1.2 Specification
\subsection{對第五章的補充}

為 \clapi{clGetMemObjectInfo} 所返回的錯誤碼添加以下內容:
\startreplacepar
\startigBase
\startitem
\cenum{CL_INVALID_D3D10_RESOURCE_KHR},
如果 \carg{param_name} 是 \cenum{CL_MEM_D3D10_RESOURCE_KHR},
並且 \carg{memobj} 不是由以下函式創建的:
\startigBase
\item \clapi{clCreateFromD3D10BufferKHR},
\item \clapi{clCreateFromD3D10Texture2DKHR},或
\item \clapi{clCreateFromD3D10Texture3DKHR}。
\stopigBase
\stopitem
\stopigBase
\stopreplacepar

將\reftab{meminfo_d3d10}中的內容添加到\reftab{meminfo}中。

\placetable[here][tab:meminfo_d3d10]
{\capi{clGetMemObjectInfo}所支持的\carg{param_names}}
{\startETD[cl_mem_info][返回类型]

\clETD{CL_MEM_D3D10_RESOURCE_KHR}{ID3D10Resource *}{
如果 \carg{memobj} 是用以下函式創建的,
則返回創建 \carg{memobj} 時所指定的引數 \carg{resource}:
\startigBase
\item \cenum{clCreateFromD3D10BufferKHR},
\item \cenum{clCreateFromD3D10Texture2DKHR},或
\item \cenum{clCreateFromD3D10Texture3DKHR}。
\stopigBase
}

\stopETD

}

為 \clapi{clGetImageInfo} 所返回的錯誤碼添加以下內容:
\startreplacepar
\startigBase
\startitem
\cenum{CL_INVALID_D3D10_RESOURCE_KHR},
如果 \carg{param_name} 是 \cenum{CL_MEM_D3D10_SUBRESOURCE_KHR},
並且 \carg{image} 不是由以下函式創建的:
\startigBase
\item \clapi{clCreateFromD3D10Texture2DKHR},或
\item \clapi{clCreateFromD3D10Texture3DKHR}。
\stopigBase
\stopitem
\stopigBase
\stopreplacepar

將\reftab{clgetimginfo_d3d10}中的內容添加到\reftab{clgetimginfo}中。

\placetable[here][tab:clgetimginfo_d3d10]
{\capi{clGetImageInfo}所支持的\carg{param_names}}
{\startETD[cl_image_info][返回型別]

\clETD{CL_MEM_D3D10_SUBRESOURCE_KHR}{ID3D10Resource *}{
如果 \carg{image} 是用以下函式創建的,
則返回創建 \carg{image} 時所指定的引數 \carg{subresource}:
\startigBase
\item \cenum{clCreateFromD3D10Texture2DKHR},或
\item \cenum{clCreateFromD3D10Texture3DKHR}。
\stopigBase
}

\stopETD
}

在\reftab{clGetEventInfo}中,
當 \carg{cl_event_info} = \cenum{CL_EVENT_COMMAND_TYPE} 時,
對其返回值的描述增加以下內容:
\startreplacepar
\startigBase
\item \cenum{CL_COMMAND_ACQUIRE_D3D10_OBJECTS_KHR}
\item \cenum{CL_COMMAND_RELEASE_D3D10_OBJECTS_KHR}
\stopigBase
\stopreplacepar

