\section{可選擴展的編譯指示}

編譯指示 \cdrtemp{#pragma OPENCL EXTENSION} 控制着 OpenCL 編譯器關於擴展的行為。
此指示定義如下:
\startclc[indentnext=no]
#pragma OPENCL EXTENSION /BTEX\cref{extension_name}/ETEX : /BTEX\cref{behavior}/ETEX
#pragma OPENCL EXTENSION all : /BTEX\cref{behavior}/ETEX
\stopclc
其中 \cref{extension_name} 就是擴展的名字,
形如 \clext{cl_khr_<名字>}(OpenCL 工作組所批准的擴展)或者 \clext{cl_<供應商的名字>_<名字>}(供應商的擴展)。
而符記 \cemp{all} 則意味着在編譯器所支持的所有擴展上應用此行為。
可以將 \cref{behavior} 設置為\reftab{behaviorDsc}中的值。

\placetable[here][tab:behaviorDsc]
{可選擴展的行為}
{\startbuffer[enExtDsc]
其行為遵守擴展 \cref{extension_name} 的規定。

如果指定了 \cemp{all},或者不支持 \cref{extension_name},
則會在 \cdrtemp{#pragma OPENCL EXTENSION} 上報告錯誤。
\stopbuffer

\startbuffer[disExtDsc]
其行為(包括發出錯誤、警告)如同語言定義中沒有擴展 \cref{extension_name} 那樣。

如果指定了 \cemp{all},則其行為必須回退到所編譯語言的無擴展核心版本。

如果不支持 \cref{extension_name},
則會在 \cdrtemp{#pragma OPENCL EXTENSION} 上發出警告。
\stopbuffer

\startCLOD[behavior][描述]
\clOD{enable}{\getbuffer[enExtDsc]}
\clOD{disable}{\getbuffer[disExtDsc]}
\stopCLOD
}

就設定每個擴展的行為而言,
編譯指示 \cdrtemp{#pragma OPENCL EXTENSION} 是一種簡單、底層的機制。
但是他並沒有定義任何策略,比如哪種組合比較恰當;
這些必須在其他地方定義。
在設定每個擴展的行為時,指示的順序非常重要。
後出現的指示會覆蓋早出現的。
而變體 \cemp{all} 會設定所有擴展的行為,將覆蓋前面出現的所有擴展指示,
但 \cref{behavior} 只能設置為 \cemp{disable}。

編譯器的初始狀態相當於如下指示:
\startclc[indentnext=no]
#pragma OPENCL EXTENSION all : disable
\stopclc
告訴編譯器必須按照此規格來報告所有錯誤和警告,忽略所有擴展。

每個擴展,只要會影響 OpenCL 語言的語義、文法或者為語言增加了內建函式,
都必須創建一個預處理 \ccmm{#define} 來匹配擴展名。
當且僅當實作支持此擴展時,這個 \ckey{#define} 才可用。

例:

如果一個擴展增加了擴展字串“\clext{cl_khr_3d_image_writes}”,
那麼同時應當增加相應的預處理 \cemp{#define}。
現在\cnglo{kernel}就可以像這樣來使用這個預處理 \ccmm{#define}:
\startclc
#ifdef cl_khr_3d_image_writes
	// do something using the extension
#else
	// do something else or #error!
#endif
\stopclc
