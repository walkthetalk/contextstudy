\subsection{由 GL 緩衝對象創建 CL 緩衝對象}

\topclfunc{clCreateFromGLBuffer}

\startCLFUNC
cl_mem clCreateFromGLBuffer (
			cl_context context,
			cl_mem_flags flags,
			GLuint bufobj,
			cl_int *errcode_ret)
\stopCLFUNC

此函式可以由 OpenGL 緩衝對象創建 OpenCL 緩衝對象。

\carg{context} 是一個由 OpenGL \cnglo{context}創建的 OpenCL \cnglo{context}。

\carg{flags} 是位欄,用於指定用法資訊。參見\reftab{clmemflags}。
此處只能使用 \cenum{CL_MEM_READ_ONLY}、 \cenum{CL_MEM_WRITE_ONLY} 和 \cenum{CL_MEM_READ_WRITE}。

\carg{bufobj} 是 GL \cnglo{bufobj}的名字。
其數據存儲必須在之前已經通過調用 \capi{glBufferData} 創建好了,
但其內容可以不必初始化。
數據存儲的大小將用來確定 CL \cnglo{bufobj}的大小。

\carg{errcode_ret} 會返回相應的錯誤碼,見下文。
如果他是 \cenum{NULL},則不會返回錯誤碼。

如果成功創建了\cnglo{bufobj},則 \clapi{clCreateFromGLBuffer} 會將其返回,
並將 \carg{errcode_ret} 置為 \cenum{CL_SUCCESS}。
否則,返回 \cenum{NULL},並將 \carg{errcode_ret} 置為下列錯誤碼之一:
\startigBase
\item \cenum{CL_INVALID_CONTEXT},
如果 \carg{context} 無效,或者不是由 GL \cnglo{context}創建的。

\item \cenum{CL_INVALID_VALUE},如果 \carg{flags} 的值無效。

\item \cenum{CL_INVALID_GL_OBJECT},
如果 \carg{bufobj} 不是 GL \cnglo{bufobj},
或者是 GL \cnglo{bufobj},但沒有數據存儲,或者其大小是 0。

\item \cenum{CL_OUT_OF_RESOURCES},如果\scdevfailres。

\item \cenum{CL_OUT_OF_HOST_MEMORY},如果\schostfailres。
\stopigBase

調用 \clapi{clCreateFromGLBuffer} 時,
會將 GL \cnglo{bufobj}數據存儲的大小用作所返回\cnglo{bufobj}的大小。
如果存在對應的 CL \cnglo{bufobj},
但是通過 GL API (如 \capi{glBufferData})修改了 GL \cnglo{bufobj}的狀態,
則後續使用 CL \cnglo{bufobj}時會導致\cnglo{undef}的行為。

可使用函式 \clapi{clRetainMemObject} 和 \clapi{clReleaseMemObject} 來
\cnglo{retain}和\cnglo{release}此\cnglo{bufobj}。

用 \clapi{clCreateFromGLBuffer} 創建的 CL \cnglo{bufobj}可用來創建 CL 1D \cnglo{imgobj}。

