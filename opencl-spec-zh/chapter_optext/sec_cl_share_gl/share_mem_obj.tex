\subsection{在 GL 和 CL 上下文間共享已映射到 GL 對象的內存對象}

\topclfunc{clEnqueueAcquireGLObjects}

\startCLFUNC
cl_int clEnqueueAcquireGLObjects (
			cl_command_queue command_queue,
			cl_uint num_objects.
			const cl_mem *mem_objects,
			cl_uint num_events_in_wait_list,
			const cl_event *event_wait_list,
			cl_event *event)
\stopCLFUNC

此函式可用來獲取(acquire)由 OpenGL 對象創建的 OpenCL \cnglo{memobj}。
\cnglo{cmdq}中的任一 OpenCL \cnglo{cmd}要想使用這些對象,都必須先獲取他們。
\carg{command_queue} 所在 OpenCL \cnglo{context}獲取 OpenGL 對象後,
此\cnglo{context}中的所有\cnglo{cmdq}都可以使用這些 OpenGL 對象。

\carg{command_queue} 是一個有效的\cnglo{cmdq}。
用來創建 \carg{command_queue} 所在 OpenCL \cnglo{context}的所有\cnglo{device}
都必須支持獲取共享的 CL/GL 對象。
在創建\cnglo{context}時就要如此。

\carg{num_objects} 即 \carg{mem_objects} 中\cnglo{memobj}的數目。

\carg{mem_objects} 指向 GL 對象所對應的一組 CL \cnglo{memobj}。

\carg{event_wait_list} 和 \carg{num_events_in_wait_list} 中
列出了執行此\cnglo{cmd}前要等待的事件。
如果 \carg{event_wait_list} 是 \cmacro{NULL},
則無須等待任何事件,並且 \carg{num_events_in_wait_list} 必須是0。
如果 \carg{event_wait_list} 不是 \cmacro{NULL},
則其中所有事件都必須是有效的,並且 \carg{num_events_in_wait_list} 必須大於 0。
\carg{event_wait_list} 中的事件充當同步點。

\carg{event} 會返回一個\cnglo{evtobj},
用來識別此\cnglo{cmd},可用來查詢或等待此\cnglo{cmd}完成。
而如果 \carg{event} 是 \cmacro{NULL},就沒辦法查詢此\cnglo{cmd}的狀態或等待其完成了。
如果 \carg{event_wait_list} 和 \carg{event} 都不是 \cmacro{NULL},
則引數 \carg{event} 不能屬於 \carg{event_wait_list}。

如果執行成功, \clapi{clEnqueueAcquireGLObjects} 會返回 \cenum{CL_SUCCESS}。
如果 \carg{num_objects} 是 0,並且 \carg{mem_objects} 是 \cmacro{NULL},
則此函式什麼都不做,並直接返回 \cenum{CL_SUCCESS}。
否則,返回下列錯誤碼之一:
\startigBase
\item \cenum{CL_INVALID_VALUE},
如果 \carg{num_objects} 是零,而 \carg{mem_objects} 不是 \cmacro{NULL};
或者 \carg{num_objects} > 0 而 \carg{mem_objects} 是 \cmacro{NULL}。

\item \cenum{CL_INVALID_MEM_OBJECT},
如果 \carg{mem_objects} 中的\cnglo{memobj}無效。

\item \cenum{CL_INVALID_COMMAND_QUEUE},
如果 \carg{command_queue} 無效。

\item \cenum{CL_INVALID_CONTEXT},
如果 \carg{command_queue} 所在\cnglo{context}不是由 OpenGL \cnglo{context}創建的。

\item \cenum{CL_INVALID_GL_OBJECT},
如果 \carg{mem_objects} 中的\cnglo{memobj}不是由 GL 對象創建的。

\startitem
\cenum{CL_INVALID_EVENT_WAIT_LIST},如果滿足下列條件中的任一項:
\startigBase
\item \carg{event_wait_list} 是 \cmacro{NULL},但 \carg{num_events_in_wait_list} > 0;
\item 或者 \carg{event_wait_list} 不是 \cmacro{NULL},但 \carg{num_events_in_wait_list} 是 0;
\item 或者 \carg{event_wait_list} 中有無效的事件。
\stopigBase
\stopitem

\item \cenum{CL_OUT_OF_RESOURCES},如果\scdevfailres。

\item \cenum{CL_OUT_OF_HOST_MEMORY},如果\schostfailres。
\stopigBase

\topclfunc{clEnqueueReleaseGLObjects}

\startCLFUNC
cl_int clEnqueueReleaseGLObjects (
			cl_command_queue command_queue,
			cl_uint num_objects.
			const cl_mem *mem_objects,
			cl_uint num_events_in_wait_list,
			const cl_event *event_wait_list,
			cl_event *event)
\stopCLFUNC

此函式可用來釋放(release)由 OpenGL 對象創建的 OpenCL \cnglo{memobj}。
要先將其釋放,然後 OpenGL 才能使用他們。
OpenGL 對象由 \carg{command_queue} 所在 OpenCL \cnglo{context} 釋放。

\carg{command_queue} 是一個有效的\cnglo{cmdq}。
用來創建 \carg{command_queue} 所在 OpenCL \cnglo{context}的所有\cnglo{device}
都必須支持獲取共享的 CL/GL 對象。
在創建\cnglo{context}時就要如此。

\carg{num_objects} 即 \carg{mem_objects} 中\cnglo{memobj}的數目。

\carg{mem_objects} 指向 GL 對象所對應的一組 CL \cnglo{memobj}。

\carg{event_wait_list} 和 \carg{num_events_in_wait_list} 中
列出了執行此\cnglo{cmd}前要等待的事件。
如果 \carg{event_wait_list} 是 \cmacro{NULL},
則無須等待任何事件,並且 \carg{num_events_in_wait_list} 必須是0。
如果 \carg{event_wait_list} 不是 \cmacro{NULL},
則其中所有事件都必須是有效的,並且 \carg{num_events_in_wait_list} 必須大於 0。
\carg{event_wait_list} 中的事件充當同步點。

\carg{event} 會返回一個\cnglo{evtobj},
用來識別此\cnglo{cmd},可用來查詢或等待此\cnglo{cmd}完成。
而如果 \carg{event} 是 \cmacro{NULL},就沒辦法查詢此\cnglo{cmd}的狀態或等待其完成了。
如果 \carg{event_wait_list} 和 \carg{event} 都不是 \cmacro{NULL},
則引數 \carg{event} 不能屬於 \carg{event_wait_list}。

如果執行成功, \clapi{clEnqueueReleaseGLObjects} 會返回 \cenum{CL_SUCCESS}。
如果 \carg{num_objects} 是 0,並且 \carg{mem_objects} 是 \cmacro{NULL},
則此函式什麼都不做,並直接返回 \cenum{CL_SUCCESS}。
否則,返回下列錯誤碼之一:
\startigBase
\item \cenum{CL_INVALID_VALUE},
如果 \carg{num_objects} 是零,而 \carg{mem_objects} 不是 \cmacro{NULL};
或者 \carg{num_objects} > 0 而 \carg{mem_objects} 是 \cmacro{NULL}。

\item \cenum{CL_INVALID_MEM_OBJECT},
如果 \carg{mem_objects} 中的\cnglo{memobj}無效。

\item \cenum{CL_INVALID_COMMAND_QUEUE},
如果 \carg{command_queue} 無效。

\item \cenum{CL_INVALID_CONTEXT},
如果 \carg{command_queue} 所在\cnglo{context}不是由 OpenGL \cnglo{context}創建的。

\item \cenum{CL_INVALID_GL_OBJECT},
如果 \carg{mem_objects} 中的\cnglo{memobj}不是由 GL 對象創建的。

\startitem
\cenum{CL_INVALID_EVENT_WAIT_LIST},如果滿足下列條件中的任一項:
\startigBase
\item \carg{event_wait_list} 是 \cmacro{NULL},但 \carg{num_events_in_wait_list} > 0;
\item 或者 \carg{event_wait_list} 不是 \cmacro{NULL},但 \carg{num_events_in_wait_list} 是 0;
\item 或者 \carg{event_wait_list} 中有無效的事件。
\stopigBase
\stopitem

\item \cenum{CL_OUT_OF_RESOURCES},如果\scdevfailres。

\item \cenum{CL_OUT_OF_HOST_MEMORY},如果\schostfailres。
\stopigBase

% Synchronizing OpenCL and OpenGL Access to Shared Objects
\subsubsection{OpenCL 和 OpenGL 存取共享對象時的同步}

為確保數據完整性,\cnglo{app}要負責通過各自的 API 對共享 CL/GL 對象的存取進行同步。
如果沒有提供同步,則會導致竟態以及其他\cnglo{undef}的行為,包括不可在實作間進行移植。

在調用 \clapi{clEnqueueAcquireGLObjects} 之前,
\cnglo{app}必須確保所有會存取 \carg{mem_objects} 中對象的處於擱置狀態的 GL 操作都完成了。
這可以通過在所有引用這些對象的 GL \cnglo{context}上發起 \clapi{glFinish} 並等待其完成來實現,
同時這種方式也是可移植的。
實作可能會提供其他更加有效的同步方法;
例如,在一些\cnglo{platform}上,調用 \clapi{glFlush} 可能就已經足夠了,
或者在線程中已經有隱式的同步,
或者在特定供應商擴展中可以在 GL 命令流間放置隔柵(fence)以等待 CL \cnglo{cmdq}中的此隔柵完成。
注意:目前只有 \clapi{glFinshi} 才是可移植的,其他方法都不是。

同樣,在調用 \clapi{clEnqueueReleaseGLObjects} 之後,
要想執行後續引用這些對象的 GL 命令,
\cnglo{app}必須確保所有會存取 \carg{mem_objects} 中對象的處於擱置狀態的 CL 操作都完成了。
這可以通過調用 \clapi{clWaitForEvents} 來等待 \clapi{clEnqueueReleaseGLObjects} 所返回的\cnglo{evtobj},或者調用 \clapi{clFinish} 來實現,
這兩種方式都是可移植的。
跟上面一樣,一些實作可能會提供更加有效的方式。

如果 CL 和 GL \cnglo{context} 處於不同的線程中,則\cnglo{app}需要負責維護操作間的正確次序。

如果 GL \cnglo{context}所綁定的線程不是調用 \clapi{clEnqueueReleaseGLObjects} 的線程,
在\cnglo{app}沒有執行其他附加步驟的情況下,
任何 \carg{mem_objects} 中對象的改變對那個\cnglo{context}都是不可見的。
對於 OpenGL 3.1 (或者更高版本)中的\cnglo{context}而言,
OpenGL 3.1 規範中的{\ftRef{附錄 D}}(共享對象以及多個\cnglo{context})描述了相關的要求。
而對於 OpenGL 的早期版本,相關要求\cnglo{impdef}。

在 OpenCL 獲取 OpenGL 對象後並將其釋放之前,
如果 OpenGL 嘗試存取此對象的數據會導致\cnglo{undef}的行為。
類似的,在 OpenCL 獲取共享的 CL/GL 對象前,或者將其釋放後,
如果 OpenCL 嘗試存取此對象也會導致\cnglo{undef}的行為。

