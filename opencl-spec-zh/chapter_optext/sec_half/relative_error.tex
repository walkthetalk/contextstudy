% Relative Error as ULPs
\subsection[sec:relativeErrorHalf]{相對誤差即 ULP}

本節中,我們將討論相對誤差(定義為 \ccmm{ulp},即 units in the last place,
浮點數間的最小間隔)的最大值。
對於半精度浮點數,
如果支持 \cenum{CL_FP_ROUND_TO_NEAREST},則缺省捨入模式為捨入為最近偶數;
否則缺省捨入模式為向零捨入。
而對於半精度浮點運算,如加、減、乘、積和熔加,則要求用缺省捨入模式正確捨入。

轉換為半精度浮點格式時,
如果指定了捨入模式 \ccmm{convert_},則用此模式進行捨入;
否則用缺省捨入模式進行捨入,或者 C 風格的轉型。

而將 \cldt{half} 轉換為整數格式時,
如果指定了捨入模式 \ccmm{convert_},則用此模式進行捨入;
否則向零捨入,或者使用 C 風格的轉型。

由 \cldt{half} 轉換為浮點格式時都是無損的。

\reftab{hpMathUlp}描述的是半精度浮點算術運算的最小精度,以 ULP 為單位。
計算 ULP 值時所參考的是無限精確的結果。
其中 0 ulp 表示相應函式無需捨入。

\placetable[here,split][tab:hpMathUlp]
{半精度內建數學函式的 ULP 值}
{\startCLFA[函式][最小精度—— ULP 值]

\clFAM{x+y}{正確捨入}
\clFAM{x-y}{正確捨入}
\clFAM{x*y}{正確捨入}
\clFAM{1.0/y}{正確捨入}
\clFAM{x/y}{正確捨入}

\clFAA{acos}{<= 2 ulp}
\clFAA{acospi}{<= 2 ulp}
\clFAA{asin}{<= 2 ulp}
\clFAA{asinpi}{<= 2 ulp}
\clFAA{atan}{<= 2 ulp}
\clFAA{atan2}{<= 2 ulp}
\clFAA{atanpi}{<= 2 ulp}
\clFAA{atan2pi}{<= 2 ulp}
\clFAA{acosh}{<= 2 ulp}
\clFAA{asinh}{<= 2 ulp}
\clFAA{atanh}{<= 2 ulp}
\clFAA{cbrt}{<= 2 ulp}
\clFAA{ceil}{正確捨入}
\clFAA{copysign}{0 ulp}
\clFAA{cos}{<= 2 ulp}
\clFAA{cosh}{<= 2 ulp}
\clFAA{cospi}{<= 2 ulp}
\clFAA{erfc}{<= 4 ulp}
\clFAA{erf}{<= 4 ulp}
\clFAA{exp}{<= 2 ulp}
\clFAA{exp2}{<= 2 ulp}
\clFAA{exp10}{<= 2 ulp}
\clFAA{expm1}{<= 2 ulp}
\clFAA{fabs}{0 ulp}
\clFAA{fdim}{正確捨入}
\clFAA{floor}{正確捨入}
\clFAA{fma}{正確捨入}
\clFAA{fmax}{0 ulp}
\clFAA{fmin}{0 ulp}
\clFAA{fmod}{0 ulp}
\clFAA{fract}{正確捨入}
\clFAA{frexp}{0 ulp}
\clFAA{hypot}{<= 2 ulp}
\clFAA{ilogb}{0 ulp}
\clFAA{ldexp}{正確捨入}
\clFAA{log}{<= 2 ulp}
\clFAA{log2}{<= 2 ulp}
\clFAA{log10}{<= 2 ulp}
\clFAA{log1p}{<= 2 ulp}
\clFAA{logb}{0 ulp}
\clFAA{mad}{所允許的任何值(無窮 ulp)}
\clFAA{maxmag}{0 ulp}
\clFAA{minmag}{0 ulp}
\clFAA{modf}{0 ulp}
\clFAA{nan}{0 ulp}
\clFAA{nextafter}{0 ulp}
\clFAA{pow(x, y)}{<= 4 ulp}
\clFAA{pown(x, y)}{<= 4 ulp}
\clFAA{powr(x, y)}{<= 4 ulp}
\clFAA{remainder}{0 ulp}
\clFAA{remquo}{0 ulp}
\clFAA{rint}{正確捨入}
\clFAA{rootn}{<= 4 ulp}
\clFAA{round}{正確捨入}
\clFAA{rsqrt}{<= 1 ulp}
\clFAA{sin}{<= 2 ulp}
\clFAA{sincos}{正弦值和餘弦值都是 <= 2 ulp}
\clFAA{sinh}{<= 2 ulp}
\clFAA{sinpi}{<= 2 ulp}
\clFAA{sqrt}{正確捨入}
\clFAA{tan}{<= 2 ulp}
\clFAA{tanh}{<= 2 ulp}
\clFAA{tanpi}{<= 2 ulp}
\clFAA{tgamma}{<= 4 ulp}
\clFAA{trunc}{正確捨入}

\stopCLFA

}

\startnotepar
在 \cldt{half} 標量或矢量數據型別上實施運算時,
實作可能將 \cldt{half} 值轉換為 \cldt{float} 值,
並在 \cldt{float} 值上實施運算。
這種情況下,實作僅將 \cldt{half} 標量或矢量數據型別作為存儲格式。
\stopnotepar

