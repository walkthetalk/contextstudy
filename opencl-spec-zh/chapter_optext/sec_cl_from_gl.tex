\section{由 GL 上下文或共享組創建 CL 上下文}

% Overview
\subsection{概覽}

\refsec{clShareGl}中定義了如何與 OpenGL 實作中的材質和緩衝對象共享數據,
但對於如何在 OpenCL \cnglo{context}和 OpenGL \cnglo{context}或共享組間建立聯繫,
則沒有定義。
此擴展為創建 OpenCL \cnglo{context}的例程定義了一個可選特性,
可以將 GL \cnglo{context}或共享組對象與新建的 OpenCL \cnglo{context}關聯起來。
如果實作支持此擴展,則\reftab{cldevquery}中所描述的 \cenum{CL_PLATFORM_EXTENSIONS} 或
\cenum{CL_DEVICE_EXTENSIONS} 中將包含字串 \clext{cl_khr_gl_sharing}。

此擴展要求 OpenGL 實作支持\cnglo{bufobj},以及與 OpenCL 共享材質和\cnglo{bufobj}圖像。

% New Procedures and Functions
\subsection{新的程序和函式}

\topclfunc{clGetGLContextInfoKHR}

\startclc
cl_intclGetGLContextInfoKHR (
		const cl_context_properties *properties,
		cl_gl_context_info param_name,
		size_t param_value_size,
		void *param_value,
		size_t *param_value_size_ret);
\stopclc

% New Tokens
\subsection{新的符記}

如果 \carg{properties} 中所指定的 OpenGL \cnglo{context}或共享組對象句柄無效,
則 \clapi{clCreateContext}、 \clapi{clCreateContextFromType} 和 \clapi{clGetGLContextInfoKHR} 會返回:
\startclc
CL_INVALID_GL_SHAREGROUP_REFERENCE_KHR		-1000
\stopclc

\clapi{clGetGLContextInfoKHR} 的引數 \carg{param_name} 接受下列值:
\startclc
CL_CURRENT_DEVICE_FOR_GL_CONTEXT_KHR		0x2006
CL_DEVICES_FOR_GL_CONTEXT_KHR			0x2007
\stopclc

\clapi{clCreateContext} 和 \clapi{clCreateContextFromType} 的引數 \carg{properties} 接受下列特性名:
\startclc
CL_GL_CONTEXT_KHR	0x2008
CL_EGL_DISPLAY_KHR	0x2009
CL_GLX_DISPLAY_KHR	0x200A
CL_WGL_HDC_KHR		0x200B
CL_CGL_SHAREGROUP_KHR	0x200C
\stopclc

%  Additions to Chapter 4 of the OpenCL 1.2 Specification
\subsection{對第 4 章的補充}

\refsec{contexts}中,用下列內容取代 \clapi{clCreateContext} 後面對 \carg{properties} 的描述:

\carg{properties} 指向一個特性列,即 \ccmm{<特性名, 值>} 的陣列,此陣列已排好序,以零終止。
如果此陣列中沒有某個特性,則使用其缺省值,參見\reftab{prptForclCreateContext}。
如果 \carg{properties} 是 \cenum{NULL} 或者是空的(第一個值就是零),所有特性都使用缺省值。

\refsec{clShareGl}中定義了一些特性,
用來控制如何與 OpenGL 緩衝、材質和渲染緩衝對象共享 OpenCL \cnglo{memobj}。

可以設置下列特性來識別 OpenGL \cnglo{context},當然,
這取決於一些特定\cnglo{platform}的 API (用來將 OpenGL \cnglo{context}綁定到視窗系統上):
\startigBase
\item 如果支持 CGL 綁定 API,
應當將特性 \cenum{CL_CGL_SHAREGROUP_KHR} 設置為一個 CGLShareGroup 句柄,
指向一個 CGL 共享組對象。

\item 如果支持 EGL 綁定 API,
應當將特性 \cenum{CL_GL_CONTEXT_KHR} 設置為一個 EGLContext 句柄,
指向一個 OpenGL ES 或 OpenGL \cnglo{context},
而將特性 \cenum{CL_EGL_DISPLAY_KHR} 設置為一個 EGLDisplay 句柄,
指向用於創建這個 OpenGL ES 或 OpenGL \cnglo{context}的顯示屏。

\item 如果支持 GLX 綁定 API,
應當將特性 \cenum{CL_GL_CONTEXT_KHR} 設置為一個 GLXContext 句柄,
指向一個 OpenGL \cnglo{context},
而將特性 \cenum{CL_GLX_DISPLAY_KHR} 設置為一個 Display 句柄,
指向用於創建這個 OpenGL \cnglo{context}的 X 視窗系統顯示屏。

\item 如果支持 WGL 綁定 API,
應當將特性 \cenum{CL_GL_CONTEXT_KHR} 設置為一個 HGLRC 句柄,
指向一個 OpenGL \cnglo{context},
而將特性 \cenum{CL_WGL_HDC_KHR} 設置為一個 HDC 句柄,
指向用於創建這個 OpenGL \cnglo{context}的顯示屏。
\stopigBase

