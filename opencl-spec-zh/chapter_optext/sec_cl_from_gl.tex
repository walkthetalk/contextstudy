\section{由 GL 上下文或共享組創建 CL 上下文}

% Overview
\subsection{概覽}

\refsec{clShareGl}中定義了如何與 OpenGL 實作中的材質和緩衝對象共享數據,
但對於如何在 OpenCL \cnglo{context}和 OpenGL \cnglo{context}或共享組間建立聯繫,
則沒有定義。
此擴展為創建 OpenCL \cnglo{context}的例程定義了一個可選特性,
可以將 GL \cnglo{context}或共享組對象與新建的 OpenCL \cnglo{context}關聯起來。
如果實作支持此擴展,則\reftab{cldevquery}中所描述的 \cenum{CL_PLATFORM_EXTENSIONS} 或
\cenum{CL_DEVICE_EXTENSIONS} 中將包含字串 \clext{cl_khr_gl_sharing}。

此擴展要求 OpenGL 實作支持\cnglo{bufobj},以及與 OpenCL 共享材質和\cnglo{bufobj}圖像。

% New Procedures and Functions
\subsection{新的程序和函式}

\topclfunc{clGetGLContextInfoKHR}

\startclc
cl_intclGetGLContextInfoKHR (
		const cl_context_properties *properties,
		cl_gl_context_info param_name,
		size_t param_value_size,
		void *param_value,
		size_t *param_value_size_ret);
\stopclc

% New Tokens
\subsection{新的符記}

如果 \carg{properties} 中所指定的 OpenGL \cnglo{context}或共享組對象句柄無效,
則 \clapi{clCreateContext}、 \clapi{clCreateContextFromType} 和 \clapi{clGetGLContextInfoKHR} 會返回:
\startclc
CL_INVALID_GL_SHAREGROUP_REFERENCE_KHR		-1000
\stopclc

\clapi{clGetGLContextInfoKHR} 的引數 \carg{param_name} 接受下列值:
\startclc
CL_CURRENT_DEVICE_FOR_GL_CONTEXT_KHR		0x2006
CL_DEVICES_FOR_GL_CONTEXT_KHR			0x2007
\stopclc

\clapi{clCreateContext} 和 \clapi{clCreateContextFromType} 的引數 \carg{properties} 接受下列特性名:
\startclc
CL_GL_CONTEXT_KHR	0x2008
CL_EGL_DISPLAY_KHR	0x2009
CL_GLX_DISPLAY_KHR	0x200A
CL_WGL_HDC_KHR		0x200B
CL_CGL_SHAREGROUP_KHR	0x200C
\stopclc

%  Additions to Chapter 4 of the OpenCL 1.2 Specification
\subsection{對第 4 章的補充}

\refsec{contexts}中,用下列內容取代 \clapi{clCreateContext} 後面對 \carg{properties} 的描述:

\carg{properties} 指向一個特性列,即 \ccmm{<特性名, 值>} 的陣列,此陣列已排好序,以零終止。
如果此陣列中沒有某個特性,則使用其缺省值,參見\reftab{prptForclCreateContext}。
如果 \carg{properties} 是 \cenum{NULL} 或者是空的(第一個值就是零),所有特性都使用缺省值。

\refsec{clShareGl}中定義了一些特性,
用來控制如何與 OpenGL 緩衝、材質和渲染緩衝對象共享 OpenCL \cnglo{memobj}。

可以設置下列特性來識別 OpenGL \cnglo{context},當然,
這取決於一些特定\cnglo{platform}的 API (用來將 OpenGL \cnglo{context}綁定到視窗系統上):
\startigBase
\item 如果支持 CGL\footnote{CGL 是 Mac OS X 的 OpenGL 接口。} 綁定 API,
應當將特性 \cenum{CL_CGL_SHAREGROUP_KHR} 設置為一個 CGLShareGroup 句柄,
指向一個 CGL 共享組對象。

\item 如果支持 EGL\footnote{%
EGL 是 Khronos 渲染 API (如 OpenGL ES 或 OpenVG)和底層原生平台視窗系統之間的接口。%
} 綁定 API,
應當將特性 \cenum{CL_GL_CONTEXT_KHR} 設置為一個 EGLContext 句柄,
指向一個 OpenGL ES 或 OpenGL \cnglo{context},
而將特性 \cenum{CL_EGL_DISPLAY_KHR} 設置為一個 EGLDisplay 句柄,
指向用於創建這個 OpenGL ES 或 OpenGL \cnglo{context}的顯示屏。

\item 如果支持 GLX\footnote{GLX 是 X11 的 OpenGL 接口。} 綁定 API,
應當將特性 \cenum{CL_GL_CONTEXT_KHR} 設置為一個 GLXContext 句柄,
指向一個 OpenGL \cnglo{context},
而將特性 \cenum{CL_GLX_DISPLAY_KHR} 設置為一個 Display 句柄,
指向用於創建這個 OpenGL \cnglo{context}的 X 視窗系統顯示屏。

\item 如果支持 WGL\footnote{WGL 是 Microsoft Windows 的 OpenGL 接口。} 綁定 API,
應當將特性 \cenum{CL_GL_CONTEXT_KHR} 設置為一個 HGLRC 句柄,
指向一個 OpenGL \cnglo{context},
而將特性 \cenum{CL_WGL_HDC_KHR} 設置為一個 HDC 句柄,
指向用於創建這個 OpenGL \cnglo{context}的顯示屏。
\stopigBase

如果是在這樣的\cnglo{context}中創建的\cnglo{memobj},
那麼他可以被指定的 OpenGL 或 OpenGL ES \cnglo{context}
(也包括此\cnglo{context}的共享列中其他 OpenGL \cnglo{context},參見 GLX 1.4 和 EGL 1.4 規範,以及 Microsoft Windows 上 OpenGL 實作 WGL 的文檔)、
或者 CGL 共享組所共享。

如果特性列中沒有指定 OpenGL 或 OpenGL ES \cnglo{context}或者共享組,
那麼就不能共享\cnglo{memobj},
並且調用\refsec{clShareGl}中的\cnglo{cmd}時
會導致錯誤 \cenum{CL_INVALID_GL_SHAREGROUP_REFERENCE_KHR}。

OpenCL / OpenGL 間的共享不支持屬性 \cenum{CL_CONTEXT_INTEROP_USER_SYNC}
 (參見\reftab{prptForclCreateContext})。
如果創建帶有 OpenCL / OpenGL 共享的\cnglo{context}時指定了此屬性,則會返回錯誤。

\reftab{prptForclCreateContextPF}是對\reftab{prptForclCreateContext}的補充。

\placetable[here][tab:prptForclCreateContextPF]
{創建上下文時所用特性}
{\startETD[cl_context_properties][屬性值]

\clETD{CL_GL_CONTEXT_KHR}{OpenGL \cnglo{context}句柄}{
OpenCL \cnglo{context}所關聯的 OpenGL \cnglo{context}。
缺省值為 \cenumemp{0}。
}

\clETD{CL_CGL_SHAREGROUP_KHR}{OpenGL 共享組句柄}{
OpenCL \cnglo{context}所關聯的 CGL 共享組。
缺省值為 \cenumemp{0}。
}

\clETD{CL_EGL_DISPLAY_KHR}{EGLDisplay 句柄}{
OpenGL \cnglo{context}所對應的 EGLDisplay。
缺省值為 \cenumemp{EGL_NO_DISPLAY}。
}

\clETD{CL_GLX_DISPLAY_KHR}{X 句柄}{
OpenGL \cnglo{context}所對應的 X Display。
缺省值為 \cenumemp{None}。
}

\clETD{CL_WGL_HDC_KHR}{HDC 句柄}{
OpenGL \cnglo{context}所對應的 HDC。
缺省值為 \cenumemp{0}。
}

\stopETD

}

下列內容取代 \clapi{clCreateContext} 所返回的錯誤列中的第一個:
\startreplacepar
如果\cnglo{context}由下列任一方式指定:
\startigBase[indentnext=no]
\item 通過設置特性 \cenum{CL_GL_CONTEXT_KHR} 和 \cenum{CL_EGL_DISPLAY_KHR} 為
基於 EGL 的 OpenGL ES 或 OpenGL 實作指定了一個\cnglo{context}。

\item 通過設置特性 \cenum{CL_GL_CONTEXT_KHR} 和 \cenum{CL_GLX_DISPLAY_KHR} 為
基於 GLX 的 OpenGL 實作指定了一個\cnglo{context}。

\item 通過設置特性 \cenum{CL_GL_CONTEXT_KHR} 和 \cenum{CL_WGL_HDC_KHR} 為
基於 WGL 的 OpenGL 實作指定了一個\cnglo{context}。
\stopigBase
並且滿足下列任一條件:
\startigBase[indentnext=no]
\item 所指定的 display 和\cnglo{context}特性不能標識一個有效的 OpenGL 或 OpenGL ES \cnglo{context}。

\item 所指定的\cnglo{context}不支持\cnglo{bufobj}和渲染緩衝對象。

\item 所指定的\cnglo{context}與所創建的 OpenCL \cnglo{context}不兼容。
例如,位於物理上不同的位址空間內,如另一個硬件設備上;或者由於實作的局限不支持與 OpenCL 共享數據。
\stopigBase
則 \carg{errcode_ret} 會返回 \cenum{CL_INVALID_GL_SHAREGROUP_REFERENCE_KHR}。

如果通過設置特性 \cenum{CL_CGL_SHAREGROUP_KHR} 為基於 CGL 的 OpenGL 實作指定了一個共享組,
但是所指定的共享組不能標誌一個有效的 CGL 共享組對象,
那麼 \carg{errcode_ret} 會返回 \cenum{CL_INVALID_GL_SHAREGROUP_REFERENCE_KHR}。

如果按上面所描述的那樣指定了一個\cnglo{context},並且滿足下列任一條件:
\startigBase[indentnext=no]
\item 為 CGL、 EGL、 GLX 或 WGL 其中之一指定了一個\cnglo{context}或共享組對象,
但是 OpenGL 實作不支持視窗系統綁定 API。

\item 為 \cenum{CL_CGL_SHAREGROUP_KHR}、 \cenum{CL_EGL_DISPLAY_KHR}、
 \cenum{CL_GLX_DISPLAY_KHR} 以及 \cenum{CL_WGL_HDC_KHR} 中一個以上的特性設置了非缺省值。

\item 為特性 \cenum{CL_CGL_SHAREGROUP_KHR} 和 \cenum{CL_GL_CONTEXT_KHR} 都設置了非缺省值。

\item 引數 \carg{devices} 中任一\cnglo{device}不支持 OpenCL 對象與 OpenGL 對象共享數據存儲,
參見\refsec{clShareGl}。
\stopigBase
則 \carg{errcode_ret} 會返回 \cenum{CL_INVALID_OPERATION}。

如果 \carg{properties} 中任一特性名無效
或者 \carg{properties} 中有特性 \cenum{CL_CONTEXT_INTEROP_USER_SYNC},
則 \carg{errcode_ret} 會返回 \cenum{CL_INVALID_PROPERTY}。
\stopreplacepar

下列內容取代 \clapi{clCreateContextFromType} 中對 \carg{properties} 的描述:
\startreplacepar
\carg{properties} 指向一個特性列,
其格式以及有效內容與 \clapi{clCreateContext} 的引數 \carg{properties} 相同。
\stopreplacepar

用上面所描述的兩個新錯誤取代 \clapi{clCreateContextFromType} 的錯誤列中的第一個。

% Additions to section 9.7 of the OpenCL 1.2 Extension Specification
\subsection{對節 9.7 的補充}

