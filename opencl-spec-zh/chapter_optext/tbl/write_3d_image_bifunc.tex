% atomic_add
\startbuffer[funcproto:write_image_3d]
void write_imagef (image3d_t image,
		int4 coord,
		float4 color)

void write_imagei (image3d_t image,
		int4 coord,
		int4 color)

void write_imageui (image3d_t image,
		int4 coord,
		uint4 color)
\stopbuffer
\startbuffer[funcdesc:write_image_3d]
將 \carg{color} 的值寫入 3D \cnglo{imgobj} \carg{image}中坐標 \math{(x,y,z)} 處。
寫入前會對顏色值進行恰當的數據格式轉換。
會將 \carg{coord.x}、 \carg{coord.y} 和 \carg{coord.z} 視為非歸一化坐標,
且其值必須分別位於區間 \math{0\cdots\mvar{圖像寬度}-1}、 \math{0\cdots\mvar{圖像高度}-1} 和 \math{0\cdots\mvar{圖像深度}-1} 內。

對於 \capi{write_imagef},
創建\cnglo{imgobj}時所用的 \carg{image_channel_data_type} 必須是預定義壓縮過的格式
或者 \cenum{CL_SNORM_INT8}、 \cenum{CL_UNORM_INT8}、 \cenum{CL_SNORM_INT16}、
 \cenum{CL_UNORM_INT16}、 \cenum{CL_HALF_FLOAT} 或 \cenum{CL_FLOAT}。
會將通道數據由浮點值轉換成存儲數據所用的實際數據格式。

對於 \capi{write_imagei} 而言,
創建\cnglo{imgobj}時所用的 \carg{image_channel_data_type} 必須是下列值之一:
\startigBase
\item \cenum{CL_SIGNED_INT8}
\item \cenum{CL_SIGNED_INT16}
\item \cenum{CL_SIGNED_INT32}
\stopigBase

對於 \capi{write_imageui} 而言,
創建\cnglo{imgobj}時所用的 \carg{image_channel_data_type} 必須是下列值之一:
\startigBase
\item \cenum{CL_UNSIGNED_INT8}
\item \cenum{CL_UNSIGNED_INT16}
\item \cenum{CL_UNSIGNED_INT32}
\stopigBase

如果創建\cnglo{imgobj}時所用的 \carg{image_channel_data_type} 不再上述所列範圍內,
或者坐標 \math{(x, y, z)} 不在 \math{(0 \cdots \mvar{圖像寬度}-1, 0 \cdots \mvar{圖像高度}-1, 0 \cdots \mvar{圖像深度}-1)} 範圍內,
則 \capi{write_imagef}、 \capi{write_imagei} 和 \capi{write_imageui} 的行為\cnglo{undef}。
\stopbuffer

% begin table
\startCLFD
\clFD{write_image_3d}
\stopCLFD
