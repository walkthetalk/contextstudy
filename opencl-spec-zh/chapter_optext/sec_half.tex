\section{半精度浮點數}

此擴展增加了對 \cldt{half} 標量和矢量型別的支持,
可以 \cldt{half} 作為內建型別進行算術運算、轉換等。
\cnglo{app}要想使用型別 \cldt{half} 和 \cldt[n]{half},
必須包含編譯指示 \cemp{#pragma OPENCL EXTENSION cl_khr_fp16 : enable}。

\reftab{builtInScalarDataTypes}和\reftab{builtInVectorDataTypes}中所列內建標量、矢量數據型別又做了如下擴充:

\placetable[here][tab:half_type_dsc]
{\cldt{half} 相關數據型別}
{\startCLOD[型別][描述]

\clOD{\cldt{half2}}{2 組件半精度浮點矢量。}

\clOD{\cldt{half3}}{3 組件半精度浮點矢量。}

\clOD{\cldt{half4}}{4 組件半精度浮點矢量。}

\clOD{\cldt{half8}}{8 組件半精度浮點矢量。}

\clOD{\cldt{half16}}{16 組件半精度浮點矢量。}

\stopCLOD
}

在 OpenCL API(以及頭檔)中,內建矢量數據型別 \cldt[n]{half} 被聲明為其他型別,
以更好的為\cnglo{app}所用。
\reftab{bihalf2appdt}中列出了 OpenCL C 編程語言中
所定義的內建矢量數據型別 \cldt[n]{half} 與\cnglo{app}所用型別間的對應關係。

\placetable[here][tab:bihalf2appdt]
{內建矢量數據型別與應用程式所用型別的對應關係}
{\startCLOO[OpenCL 語言中的型別][\cnglo{app}所用 API 中的型別]

\clOO{\cldt{half2}}{\cldt{cl_half2}}
\clOO{\cldt{half3}}{\cldt{cl_half3}}
\clOO{\cldt{half4}}{\cldt{cl_half4}}
\clOO{\cldt{half8}}{\cldt{cl_half8}}
\clOO{\cldt{half16}}{\cldt{cl_half16}}

\stopCLOO


}

\refsec{operator}中所描述的關係、相等、邏輯以及邏輯單元算子
均可用於 \cldt{half} 標量和 \cldt[n]{half} 矢量型別,
所產生的結果分別為標量 \cldt{int} 和矢量 \cldt[n]{short}。

可以為浮點常值添加後綴 \ccmm{h} 或 \ccmm{H},
以表明此常值型別為 \cldt{half}。

\subsection{轉換}

現在,\refsec{implicityConversion}中的隱式轉換規則也適用於 \cldt{half} 標量和 \cldt[n]{half} 矢量數據型別。

\refsec{explicitCast}中的顯式轉型也做了擴充,
適用於 \cldt{half} 標量數據型別和 \cldt[n]{half} 矢量數據型別。

\refsec{explicitConversion}中所描述的顯式轉換函式也做了擴充,
適用於 \cldt{half} 標量數據型別和 \cldt[n]{half} 矢量數據型別。

\refsec{as_typen}中所描述的用於重釋型別的函式 \clapi[n]{as_type} 也做了擴充,
允許在 \cldt[n]{short}、 \cldt[n]{ushort} 和 \cldt[n]{half} 標量、矢量數據型別間進行無需轉換的轉型。

\subsection{數學函式}

對\reftab{svMathFunc}中所列內建數學函式作了擴充,
函式引數和返回值也可以是 \cldt{half} 和 \cldt[n]{half},
參見\reftab{svMathFuncHalf}。
現在, \cldt{gentype} 也包含 \cldt{half} 和 \cldt[n]{half},
其中 \ccmmsuffix{n} 可以是 2、 3、 4、 8、 16。

對於函式的任一特定用法,所有引數以及返回值的實際型別必須相同。

\placetable[here,split][tab:svMathFuncHalf]
{標量和矢量引數內建數學函式表}
{\startCLFD

\clFD{acos}
\clFD{acosh}
\clFD{acospi}
\clFD{asin}
\clFD{asinh}
\clFD{asinpi}
\clFD{atan}
\clFD{atan2}
\clFD{atanh}
\clFD{atanpi}
\clFD{atan2pi}
\clFD{cbrt}
\clFD{ceil}
\clFD{copysign}
\clFD{cos}
\clFD{cosh}
\clFD{cospi}
\clFD{erfc}
\clFD{erf}
\clFD{exp}
\clFD{exp2}
\clFD{exp10}
\clFD{expm1}
\clFD{fabs}
\clFD{fdim}
\clFD{floor}
\clFD{fma}
\clFD{fmaxh}
\clFD{fminh}
\clFD{fmod}
\clFD{fract}
\clFD{frexph}
\clFD{hypot}
\clFD{ilogbh}
\clFD{ldexph}
\clFD{lgammah}
\clFD{log}
\clFD{log2}
\clFD{log10}
\clFD{log1p}
\clFD{logb}
\clFD{mad}
\clFD{maxmag}
\clFD{minmag}
\clFD{modf}
\clFD{nanh}
\clFD{nextafter}
\clFD{pow}
\clFD{pownh}
\clFD{powr}
\clFD{remainder}
\clFD{remquoh}
\clFD{rint}
\clFD{rootnh}
\clFD{rsqrt}
\clFD{sin}
\clFD{sincos}
\clFD{sinh}
\clFD{sinpi}
\clFD{sqrt}
\clFD{tan}
\clFD{tanh}
\clFD{tanpi}
\clFD{tgamma}
\clFD{trunc}

\stopCLFD
}

巨集 \cmacroemp{FP_FAST_FMAF} 用來指明對於半精度浮點數,
 \capi{fma} 函式族是否比直接編碼更快。
如果定義了此巨集,則表明對算元為 \ctype{float} 的乘、加運算,
函式 \capi{fma} 一般跟直接編碼一樣快,或者更快。

下列巨集必須使用指定的值。
可以在預處理指示 \ccmm{#if} 中使用這些常量算式。
\startclc
#define HALF_DIG		3
#define HALF_MANT_DIG		11
#define HALF_MAX_10_EXP		+4
#define HALF_MAX_EXP		+16
#define HALF_MIN_10_EXP		-4
#define HALF_MIN_EXP		-13
#define HALF_RADIX		2
#define HALF_MAX		0x1.ffcp15h
#define HALF_MIN		0x1.0p-14h
#define HALF_EPSILON		0x1.0p-10f
\stopclc

\reftab{tblHalfMacroAndApp}中給出了上面所列巨集與\cnglo{app}所用的巨集名字之間的對應關係。

\placetable[here][tab:tblHalfMacroAndApp]
{半精度浮點巨集與應用程式所用巨集的對應關係}
{\startCLOO[OpenCL 語言中的巨集][\cnglo{app}所用的巨集]

\clMMH{DIG}
\clMMH{MANT_DIG}
\clMMH{MAX_10_EXP}
\clMMH{MAX_EXP}
\clMMH{MIN_10_EXP}
\clMMH{MIN_EXP}
\clMMH{RADIX}
\clMMH{MAX}
\clMMH{MIN}
\clMMH{EPSILSON}

\stopCLOO
}

除此之外,還有一些常量可用,如\reftab{tblHalfMacroConst}所示。
他們的型別都是 \ctype{float},在 \ctype{float} 型別的精度內是準確的。

\placetable[here][tab:tblHalfMacroConst]
{半精度浮點常量}
{\startCLOO[常量][描述]

\clCM{M_E_H}{e}
\clCM{M_LOG2E_H}{log_{2}e}
\clCM{M_LOG10E_H}{log_{10}e}
\clCM{M_LN2_H}{log_{e}2}
\clCM{M_LN10_H}{log_{e}10}
\clCM{M_PI_H}{\pi}
\clCM{M_PI_2_H}{\pi/2}
\clCM{M_PI_4_H}{\pi/4}
\clCM{M_1_PI_H}{1/\pi}
\clCM{M_2_PI_H}{2/\pi}
\clCM{M_2_SQRTPI_H}{2/\sqrt{\pi}}
\clCM{M_SQRT2_H}{\sqrt{2}}
\clCM{M_SQRT1_2_H}{1/\sqrt{2}}

\stopCLOO
}

\subsection{公共函式}

對\reftab{svCommonFunc}中所列的內建公共函式作了擴充,
函式引數和返回值也可以是 \cldt{half} 和 \cldt[n]{half},
參見\reftab{svCommonFuncHalf}。
現在, \cldt{gentype} 也包含 \cldt{half} 和 \cldt[n]{half},
其中 \ccmmsuffix{n} 可以是 2、 3、 4、 8、 16。

\startnotepar
可以使用化簡(如 \capi{mad} 或 \capi{fma})來實現 \capi{mix} 和 \capi{smoothstep}。
\stopnotepar

\placetable[here][tab:svCommonFuncHalf]
{內建公共函式}
{\startCLFD

\clFD{clamp_half}
\clFD{degrees}
\clFD{max_half}
\clFD{min_half}
\clFD{mix_half}
\clFD{radians}
\clFD{step_half}
\clFD{smoothstep_half}
\clFD{sign}

\stopCLFD
}

\subsection[sec:geomtricFunc]{幾何函式}

對\reftab{svGeometricFunc}中所列的內建幾何函式作了擴充,
函式引數和返回值也可以是 \cldt{half} 和 \cldt[n]{half},
參見\reftab{svGeometricFuncHalf}。
現在, \cldt{gentype} 也包含 \cldt{half} 和 \cldt[n]{half},
其中 \ccmmsuffix{n} 可以是 2、 3、 4。

\startnotepar
可以使用化簡(如 \capi{mad} 或 \capi{fma})來實現幾何函式。
\stopnotepar

\placetable[here][tab:svGeometricFuncHalf]
{內建幾何函式}
{\startCLFD

\clFD{cross_half}
\clFD{dot_half}
\clFD{distance_half}
\clFD{length_half}
\clFD{normalize_half}

\stopCLFD
}

\subsection[sec:relationFunc]{關係函式}

對\reftab{svRelationalFunc}中所列內建關係函式作了擴充,
引數可以為 \cldt{half} 和 \cldt[n]{half},
其中 \ccmmsuffix{n} 可以是 2、 3、 4、 8、 16,
參見\reftab{relationalFuncHalf}。

關係算子和相等算子(<、 <=、 >、 >=、 !=、 ==)也可用於矢量型別 \cldt[n]{half},
所產生的結果為 \cldt[n]{short},參見\refsec{operator}。

對於標量型別的引數,如果所指定的關係為 {\ftRef{false}},則下列函式(參見\reftab{svRelationalFunc})會返回 0,否則返回 1:
\startigBase[indentnext=no]
\item \capi{isequal}、 \capi{isnotequal}、
\item \capi{isgreater}、 \capi{isgreaterequal}、
\item \capi{isless}、 \capi{islessequal}、
\item \capi{islessgreater}、
\item \capi{isfinite}、 \capi{isinf}、
\item \capi{isnan}、 \capi{isnormal}、
\item \capi{isordered}、 \capi{isunordered} 和
\item \capi{signbit}。
\stopigBase
而對於矢量型別的引數,如果所指定的關係為 {\ftRef{false}},則返回 0,
否則返回 -1 (即所有位都是 1)。

如果任一引數為 NaN,則下列關係函式返回 0:
\startigBase[indentnext=no]
\item \capi{isequal}、
\item \capi{isgreater}、 \capi{isgreaterequal}、
\item \capi{isless}、 \capi{islessequal} 和
\item \capi{islessgreater}。
\stopigBase
如果引數為標量,則當任一引數為 NaN 時, \capi{isnotequal} 返回 1;
而如果引數為矢量,則當任一引數為 NaN 時, \capi{isnotequal} 返回 -1。

\placetable[here,split][tab:relationalFuncHalf]
{內建關係函式}
{\startCLFD

\clFD{isequal_half}
\clFD{isnotequal_half}
\clFD{isgreater_half}
\clFD{isgreaterequal_half}
\clFD{isless_half}
\clFD{islessequal_half}
\clFD{islessgreater_half}
\clFD{isfinite_half}
\clFD{isinf_half}
\clFD{isnan_half}
\clFD{isnormal_half}
\clFD{isordered_half}
\clFD{isunordered_half}
\clFD{signbit_half}
\clFD{bitselect_half}
\clFD{select_half}

\stopCLFD
}

\subsection[sec:vectorLsFuncHalf]{矢量數據裝載和存儲函式}

對\reftab{vectorLsFunc}中所列的矢量數據裝載(\clapi[n]{vload})和存儲(\clapi[n]{vstore})函式作了擴充,
可以讀寫 \cldt{half} 標量和矢量值,
參見\reftab{vectorLsFuncHalf}

對泛型 \cldt{gentype} 也作了擴充,包含 \cldt{half}。
而對泛型 \cldt[n]{gentype} 也作了擴充,包含了 \cldt[n]{half},
其中 \ccmmsuffix{n} 為 2、 3、 4、 8 或 16。

\startnotepar
\capi{vload3} 和 \capi{vstore3}
均由位址 \math{(\marg{p} + (\marg{offset}\times 3))} 讀寫矢量組件 \ccmm{x}、 \ccmm{y}、 \ccmm{z}。
\stopnotepar

\placetable[here,split][tab:vectorLsFuncHalf]
{矢量數據裝載、存儲函式表}
{\startCLFD
\clFD{vloadn}
\clFD{vstoren}
\stopCLFD
}

\subsection[sec:asyncCopyPrefetch]{在全局內存和局部內存間的異步拷貝以及預取}

OpenCL C 編程語言實現了\reftab{asyncCopyPrefetch}中所列函式,
可在\cnglo{glbmem}和\cnglo{locmem}間進行異步拷貝,
以及從\cnglo{glbmem}中預取(prefetch)。

對泛型 \ctype{gentype} 作了擴充,
包含 \cldt{half} 和 \cldt[n]{half},
其中 \ccmmsuffix{n} 可以是 2、 3、 4、 8、 16。

%\placetable[here][tab:asyncCopyPrefetch]
%{內建異步拷貝和預取函式}
%{% async_work_group_copy
\startbuffer[funcproto:async_work_group_copy]
event_t async_work_group_copy ( 
	__local gentype *dst, 
	const __global gentype *src, 
	size_t num_gentypes, 
	event_t event) 
event_t async_work_group_copy (
	__global gentype *dst,
	const __local gentype *src,
	size_t num_gentypes,
	event_t event)
\stopbuffer
\startbuffer[funcdesc:async_work_group_copy]
從 \carg{src} 異步拷貝 \carg{num_gentypes} 個 \ctype{gentype} 元素
到 \carg{dst} 中。
\cnglo{workgrp}中的所有\cnglo{workitem}都會實施此異步拷貝,
因此在\cnglo{workgrp}中,
使用相同引數值執行\cnglo{kernel}的所有\cnglo{workitem}必須都能執行到此函式,
否則結果未定義。

返回的\cnglo{evtobj}可由 \capi{wait_group_events} 用來等待異步拷貝完畢。
可以使用引數 \carg{event} 將 \capi{async_work_group_copy} 與之前的異步拷貝關聯在一起,
從而使得多個異步拷貝間可共享同一事件;否則 \carg{event} 必須是零。

如果引數 \carg{event} 非零,則會將其中的\cnglo{evtobj}返回。

此函式不會對源數據實施隱式同步,如在拷貝前執行 \capi{barrier}。
\stopbuffer

% async_work_group_strided_copy
\startbuffer[funcproto:async_work_group_strided_copy]
event_t async_work_group_strided_copy (
	__local gentype *dst,
	const __global gentype *src,
	size_t num_gentypes,
	size_t src_stride,
	event_t event)
event_t async_work_group_strided_copy (
	__global gentype *dst,
	const __local gentype *src,
	size_t num_gentypes,
	size_t dst_stride,
	event_t event)
\stopbuffer
\startbuffer[funcdesc:async_work_group_strided_copy]
從 \carg{src} 異步採集 \carg{num_gentypes} 個 \ctype{gentype} 元素
到 \carg{dst} 中。
參數 \carg{src_stride} 為從 \carg{src} 中讀取元素時所用跨距。
參數 \carg{dst_stride} 為將元素寫入 \carg{dst} 中時所用跨距。
\cnglo{workgrp}中的所有\cnglo{workitem}都會實施此異步採集,
因此在\cnglo{workgrp}中,
使用相同引數值執行\cnglo{kernel}的所有\cnglo{workitem}必須都能執行到此函式,
否則結果未定義。

返回的\cnglo{evtobj}可由 \capi{wait_group_events} 用來等待異步拷貝完畢。
可以使用引數 \carg{event} 將 \capi{async_work_group_strided_copy} 與之前的異步拷貝
關聯在一起,
從而使得多個異步拷貝間可共享同一事件;
否則 \carg{event} 必須是零。

如果引數 \carg{event} 非零,則會將其中的\cnglo{evtobj}返回。

此函式不會對源數據實施隱式同步,如在拷貝前執行 \capi{barrier}。

如果 \carg{src_stride} 或 \carg{dst_stride} 是 0,
或者拷貝時, \carg{src_stride} 或 \carg{dst_stride} 使得
 \carg{src} 或 \carg{dst} 指針超過了位址空間的上界,
則 \capi{async_work_group_strided_copy} 的行為未定義。
\stopbuffer

% wait_group_events
\startbuffer[funcproto:wait_group_events]
void wait_group_events (
	int num_events,
	event_t *event_list)
\stopbuffer
\startbuffer[funcdesc:wait_group_events]
等待用來表示 \capi{async_work_group_copy} 操作完成的事件。
實施等待後會釋放 \carg{event_list} 中的\cnglo{evtobj}。
對於某個\cnglo{workgrp}中的\cnglo{workitem}而言,
如果執行\cnglo{kernel}時
用的 \carg{num_events} 以及 \carg{event_list} 中的\cnglo{evtobj}一樣,
則它們必須都能執行到此函式;
否則結果未定義。
\stopbuffer

% prefetch
\startbuffer[funcproto:prefetch]
void prefetch (
	const __global gentype *p,
	size_t num_gentypes)
\stopbuffer
\startbuffer[funcdesc:prefetch]
預取 \math{\text{\carg{num_gentypes}}
 \times \text{\capi{sizeof}}(\text{\ctype{gentype}})} 字節到全局緩存中。
預取指令會作用到\cnglo{workgrp}中的\cnglo{workitem}上,
不會影響\cnglo{kernel}的功能性行為。
\stopbuffer


% begin table
\startCLFD
\clFD{async_work_group_copy}
\clFD{async_work_group_strided_copy}
\clFD{wait_group_events}
\clFD{prefetch}
\stopCLFD
}

