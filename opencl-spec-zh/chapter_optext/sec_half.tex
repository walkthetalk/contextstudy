\section{半精度浮點數}

此擴展增加了對 \cldt{half} 標量和矢量型別的支持,
可以 \cldt{half} 作為內建型別進行算術運算、轉換等。
\cnglo{app}要想使用型別 \cldt{half} 和 \cldt[n]{half},
必須包含編譯指示 \cemp{#pragma OPENCL EXTENSION cl_khr_fp16 : enable}。

\reftab{builtInScalarDataTypes}和\reftab{builtInVectorDataTypes}中所列內建標量、矢量數據型別又做了如下擴充:

\placetable[here][tab:half_type_dsc]
{\cldt{half} 相關數據型別}
{\startCLOD[型別][描述]

\clOD{\cldt{half2}}{2 組件半精度浮點矢量。}

\clOD{\cldt{half3}}{3 組件半精度浮點矢量。}

\clOD{\cldt{half4}}{4 組件半精度浮點矢量。}

\clOD{\cldt{half8}}{8 組件半精度浮點矢量。}

\clOD{\cldt{half16}}{16 組件半精度浮點矢量。}

\stopCLOD
}

在 OpenCL API(以及頭檔)中,內建矢量數據型別 \cldt[n]{half} 被聲明為其他型別,
以更好的為\cnglo{app}所用。
\reftab{bihalf2appdt}中列出了 OpenCL C 編程語言中
所定義的內建矢量數據型別 \cldt[n]{half} 與\cnglo{app}所用型別間的對應關係。

\placetable[here][tab:bihalf2appdt]
{內建矢量數據型別與應用程式所用型別的對應關係}
{\startCLOO[OpenCL 語言中的型別][\cnglo{app}所用 API 中的型別]

\clOO{\cldt{half2}}{\cldt{cl_half2}}
\clOO{\cldt{half3}}{\cldt{cl_half3}}
\clOO{\cldt{half4}}{\cldt{cl_half4}}
\clOO{\cldt{half8}}{\cldt{cl_half8}}
\clOO{\cldt{half16}}{\cldt{cl_half16}}

\stopCLOO


}

\refsec{operator}中所描述的關係、相等、邏輯以及邏輯單元算子
均可用於 \cldt{half} 標量和 \cldt[n]{half} 矢量型別,
所產生的結果分別為標量 \cldt{int} 和矢量 \cldt[n]{short}。

可以為浮點常值添加後綴 \ccmm{h} 或 \ccmm{H},
以表明此常值型別為 \cldt{half}。

\subsection{轉換}

現在,\refsec{implicityConversion}中的隱式轉換規則也適用於 \cldt{half} 標量和 \cldt[n]{half} 矢量數據型別。

\refsec{explicitCast}中的顯式轉型也做了擴充,
適用於 \cldt{half} 標量數據型別和 \cldt[n]{half} 矢量數據型別。

\refsec{explicitConversion}中所描述的顯式轉換函式也做了擴充,
適用於 \cldt{half} 標量數據型別和 \cldt[n]{half} 矢量數據型別。

\refsec{as_typen}中所描述的用於重釋型別的函式 \clapi[n]{as_type} 也做了擴充,
允許在 \cldt[n]{short}、 \cldt[n]{ushort} 和 \cldt[n]{half} 標量、矢量數據型別間進行無需轉換的轉型。

\subsection{數學函式}

對\reftab{svMathFunc}中所列內建數學函式作了擴充,
函式引數和返回值也可以是 \cldt{half} 和 \cldt[n]{half},
參見\reftab{svMathFuncHalf}。
現在, \cldt{gentype} 也包含 \cldt{half} 和 \cldt[n]{half},
其中 \ccmmsuffix{n} 可以是 2、 3、 4、 8、 16。

對於函式的任一特定用法,所有引數以及返回值的實際型別必須相同。

\placetable[here,split][tab:svMathFuncHalf]
{標量和矢量引數內建數學函式表}
{\startCLFD

\clFD{acos}
\clFD{acosh}
\clFD{acospi}
\clFD{asin}
\clFD{asinh}
\clFD{asinpi}
\clFD{atan}
\clFD{atan2}
\clFD{atanh}
\clFD{atanpi}
\clFD{atan2pi}
\clFD{cbrt}
\clFD{ceil}
\clFD{copysign}
\clFD{cos}
\clFD{cosh}
\clFD{cospi}
\clFD{erfc}
\clFD{erf}
\clFD{exp}
\clFD{exp2}
\clFD{exp10}
\clFD{expm1}
\clFD{fabs}
\clFD{fdim}
\clFD{floor}
\clFD{fma}
\clFD{fmaxh}
\clFD{fminh}
\clFD{fmod}
\clFD{fract}
\clFD{frexph}
\clFD{hypot}
\clFD{ilogbh}
\clFD{ldexph}
\clFD{lgammah}
\clFD{log}
\clFD{log2}
\clFD{log10}
\clFD{log1p}
\clFD{logb}
\clFD{mad}
\clFD{maxmag}
\clFD{minmag}
\clFD{modf}
\clFD{nanh}
\clFD{nextafter}
\clFD{pow}
\clFD{pownh}
\clFD{powr}
\clFD{remainder}
\clFD{remquoh}
\clFD{rint}
\clFD{rootnh}
\clFD{rsqrt}
\clFD{sin}
\clFD{sincos}
\clFD{sinh}
\clFD{sinpi}
\clFD{sqrt}
\clFD{tan}
\clFD{tanh}
\clFD{tanpi}
\clFD{tgamma}
\clFD{trunc}

\stopCLFD
}

巨集 \cmacroemp{FP_FAST_FMAF} 用來指明對於半精度浮點數,
 \capi{fma} 函式族是否比直接編碼更快。
如果定義了此巨集,則表明對算元為 \ctype{float} 的乘、加運算,
函式 \capi{fma} 一般跟直接編碼一樣快,或者更快。

下列巨集必須使用指定的值。
可以在預處理指示 \ccmm{#if} 中使用這些常量算式。
\startclc
#define HALF_DIG		3
#define HALF_MANT_DIG		11
#define HALF_MAX_10_EXP		+4
#define HALF_MAX_EXP		+16
#define HALF_MIN_10_EXP		-4
#define HALF_MIN_EXP		-13
#define HALF_RADIX		2
#define HALF_MAX		0x1.ffcp15h
#define HALF_MIN		0x1.0p-14h
#define HALF_EPSILON		0x1.0p-10f
\stopclc

\reftab{tblHalfMacroAndApp}中給出了上面所列巨集與\cnglo{app}所用的巨集名字之間的對應關係。

\placetable[here][tab:tblHalfMacroAndApp]
{半精度浮點巨集與應用程式所用巨集的對應關係}
{\startCLOO[OpenCL 語言中的巨集][\cnglo{app}所用的巨集]

\clMMH{DIG}
\clMMH{MANT_DIG}
\clMMH{MAX_10_EXP}
\clMMH{MAX_EXP}
\clMMH{MIN_10_EXP}
\clMMH{MIN_EXP}
\clMMH{RADIX}
\clMMH{MAX}
\clMMH{MIN}
\clMMH{EPSILSON}

\stopCLOO
}

除此之外,還有一些常量可用,如\reftab{tblHalfMacroConst}所示。
他們的型別都是 \ctype{float},在 \ctype{float} 型別的精度內是準確的。

\placetable[here][tab:tblHalfMacroConst]
{半精度浮點常量}
{\startCLOO[常量][描述]

\clCM{M_E_H}{e}
\clCM{M_LOG2E_H}{log_{2}e}
\clCM{M_LOG10E_H}{log_{10}e}
\clCM{M_LN2_H}{log_{e}2}
\clCM{M_LN10_H}{log_{e}10}
\clCM{M_PI_H}{\pi}
\clCM{M_PI_2_H}{\pi/2}
\clCM{M_PI_4_H}{\pi/4}
\clCM{M_1_PI_H}{1/\pi}
\clCM{M_2_PI_H}{2/\pi}
\clCM{M_2_SQRTPI_H}{2/\sqrt{\pi}}
\clCM{M_SQRT2_H}{\sqrt{2}}
\clCM{M_SQRT1_2_H}{1/\sqrt{2}}

\stopCLOO
}

\subsection{公共函式}

對\reftab{svCommonFunc}中所列的內建公共函式作了擴充,
函式引數和返回值也可以是 \cldt{half} 和 \cldt[n]{half},
參見\reftab{svCommonFuncHalf}。
現在, \cldt{gentype} 也包含 \cldt{half} 和 \cldt[n]{half},
其中 \ccmmsuffix{n} 可以是 2、 3、 4、 8、 16。

\startnotepar
可以使用化簡(如 \capi{mad} 或 \capi{fma})來實現 \capi{mix} 和 \capi{smoothstep}。
\stopnotepar

\placetable[here][tab:svCommonFuncHalf]
{內建公共函式}
{\startCLFD

\clFD{clamp_half}
\clFD{degrees}
\clFD{max_half}
\clFD{min_half}
\clFD{mix_half}
\clFD{radians}
\clFD{step_half}
\clFD{smoothstep_half}
\clFD{sign}

\stopCLFD
}

