\section{半精度浮點數}

此擴展增加了對 \cldt{half} 標量和矢量型別的支持,
可以 \cldt{half} 作為內建型別進行算術運算、轉換等。
\cnglo{app}要想使用型別 \cldt{half} 和 \cldt[n]{half},
必須包含編譯指示 \cemp{#pragma OPENCL EXTENSION cl_khr_fp16 : enable}。

\reftab{builtInScalarDataTypes}和\reftab{builtInVectorDataTypes}中所列內建標量、矢量數據型別又做了如下擴充:

\placetable[here][tab:half_type_dsc]
{\cldt{half} 相關數據型別}
{\startCLOD[型別][描述]

\clOD{\cldt{half2}}{2 組件半精度浮點矢量。}

\clOD{\cldt{half3}}{3 組件半精度浮點矢量。}

\clOD{\cldt{half4}}{4 組件半精度浮點矢量。}

\clOD{\cldt{half8}}{8 組件半精度浮點矢量。}

\clOD{\cldt{half16}}{16 組件半精度浮點矢量。}

\stopCLOD
}

在 OpenCL API(以及頭檔)中,內建矢量數據型別 \cldt[n]{half} 被聲明為其他型別,
以更好的為\cnglo{app}所用。
\reftab{bihalf2appdt}中列出了 OpenCL C 編程語言中
所定義的內建矢量數據型別 \cldt[n]{half} 與\cnglo{app}所用型別間的對應關係。

\placetable[here][tab:bihalf2appdt]
{內建矢量數據型別與應用程式所用型別的對應關係}
{\startCLOO[OpenCL 語言中的型別][\cnglo{app}所用 API 中的型別]

\clOO{\cldt{half2}}{\cldt{cl_half2}}
\clOO{\cldt{half3}}{\cldt{cl_half3}}
\clOO{\cldt{half4}}{\cldt{cl_half4}}
\clOO{\cldt{half8}}{\cldt{cl_half8}}
\clOO{\cldt{half16}}{\cldt{cl_half16}}

\stopCLOO


}

\refsec{operator}中所描述的關係、相等、邏輯以及邏輯單元算子
均可用於 \cldt{half} 標量和 \cldt[n]{half} 矢量型別,
所產生的結果分別為標量 \cldt{int} 和矢量 \cldt[n]{short}。

OpenCL 編譯器可接受為浮點常值添加後綴 \ccmm{h} 或 \ccmm{H},
表明此常值型別為 \cldt{half}。

\subsection{轉換}

現在,\refsec{implicityConversion}中的隱式轉換規則也適用於 \cldt{half} 標量和 \cldt[n]{half} 矢量數據型別。

\refsec{explicitCast}中的顯式轉型也做了擴充,
適用於 \cldt{half} 標量數據型別和 \cldt[n]{half} 矢量數據型別。

\refsec{explicitConversion}中所描述的顯式轉換函式也做了擴充,
適用於 \cldt{half} 標量數據型別和 \cldt[n]{half} 矢量數據型別。

\refsec{as_typen}中所描述的用於重釋型別的函式 \clapi[n]{as_type} 也做了擴充,
允許在 \cldt[n]{short}、 \cldt[n]{ushort} 和 \cldt[n]{half} 標量、矢量數據型別間進行無需轉換的轉型。

