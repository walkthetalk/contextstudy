% Edge Case Behavior
\section[sec:edgeCaseBehavior]{邊界條件下的行為}

數學函式(\refsec{mathFunc})在邊界條件下的行為
要符合 ISO/IEC 9899:TC2 (通常稱作 C99, TC2)中的節 F.9 和節 G.6,
\refsec{addReqBeyondC99}中所列之處除外。

% Additional Requirements Beyond C99 TC2
\subsection[sec:addReqBeyondC99]{C99 TC2 之外的附加要求}

如果函式會返回 NaN,且有多個算元為 NaN,則應當返回其中一個 NaN 算元。
如果是 sNaN,會返回 NaN 算元的函式可能會使其消音。
非 sNaN 轉換後應當還是非 sNaN。
sNaN 轉換後應當是 NaN,但可能被轉換成非 sNaN。
而 NaN 的凈荷位和正負號如何轉換則未定義。

函式 \capi{half_<funcname>} 所起作用與不帶前綴 \capi{half_} 的同名函式一樣。
他們必須符合同一邊界條件需求(參見C99, TC2 中的節 F.9 和節 G.6)。
對於其他情況,除了明確提到的地方,准許單精度函式的最大誤差為 8192 ulp
(以單精度結果來衡量),儘管鼓勵更高的精確度。

對於那些明確規定了其結果的值(參見 C99, TC2 中的節 F.9,
或者\refsec{addReqBeyondC99}以及後面的\refsec{ftzmECB})
(如: \ccmm{ceil(-1 < x < 0)} 會返回 \ccmm{-0}),
捨入誤差(\refsec{relativeError})或刷新行為(\refsec{ftzmECB})的容差不會起作用。
那些值肯定產生規定的答案,不會是其他的。
使用 \math{\pm} 的地方,正負號會保留下來。
例如, \math{sin(\pm{0}) = \pm{0}} 的意思就是:
\math{sin(+0) = +0} 以及 \math{sin(-0) = -0}。

% func description

acospi ( 1 ) = +0.\par
acospi ( x ) returns a NaN for | x | > 1.\par

asinpi ( ±0 ) = ±0.\par
asinpi ( x ) returns a NaN for | x | > 1.\par

atanpi ( ±0 ) = ±0.\par
atanpi ( ±∞ ) = ±0.5.\par

atan2pi ( ±0, -0 ) = ±1.\par
atan2pi ( ±0, +0 ) = ± 0.\par
atan2pi ( ±0, x ) returns ± 1 for x < 0.\par
atan2pi ( ±0, x ) returns ± 0 for x > 0.\par
atan2pi ( y, ±0 ) returns -0.5 for y < 0.\par
atan2pi ( y, ±0 ) returns 0.5 for y > 0.\par
atan2pi ( ±y, -∞ ) returns ± 1 for finite y > 0.\par
atan2pi ( ±y, +∞ ) returns ± 0 for finite y > 0.\par
atan2pi ( ±∞, x ) returns ± 0.5 for finite x.\par
atan2pi ( ±∞, -∞ ) returns ±0.75.\par
atan2pi ( ±∞, +∞ ) returns ±0.25.\par

ceil ( -1 < x < 0 ) returns -0.\par

cospi ( ±0 ) returns 1\par
cospi ( n + 0.5 ) is +0 for any integer n where n + 0.5 is representable.\par
cospi ( ±∞ ) returns a NaN.\par

exp10 ( ±0 ) returns 1.\par
exp10 ( -∞ ) returns +0.\par
exp10 ( +∞ ) returns +∞.\par

distance (x, y) calculates the distance from x to y without overflow or extraordinary
precision loss due to underflow.\par

fdim ( any, NaN ) returns NaN.\par
fdim ( NaN, any ) returns NaN.\par

fmod ( ±0, NaN ) returns NaN.\par

frexp ( ±∞, exp ) returns ±∞ and stores 0 in exp.\par
frexp ( NaN, exp ) returns the NaN and stores 0 in exp.\par

fract ( x, iptr) shall not return a value greater than or equal to 1.0,
and shall not return a value less than 0.\par

fract ( +0, iptr ) returns +0 and +0 in iptr.\par
fract ( -0, iptr ) returns -0 and -0 in iptr.\par
fract ( +inf, iptr ) returns +0 and +inf in iptr.\par
fract ( -inf, iptr ) returns -0 and -inf in iptr.\par
fract ( NaN, iptr ) returns the NaN and NaN in iptr.\par

length calculates the length of a vector without overflow or extraordinary precision loss
due to underflow.\par

lgamma_r (x, signp) returns 0 in signp if x is zero or a negative integer.\par

nextafter ( -0, y > 0 ) returns smallest positive denormal value.\par
nextafter ( +0, y < 0 ) returns smallest negative denormal value.\par

normalize shall reduce the vector to unit length, pointing in the same direction without
overflow or extraordinary precision loss due to underflow.\par
normalize ( v ) returns v if all elements of v are zero.\par
normalize ( v ) returns a vector full of NaNs if any element is a NaN.\par
normalize ( v ) for which any element in v is infinite shall proceed as if the elements in v
were replaced as follows:\par
\startclc
for( i = 0; i < sizeof(v) / sizeof(v[0] ); i++ )
	v[i] = isinf(v[i] ) ? copysign(1.0, v[i]) : 0.0 * v [i];
\stopclc

pow ( ±0, -∞ ) returns +∞\par

pown ( x, 0 ) is 1 for any x, even zero, NaN or infinity.\par
pown ( ±0, n ) is ±∞ for odd n < 0.\par
pown ( ±0, n ) is +∞ for even n < 0.\par
pown ( ±0, n ) is +0 for even n > 0.\par
pown ( ±0, n ) is ±0 for odd n > 0.\par

powr ( x, ±0 ) is 1 for finite x > 0.\par
powr ( ±0, y ) is +∞ for finite y < 0.\par
powr ( ±0, -∞) is +∞.\par
powr ( ±0, y ) is +0 for y > 0.\par
powr ( +1, y ) is 1 for finite y.\par
powr ( x, y ) returns NaN for x < 0.\par
powr ( ±0, ±0 ) returns NaN.\par
powr ( +∞, ±0 ) returns NaN.\par
powr ( +1, ±∞ ) returns NaN.\par
powr ( x, NaN ) returns the NaN for x >= 0.\par
powr ( NaN, y ) returns the NaN.\par

rint ( -0.5 <= x < 0 ) returns -0.\par

remquo (x, y, &quo) returns a NaN and 0 in quo if x is ±∞,
or if y is 0 and the other argument is non-NaN or if either argument is a NaN.\par

rootn ( ±0, n ) is ±∞ for odd n < 0.\par
rootn ( ±0, n ) is +∞ for even n < 0.\par
rootn ( ±0, n ) is +0 for even n > 0.\par
rootn ( ±0, n ) is ±0 for odd n > 0.\par
rootn ( x, n ) returns a NaN for x < 0 and n is even.\par
rootn ( x, 0 ) returns a NaN.\par

round ( -0.5 < x < 0 ) returns -0.\par

sinpi ( ±0 ) returns ±0.\par
sinpi ( +n) returns +0 for positive integers n.\par
sinpi ( -n ) returns -0 for negative integers n.\par
sinpi ( ±∞ ) returns a NaN.\par

tanpi ( ±0 ) returns ±0.\par
tanpi ( ±∞ ) returns a NaN.\par
tanpi ( n ) is copysign( 0.0, n) for even integers n.\par
tanpi ( n ) is copysign( 0.0, - n) for odd integers n.\par
tanpi ( n + 0.5 ) for even integer n is +∞ where n + 0.5 is representable.\par
tanpi ( n + 0.5 ) for odd integer n is -∞ where n + 0.5 is representable.\par

trunc ( -1 < x < 0 ) returns -0.\par



% Changes to C99 TC2 Behavior
\subsection{對 C99 TC2 的行為做出的改變}

\capi{modf} 的作用如同如下實作:
\startclc[indentnext=no]
gentype modf ( gentype value, gentype *iptr )
{
	*iptr = trunc( value );
	return copysign( isinf( value ) ? 0.0 : value – *iptr, value );
}
\stopclc
\capi{rint} 始終捨入為最近偶數,即使其調用者處於其他捨入模式下。

% Edge Case Behavior in Flush To Zero Mode
\subsection[sec:ftzmECB]{Flush-To-Zero 模式中,邊界條件下的行為}

如果去規格化數被刷成了零,則函式可能返回下列結果之一:
\startigNum
\item[item:nftz]任何符合 non-flush-to-zero 模式的結果;

\item 如果捨入前,\refitem{nftz}所給出的結果是次規格化數,則可能會將其刷成零;

\item[item:nfcr] 如果有一個或多個次規格化算元被刷成了零,則結果為符合此函式的任意未刷新的值。

\item 如果捨入前,\refitem{nfcr}的結果是次規格化數,則可能將其刷成零。
\stopigNum

在上述任一情況中,如果某個算元或結果被刷成了零,零的正負號未定義。

如果次規格化數被刷成了零,則\cnglo{device}可能選擇使用下列方式來處理 \capi{nextafter}
的邊界情況,以取代\refsec{addReqBeyondC99}中的那些:

nextafter ( +smallest normal, y < +smallest normal ) = +0.\par
nextafter ( -smallest normal, y > -smallest normal ) = -0.\par
nextafter ( -0, y > 0 ) returns smallest positive normal value.\par
nextafter ( +0, y < 0 ) returns smallest negative normal value.\par

清晰起見,將次規格化數或去規格化數定義為位於區間 \ccmm{0 < x < TYPE_MIN}
和 \ccmm{-TYPE_MIN < x < -0} 內的一組可表示的數。
他們不包括 \math{\pm{0}}。
如果在捨入前,一個非零數規格化後,其以 2 為底的指數小於 \ccmm{(TYPE_MIN_EXP - 1)},
就說他是次規格化數。\footnote{%
此處要用相應浮點型別常量來替換 \cenum{TYPE_MIN} 和 \cenum{TYPE_MIN_EXP},
如對於 \ctype{float} 這兩個常量就應該是 \cenum{FLT_MIN} 和 \cenum{FLT_MIN_EXP}。}