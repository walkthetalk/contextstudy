% Relative Error as ULPs
\section[sec:relativeError]{相對誤差即 ULP}

本節中,我們將討論相對誤差(定義為 \ccmm{ulp},即 units in the last place,
浮點數間的最小間隔)的最大值。
整數和單精度浮點數間的加、減、乘、積和熔加以及轉換\problem{沒有除?}都符合 IEEE 754,
因此可以正確捨入。
浮點格式間的轉換以及\refsec{explicitConversion}中的顯式轉換都必須正確捨入。

ULP 的定義如下:

{\ftRef%
如果 x 是位於兩個有限連續浮點數 a 和 b 之間的實數,
並且與 a 和 b 都不相等,則:
\startformula[indentnext=no]
ulp(x) = |b - a|
\stopformula
。否則 \math{ulp(x)} 是這兩個離 x 最近且互不相等的有限浮點數間的距離。
此外, \math{ulp(NaN)} 就是 NaN。}

\useurl[rr5504][ftp://ftp.inria.fr/INRIA/publication/publi-pdf/RR/RR-5504.pdf]
{\ftRef%
歸因:此定義獲得了 Jean-Michel Muller 的認可,不過他對在零處的行為做了一點澄清。
請參考 \from[rr5504]。}

\reftab{spMathUlp}\footnote{%
內建數學函式 \capi{lgamma} 和 \capi{lgamma_r} 的 ULP 值目前還未定義。}%
中以 ULP 的形式給出了單精度浮點算術運算的最小精確度。
計算 ULP 值時參考的是無限精確的結果。
其中 0 ulp 表示相應函式無需捨入。

\placetable[here,split][tab:spMathUlp]
{單精度內建數學函式的 ULP 值}
{\startCLFA[函式][最小精度—— ULP 值]

\clFAM{x+y}{正確捨入}
\clFAM{x-y}{正確捨入}
\clFAM{x*y}{正確捨入}
\clFAM{1.0/y}{正確捨入}
\clFAM{x/y}{正確捨入}

\clFAA{acos}{<= 4 ulp}
\clFAA{acospi}{<= 5 ulp}
\clFAA{asin}{<= 4 ulp}
\clFAA{asinpi}{<= 5 ulp}
\clFAA{atan}{<= 5 ulp}
\clFAA{atan2}{<= 6 ulp}
\clFAA{atanpi}{<= 5 ulp}
\clFAA{atan2pi}{<= 6 ulp}
\clFAA{acosh}{<= 4 ulp}
\clFAA{asinh}{<= 4 ulp}
\clFAA{atanh}{<= 5 ulp}
\clFAA{cbrt}{<= 2 ulp}
\clFAA{ceil}{正確捨入}
\clFAA{copysign}{0 ulp}
\clFAA{cos}{<= 4 ulp}
\clFAA{cosh}{<= 4 ulp}
\clFAA{cospi}{<= 4 ulp}
\clFAA{erfc}{<= 16 ulp}
\clFAA{erf}{<= 16 ulp}
\clFAA{exp}{<= 3 ulp}
\clFAA{exp2}{<= 3 ulp}
\clFAA{exp10}{<= 3 ulp}
\clFAA{expm1}{<= 3 ulp}
\clFAA{fabs}{0 ulp}
\clFAA{fdim}{正確捨入}
\clFAA{floor}{正確捨入}
\clFAA{fma}{正確捨入}
\clFAA{fmax}{0 ulp}
\clFAA{fmin}{0 ulp}
\clFAA{fmod}{0 ulp}
\clFAA{fract}{正確捨入}
\clFAA{frexp}{0 ulp}
\clFAA{hypot}{<= 4 ulp}
\clFAA{ilogb}{0 ulp}
\clFAA{ldexp}{正確捨入}
\clFAA{log}{<= 3 ulp}
\clFAA{log2}{<= 3 ulp}
\clFAA{log10}{<= 3 ulp}
\clFAA{log1p}{<= 2 ulp}
\clFAA{logb}{0 ulp}
\clFAA{mad}{所允許的任何值(無窮 ulp)}
\clFAA{maxmag}{0 ulp}
\clFAA{minmag}{0 ulp}
\clFAA{modf}{0 ulp}
\clFAA{nan}{0 ulp}
\clFAA{nextafter}{0 ulp}
\clFAA{pow(x, y)}{<= 16 ulp}
\clFAA{pown(x, y)}{<= 16 ulp}
\clFAA{powr(x, y)}{<= 16 ulp}
\clFAA{remainder}{0 ulp}
\clFAA{remquo}{0 ulp}
\clFAA{rint}{正確捨入}
\clFAA{rootn}{<= 16 ulp}
\clFAA{round}{正確捨入}
\clFAA{rsqrt}{<= 2 ulp}
\clFAA{sin}{<= 4 ulp}
\clFAA{sincos}{正弦值和餘弦值都是 <= 4 ulp}
\clFAA{sinh}{<= 4 ulp}
\clFAA{sinpi}{<= 4 ulp}
\clFAA{sqrt}{正確捨入}
\clFAA{tan}{<= 5 ulp}
\clFAA{tanh}{<= 5 ulp}
\clFAA{tanpi}{<= 6 ulp}
\clFAA{tgamma}{<= 16 ulp}
\clFAA{trunc}{正確捨入}

\clFAA{half_cos}{<= 8192 ulp}
\clFAA{half_divide}{<= 8192 ulp}
\clFAA{half_exp}{<= 8192 ulp}
\clFAA{half_exp2}{<= 8192 ulp}
\clFAA{half_exp10}{<= 8192 ulp}
\clFAA{half_log}{<= 8192 ulp}
\clFAA{half_log2}{<= 8192 ulp}
\clFAA{half_log10}{<= 8192 ulp}
\clFAA{half_powr}{<= 8192 ulp}
\clFAA{half_recip}{<= 8192 ulp}
\clFAA{half_rsqrt}{<= 8192 ulp}
\clFAA{half_sin}{<= 8192 ulp}
\clFAA{half_sqrt}{<= 8192 ulp}
\clFAA{half_tan}{<= 8192 ulp}

\clFAA{native_cos}{\cnglo{impdef}}
\clFAA{native_divide}{\cnglo{impdef}}
\clFAA{native_exp}{\cnglo{impdef}}
\clFAA{native_exp2}{\cnglo{impdef}}
\clFAA{native_exp10}{\cnglo{impdef}}
\clFAA{native_log}{\cnglo{impdef}}
\clFAA{native_log2}{\cnglo{impdef}}
\clFAA{native_log10}{\cnglo{impdef}}
\clFAA{native_powr}{\cnglo{impdef}}
\clFAA{native_recip}{\cnglo{impdef}}
\clFAA{native_rsqrt}{\cnglo{impdef}}
\clFAA{native_sin}{\cnglo{impdef}}
\clFAA{native_sqrt}{\cnglo{impdef}}
\clFAA{native_tan}{\cnglo{impdef}}

\stopCLFA
}

\reftab{dpMathUlp}中以 ULP 的形式給出了雙精度浮點算術運算的最小精確度。
計算 ULP 值時參考的是無限精確的結果。
其中 0 ulp 表示相應函式無需捨入。

\placetable[here,split][tab:dpMathUlp]
{雙精度內建數學函式的 ULP 值}
{\startCLFA[函式][最小精度—— ULP 值]

\clFAM{x+y}{正確捨入}
\clFAM{x-y}{正確捨入}
\clFAM{x*y}{正確捨入}
\clFAM{1.0/y}{正確捨入}
\clFAM{x/y}{正確捨入}

\clFAA{acos}{<= 4 ulp}
\clFAA{acospi}{<= 5 ulp}
\clFAA{asin}{<= 4 ulp}
\clFAA{asinpi}{<= 5 ulp}
\clFAA{atan}{<= 5 ulp}
\clFAA{atan2}{<= 6 ulp}
\clFAA{atanpi}{<= 5 ulp}
\clFAA{atan2pi}{<= 6 ulp}
\clFAA{acosh}{<= 4 ulp}
\clFAA{asinh}{<= 4 ulp}
\clFAA{atanh}{<= 5 ulp}
\clFAA{cbrt}{<= 2 ulp}
\clFAA{ceil}{正確捨入}
\clFAA{copysign}{0 ulp}
\clFAA{cos}{<= 4 ulp}
\clFAA{cosh}{<= 4 ulp}
\clFAA{cospi}{<= 4 ulp}
\clFAA{erfc}{<= 16 ulp}
\clFAA{erf}{<= 16 ulp}
\clFAA{exp}{<= 3 ulp}
\clFAA{exp2}{<= 3 ulp}
\clFAA{exp10}{<= 3 ulp}
\clFAA{expm1}{<= 3 ulp}
\clFAA{fabs}{0 ulp}
\clFAA{fdim}{正確捨入}
\clFAA{floor}{正確捨入}
\clFAA{fma}{正確捨入}
\clFAA{fmax}{0 ulp}
\clFAA{fmin}{0 ulp}
\clFAA{fmod}{0 ulp}
\clFAA{fract}{正確捨入}
\clFAA{frexp}{0 ulp}
\clFAA{hypot}{<= 4 ulp}
\clFAA{ilogb}{0 ulp}
\clFAA{ldexp}{正確捨入}
\clFAA{log}{<= 3 ulp}
\clFAA{log2}{<= 3 ulp}
\clFAA{log10}{<= 3 ulp}
\clFAA{log1p}{<= 2 ulp}
\clFAA{logb}{0 ulp}
\clFAA{mad}{所允許的任何值(無窮 ulp)}
\clFAA{maxmag}{0 ulp}
\clFAA{minmag}{0 ulp}
\clFAA{modf}{0 ulp}
\clFAA{nan}{0 ulp}
\clFAA{nextafter}{0 ulp}
\clFAA{pow(x, y)}{<= 16 ulp}
\clFAA{pown(x, y)}{<= 16 ulp}
\clFAA{powr(x, y)}{<= 16 ulp}
\clFAA{remainder}{0 ulp}
\clFAA{remquo}{0 ulp}
\clFAA{rint}{正確捨入}
\clFAA{rootn}{<= 16 ulp}
\clFAA{round}{正確捨入}
\clFAA{rsqrt}{<= 2 ulp}
\clFAA{sin}{<= 4 ulp}
\clFAA{sincos}{正弦值和餘弦值都是 <= 4 ulp}
\clFAA{sinh}{<= 4 ulp}
\clFAA{sinpi}{<= 4 ulp}
\clFAA{sqrt}{正確捨入}
\clFAA{tan}{<= 5 ulp}
\clFAA{tanh}{<= 5 ulp}
\clFAA{tanpi}{<= 6 ulp}
\clFAA{tgamma}{<= 16 ulp}
\clFAA{trunc}{正確捨入}

\stopCLFA
}
