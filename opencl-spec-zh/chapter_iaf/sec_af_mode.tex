% Addressing and Filter Modes
\section{尋址模式和濾波模式}

我們先來描述尋址模式既不是 \cenum{CLK_ADDRESS_REPEAT}
也不是 \cenum{CLK_ADDRESS_MIRRORED_REPEAT} 的情況下,
怎樣利用尋址模式和濾波模式來生成恰當的採樣位置以讀取圖像。

生成圖像坐標 \math{(u, v, w)} 後,
我們會使用恰當的尋址模式和濾波模式來生成恰當的採樣區以讀取圖像。

如果 \math{(u, v, w)} 中的值有 INF 或 NaN,
則 \capi{read_image{f|i|ui}} 的行為未定義。

% nearest
{\ftEmp{Filter Mode = CLK_FILTER_NEAREST}}

如果濾波模式為 \cenum{CLK_FILTER_NEAREST},
則會得到離 \math{(u, v, w)} 最近(Manhattan 距離)的圖像元素。
即返回坐標為 \math{(i, j, k)} 的元素,其中
\startclc[indentnext=no]
i = address_mode((int)floor(u))
j = address_mode((int)floor(v))
k = address_mode((int)floor(w))
\stopclc
對於 3D 圖像, \math{(i,j,k)} 處的圖像元素即為所求。
而對於 2D 圖像, \math{(i,j)} 處的圖像元素即為所求。

\reftab{address_mode}中描述了函式 \capi{address_mode}。

\placetable[here,force][tab:address_mode]
{用來生成紋理位置的尋址模式}
{\startCLOD[尋址模式][address_mode(coord) 的結果]

\clOD{\cenum{CLK_ADDRESS_CLAMP_TO_EDGE}}{\ccmm{clamp (coord, 0, size – 1)}}
\clOD{\cenum{CLK_ADDRESS_CLAMP}}{\ccmm{clamp (coord, -1, size)}}
\clOD{\cenum{CLK_ADDRESS_NONE}}{\ccmm{coord}}

\stopCLOD
}

對於 \math{u}、 \math{v} 和 \math{w} 而言,
\reftab{address_mode}中的 \ccmm{size} 分別為 \math{w_t}、 \math{h_t} 和 \math{d_t}。

\reftab{address_mode}中的 \ccmm{clamp} 定義為:
\startclc
clamp(a, b, c) = return (a < b) ? b : ((a > c) ? c : a)
\stopclc

如果紋理位置 \math{(i, j, k)} 落到了圖像的外面,則用顏色極值作為此紋理的顏色。

% linear
{\ftEmp{Filter Mode = CLK_FILTER_LINEAR}}

如果濾波模式為 \cenum{CLK_FILTER_LINEAR},
則對於 2D 圖像會選擇一個 \math{2\times 2} 的方陣中的圖像元素,
而對於 3D 圖像,則會選擇一個 \math{2\times 2\times 2} 的立方體中的圖像元素。
得到的 \math{2\times 2} 方陣或 \math{2\times 2\times 2} 立方體如下所示。

設
\startclc[indentnext=no]
i0 = address_mode((int)floor(u - 0.5))
j0 = address_mode((int)floor(v - 0.5))
k0 = address_mode((int)floor(w - 0.5))
i1 = address_mode((int)floor(u - 0.5) + 1)
j1 = address_mode((int)floor(v - 0.5) + 1)
k1 = address_mode((int)floor(w - 0.5) + 1)
a  = frac(u – 0.5)
b  = frac(v – 0.5)
c  = frac(w – 0.5)
\stopclc
其中 \ccmm{frac(x)} 為 \ccmm{x} 的小數部分,相當於 \ccmm{x - floor(x)}。

對於 3D 圖像,用如下方式得到圖像元素:
\startclc[indentnext=no]
T = (1 – a) * (1 – b) * (1 – c) * T/BTEX\low{i0j0k0}/ETEX
    + a * (1 – b) * (1 – c) * T/BTEX\low{i1j0k0}/ETEX
    + (1 – a) * b * (1 – c) * T/BTEX\low{i0j1k0}/ETEX
    + a * b * (1 – c) * T/BTEX\low{i1j1k0}/ETEX
    + (1 – a) * (1 – b) * c * T/BTEX\low{i0j0k1}/ETEX
    + a * (1 – b) * c * T/BTEX\low{i1j0k1}/ETEX
    + (1 – a) * b * c * T/BTEX\low{i0j1k1}/ETEX
    + a * b * c * T/BTEX\low{i1j1k1}/ETEX
\stopclc
其中 \math{T_{ijk}} 就是此 3D 圖像中位置 \math{(i, j, k)} 處的元素。

對於 2D 圖像,用如下方式得到圖像元素:
\startclc[indentnext=no]
T = (1 – a) * (1 – b) * T/BTEX\low{i0j0}/ETEX
    + a * (1 – b) * T/BTEX\low{i1j0}/ETEX
    + (1 – a) * b * T/BTEX\low{i0j1}/ETEX
    + a * b * T/BTEX\low{i1j1}/ETEX
\stopclc
其中 \math{T_{ij}} 就是此 2D 圖像中位置 \math{(i, j)} 處的元素。

上面方程中,如果 \math{T_{ijk}} 或 \math{T_{ij}} 中任意一個所指代的位置落到了圖像外面,
則用顏色極值作為 \math{T_{ijk}} 或 \math{T_{ij}} 處的顏色。


現在我們來討論尋址模式是 \cenum{CLK_ADDRESS_REPEAT} 的情況下,
怎樣利用尋址模式和濾波模式來生成恰當的採樣位置以讀取圖像。

如果 \math{(s, t, r)} 中的值有 INF 或 NaN,
則內建圖像讀取函式的行為未定義。

% nearest
{\ftEmp{Filter Mode = CLK_FILTER_NEAREST}}

如果濾波模式為 \cenum{CLK_FILTER_NEAREST},
則使用位置 \math{(i, j, k)} 處的元素,
其中 \math{i}、 \math{j} 和 \math{k} 的計算方式如下:
\startclc[indentnext=no]
u = (s – floor(s)) * w/BTEX\low{t}/ETEX
i = (int)floor(u)
if (i > w/BTEX\low{t}/ETEX – 1)
	i = i – w/BTEX\low{t}/ETEX

v = (t – floor(t)) * h/BTEX\low{t}/ETEX
j = (int)floor(v)
if (j > h/BTEX\low{t}/ETEX – 1)
	j = j – h/BTEX\low{t}/ETEX

w = (r – floor(r)) * d/BTEX\low{t}/ETEX
k = (int)floor(w)
if (k > d/BTEX\low{t}/ETEX – 1)
	k = k - d/BTEX\low{t}/ETEX
\stopclc
對於 3D 圖像, \math{(i,j,k)} 處的圖像元素即為所求。
而對於 2D 圖像, \math{(i,j)} 處的圖像元素即為所求。

% linear
{\ftEmp{Filter Mode = CLK_FILTER_LINEAR}}

如果濾波模式為 \cenum{CLK_FILTER_LINEAR},
則對於 2D 圖像會選擇一個 \math{2\times 2} 的方陣中的圖像元素,
而對於 3D 圖像,則會選擇一個 \math{2\times 2\times 2} 的立方體中的圖像元素。
得到的 \math{2\times 2} 方陣或 \math{2\times 2\times 2} 立方體如下所示。

設
\startclc[indentnext=no]
u = (s – floor(s)) * w/BTEX\low{t}/ETEX
i0 = (int)floor(u – 0.5)
i1 = i0 + 1
if (i0 < 0)
	i0 = w/BTEX\low{t}/ETEX + i0
if (i1 > w/BTEX\low{t}/ETEX – 1)
	i1 = i1 – w/BTEX\low{t}/ETEX

v = (t – floor(t)) * h/BTEX\low{t}/ETEX
j0 = (int)floor(v – 0.5)
j1 = j0 + 1
if (j0 < 0)
	j0 = h/BTEX\low{t}/ETEX + j0
if (j1 > h/BTEX\low{t}/ETEX – 1)
	j1 = j1 – h/BTEX\low{t}/ETEX

w = (r – floor(r)) * d/BTEX\low{t}/ETEX
k0 = (int)floor(w – 0.5)
k1 = k0 + 1
if (k0 < 0)
	k0 = d/BTEX\low{t}/ETEX + k0
if (k1 > dt – 1)
	k1 = k1 – d/BTEX\low{t}/ETEX

a = frac(u – 0.5)
b = frac(v – 0.5)
c = frac(w – 0.5)
\stopclc
其中 \ccmm{frac(x)} 為 \ccmm{x} 的小數部分,相當於 \ccmm{x - floor(x)}。

對於 3D 圖像,用如下方式得到圖像元素:
\startclc[indentnext=no]
T = (1 – a) * (1 – b) * (1 – c) * T/BTEX\low{i0j0k0}/ETEX
    + a * (1 – b) * (1 – c) * T/BTEX\low{i1j0k0}/ETEX
    + (1 – a) * b * (1 – c) * T/BTEX\low{i0j1k0}/ETEX
    + a * b * (1 – c) * T/BTEX\low{i1j1k0}/ETEX
    + (1 – a) * (1 – b) * c * T/BTEX\low{i0j0k1}/ETEX
    + a * (1 – b) * c * T/BTEX\low{i1j0k1}/ETEX
    + (1 – a) * b * c * T/BTEX\low{i0j1k1}/ETEX
    + a * b * c * T/BTEX\low{i1j1k1}/ETEX
\stopclc
其中 \math{T_{ijk}} 就是此 3D 圖像中位置 \math{(i, j, k)} 處的元素。

對於 2D 圖像,用如下方式得到圖像元素:
\startclc[indentnext=no]
T = (1 – a) * (1 – b) * T/BTEX\low{i0j0}/ETEX
    + a * (1 – b) * T/BTEX\low{i1j0}/ETEX
    + (1 – a) * b * T/BTEX\low{i0j1}/ETEX
    + a * b * T/BTEX\low{i1j1}/ETEX
\stopclc
其中 \math{T_{ij}} 就是此 2D 圖像中位置 \math{(i, j)} 處的元素。


現在我們來討論尋址模式為 \cenum{CLK_ADDRESS_MIRRORED_REPEAT} 的情況下,
怎樣利用尋址模式和濾波模式來生成恰當的採樣位置以讀取圖像。
這種情況下讀取圖像時,就如同圖像數據在整數處會翻轉平鋪一樣。
例如, 2 和 3 之間的坐標 \math{(s, t, r)} 如同從 1 降到 0 的坐標一樣。
如果 \math{(s, t, r)} 中的值有 INF 或 NaN,則內建圖像讀取函式的行為未定義。

% nearest
{\ftEmp{Filter Mode = CLK_FILTER_NEAREST}}

如果濾波模式為 \cenum{CLK_FILTER_NEAREST},
則使用位置 \math{(i, j, k)} 處的元素,
其中 \math{i}、 \math{j} 和 \math{k} 的計算方式如下:
\startclc[indentnext=no]
s’ = 2.0f * rint(0.5f * s)
s’ = fabs(s – s’)
u = s’ * w/BTEX\low{t}/ETEX
i = (int)floor(u)
i = min(i, w/BTEX\low{t}/ETEX – 1)

t’ = 2.0f * rint(0.5f * t)
t’ = fabs(t – t’)
v = t’ * h/BTEX\low{t}/ETEX
j = (int)floor(v)
j = min(j, h/BTEX\low{t}/ETEX – 1)

r’ = 2.0f * rint(0.5f * r)
r’ = fabs(r – r’)
w = r’ * d/BTEX\low{t}/ETEX
k = (int)floor(w)
k = min(k, d/BTEX\low{t}/ETEX – 1)
\stopclc
對於 3D 圖像, \math{(i,j,k)} 處的圖像元素即為所求。
而對於 2D 圖像, \math{(i,j)} 處的圖像元素即為所求。

% linear
{\ftEmp{Filter Mode = CLK_FILTER_LINEAR}}

如果濾波模式為 \cenum{CLK_FILTER_LINEAR},
則對於 2D 圖像會選擇一個 \math{2\times 2} 的方陣中的圖像元素,
而對於 3D 圖像,則會選擇一個 \math{2\times 2\times 2} 的立方體中的圖像元素。
得到的 \math{2\times 2} 方陣或 \math{2\times 2\times 2} 立方體如下所示。

設
\startclc[indentnext=no]
s’ = 2.0f * rint(0.5f * s)
s’ = fabs(s – s’)
u = s’ * w/BTEX\low{t}/ETEX
i0 = (int)floor(u – 0.5f)
i1 = i0 + 1
i0 = max(i0, 0)
i1 = min(i1, w/BTEX\low{t}/ETEX – 1)

t’ = 2.0f * rint(0.5f * t)
t’ = fabs(t – t’)
v = t’ * h/BTEX\low{t}/ETEX
j0 = (int)floor(v – 0.5f)
j1 = j0 + 1
j0 = max(j0, 0)
j1 = min(j1, h/BTEX\low{t}/ETEX – 1)

r’ = 2.0f * rint(0.5f * r)
r’ = fabs(r – r’)
w = r’ * d/BTEX\low{t}/ETEX
k0 = (int)floor(w – 0.5f)
k1 = k0 + 1
k0 = max(k0, 0)
k1 = min(k1, d/BTEX\low{t}/ETEX – 1)

a = frac(u – 0.5)
b = frac(v – 0.5)
c = frac(w – 0.5)
\stopclc
其中 \ccmm{frac(x)} 為 \ccmm{x} 的小數部分,相當於 \ccmm{x - floor(x)}。

對於 3D 圖像,用如下方式得到圖像元素:
\startclc[indentnext=no]
T = (1 – a) * (1 – b) * (1 – c) * T/BTEX\low{i0j0k0}/ETEX
    + a * (1 – b) * (1 – c) * T/BTEX\low{i1j0k0}/ETEX
    + (1 – a) * b * (1 – c) * T/BTEX\low{i0j1k0}/ETEX
    + a * b * (1 – c) * T/BTEX\low{i1j1k0}/ETEX
    + (1 – a) * (1 – b) * c * T/BTEX\low{i0j0k1}/ETEX
    + a * (1 – b) * c * T/BTEX\low{i1j0k1}/ETEX
    + (1 – a) * b * c * T/BTEX\low{i0j1k1}/ETEX
    + a * b * c * T/BTEX\low{i1j1k1}/ETEX
\stopclc
其中 \math{T_{ijk}} 就是此 3D 圖像中位置 \math{(i, j, k)} 處的元素。

對於 2D 圖像,用如下方式得到圖像元素:
\startclc[indentnext=no]
T = (1 – a) * (1 – b) * T/BTEX\low{i0j0}/ETEX
    + a * (1 – b) * T/BTEX\low{i1j0}/ETEX
    + (1 – a) * b * T/BTEX\low{i0j1}/ETEX
    + a * b * T/BTEX\low{i1j1}/ETEX
\stopclc
其中 \math{T_{ij}} 就是此 2D 圖像中位置 \math{(i, j)} 處的元素。


注意:

如果在\cnglo{sampler}中指明使用的是非歸一化坐標(浮點數或整數坐標),
濾波模式為 \cenum{CLK_FILTER_NEAREST},
尋址模式為 \cenum{CLK_ADDRESS_NONE}、 \cenum{CLK_ADDRESS_CLAMP_TO_EDGE}
或 \cenum{CLK_ADDRESS_CLAMP},
則本節中計算圖像元素的位置 \math{(i,j,k)} 時不會損失精度。

如果\cnglo{sampler}採用的是其他組合方式(歸一化或非歸一化坐標、濾波模式、尋址模式),
則對於尋址模式的計算以及圖像濾波的運算而言,
在 OpenCL 規範的這個修訂版中沒有定義其相對誤差或精度。
為了確保任何 OpenCL \cnglo{device}在進行圖像尋址和濾波計算時都至少具有一個最小精度,
對於\cnglo{sampler}的這些組合方式,
開發人員應當在\cnglo{kernel}中將坐標去歸一化,
並使用由非歸一化坐標、濾波模式為 \cenum{CLK_FILTER_NEAREST}、
尋址模式為 \cenum{CLK_ADDRESS_NONE}、 \cenum{CLK_ADDRESS_CLAMP_TO_EDGE}
或 \cenum{CLK_ADDRESS_CLAMP} 組合而成的
\cnglo{sampler}調用 \capi{read_image{f|i|ui}},
最後對從圖像中讀到的顏色進行插值來生成經過濾波的顏色值。
