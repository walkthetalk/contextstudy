\section[sec:conversionRule]{轉換規則}

本節中我們將討論\cnglo{kernel}中讀寫圖像時所用的轉換規則。

% Conversion rules for normalized integer channel data types
\subsection{歸一化整數通道數據類型的轉換規則}

本節我們將討論歸一化整數通道數據類型與浮點值之間的相互轉換。

% Converting normalized integer channel data types to floating-point values
\subsubsection[sec:normIntToFloat]{將歸一化整數通道數據類型轉換為浮點值}

如果通道數據類型為 \cenum{CL_UNORM_INT8} 和 \cenum{CL_UNORM_INT16}。
\capi{read_imagef} 將把 8 位或 16 位無符號整數轉換為歸一化浮點值,
其所在區間為 \math{[0.0f \cdots 1.0]}。

如果通道數據類型為 \cenum{CL_SNORM_INT8} 和 \cenum{CL_SNORM_INT16}。
\capi{read_imagef} 將把 8 位或 16 位帶符號整數轉換為歸一化浮點值,
其所在區間為 \math{[-1.0 \cdots 1.0]}。

轉換方式如下:

\startclCmmDesc{\cenum{CL_UNORM_INT8} (8 位無符號整數)-> \ctype{float}}
歸一化浮點值 \ccmm{= (float) c / 255.0f}
\stopclCmmDesc

\startclCmmDesc{\cenum{CL_UNORM_INT_101010} (10 位無符號整數)-> \ctype{float}}
歸一化浮點值 \ccmm{= (float) c / 1023.0f}
\stopclCmmDesc

\startclCmmDesc{\cenum{CL_UNORM_INT16} (16 位無符號整數)-> \ctype{float}}
歸一化浮點值 \ccmm{= (float) c / 65535.0f}
\stopclCmmDesc

\startclCmmDesc{\cenum{CL_SNORM_INT8} (8 位帶符號整數)-> \ctype{float}}
歸一化浮點值 \ccmm{= max(-1.0f, (float)c / 127.0f)}
\stopclCmmDesc

\startclCmmDesc{\cenum{CL_SNORM_INT16} (16 位帶符號整數)-> \ctype{float}}
歸一化浮點值 \ccmm{= max(-1.0f, (float)c / 32767.0f)}
\stopclCmmDesc

除了下面所列情況,上述轉換的精度均 \math{\leq 1.5 ulp}:
\startclCmmDesc{對於 \cenum{CL_UNORM_INT8}}
\ccmm{0} 必須轉換為 \ccmm{0.0f}

\ccmm{255} 必須轉換為 \ccmm{1.0f}
\stopclCmmDesc

\startclCmmDesc{對於 \cenum{CL_UNORM_INT_101010}}
\ccmm{0} 必須轉換為 \ccmm{0.0f}

\ccmm{1023} 必須轉換為 \ccmm{1.0f}
\stopclCmmDesc

\startclCmmDesc{對於 \cenum{CL_UNORM_INT16}}
\ccmm{0} 必須轉換為 \ccmm{0.0f}

\ccmm{65535} 必須轉換為 \ccmm{1.0f}
\stopclCmmDesc

\startclCmmDesc{對於 \cenum{CL_SNORM_INT8}}
\ccmm{-128} 和 \ccmm{-127} 必須轉換為 \ccmm{-1.0f}
\ccmm{0} 必須轉換為 \ccmm{0.0f}

\ccmm{127} 必須轉換為 \ccmm{1.0f}
\stopclCmmDesc

\startclCmmDesc{對於 \cenum{CL_SNORM_INT16}}
\ccmm{-32768} 和 \ccmm{-32767} 必須轉換為 \ccmm{-1.0f}

\ccmm{0} 必須轉換為 \ccmm{0.0f}

\ccmm{32767} 必須轉換為 \ccmm{1.0f}
\stopclCmmDesc

% Converting floating-point values to normalized integer channel data types
\subsubsection{將浮點值轉換為歸一化整數通道數據類型}

如果通道數據類型為 \cenum{CL_UNORM_INT8} 和 \cenum{CL_UNORM_INT16}。
\capi{write_imagef} 將把浮點顏色值轉換為 8 位或 16 位無符號整數。

如果通道數據類型為 \cenum{CL_SNORM_INT8} 和 \cenum{CL_SNORM_INT16}。
\capi{write_imagef} 將把浮點顏色值轉換為 8 位或 16 位帶符號整數。

建議按如下方式轉換:

\startclCmmDesc{\ctype{float} -> \cenum{CL_UNORM_INT8} (8 位無符號整數)}
\ccmm{convert_uchar_sat_rte(f * 255.0f)}
\stopclCmmDesc

\startclCmmDesc{\ctype{float} -> \cenum{CL_UNORM_INT_101010} (10 位無符號整數)}
\ccmm{min(convert_ushort_sat_rte(f * 1023.0f), 0x3ff)}
\stopclCmmDesc

\startclCmmDesc{\ctype{float} -> \cenum{CL_UNORM_INT16} (16 位無符號整數)}
\ccmm{convert_ushort_sat_rte(f * 65535.0f)}
\stopclCmmDesc

\startclCmmDesc{\ctype{float} -> \cenum{CL_SNORM_INT8} (8 位帶符號整數)}
\ccmm{convert_char_sat_rte(f * 127.0f)}
\stopclCmmDesc

\startclCmmDesc{\ctype{float} -> \cenum{CL_SNORM_INT16} (16 位帶符號整數)}
\ccmm{convert_short_sat_rte(f * 32767.0f)}
\stopclCmmDesc

對於溢出時的行為以及飽和轉換規則請參考\refsec{oorbasc}。

OpenCL 實作也可能選擇其他捨入模式來逼近上述轉換結果。
如果使用的捨入模式不是捨入為最近偶數(\ccmm{_rte}),
則實作所產生的結果與捨入模式 \ccmm{_rte} 所產生結果的絕對誤差必須 \math{\leq 0.6}。

\startclc
float -> CL_UNORM_INT8 (8-bit unsigned integer)
	Let f/BTEX\low{preferred}/ETEX = convert_uchar_sat_rte(f * 255.0f)
	Let f/BTEX\low{approx}/ETEX =
		convert_uchar_sat_<impl-rounding-mode>(f * 255.0f)
	fabs(f/BTEX\low{preferred}/ETEX – /BTEX\low{fapprox}/ETEX) must be <= 0.6
float -> CL_UNORM_INT_101010 (10-bit unsigned integer)
	Let f/BTEX\low{preferred}/ETEX = convert_ushort_sat_rte(f * 1023.0f)
	Let f/BTEX\low{approx}/ETEX =
		convert_ushort_sat_<impl-rounding-mode>(f * 1023.0f)
	fabs(f/BTEX\low{preferred}/ETEX – f/BTEX\low{approx}/ETEX) must be <= 0.6
float -> CL_UNORM_INT16 (16-bit unsigned integer)
	Let f/BTEX\low{preferred}/ETEX = convert_ushort_sat_rte(f * 65535.0f)
	Let f/BTEX\low{approx}/ETEX =
		convert_ushort_sat_<impl-rounding-mode>(f * 65535.0f)
	fabs(f/BTEX\low{preferred}/ETEX – f/BTEX\low{approx}/ETEX) must be <= 0.6
float -> CL_SNORM_INT8 (8-bit signed integer)
	Let f/BTEX\low{preferred}/ETEX = convert_char_sat_rte(f * 127.0f)
	Let f/BTEX\low{approx}/ETEX =
		convert_char_sat_<impl_rounding_mode>(f * 127.0f)
	fabs(f/BTEX\low{preferred}/ETEX – fapprox}/ETEX) must be <= 0.6
float -> CL_SNORM_INT16 (16-bit signed integer)
	Let f/BTEX\low{preferred}/ETEX = convert_short_sat_rte(f * 32767.0f)
	Let f/BTEX\low{approx}/ETEX =
		convert_short_sat_<impl-rounding-mode>(f * 32767.0f)
	fabs(f/BTEX\low{preferred}/ETEX – f/BTEX\low{approx}/ETEX) must be <= 0.6
\stopclc

% Conversion rules for half precision floating-point channel data type
\subsection{半精度浮點通道數據類型的轉換規則}

如果圖像通道數據類型為 \cenum{CL_HALF_FLOAT},
則由 \ctype{half} 到 \ctype{float} 的轉換是無損的(參見\refsec{dataTypeHalf})。
由 \ctype{float} 到 \ctype{half} 的轉換中,
捨入尾數所用的捨入模式為捨入為最近偶數或向零捨入。
型別為 \ctype{half} 的去規格化數(可能是將 \ctype{float} 轉換為 \ctype{half} 時生成)
可能會被刷成零。
型別為 \ctype{float} 的 NaN 必須轉換為型別為 \ctype{half} 的 NaN。
型別為 \ctype{float} 的 INF 必須轉換為型別為 \ctype{half} 的 INF。

% Conversion rules for floating-point channel data type
\subsection{浮點通道數據類型的轉換規則}

如果圖像通道數據類型為 \cenum{CL_FLOAT},則對其讀寫時要遵守下列規則:
\startigBase
\item NaN 必須轉換為 \cnglo{device} 所支持的 NaN 值。
\item 可以將去規格化數刷成零。
\item 所有其他值都必須保留。
\stopigBase

% Conversion rules for signed and unsigned 8-bit, 16-bit
% and 32-bit integer channel data types
\subsection{帶符號或無符號的 8/16/32 位整數通道數據類型的轉換規則}

如果圖像通道數據類型為 \cenum{CL_SIGNED_INT8}、 \cenum{CL_SIGNED_INT16}
或 \cenum{CL_SIGNED_INT32},
則 \capi{read_imagei} 會返回圖像中指定位置所存儲的原始整數值。

如果圖像通道數據類型為 \cenum{CL_UNSIGNED_INT8}、 \cenum{CL_UNSIGNED_INT16}
或 \cenum{CL_UNSIGNED_INT32},
則 \capi{read_imageui} 會返回圖像中指定位置所存儲的原始整數值。

\capi{write_imagei} 會進行下列轉換:
\startclc
32 bit signed integer -> 8-bit signed integer
	convert_char_sat(i)
32 bit signed integer -> 16-bit signed integer
	convert_short_sat(i)
32 bit signed integer -> 32-bit signed integer
	no conversion is performed
\stopclc

\capi{write_imageui} 會進行下列轉換:
\startclc
32 bit signed integer -> 8-bit signed integer
	convert_uchar_sat(i)
32 bit signed integer -> 16-bit signed integer
	convert_ushort_sat(i)
32 bit signed integer -> 32-bit signed integer
	no conversion is performed
\stopclc

對於本節所描述的轉換,必須使其正確飽和。
