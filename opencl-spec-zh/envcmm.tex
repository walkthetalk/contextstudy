%
% author:	Ni Qingliang
% date:		2011-02-11
%
\startenvironment envcmm

%%%%%%%%%%%%%%%%%%% simplefonts %%%%%%%%%%%%%%%%%%%%%%%%%%%%%%%%%%%%%%%%%%%%%%%%

%%%%%%%%%%%%%%%%%%% zhfonts     %%%%%%%%%%%%%%%%%%%%%%%%%%%%%%%%%%%%%%%%%%%%%%%%
\usemodule[zhfonts][style=rm, size=10.5pt]
%\setupzhfonts[feature][onum=yes, pnum=yes]

%% use `mtxrun --script font --list --all` to list all fonts
% set base fonts
\setupzhfonts
[serif]
[regular=adobesongstd,
bold=adobeheitistd,
italic=adobesongstd,
bolditalic=adobeheitistd]

\setupzhfonts
[sans]
[regular=adobefangsongstd,
bold=adobeheitistd,
italic=adobefangsongstd,
bolditalic=adobeheitistd]

\setupzhfonts
[mono]
[regular=adobekaitistd,
bold=adobeheitistd,
italic=adobekaitistd,
bolditalic=adobeheitistd]

% set latin fonts
\setupzhfonts
[latin, serif]
[regular=texgyrepagellaregular,
bold=texgyrepagellabold,
italic=texgyrepagellaitalic,
bolditalic=texgyrepagellabolditalic]

\setupzhfonts
[latin, mono]
[regular=texgyrecursorregular,
bold=texgyrecursorbold,
italic=texgyrecursoritalic,
bolditalic=texgyrecursorbolditalic]

\setupzhfonts
[latin, sans]
[regular=texgyreherosregular,
bold=texgyreherosbold,
italic=texgyreherositalic,
bolditalic=texgyreherosbolditalic]

% 启用中文断行
\setscript[hanzi]
\mainlanguage[cn]

% 交互
%\startallmodes[screen,solution]
\setupinteraction[
  state=start,	%start stop
  focus=standard,
  %menu	on off
  %page	yes no
  %click	yes no
  %split	yes no
  %display	new
  %openaction	reference
  %closeaction	reference
  %openpageaction	reference
  %closepageaction	reference
  %calculate	name
  %strut	yes no
  %width	dimension
  %height	dimension
  %depth	dimension
  style=\rm\it,	%normal bold slanted boldslanted type cap small...command
  color=black,	%name	link to other page
  contrastcolor=black,	%name	link to the same page
  %symbolset	name
  %title	text
  %subtitle	text
  %author	text
  %date	text
  %keyword	text
  %fieldlayer	auto name
]
%\stopallmodes

\setuptolerance[horizontal,
  strict, %stretch space verystrict strict tolerant verytolerant
]
\setuptolerance[vertical,
  tolerant, %stretch space verystrict strict tolerant verytolerant
]
%%%%%%%%%%%%%%%%%%%%%%%%%%%%font%%%%%%%%%%%%%%%%%%%%%%%%%%%%%%%%%%%%%%%%%%%%%%%%
\define\ftEmp{\rm\bf} % emphasize
\define\ftRef{\tt\it} % reference
\define\ftFmt{\ss\it} % format
\define\ftClc{\tt\tf}
\define\ftClcEmp{\ftClc\bf}
\define\ftClcRef{\ftClc\it}
\definetype[capi][ % api
  %space	on off
  %option	slanted normal none
  style=\rm\bf,	%normal bold slanted boldslanted type cap small ... command
  %color	name
]
\definetype[carg][ % argument
  style=\rm\it,
]
\definetype[ctype][ % type
  style=\tt\tf,
]
\definetype[cenum][ % enumeration
  style=\rm\tf,
]
\definetype[cqlf][ % qualifier
  style=\tt\it,
]
\definetype[cqlfemp][
  style=\tt\bf,
]
\definetype[cdrt][ % directive
  style=\tt\tf,
]
\definetype[cdrtemp][
  style=\tt\bf,
]
\definetype[cmacro][
  style=\tt\tf,
]
\definetype[cvar][ % variable
  style=\tt\tf,
]
\definetype[ccmm][ % common
  style=\tt\tf,
]
\definetype[clext][ % opencl extension
  style=\rm\bf,
]

% opencl data type scalar / vector
\definetype[cldts][
  style=\tt\tf,
]
\definetype[cldtvfix][% don't use it directly
  style=\tt\it,
]
\define[1]\cldtv{\cldts{#1}\cldtvfix{n}}
\define[1]\cldtvv{\cldts{#1}\cldtvfix{n}\cldts{x}\cldtvfix{m}}

% opencl api scalar / vector
\definetype[clapis][
  style=\rm\bf,
]
\definetype[clapivfix][% don't use it directly
  style=\rm\bi,
]
\define[1]\clapiv{\clapis{#1}\clapivfix{n}}
%%%%%%%%%%%%%%%%%%%%%%%%%%%%%%%%%%layout%%%%%%%%%%%%%%%%%%%%%%%%%%%%%%%%%%%%%%%%

\definepapersize[CD][width=20cm,height=20cm]
\setuppapersize[A4][A4]

\setuppagenumbering[
  state=start,
  way=bytext,
  alternative=doublesided,
  location=,%header,
  left="the",
  right=ok,
]
% don't use TAB, and no space between *reg=val*, and no null line,
% or it will be ommitted
\setuplayout[
  % vertical
  top=20mm,
    topdistance=2mm,
      header=\bodyfontsize,
        headerdistance=2mm,
        footerdistance=2mm,
      footer=\bodyfontsize,
    bottomdistance=2mm,
  bottom=20mm,
  topspace=\dimexpr(\topheight + \topdistance),
  bottomspace=\dimexpr(\bottomheight + \bottomdistance),
  height=fit,
  % horizontal
  leftedge=10mm,
    leftedgedistance=2mm,
      leftmargin=20mm,
        leftmargindistance=2mm,
        rightmargindistance=2mm,
      rightmargin=20mm,
    rightedgedistance=2mm,
  rightedge=10mm,
  backspace=\dimexpr(\leftedgewidth + \leftedgedistance + \leftmarginwidth + \leftmargindistance + 5mm),
  cutspace=\dimexpr(\rightedgewidth + \rightedgedistance + \rightmarginwidth + \rightmargindistance),
  width=fit,
  % misc
  location=middle,
  marking=on
]

% set the background clors
% NOTE: it is confilict with \show frame
%\setupcolors[state=start]
\setupbackgrounds[page][background=color,backgroundcolor=lightgray]
\setupbackgrounds[top, bottom][background=color,backgroundcolor=lightgray]
%\setupbackgrounds[text][background=color,backgroundcolor=green]
\setupbackgrounds[header][background=color,backgroundcolor=darkyellow]
\setupbackgrounds[footer][background=color,backgroundcolor=darkyellow]
\setupbackgrounds[text][leftedge,leftmargin,text,rightmargin,rightedge][background=color,backgroundcolor=darkcyan]

%\setupheader[state=none]
\setupheadertexts[margin]
[\leftaligned{\getmarking[chapter]}]
[\rightaligned{odd right}]
[\vtop{even \vskip5pt left}]
[even right{\currentdate[month,year]}]

\setupheadertexts[text]
[\rlap{abcdabcd} \hfill {\getmarking[chapter]}\hfill \llap{the \pagenumber page}]
%[\centerline{hello} \llap{???right}]
[]
[]
[]

\setupfootertexts[text]
[\vbox{\centerline{\pagenumber }\vskip12pt }]
[]
[]
[]

%%%%%%%%%%%%%%%%%%%%%%%%%%%title%%%%%%%%%%%%%%%%%%%%%%%%%%%%%%%%%%%%%%%%%%%%%%%%%
%两个quad空格	a \qquad b
%quad空格	a \quad b
%大空格		a\ b
%中等空格	a\;b
%小空格		a\,b
%没有空格	ab
%紧贴		a\!b
\define[1]\PartNbr{第#1卷}
\define[1]\ChapterNbr{第#1章}
\define[1]\SectionNbr{节#1}
\define[1]\SubsectionNbr{节#1}
\define[1]\SubsubsectionNbr{节#1}

\define\fontPart{\rm\bf\switchtobodyfont[big]\switchtobodyfont[big]\switchtobodyfont[big]}
\setuplabeltext[cn][part={第\;,\;卷}]
\setuphead[part][
  conversion=chinesecapnumeralscn,
  textstyle={\fontChapter}, %normal bold slanted boldslanted % 標題內容
  numberstyle={\fontChapter}, %normal bold slanted boldslanted % 編號
%  numbercommand=\PartNbr,
  header=high, %none empty high nomarking %章節首頁無頁眉
  footer=high, %none empty high nomarking
  before={\blank[2em]}, %COMMAND
  after={\blank[1em]}, %COMMAND
  alternative=middle, %normal inmargin middle TEXT
  placehead=yes,
]



\define\fontChapter{\rm\bf\switchtobodyfont[big]\switchtobodyfont[big]}
\setuplabeltext[cn][chapter={第\;,\;章}]
\setuphead[chapter][
  conversion=numbers, %chinesenumerals, chinesecapnumeralscn chineseallnumeralscn
  sectionsegments=2:100,
  indentnext=yes,
%  style=bold,%normal bold slanted boldslanted % 整体
  textstyle={\fontChapter}, %normal bold slanted boldslanted % 標題內容
  numberstyle={\fontChapter}, %normal bold slanted boldslanted % 編號
%  numbercommand=\ChapterNbr, %\...#1
%  color=red, %IDENTIFIER % 整體顏色
%  textcolor=blue, %IDENTIFIER % 內容顏色
%  numbercolor=green, %IDENTIFIER % 編號顏色
  number=yes, %yes no % 是否帶編號
  ownnumber=no, %yes no % 是否指定編號,如果是,則第一個參數就是編號
  page=right, %left right yes % 是否固定於左頁或右頁
  continue=no, %yes no % 第一個是否緊接上一層,優先於*page*
  header=high, %none empty high nomarking %章節首頁無頁眉
%  text=nomarking, %none empty high nomarking
  footer=high, %none empty high nomarking
  before={\blank[2em]}, %COMMAND
%  inbetween=, %COMMAND
  after={\blank[1em]}, %COMMAND
  alternative=middle, %normal inmargin middle TEXT
%  hang=none, %none broad fit line NUMBER
%  command=, %\...#1#2
%  textcommand=, %\...#1
%  deepnumbercommand=, %\...#1
%  deeptextcommand=, %\...#1
%  prefix=, %+ - TEXT
%  placehead=yes, %yes no empty
%  incrementnumber=yes, %yes no LIST FILE
%  resetnumber=no, %yes no
%  file=, %IDENTIFIER
%  expansion=, %yes no command
%  margintext=no, %yes no
  %inherits from \setupheads
  sectionnumber=no,
%  alternative=,
%  separator={haha},
%  stopper=,
%  align=,
%  aligntitle=,
%  tolerance=,
%  indentnext=,
%  command=,
%  margin=,
]


\define\fontSection{\rm\bf\switchtobodyfont[big]}
%\setuplabeltext[cn][section={第,節}]
\setuplabeltext[cn][section={节\;,}]
\setuphead[section][
  conversion=numbers, %chinesecapnumeralscn chineseallnumeralscn
  sectionsegments=2:100,
  indentnext=yes,
%  style=bold,%normal bold slanted boldslanted % 整体
  textstyle={\fontSection}, %normal bold slanted boldslanted % 標題內容
  numberstyle={\fontSection}, %normal bold slanted boldslanted % 編號
%  numbercommand=\SectionNbr,
  number=yes, %yes no % 是否帶編號
  ownnumber=no, %yes no % 是否指定編號,如果是,則第一個參數就是編號
  before={\blank[1.5em]}, %COMMAND
  after={\blank[1em]}, %COMMAND
  alternative=inmargin, %normal inmargin middle TEXT
]

\define\fontSubsection{\rm\bf}
%\setuplabeltext[cn][subsection={,節}]
\setuphead[subsection][
  conversion=numbers,
  indentnext=yes,
%  style=bold,%normal bold slanted boldslanted % 整体
  textstyle={\fontSubsection}, %normal bold slanted boldslanted % 標題內容
  numberstyle={\fontSubsection}, %normal bold slanted boldslanted % 編號
%  numbercommand=\SubsectionNbr,
  number=yes, %yes no % 是否帶編號
  ownnumber=no, %yes no % 是否指定編號,如果是,則第一個參數就是編號
  before={\blank[1em]}, %COMMAND
  after={\blank[1em]}, %COMMAND
  alternative=inmargin, %normal inmargin middle TEXT
]

%\setupheadertexts[hello]
%%%%%%%%%%%%%%%%%%%%%%%%%%%%%setup list%%%%%%%%%%%%%%%%%%%%%%%%%%%%%%%%%%%%%%%%%
\setuplist[chapter][
  %state	start stop
  alternative=c,	%a 空格
			%b 右对齐
			%c ...page
			%d 大空格 紧接下一个list
			%e 编号加方框,标题向左缩
			%f 
			%g 标题居中... none command
  %coupling	on off
  %criterium	section local previous current all
  %pageboundaries	list
  %style	normal bold slanted boldslanted type cap small... command
  %numberstyle	normal bold slanted boldslanted type cap small... command
  %textstyle	normal bold slanted boldslanted type cap small... command
  %pagestyle	normal bold slanted boldslanted type cap small... command
  %color	name
  %command	threearguments
  numbercommand=\ChapterNbr,	%oneargument
  %textcommand={},  %oneargument
  %pagecommand	oneargument
  interaction=all, %sectionnumber text pagenumber all
  %before=\right,	%command
  %after	command
  %inbetween	command
  %left={haha},	%text
  %right	text
  %label	yes no
  %prefix	yes no none
  %pagenumber	yes no
  %headnumber	yes no
  %sectionnumber	yes no
  %aligntitle	yes no
  margin=0em,	%dimension % 左侧缩进
  width=0em,	%dimension fit	% 编号宽度
  %height=2em,	%dimension fit broad
  %depth=10em,	%dimension fit broad
  distance=1.0em,	%dimension 标题与编号的距离
  %separator={\dot},	%text
  %stopper	text
  %symbol	none 1 2 3 ...
  %expansion=yes,	%yes no command
  %maxwidth	dimension
  %[...,...=...,...]	see \framed
]

\setuplist[section][
  alternative=c,	%a 空格
  interaction=all,
  %numbercommand=\SectionNbr,	%oneargument
  margin=2em,	%dimension % 左侧缩进
  width=0em,	%dimension fit	% 编号宽度
  distance=1.0em,	%dimension
]

\setuplist[subsection][
  alternative=c,	%a 空格
  %numbercommand=\SubsectionNbr,	%oneargument
  margin=4em,	%dimension % 左侧缩进
  width=0em,	%dimension fit	% 编号宽度
  distance=0.5em,	%dimension
]

%%%%%%%%%%%%%%%%%%%%%%%%%%%%footnote%%%%%%%%%%%%%%%%%%%%%%%%%%%%%%%%%%%%%%%%%%%%
\setupfootnotes[
  %conversion=set 2,	%numbers characters Characters romannumerals Romannumerals
  %way=bytext,	%bytext bysection
  %location=high,	%page text columns firstcolumn lastcolumn high none
  %background=color,
  %backgroundcolor=red,
  %rule	on off
  %before	command
  %after	command
  %width	dimension
  %height	dimension
  %bodyfont	5pt ... 12pt small big
  %style=bold,	%normal bold slanted boldslanted type cap small... command
  %distance	dimension
  %columndistance	dimension
  %margindistance	dimension
  %n=3,		%number
  %numbercommand=\high,	%oneargument 脚注编号的位置
  %textcommand	oneargument
  %split	tolerant strict verystrict number
  %textstyle	normal bold slanted boldslanted type cap small... command
  %textcolor=red,	%name
  %interaction=yes,	%yes no
  %factor	number
]

%%%%%%%%%%%%%%%%%%%%%%%%%%%%par%%%%%%%%%%%%%%%%%%%%%%%%%%%%%%%%%%%%%%%%%%%%%%%%%

% 首行缩进
\setupindenting[yes, 2em]
% 段落间距
\setupwhitespace[none] %none small medium big line fixed fix dimension
% 行间距
\setupinterlinespace[medium]
%\setupinterlinespace[big, on][line=3ex]
%\setuprelativeinterlinespace[1.2]

% code
%\installprettytype[C][C]
\setuplinenumbering[
% conversion	numbers characters Characters romannumerals Romannumerals text
% start	number
% step	number
%  width=1em,	%dimension
%  location=text,	%intext inmargin
% style	normal bold slanted boldslanted type cap small... command
% prefix	text
% referencing	on off
]
\definetextbackground
  [verb]
  [frame=on,location=paragraph,
   leftoffset=0.5em,rightoffset=0.5em,
   topoffset=0.25em,bottomoffset=0.25em,
   rulethickness=0.75pt]
\definetyping[clc][%option=c,
  bodyfont=9pt,
  %space	on off
  %page	yes no
  %option	slanted normal commands color none
  %text	yes no
  %icommand	command
  %vcommand	command
  %ccommand	command
  before={\blank\starttextbackground[verb]},	%command
  after={\stoptextbackground\blank},	%command
  margin=no,	%dimension standard yes no
  %evenmargin	dimension
  %oddmargin	dimension
  %blank	dimension small medium big standard halfline line
  escape=yes,	%character %{[[,]]}
  %space	on off
  tab=8,	%number yes no
  %page	yes no
  %indentnext	yes no
  style=\ftClc,		%normal bold slanted boldslanted type cap small... command
  %color	name
  %palet	name
  %lines=yes,	%yes no hyphenated
  %empty	yes all no
  numbering=line,	%line file no
]
%%%%%%%%%%%%%%%%%%%%%%%%%%%%enum%%%%%%%%%%%%%%%%%%%%%%%%%%%%%%%%%%%%%%%%%%%%%%%%

\defineenumeration[example][
  text={例},
%  inbetween=\blank,
%  after=\blank,
%  margin=2em,
  headstyle=\rm\bf,	%normal bold slanted boldslanted type cap small... command
%  headcolor=blue:7,
%style	normal bold slanted boldslanted type cap small... command
%color	name
%width	fit broad dimension
%distance	dimension
%sample	text
%text	text
%align	flushleft middle flushright
%margin	standard yes no dimension
  alternative=top,	%left right top serried inmargin inleft inright hanging
%headcommand	command
%hang	fit broad number
%before	command
%inbetween	command
%after	command
%indentnext	yes no
%indenting	never not no yes always first next
  prefix=yes,
  prefixsegments=chapter, %chapter:section
]

% no number
\defineitemgroup[igBase][levels=1]
\setupitemgroup[igBase]
[each]
[packed,joinedup,]
%standard broad serried packed unpacked stopper joinedup atmargin inmargin autointro loose repeat section paragraph intext random columns
%standard: default setup
%n*broad:  extra horizontal white space after symbol
%n*serried:little horizontal white space after symbol
%packed:   no whitespace between items
%stopper:  punctuation after item separator
%joinedup: no white space before and after itemization
%atmargin: item separator at the margin
%inmargin: item separator in margin
[
%margin	no standard dimension
  leftmargin=2em,	%no standard dimension
%rightmargin	no standard dimension
%width	dimension
%distance	dimension
%factor	number
%items	number
%start	number
  before=,	%command
  inbetween=,	%command
  after=,	%command
%left	text
%right	text
%beforehead	command
%afterhead	command
%headstyle=boldslanted, %normal bold slanted boldslanted type cap small... command
%marstyle=boldslanted, %normal bold slanted boldslanted type cap small... command
%symstyle=boldslanted, %normal bold slanted boldslanted type cap small... command
%stopper	text
%n	number
%symbol	number
%align	left right normal
%indentnext	yes no
]
\setupitemgroup[igBase][1][1] % 枚举的标识必须单独定义,否则无效,可选的
%m	A numbered list, with lowercase (“medieval”, aka “oldstyle”) numbers.
%1 … 8	Different kinds of bullets. All items get the same symbol.
%a	Items are numbered a., b., c., …
%A	Items are numbered A., B., C., …
%AK	Items are numbered A., B., C., …, in small caps.
%r	Items are numbered in lowercase Roman numerals.
%R	Items are numbered in uppercase Roman numerals.
%KR	Items are numbered in uppercase Roman numerals, small caps style.


% with number
\defineitemgroup[igNum][levels=1]
\setupitemgroup[igNum]
[each]
[packed,joinedup,]
[
  leftmargin=2em,	%no standard dimension
  before=,	%command
  inbetween=,	%command
  after=,	%command
]
\setupitemgroup[igNum][1][n]

% big item
\defineitemgroup[igBig][levels=2]
\setupitemgroup[igBig][1]
[packed,joinedup,]
[
  leftmargin=0em,
  before=,
  inbetween=,
  after=,
  width=2em,
]
\setupitemgroup[igBig][1][a]
\setupitemgroup[igBig][2]
[packed,joinedup,]
[
  leftmargin=2em,
  before=,
  inbetween=,
  after=,
  width=2em,
]
\setupitemgroup[igBig][2][1]

%%%%%%%%%%%%%%%%%%%%%%%%%%table%%%%%%%%%%%%%%%%%%%%%%%%%%%%%%%%%%%%%%%%%%%%%%%%%
\setupTABLE[%
    %frameoffset=.5\linewidth,
    %backgroundoffset=\v!frame,
    %framecolor=\s!black,
    %width=\v!fit,
    %height=\v!fit,
    %autowidth=\v!yes,
    %rulethickness=\linewidth,
    %strut=\v!yes,
    %autostrut=\v!no,
    %
    %color=,
    style={\rmx},
    %headstyle=\rmx\bf,
    %headcolor=,
    %aligncharacter=\v!no,
    %alignmentcharacter={,},
    %option=, % \v!stretch
    %header=,
    %spaceinbetween=,
    %maxwidth=8em,
    %textwidth=\hsize,
    %split=repeat,
    %splitoffset=0pt,
    %distance=\zeropoint,           % individual column
    %columndistance=\zeropoint,     % each column (whole table)
    %leftmargindistance=\zeropoint, % whole table
    %rightmargindistance=\zeropoint,% whole table
    %left=,
    %right=,
    %setups=,
    %splitmethod=a%
]
\setupTABLE[header][
    style={\rmx\bf},
    background=color,
    backgroundcolor=gray,
]

%%%%%%%%%%%%%%%%%%% ETD: enum / type / desc
\def\startETD{\dodoubleempty\dostartETD}

\long\def\dostartETD[#1][#2]#3\stopETD{%
  \startlocalfootnotes
  \bTABLE[option=stretch]
  \bTABLEhead
    \bTR[background=color,backgroundcolor=gray]
        \bTH \ctype{#1} \eTH
        \bTH #2 \eTH
    \eTR
  \eTABLEhead
  \bTABLEbody
    #3
  \eTABLEbody
  \eTABLE
  \placelocalfootnotes[style=\rmx,before=,after=,]
  \stoplocalfootnotes
}

\define[3]\clETD{
\bTR[background=color,backgroundcolor=gray]
  \bTC \cenum{#1} \eTC
  \bTC \ctype{#2} \eTC
\eTR
\bTR
  \bTC[nc=2]
    \parindent2em
    #3
  \eTC
\eTR
}
%%%%%%%%%%%%%%%%%%% ED: enum / desc
\def\startED{\dodoubleempty\dostartED}

\long\def\dostartED[#1]#2\stopED{%
  \startlocalfootnotes
  \bTABLE[option=stretch]
  \bTABLEhead
    \bTR[background=color,backgroundcolor=gray]
        \bTH #1 \eTH
    \eTR
  \eTABLEhead
  \bTABLEbody
    #2
  \eTABLEbody
  \eTABLE
  \placelocalfootnotes[style=\rmx,before=,after=,]
  \stoplocalfootnotes
}

\define[2]\clED{
\bTR[background=color,backgroundcolor=gray]
  \bTC \cenum{#1} \eTC
\eTR
\bTR
  \bTC
    \parindent2em
    #2
  \eTC
\eTR
}
%%%%%%%%%%%%%%%%%%% TD: object / desc
\def\startCLOD{\dodoubleempty\dostartCLOD}

\long\def\dostartCLOD[#1][#2]#3\stopCLOD{%
  \startlocalfootnotes
  \bTABLE[option=stretch]
  \bTABLEhead
    \bTR[background=color,backgroundcolor=gray]
        \bTH #1 \eTH
        \bTH #2 \eTH
    \eTR
  \eTABLEhead
  \bTABLEbody
    #3
  \eTABLEbody
  \eTABLE
  \placelocalfootnotes[style=\rmx,before=,after=,]
  \stoplocalfootnotes
}

\define[2]\clOD{
\bTR
  \bTD #1 \eTD
  \bTD #2 \eTD
\eTR
}
%%%%%%%%%%%%%%%%%%% TD: type / type
\def\startCLTT{\dostartCLTT}

\long\def\dostartCLTT#1\stopCLTT{%
  \midaligned{
  \bTABLE
  \setupTABLE[c][each][align={middle,lohi}]
  \bTABLEhead
    \bTR[background=color,backgroundcolor=gray]
        \bTH OpenCL 語言中的型別 \eTH
        \bTH 應用所用 API 中的型別 \eTH
    \eTR
  \eTABLEhead
  \bTABLEbody
    #1
  \eTABLEbody
  \eTABLE
  }
}

\define[2]\clTT{
\bTR
  \bTD #1 \eTD
  \bTD #2 \eTD
\eTR
}

%%%%%%%%%%%%%%%%%%%%%%%%%%shotcut%%%%%%%%%%%%%%%%%%%%%%%%%%%%%%%%%%%%%%%%%%%%%%%
\define\scver{1.2}
\define\scclver{OpenCL 1.2}
\define\schostfailres{在為\cnglo{host}上的 OpenCL 實作分配資源時失敗}
\define\scdevfailres{在為\cnglo{device}上的 OpenCL 實作分配資源時失敗}
\define\scinf{INF}
\define\scqnan{Quiet NaN}
\define\scfma{FMA}
\define\scnfma{積和熔加运算}
%%%%%%%%%%%%%%%%%%%%%%%%%%function%%%%%%%%%%%%%%%%%%%%%%%%%%%%%%%%%%%%%%%%%%%%%%
%% clOption
\definedescription[clOption][
  %headstyle	normal bold slanted boldslanted type cap small... command
  %style	normal bold slanted boldslanted type cap small... command
  %color	name
  %width	fit broad dimension
  %distance	dimension
  %sample	text
  %text	text
  %align	flushleft middle flushright
  margin=2em,	%standard yes no dimension
  alternative=top,	%left right top serried inmargin inleft inright hanging
  %headcommand=,	%command
  command=\hskip-2em,	%command
  %hang=,	%fit broad number
  before=\vskip.0em,		%command
  inbetween=\vskip.0em,		%command
  after=\vskip.0em,		%command
  %indentnext	yes no
  indenting=yes,	%never not no yes always first next
]
%%%%%%%%%%%%%%%%%%%%%%%%%%%%%%%%%%%%%%%%%%%%%%%%%%%%%%%%%%%%%%%%%%%%%%% glossary
\definedescription[clglodsc][
  headstyle={\rm\bfa},	%normal bold slanted boldslanted type cap small... command
  %style=normal,	%normal bold slanted boldslanted type cap small... command
  %color=,		%name
  %width=fit,		%fit broad dimension
  %distance=,		%dimension
  %sample=,		%text
  %text=,		%text
  %align=,		%flushleft middle flushright
  %margin=2em,		%standard yes no dimension
  alternative=top,		%left right top serried inmargin inleft inright hanging
  %headcommand=\vskip.0em,		%command
  %command=\hskip-2em,	%command
  %hang=,		%fit broad number
  before=\vskip.0em,		%command
  inbetween=\vskip.0em,		%command
  after=\vskip.5em,		%command
  %indentnext=,		%yes no
  indenting=yes,	%never not no yes always first next
]

%如果reference放到\glossary前面,会多加一个空行
\define[3]\clglohead{
{#2}(#3)
\reference[clglo:#1]{#2}
\reference[clglo:en#1]{#3}
}

\define[3]\clglo{
\clglodsc{\clglohead{#1}{#2}{#3}}
}

\define[3]\startclglo{
\startclglodsc{\clglohead{#1}{#2}{#3}}
}

\define\stopclglo{
\stopclglodsc
}

\define[1]\refglo{
\goto{\ftRef\ref[t][clglo:#1]}[clglo:#1]
}
\define[1]\empglo{
\goto{\ftEmp{\ref[t][clglo:#1]}}[clglo:#1]
}

\define[1]\cnglo{\ref[t][clglo:#1]}
\define[1]\cngloemp{{\ftEmp{\cnglo{#1}}}}
\define[1]\englo{\ref[t][clglo:en#1]}
%%%%%%%%%%%%%%%%%%%%%%%%%%%%%%%%%%%%%%%%%%%%%%%%%%%%%%%%%%%%%%%%%%%%%%%%%%ref

\if 0
\definereferenceformat[infig][
  left={图},	%text
  %right=,	%text
  %text=,	%text
  %label=,	%name
]
\setupreferencing[
  %state	start stop
  %sectionnumber=yes,	%yes no
  %prefix	+ - text
  %interaction	label text all symbol
  %width	dimension
  left=\ftRef,	%command	\about会用到
  right=,	%command
  %convertfile	yes no small big
  %separator	text
  %autofile	yes no page
  %global	yes no
]
\fi

\define[1]\refsec{
\in{節}{}[sec:#1]
}

\define[1]\reffig{
\in{图}{}[fig:#1]
}
\define[1]\reftab{
\in{表}{}[tab:#1]
}
%%%%%%%%%%%%%%%%%%%%%%%%%%%%%%%%floats%%%%%%%%%%%%%%%%%%%%%%%%%%%%%%%%%%%%%%%%%
\setupframed[
  %height	fit broad dimension
  %width	fit broad fixed local dimension
  %autowidth	yes no force
  %offset	none overlay default dimension
  %location	depth hanging high lohi low top middle bottom
  %option	none empty
  %strut	yes no global local
  %align	no flushleft flushright middle normal high low lohi
  %bottom	command
  %top	command
  %frame	on off none overlay
  %topframe	on off
  %bottomframe	on off
  %leftframe	on off
  %rightframe	on off
  %frameoffset	dimension
  %framedepth	dimension
  %framecorner	round rectangular
  %frameradius	dimension
  %framecolor	name
  %background	screen color none foreground name
  %backgroundscreen	number
  %backgroundcolor	name
  %backgroundoffset	frame dimension
  %backgrounddepth	dimension
  %backgroundcorner	round rectangular
  %backgroundradius	dimension
  %depth	dimension
  %corner	round rectangular
  %radius	dimension
  %empty	yes no
  %foregroundcolor	name
  %foregroundstyle	name
  %rulethickness	dimension
]
\setupfloats[
  %location	left right middle
  %width	fit dimension
  before=,	%command
  after=,	%command
  %margin	dimension
  %spacebefore	small medium big none
  %spaceafter	small medium big none
  %sidespacebefore	small medium big none
  %sidespaceafter	small medium big none
  indentnext=yes,	%yes no
  %ntop	number
  %nbottom	number
  %nlines	number
  %default	name
  %tolerance	0 1 2
  %leftmargindistance	dimension
  %rightmargindistance	dimension
  %sidealign	normal line
  %numbering	yes nocheck
  % to see \setupframed
]

\setupcaptions[align={middle}]

\setupcaption[figure][
  %location	top bottom none high low middle
  %width=max,	%fit broad max dimension
  %minwidth	fit dimension
  headstyle=\rmx\bf,	%normal bold slanted boldslanted type cap small... command
  style=\rmx\bf,	%normal bold slanted boldslanted type cap small... command
  %number	yes no
  %inbetween	command
  %align	left middle right no
  conversion=numbers,	%numbers characters Characters romannumerals Romannumerals
  way=bychapter,	%bytext bysection bychapter 每一章都重新编号
  %separator=a,	%text
  %stopper	text
  %command	command
  %distance	dimension
  prefixsegments=chapter,	%chapter:section
]
\setupcaption[table][
  location=top,	%top bottom none high low middle
  %width=max,	%fit broad max dimension
  %minwidth	fit dimension
  headstyle={\rmx\bf},	%normal bold slanted boldslanted type cap small... command
  style={\rmx\bf},	%normal bold slanted boldslanted type cap small... command
  %number	yes no
  %inbetween	command
  %align	left middle right no
  conversion=numbers,	%numbers characters Characters romannumerals Romannumerals
  way=bychapter,	%bytext bysection bychapter 每一章都重新编号
  %separator=a,	%text
  %stopper	text
  %command	command
  %distance	dimension
  prefixsegments=chapter,	%chapter:section
]

\setupfloatsplitting[lines=1,conversion=greek]
%%%%%%%%%%%%%%%%%%%%%%%%%%%%%%formula%%%%%%%%%%%%%%%%%%%%%%%%%%%%%%%%%%%%%%%%%%%
\setupformulas[
  %location	left right
  %left	text
  %right	text
  %align	flushleft middle flushright
  %option	middle
  %strut	yes no
  %distance=0.01ex,	%dimension
  %margin	dimension standard yes no
  %leftmargin	dimension
  %rightmargin	dimension
  %indentnext=yes,	%yes no
  %alternative	name
  %spacebefore=0.01ex,	%dimension
  %after	dimension
  %separator	text
  %conversion	numbers characters Characters romannumerals Romannumerals text
]
%%%%%%%%%%%%%%%%%%%%%%%%%%%%%%misc%%%%%%%%%%%%%%%%%%%%%%%%%%%%%%%%%%%%%%%%%%%%%%
% the textbackgournd will extends to the page boundary,
% which include the footnote and not as expected
% workaround is:
\chardef\kindofpagetextareas\plusone

% the footnote can't process linebreak correctly
% workaround is:
\setupfootnotes[style={\setscript[hanzi]}]

% \postponenotes\footnote{...}

\define[1]\todo{#1}
%\define[1]\todo{\error}
\stopenvironment

