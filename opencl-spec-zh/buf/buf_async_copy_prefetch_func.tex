% async_work_group_copy
\startbuffer[funcproto:async_work_group_copy]
event_t async_work_group_copy (
	__local gentype *dst,
	const __global gentype *src,
	size_t num_gentypes,
	event_t event)
event_t async_work_group_copy (
	__global gentype *dst,
	const __local gentype *src,
	size_t num_gentypes,
	event_t event)
\stopbuffer
\startbuffer[funcdesc:async_work_group_copy]
從 \carg{src} 異步拷貝 \carg{num_gentypes} 個 \ctype{gentype} 元素
到 \carg{dst} 中。
\cnglo{workgrp}中的所有\cnglo{workitem}都會實施此異步拷貝,
因此在\cnglo{workgrp}中,
使用相同引數值執行\cnglo{kernel}的所有\cnglo{workitem}必須都能執行到此函式,
否則結果未定義。

返回的\cnglo{evtobj}可由 \capi{wait_group_events} 用來等待異步拷貝完畢。
可以使用引數 \carg{event} 將 \capi{async_work_group_copy} 與之前的異步拷貝關聯在一起,
從而使得多個異步拷貝間可共享同一事件;否則 \carg{event} 必須是零。

如果引數 \carg{event} 非零,則會將其中的\cnglo{evtobj}返回。

此函式不會對源數據實施隱式同步,如在拷貝前執行 \capi{barrier}。
\stopbuffer

% async_work_group_strided_copy
\startbuffer[funcproto:async_work_group_strided_copy]
event_t async_work_group_strided_copy (
	__local gentype *dst,
	const __global gentype *src,
	size_t num_gentypes,
	size_t src_stride,
	event_t event)
event_t async_work_group_strided_copy (
	__global gentype *dst,
	const __local gentype *src,
	size_t num_gentypes,
	size_t dst_stride,
	event_t event)
\stopbuffer
\startbuffer[funcdesc:async_work_group_strided_copy]
從 \carg{src} 異步採集 \carg{num_gentypes} 個 \ctype{gentype} 元素
到 \carg{dst} 中。
參數 \carg{src_stride} 為從 \carg{src} 中讀取元素時所用跨距。
參數 \carg{dst_stride} 為將元素寫入 \carg{dst} 中時所用跨距。
\cnglo{workgrp}中的所有\cnglo{workitem}都會實施此異步採集,
因此在\cnglo{workgrp}中,
使用相同引數值執行\cnglo{kernel}的所有\cnglo{workitem}必須都能執行到此函式,
否則結果未定義。

返回的\cnglo{evtobj}可由 \capi{wait_group_events} 用來等待異步拷貝完畢。
可以使用引數 \carg{event} 將 \capi{async_work_group_strided_copy} 與之前的異步拷貝
關聯在一起,
從而使得多個異步拷貝間可共享同一事件;
否則 \carg{event} 必須是零。

如果引數 \carg{event} 非零,則會將其中的\cnglo{evtobj}返回。

此函式不會對源數據實施隱式同步,如在拷貝前執行 \capi{barrier}。

如果 \carg{src_stride} 或 \carg{dst_stride} 是 0,
或者拷貝時, \carg{src_stride} 或 \carg{dst_stride} 使得
 \carg{src} 或 \carg{dst} 指位器超過了位址空間的上界,
則 \capi{async_work_group_strided_copy} 的行為未定義。
\stopbuffer

% wait_group_events
\startbuffer[funcproto:wait_group_events]
void wait_group_events (
	int num_events,
	event_t *event_list)
\stopbuffer
\startbuffer[funcdesc:wait_group_events]
等待用來表示 \capi{async_work_group_copy} 操作完成的事件。
實施等待後會釋放 \carg{event_list} 中的\cnglo{evtobj}。
對於某個\cnglo{workgrp}中的\cnglo{workitem}而言,
如果執行\cnglo{kernel}時
用的 \carg{num_events} 以及 \carg{event_list} 中的\cnglo{evtobj}一樣,
則它們必須都能執行到此函式;
否則結果未定義。
\stopbuffer

% prefetch
\startbuffer[funcproto:prefetch]
void prefetch (
	const __global gentype *p,
	size_t num_gentypes)
\stopbuffer
\startbuffer[funcdesc:prefetch]
預取 \math{\marg{num_gentypes}
 \times \mapiemp{sizeof}(\mtype{gentype})} 字節到全局緩存中。
預取指令會作用到\cnglo{workgrp}中的\cnglo{workitem}上,
不會影響\cnglo{kernel}的功能性行為。
\stopbuffer
