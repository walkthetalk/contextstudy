\section[sec:pp_macro]{預處理指示和巨集}

支持 C99 規範中定義的預處理指示。

其中是這樣描述 \cdrtemp{# pragma} 指示的:
\startclc
/BTEX\ftEmp{\# pragma}/ETEX /BTEX{\ftRef{pp-tokens\low{opt} new-line}}/ETEX
\stopclc

在進行巨集代換(macro replacement)之前,如果在指令 \cdrtemp{# pragma} 中,
 \cdrtemp{pragma} 後面緊跟的不是預處理符記(token) {\ftEmp{OpenCL}}
(用來取代 {\ftEmp{STDC}}),
則其行為\cnglo{impdef}。
此行為可能會導致翻譯失敗,也可能導致翻譯器或最終\cnglo{program}的行為不符合規範。
如果實作不能識別這樣的 \cdrtemp{pragma},則將其忽略。
在進行巨集代換之前,如果 \cdrtemp{pragma} 後面緊跟的是預處理符記 {\ftEmp{OpenCL}},
則不會在此指示之上實施任何巨集代換,並且此指示只能是下列形式之一(其意義在別處有描述):
\startclc
/BTEX\ftEmp{\#pragma OPENCL FP_CONTRACT}/ETEX /BTEX\ftRef{on-off-switch}/ETEX
	/BTEX\ftRef{on-off-switch}/ETEX: one of /BTEX\ftEmp{ON OFF DEFAULT}/ETEX

/BTEX\ftEmp{\#pragma OPENCL EXTENSION}/ETEX extensionname : behavior
/BTEX\ftEmp{\#pragma OPENCL EXTENSION}/ETEX all : behavior
\stopclc

可用的預定義巨集的名字如下所示:

\cmacro{__FILE__}
當前源檔的名字(常值字串)。

\cmacro{__LINE__}
當前源檔中當前源碼行的行號(整形常數)。

\cmacro{__OPENCL_VERSION__}
一個整數,用來反映 OpenCL \cnglo{device} 所支持的 OpenCL 規範的版本號。
如果是本文所述的 OpenCL,則 \cmacro{__OPENCL_VERSION__} 所對應的整數是 120。

\cmacro{CL_VERSION_1_0}
整數 100,反映的是 OpenCL 1.0 規範。

\cmacro{CL_VERSION_1_1}
整數 110,反映的是 OpenCL 1.1 規範。

\cmacro{CL_VERSION_1_2}
整數 120,反映的是 OpenCL 1.2 規範。

\cmacro{__OPENCL_C_VERSION__}
一個整數,用來反映 OpenCL C 的版本,
此版本號由 \capi{clBuildProgram} 或 \capi{clCompileProgram}
 所用構建選項 \ccmm{-cl-std} 來指定。
如果沒有指定此構建選項,
則使用這個 OpenCL \cnglo{device} 所用編譯器所支持的 OpenCL C 的版本。
如果是本文所述的 OpenCL C,則 \cmacro{__OPENCL_C_VERSION__} 所對應的整數是 120。

\cmacro{__ENDIAN_LITTLE__}
用來確定 OpenCL \cnglo{device} 的架構是大端的還是小端的(
如果是小端的,則為 1,否則未定義)。
參見\reftab{cldevquery}中的 \cenum{CL_DEVICE_ENDIAN_LITTLE}。

\ccmm{__kernel_exec(X, }\cldt[n]{type}\ccmm{)}
(和 \ccmm{kernel_exec(X, }\cldt[n]{type}\ccmm{)})
定義為:
\startclc
__kernel __attribute__((work_group_size_hint(X, 1, 1))) \
	 __attribute__((vec_type_hint(type/BTEX\ftRef{n}/ETEX)))
\stopclc

\cmacro{__IMAGE_SUPPORT__}
用來確定 OpenCL \cnglo{device}是否支持圖像。
如果支持圖像則為整形常數 1,否則未定義。
參見\reftab{cldevquery}中的 \cenum{CL_DEVICE_IMAGE_SUPPORT}。

\cmacro{__FAST_RELAXED_MATH__}
用來確定給 \capi{clBuildProgram} 或 \capi{clCompileProgram}
 所指定的構建選項中是否有優化選項 \ccmm{-cl-fast-relaxed-math}。
如果有則為整形常數 1,否則未定義。

對於那些 C99 規範中有定義,但目前 OpenCL 還不支持的巨集名字,全部保留,以備將來使用。

預定義的標識符 \cmacro{__func__} 可用。

