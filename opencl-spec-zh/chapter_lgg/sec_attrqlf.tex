\section{特性限定符}

本節描述 \cqlf{__attribute__} 的相關語法以及他所綁定的構件。

特性限定符的形式為 \ccmm{__attribute__ ((attribute-list))}。

特性列定義如下:
\startclc
attribute-list:
	attribute/BTEX\low{opt}/ETEX
	attribute-list , attribute/BTEX\low{opt}/ETEX

attribute:
	attribute-token attribute-argument-clause/BTEX\low{opt}/ETEX

attribute-token:
	identifier

attribute-argument-clause:
	( attribute-argument-list )

attribute-argument-list:
	attribute-argument
	attribute-argument-list, attribute-argument

attribute-argument:
	assignment-expression
\stopclc

此語法直接取自 GCC,但跟 GCC 又有不同。
在 GCC 中,特性只能用在函式、型別和變量上,而 OpenCL 特性可與下列項相關聯:
\startigBase
\item 型別
\item 函式
\item 變量
\item 塊
\item 控制流語句
\stopigBase

通常,對於在給定的上下文中如何綁定特性,其規則有很多值得仔細研究的地方,
關於細節讀者可以參考 GCC 的文檔以及 Maurer 和 Wong 的論文(
\todo{第 11 章參考文獻中的第16項和第17項)}。

% specifying attributes of types
\subsection{用於型別的特性}

% Specifying Attributes of Functions
\subsection{用於函式的特性}

% Specifying Attributes of Variables
\subsection{用於變量的特性}

% Specifying Attributes of Blocks and Control-Flow-Statements
\subsection{用於塊和控制流語句的特性}

% Extending Attribute Qualifiers
\subsection{對特性限定符的擴展}

