\section{特性限定符}

本節描述 \cqlf{__attribute__} 的相關語法以及他所綁定的構件。

特性限定符的形式為 \ccmm{__attribute__ ((attribute-list))}。

特性列定義如下:
\startclc
attribute-list:
	attribute/BTEX\low{opt}/ETEX
	attribute-list , attribute/BTEX\low{opt}/ETEX

attribute:
	attribute-token attribute-argument-clause/BTEX\low{opt}/ETEX

attribute-token:
	identifier

attribute-argument-clause:
	( attribute-argument-list )

attribute-argument-list:
	attribute-argument
	attribute-argument-list, attribute-argument

attribute-argument:
	assignment-expression
\stopclc

此語法直接取自 GCC,但跟 GCC 又有不同。
在 GCC 中,特性只能用在函式、型別和變量上,而 OpenCL 特性可與下列項相關聯:
\startigBase
\item 型別
\item 函式
\item 變量
\item 塊
\item 控制流語句
\stopigBase

通常,對於在給定的上下文中如何綁定特性,其規則有很多值得仔細研究的地方,
關於細節讀者可以參考 GCC 的文檔以及 Maurer 和 Wong 的論文(
\todo{第 11 章參考文獻中的第16項和第17項)}。

% specifying attributes of types
\subsection{用於型別的特性}

定義 \ckey{enum}、 \ckey{struct} 和 \ckey{union} 型別時,
可以用關鍵字 \cqlf{__atrribute__} 為其指定某些特性。
此關鍵字後緊跟用雙層括號括起來的特性規格。
目前型別有兩種特性: \ccmm{aligned} 和 \ccmm{packed}。

在聲明或定義 \ckey{enum}、 \ckey{struct} 或 \ckey{union} 時,
或者 \ckey{typedef} 其他型別時都可以指定型別特性。

對於 \ckey{enum}、 \ckey{struct} 和 \ckey{union} 型別,
特性限定符可以位於標籤 \ckey{enum}、 \ckey{struct} 或 \ckey{union} 和型別名字中間,
也可以跟在右大括號之後。優選前者。

\startclOption{\ccmm{aligned (}\carg{alignment}\ccmm{)}}
此特性用來指定某種型別變量的最小對齊字節數。
\startclc
struct S { short f[3]; } __attribute__ ((aligned (8)));
typedef int more_aligned_int __attribute__ ((aligned (8)));
\stopclc

迫使編譯器保證(盡其所能)
在分配型別為 \ccmm{struct S} 或 \ccmm{more_aligned_int} 的變量時
{\ftEmp{至少}}按 8 字節邊界進行對齊。

需要注意的是, ISO C 標準要求任一給定的 \ckey{struct} 或 \ckey{union} 型別的齊位
都至少要是其所有成員齊位(alignment)的最小公倍數的整數倍,同時必須是二的冪。
這意味着對於 \ckey{struct} 或 \ckey{union} 型別,
你{\ftEmp{可以}}通過給其任一成員附加特性 \ccmm{aligned}
 來有效的調整整個 \ckey{struct} 或 \ckey{union} 的齊位。
不過如果要想調整整個 \ckey{struct} 或 \ckey{union} 型別的齊位,
上面示例所演示的方式無疑更明顯、更直觀、可讀性更好。
\stopclOption

% Specifying Attributes of Functions
\subsection{用於函式的特性}

% Specifying Attributes of Variables
\subsection{用於變量的特性}

% Specifying Attributes of Blocks and Control-Flow-Statements
\subsection{用於塊和控制流語句的特性}

% Extending Attribute Qualifiers
\subsection{對特性限定符的擴展}

