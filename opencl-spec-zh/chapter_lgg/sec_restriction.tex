\section[sec:restrictions]{限制\footnote{
加了刪除線的項是 OpenCL 1.0 中的限制,但在 OpenCL 1.1 中已經移除。}}

\startigBig

% pointer
\startitem
對於指針的使用有一點限制。規定如下:
\startigBig

\startitem
對於\cnglo{kernel}函式,\cnglo{program}中聲明其參數時必須帶有限定符
 \cqlf{__global}、 \cqlf{__constant} 或 \cqlf{__local}。
\stopitem

\startitem
兩個指針相互賦值時,在聲明他們時其限定符必須同為
 \cqlf{__global}、 \cqlf{__constant} 或 \cqlf{__local}。
\stopitem

\startitem
不允許指針指向函式。
\stopitem

\startitem
在\cnglo{program}中,\cnglo{kernel}函式的參數不能聲明為指針的指針。
但是函式內部的變量或者非\cnglo{kernel}函式的參數可以是指針的指針。
\stopitem

\stopigBig
\stopitem

% image
\startitem
只有函式參數的型別才能是圖像(image2d_t, image3d_t, image2d_array_t, image1d_t,
image1d_buffer_t or image1d_array_t)。
不能修改這種參數。
並且只能用內建函式來訪問其元素(參見\todo{節 6.12.14})。

圖像型別不能用來聲明變量、結構體或聯合體的欄位,
也不能聲明圖像陣列或圖像指針,
函式返回值的型別也不能是圖像。
圖像型別不能與位址空間限定符
 \cqlf{__private}、 \cqlf{__local} 和 \cqlf{__constant} 一起使用。
型別 \ctype{image3d_t} 不能與訪問限定符 \cqlf{__write_only} 一起使用,
除非使能了擴展 \clext{cl_khr_3d_image_writes}。
圖像型別不能與訪問限定符 \cqlf{__read_write} 一起使用,
這種用法暫時保留,將來可能會使用。

\cnglo{sampler}型別(\ctype{sampler_t})只能作為函式參數的型別,
或者用來聲明作用域為整個\cnglo{program}或\cnglo{kernel}函式最外層的變量。
如果\cnglo{kernel}函式中所聲明的\cnglo{sampler}變量沒有位於最外層,
則其行為\cnglo{impdef}。
如果引數或變量是\cnglo{sampler},則不能修改他。

\cnglo{sampler}型別不能用來聲明結構體或聯合體的欄位,
也不能聲明\cnglo{sampler}陣列或\cnglo{sampler}指針,
函式返回值的型別也不能是\cnglo{sampler}。
\cnglo{sampler}型別不能與位址空間限定符 \cqlf{__local} 或 \cqlf{__global} 一起使用。
\stopitem

% bit-field
\startitem
不支持結構體中的位欄(bit-field)成員。
\stopitem

% variable length array
\startitem
不支持變長陣列和帶有彈性(沒有指明大小的)陣列的結構體。
\stopitem

% variadic
\startitem
不支持變參巨集和變參函式。
\stopitem

% library function
\startitem
\cnglo{program}中不能包含下列 C99 標準頭檔,也不能使用其中的庫函式:
 \ccmm{assert.h}、 \ccmm{ctype.h}、 \ccmm{complex.h}、 \ccmm{errno.h}、
 \ccmm{fenv.h}、 \ccmm{float.h}、 \ccmm{inttypes.h}、 \ccmm{limits.h}、
 \ccmm{locale.h}、 \ccmm{setjmp.h}、 \ccmm{signal.h}、 \ccmm{stdarg.h}、
 \ccmm{stdio.h}、 \ccmm{stdlib.h}、 \ccmm{string.h}、 \ccmm{tgmath.h}、
 \ccmm{time.h}、 \ccmm{wchar.h} 和 \ccmm{wctype.h}。
\stopitem

% auto and register
\startitem
不支持存儲類別限定符 \cqlf{auto} 和 \cqlf{register}。
\stopitem

% predefined identifier
\overstrike{
\startitem
不支持預定義標識符。
\stopitem
}

% recursion
\startitem
不支持遞迴(recursion)。
\stopitem

% __kernel function
\startitem
源碼中帶有限定符 \cqlf{__kernel} 的函式所返回的型別只能是 \ctype{void}。
\stopitem

% scalar argument
\startitem
\cnglo{kernel}函式參數的型別不能是內建標量型別
 \ctype{bool}、 \ctype{half}、 \ctype{size_t}、 \ctype{ptrdiff_t}、
 \ctype{intptr_t} 和 \ctype{uintptr_t},
如果是結構體或聯合體,其中的欄位也不能是這些標量型別。
這些型別的大小(除了 \ctype{half})\cnglo{impdef},
而且在 OpenCL \cnglo{device} 和 \cnglo{host} 處理器上也可能不同,
這樣如果\cnglo{kernel}參數聲明為這些型別的指針,就很難為這樣的參數分配\cnglo{bufobj}。
這裡不支持 \ctype{half},是因為他僅僅是一種存儲格式\footnote{
除非支持擴展 \clext{cl_khr_fp16}。},
不能作為一種數據型別來參與算術運算。
\stopitem

% irreducible control flow
\startitem
不可規約的控制流(irreducible control flow)是否違規\cnglo{impdef}。
\stopitem

% built-in types less than 32 bits
\overstrike{
\startitem
對於大小小於 32 位的內建型別,即
 \ctype{char}、 \ctype{uchar}、 \ctype{char2}、 \ctype{uchar2}、
 \ctype{short}、 \ctype{ushort} 和 \ctype{half},有如下限制:
\startigBig
\startitem
不能對
 \ctype{char}、 \ctype{uchar}、 \ctype{char2}、 \ctype{uchar2}、
 \ctype{short}、 \ctype{ushort} 和 \ctype{half} 的指針(或陣列)
或者型別為
 \ctype{char}、 \ctype{uchar}、 \ctype{char2}、 \ctype{uchar2}、
 \ctype{short} 或 \ctype{ushort} 的結構體成員進行寫操作。
參見節 9.9。
\stopitem
\stopigBig

下面的\cnglo{kernel}示例中展示了少於 32 位的內建型別不支持哪些內存操作。
\startclc
kernel void
do_proc (__global char *pA, short b, __global short *pB)
{
	char		x[100];
	__private char	*px = x;
	int		id = (int)get_global_id(0);
	short		f;

	f = pB[id] + b;	<- is allowed

	px[1] = pA[1];	<- error. px cannot be written.

	pB[id] = b;	<- error. pB cannot be written.
}

\stopclc
\stopitem
}

% event_t
\startitem
\cnglo{program}中\cnglo{kernel}函式引數的型別不能是 \ctype{event_t}。
\stopitem

% address space
\startitem
結構體或聯合體的元素必須屬於同一個位址空間中,否則違規。
\stopitem

% struct / union's elements as OpenCL object
\startitem
如果\cnglo{kernel}函式的引數是結構體或聯合體,
其元素不能是 OpenCL 對象。
\stopitem

% type qualifier
\startitem
支持 C99 中定義的型別限定符 \cqlf{const}、 \cqlf{restrict} 和 \cqlf{volatile}。
這些限定符不能與下列型別一起使用:
 \ctype{image2d_t}、 \ctype{image3d_t}、 \ctype{image2d_array_t}、
 \ctype{image1d_t}、 \ctype{image1d_buffer_t}、 \ctype{image1d_array_t}。
非指針型別不能使用限定符 \cqlf{restrict}。
\stopitem

% event_t
\startitem
\cnglo{kernel}函式引數的型別不能是事件(\ctype{event_t})。
事件型別不能用來聲明作用域為整個\cnglo{program}的變量。
事件型別也不能用來聲明結構體或聯合體的欄位。
事件型別不能與位址空間限定符
 \cqlf{__local}、 \cqlf{__constant} 或 \cqlf{__global} 一起使用。
\stopitem

\stopigBig

