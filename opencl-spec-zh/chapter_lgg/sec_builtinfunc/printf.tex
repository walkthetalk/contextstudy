\subsection{printf}

OpenCL C 編程語言還實現有 \capi{printf} 函式。

\placetable[here,force,split][tab:printfFunc]
{內建 printf 函式}
{\input{chapter_lgg/tbl/tbl_printffunc.tex}}

\subsubsection{printf 輸出同步}

當特定\cnglo{kernel}所關聯的事件完成後,
此\cnglo{kernel}調用 \capi{printf} 得到的輸出會刷入實作所定義的輸出數據流。
在\cnglo{cmdq}上調用 \capi{clFinish} 會將所有擱置的 \capi{printf} 輸出
(由之前入隊並完成的\cnglo{cmd}產生)刷入實作所定義的輸出數據流。
在多個\cnglo{workitem}並發執行 \capi{printf} 的情況下,寫入數據的順序沒有任何保證。
例如,\cnglo{glbid}為 \math{(0,0,1)}的\cnglo{workitem}的輸出
與\cnglo{glbid}為 \math{(0,0,4)}的\cnglo{workitem}的輸出很可能混雜在一起。

\subsubsection{printf 格式字串}

格式是一個字符序列,其開始和結束均位於其初始轉義狀態(shift state)下。
此格式可能包含零個或多個指示:普通字符(非 \%),不作改變直接拷貝到輸出數據流中;
以及轉換規約,每個都會導致讀取零個或多個後續引數,
可能還會根據對應的轉換限定符將其轉換(如果可以的話),然後將結果寫入輸出數據流。
格式字串必須位於 \cqlf{constant} 位址空間中,因此必須在編譯時即可確定,
不能由可執行程序動態創建。
