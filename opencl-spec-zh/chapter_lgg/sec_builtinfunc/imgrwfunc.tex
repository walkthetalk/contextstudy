% Image Read and Write Functions
\subsection{圖像讀寫函式}

本節中所定義的內建函式僅能與\cnglo{imgobj}一起使用。

聲明可被\cnglo{kernel}讀取的\cnglo{imgobj}時應帶有限定符 \cqlf{__read_only}。
對帶有 \cqlf{__read_only} 的\cnglo{imgobj}調用 \capi{write_image} 會造成編譯錯誤。
聲明可被\cnglo{kernel}寫入的\cnglo{imgobj}時應帶有限定符 \cqlf{__write_only}。
對帶有 \cqlf{__write_only} 的\cnglo{imgobj}調用 \capi{read_image} 會造成編譯錯誤。
不支持在同一\cnglo{kernel}中對同一\cnglo{imgobj}
同時調用 \capi{read_image} 和 \capi{write_image}。

\capi{read_image} 返回的是一個四組件矢量浮點數、整數或無符號整數顏色值。
此值用 \ccmm{x}、 \ccmm{y}、 \ccmm{z}、 \ccmm{w} 來標識,
其中 \ccmm{x} 指代紅色分量, \ccmm{y} 指代綠色分量, \ccmm{z} 指代藍色分量,
 \ccmm{w} 指代 alpha 分量,每個分量都是一個矢量組件。

% Samplers
\subsubsection{採樣器}

圖像讀取函式的引數中有一個就是\cnglo{sampler}。
\cnglo{sampler}可作為引數由 \capi{clSetKernelArg} 傳給\cnglo{kernel},
也可以在 \cqlf{kernel} 函式的最外層聲明\cnglo{sampler},
或者是\cnglo{program}源碼中聲明的型別為 \ctype{sampler_t} 的常量。

\cnglo{program}中所聲明\cnglo{sampler}變量的型別為 \ctype{sampler_t}。
這種變量必須用 32 位無符號整形常數進行初始化,
按位欄解釋此常量,其位欄指定了下列屬性:
\startigBase[indentnext=no]
\item 尋址模式
\item 濾波模式
\item 規範化坐標
\stopigBase
這些數學控制着 \capi{read_image{f|i|ui}} 如何讀取圖像中的元素。

也可在\cnglo{program}源碼中用如下幾種語法將\cnglo{sampler}聲明為全局常量:
\startclc
const sampler_t		<sampler name> = <value>
or
constant sampler_t	<sampler name> = <value>
or 
__constant sampler_t	<sampler_name> = <value>
\stopclc

注意:

在計算每個\cnglo{device}中指向常數位址空間的引數數目或常數位址空間的大小時,
不考慮帶有限定符 \cqlf{constant} 的\cnglo{sampler}(參見\reftab{cldevquery}中的
 \cenum{CL_DEVICE_MAX_CONSTANT_ARGS} 和
 \cenum{CL_DEVICE_MAX_CONSTANT_BUFFER_SIZE})。

\placetable[here,force,split][tab:samplerDesc]
{採樣器描述符}
{\startCLOD[\cnglo{sampler}屬性][描述]

\clOD{\ccmm{<normalized coords>}}{
指定所傳入的坐標 \math{x}、 \math{y} 和 \math{z} 是否已規範化。
他必須是常值,可以是下列預定義枚舉中的一個:
\startigBase
\item \cenum{CLK_NORMALIZED_COORDS_TRUE}
\item \cenum{CLK_NORMALIZED_COORDS_FALSE}
\stopigBase

在單個\cnglo{kernel}中針對同意圖像多次調用 \capi{read_image{f|i|ui}} 時,
所用\cnglo{sampler}中 \ccmm{<normalized coords>} 的值必須相同。
}

\clOD{\ccmm{<addressing mode>}}{
指定圖像的尋址模式,即圖像坐標溢出時如何處置。
他必須是常值,可以是下列預定義枚舉中的一個:
\startigBase
\item \cenum{CLK_ADDRESS_MIRRORED_REPEAT}——在整數接點處翻轉圖像坐標。
這種尋址模式只能用於規範化坐標。
如果使用的不是規範化坐標,則此模式生成的圖像坐標未定義。

\item \cenum{CLK_ADDRESS_REPEAT}——溢出的坐標會繞回到有效區間內。
這種尋址模式只能用於規範化坐標。
如果使用的不是規範化坐標,則此模式生成的圖像坐標未定義。

\item \cenum{CLK_ADDRESS_CLAMP_TO_EDGE}——溢出的坐標會被壓入有效範圍內。

\item \cenum{CLK_ADDRESS_CLAMP}\footnote{
與尋址模式\cenum{CLK_ADDRESS_CLAMP_TO_EDGE}類似。}——溢出的坐標會返回邊界的顏色。

\item \cenum{CLK_ADDRESS_NONE}——此模式下,由程序員保證坐標不會溢出,否則結果未定義。
\stopigBase

對於 1D 和 2D 圖像陣列,尋址模式僅對坐標 \math{x} 和 \math{(x,y)} 有效。
坐標中的陣列索引所用尋址模式始終是 \cenum{CLK_ADDRESS_CLAMP_TO_EDGE}。
}

\clOD{\ccmm{filter mode}}{
指定所用的濾波模式。
他必須是常值,可以是下列預定義枚舉中的一個:
\startigBase
\item \cenum{CLK_FILTER_NEAREST}
\item \cenum{CLK_FILTER_LINEAR}
\stopigBase

對於這些濾波模式的描述,請參見\todo{節 8.2}。
}

\stopCLOD

}

例:
\startclc[indentnext=no]
const sampler_t /BTEX\ftEmp{samplerA}/ETEX = CLK_NORMALIZED_COORDS_TRUE
			| CLK_ADDRESS_REPEAT
			| CLK_FILTER_NEAREST;
\stopclc
\cnglo{sampler} {\ftEmp{samplerA}} 用的是規格化坐標、重複尋址模式和最近濾波。

對於一個\cnglo{kernel}中所能聲明\cnglo{sampler}的最大數目,
可以用 \capi{clGetDeviceInfo} 以 \cenum{CL_DEVICE_MAX_SAMPLERS} 進行查詢。
