\subsection[sec:relationFunc]{關係函式}

可以使用關係算子和相等算子(<、 <=、 >、 >=、 !=、 ==)對內建標量和矢量型別進行關係運算,
所產生的結果分別為標量或矢量帶符號整形,參見\refsec{operator}。

\reftab{svRelationalFunc}中所列函式\footnote{
如果實作對規範進行了擴充,從而支持 IEEE-754 标志和異常,
則當有一個或多個算數是 NaN 時,
\reftab{svRelationalFunc}中所定義的內建函式不會引發{\ftRef{無效(invalid)}}浮點異常。}
可以內建標量或矢量型別為引數,返回的結果為標量或矢量整形。
泛型 \ctype{gentype} 指代下列內建型別:
 \cldt{char}、 \cldt[n]{char}、 \cldt{uchar}、 \cldt[n]{uchar}、
 \cldt{short}、 \cldt[n]{short}、 \cldt{ushort}、 \cldt[n]{ushort}、
 \cldt{int}、 \cldt[n]{int}、 \cldt{uint}、 \cldt[n]{uint}、
 \cldt{long}、 \cldt[n]{long}、 \cldt{ulong}、 \cldt[n]{ulong}、
 \cldt{float}、 \cldt[n]{float}、 \cldt{double} 和 \cldt[n]{double}。
泛型 \ctype{igentype} 指代內建帶符號整形,即:
 \cldt{char}、 \cldt[n]{char}、 \cldt{short}、 \cldt[n]{short}、
 \cldt{int}、 \cldt[n]{int}、 \cldt{long} 和 \cldt[n]{long}。
泛型 \ctype{ugentype} 指代內建無符號整形,即:
 \cldt{uchar}、 \cldt[n]{uchar}、 \cldt{ushort}、 \cldt[n]{ushort}、
 \cldt{uint}、 \cldt[n]{uint}、 \cldt{ulong} 和 \cldt[n]{ulong}。
其中 \ccmmsuffix{n} 為 2、 3、 4、 8 或 16。

對於標量型別的引數,如果所指定的關係為 {\ftRef{false}},則\reftab{svRelationalFunc}中的函式
 \capi{isequal}、 \capi{isnotequal}、 \capi{isgreater}、 \capi{isgreaterequal}、
 \capi{isless}、 \capi{islessequal}、 \capi{islessgreater}、 \capi{isfinite}、
 \capi{isinf}、 \capi{isnan}、 \capi{isnormal}、 \capi{isordered}、
 \capi{isunordered} 和 \capi{signbit} 會返回 0,否則返回 1。
而對於矢量型別的引數,如果所指定的關係為 {\ftRef{false}},則返回 0,
否則返回 -1 (即所有位都是 1)。

如果任一引數為 NaN,則關係函式
 \capi{isequal}、 \capi{isgreater}、 \capi{isgreaterequal}、
 \capi{isless}、 \capi{islessequal} 和 \capi{islessgreater}
返回 0。
如果引數為標量,則當任一引數為 NaN 時, \capi{isnotequal} 返回 1;
而如果引數為矢量,則當任一引數為 NaN 時, \capi{isnotequal} 返回 -1。

\placetable[here,force,split][tab:svRelationalFunc]
{標量和矢量關係函式}
{% isequal
\startbuffer[funcproto:isequal]
int isequal (float x, float y)
intn isequal (floatn x, floatn y) 
int isequal (double x, double y)
longn isequal (doublen x, doublen y)
\stopbuffer
\startbuffer[funcdesc:isequal]
按組件逐一比較 \math{x == y}。
\stopbuffer

% isnotequal
\startbuffer[funcproto:isnotequal]
int isnotequal (float x, float y)
intn isnotequal (floatn x, floatn y)
int isnotequal (double x, double y)
longn isnotequal (doublen x, doublen y)
\stopbuffer
\startbuffer[funcdesc:isnotequal]
按組件逐一比較 \math{x != y}。
\stopbuffer

% isgreater
\startbuffer[funcproto:isgreater]
int isgreater (float x, float y)
intn isgreater (floatn x, floatn y)
int isgreater (double x, double y)
longn isgreater (doublen x, doublen y)
\stopbuffer
\startbuffer[funcdesc:isgreater]
按組件逐一比較 \math{x > y}。
\stopbuffer

% isgreaterequal
\startbuffer[funcproto:isgreaterequal]
int isgreaterequal (float x, float y)
intn isgreaterequal (floatn x, floatn y)
int isgreaterequal (double x,
		double y)
longn isgreaterequal (doublen x,
		doublen y)
\stopbuffer
\startbuffer[funcdesc:isgreaterequal]
按組件逐一比較 \math{x \geq y}。
\stopbuffer

% isless
\startbuffer[funcproto:isless]
int isless (float x, float y)
intn isless (floatn x, floatn y)
int isless (double x, double y)
longn isless (doublen x, doublen y)
\stopbuffer
\startbuffer[funcdesc:isless]
按組件逐一比較 \math{x < y}。
\stopbuffer

% islessequal
\startbuffer[funcproto:islessequal]
int islessequal (float x, float y)
intn islessequal (floatn x, floatn y)
int islessequal (double x, double y)
longn islessequal (doublen x, doublen y)
\stopbuffer
\startbuffer[funcdesc:islessequal]
按組件逐一比較 \math{x \leq y}。
\stopbuffer

% islessgreater
\startbuffer[funcproto:islessgreater]
int islessgreater (float x, float y)
intn islessgreater (floatn x, floatn y)
int islessgreater (double x, double y)
longn islessgreater (doublen x, doublen y)
\stopbuffer
\startbuffer[funcdesc:islessgreater]
按組件逐一比較 \math{(x < y) || (x > y)}。
\stopbuffer

% isfinite
\startbuffer[funcproto:isfinite]
int isfinite (float)
intn isfinite (floatn)
int isfinite (double)
longn isfinite (doublen)
\stopbuffer
\startbuffer[funcdesc:isfinite]
測試引數是否為有限值。
\stopbuffer

% isinf
\startbuffer[funcproto:isinf]
int isinf (float)
intn isinf (floatn)
int isinf (double)
longn isinf (doublen)
\stopbuffer
\startbuffer[funcdesc:isinf]
測試引數是否為無限值(正數或負數)。
\stopbuffer

% isnan
\startbuffer[funcproto:isnan]
int isnan (float)
intn isnan (floatn)
int isnan (double)
longn isnan (doublen)
\stopbuffer
\startbuffer[funcdesc:isnan]
測試引數是否為 NaN。
\stopbuffer

% isnormal
\startbuffer[funcproto:isnormal]
int isnormal (float)
intn isnormal (floatn)
int isnormal (double)
longn isnormal (doublen)
\stopbuffer
\startbuffer[funcdesc:isnormal]
測試引數是否為規格化值。
\stopbuffer

% isordered
\startbuffer[funcproto:isordered]
int isordered (float x, float y)
intn isordered (floatn x, floatn y)
int isordered (double x, double y)
longn isordered (doublen x, doublen y)
\stopbuffer
\startbuffer[funcdesc:isordered]
測試引數是否規則。
相當於 \math{\text{\capi{isequal}}(x, x) \text{&&} \text{\capi{isequal}}(y, y)}。
\stopbuffer

% isunordered
\startbuffer[funcproto:isunordered]
int isunordered (float x, float y)
intn isunordered (floatn x, floatn y)
int isunordered (double x, double y)
longn isunordered (doublen x, doublen y)
\stopbuffer
\startbuffer[funcdesc:isunordered]
測試引數是否不規則。
如果引數 \carg{x} 或 \carg{y} 是 NaN,則返回非零值,否則返回零。
\stopbuffer

% signbit
\startbuffer[funcproto:signbit]
int signbit (float)
intn signbit (floatn)
int signbit (double)
longn signbit (doublen)
\stopbuffer
\startbuffer[funcdesc:signbit]
測試符號位。
對於此函式的標量版本,如果設置了符號位,則返回 1,否則返回 0。
而在此函式的標量版本中,對於矢量的每個組件,
如果設置了符號位則返回 -1 (即所有位都是 1),否則返回 0。
\stopbuffer

% any
\startbuffer[funcproto:any]
int any (igentype x)
\stopbuffer
\startbuffer[funcdesc:any]
如果 \math{x} 中任一組件的最高位是 1,則返回 1;否則返回 0。
\stopbuffer

% all
\startbuffer[funcproto:all]
int all (igentype x)
\stopbuffer
\startbuffer[funcdesc:all]
如果 \math{x} 中所有組件的最高位都是 1,則返回 1;否則返回 0。
\stopbuffer

% bitselect
\startbuffer[funcproto:bitselect]
gentype bitselect (gentype a,
		gentype b,
		gentype c)
\stopbuffer
\startbuffer[funcdesc:bitselect]
如果 \math{c} 中的某一位為 0,則選取 \math{a} 中的對應位作為結果中對應位的值;
否則選取 \math{b} 中的對應位作為結果中對應位的值。
\stopbuffer

% select
\startbuffer[funcproto:select]
gentype select (gentype a,
		gentype b,
		igentype c)
gentype select (gentype a,
		gentype b,
		ugentype c)
\stopbuffer
\startbuffer[funcdesc:select]
對於矢量型別中的每個組件,
如果 \math{c[i]} 的最高位為 1,則結果為 \math{b[i]},否則為 \math{a[i]}。

對於標量型別,\math{result = c ? b : a}。

\ctype{igentype} 和 \ctype{ugentype} 的元素數目以及元素的位數
都必須與 \ctype{gentype} 相同。
\stopbuffer

% begin table
\startCLFD

\clFD{isequal}
\clFD{isnotequal}
\clFD{isgreater}
\clFD{isgreaterequal}
\clFD{isless}
\clFD{islessequal}
\clFD{islessgreater}
\clFD{isfinite}
\clFD{isinf}
\clFD{isnan}
\clFD{isnormal}
\clFD{isordered}
\clFD{isunordered}
\clFD{signbit}
\clFD{any}
\clFD{all}
\clFD{bitselect}
\clFD{select}

\stopCLFD
}

