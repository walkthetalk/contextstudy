\subsection{關係函式}

可以使用關係算子和相等算子(<、 <=、 >、 >=、 !=、 ==)對內建標量和矢量型別進行關係運算,
所產生的結果分別為標量或矢量帶符號整形,參見\refsec{operator}。

\reftab{svRelationalFunc}中所列函式\footnote{
}可以內建標量或矢量型別為引數,返回的結果為標量或矢量整形。
泛型 \ctype{gentype} 指代下列內建型別:
 \cldts{char}、 \cldtv{char}、 \cldts{uchar}、 \cldtv{uchar}、
 \cldts{short}、 \cldtv{short}、 \cldts{ushort}、 \cldtv{ushort}、
 \cldts{int}、 \cldtv{int}、 \cldts{uint}、 \cldtv{uint}、
 \cldts{long}、 \cldtv{long}、 \cldts{ulong}、 \cldtv{ulong}、
 \cldts{float}、 \cldtv{float}、 \cldts{double} 和 \cldtv{double}。
泛型 \ctype{igentype} 指代內建帶符號整形,即:
 \cldts{char}、 \cldtv{char}、 \cldts{short}、 \cldtv{short}、
 \cldts{int}、 \cldtv{int}、 \cldts{long} 和 \cldtv{long}。
泛型 \ctype{ugentype} 指代內建無符號整形,即:
 \cldts{uchar}、 \cldtv{uchar}、 \cldts{ushort}、 \cldtv{ushort}、
 \cldts{uint}、 \cldtv{uint}、 \cldts{ulong} 和 \cldtv{ulong}。
其中 \cldtvfix{n} 為 2、 3、 4、 8 或 16。

