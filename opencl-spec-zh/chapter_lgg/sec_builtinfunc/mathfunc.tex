% Math Functions
\subsection[sec:mathFunc]{數學函式}

\reftab{svMathFunc}中列出了內建的數學函式。
內建的數學函式分為兩種:
\startigBase
\item 第一種函式有兩個版本,一個版本的引數是標量,一個版本的引數是矢量;

\item 第二種函式只有一個版本,引數為標量浮點數。
\stopigBase

矢量版本的數學函式按組件逐一進行運算。
描述也是針對單個組件的。

無論調用環境中使用哪種捨入模式,
內建數學函式始終捨入為最近偶數,返回的結果也始終如一。

\reftab{svMathFunc}中所列函式即可接受標量引數,也可接受矢量引數。
泛型 \ctype{gentype} 表示函式引數的型別可以是
 \ctype{float}、 \ctype{float2}、 \ctype{float3}、 \ctype{float4}、
 \ctype{float8}、 \ctype{float16}、 \ctype{double}、 \ctype{double2}、
 \ctype{double3}、 \ctype{double4}、 \ctype{double8} 或 \ctype{double16}。
泛型 \ctype{gentypef} 表示函式引數的型別可以是
 \ctype{float}、 \ctype{float2}、 \ctype{float3}、 \ctype{float4}、
 \ctype{float8} 或 \ctype{float16}。
泛型 \ctype{gentyped} 表示函式引數的型別可以是
 \ctype{double}、 \ctype{double2}、 \ctype{double3}、 \ctype{double4}、
 \ctype{double8} 或 \ctype{double16}。
如果沒有特殊說明,函式的返回值與引數的型別都相同。

\placetable[here,force,split][tab:svMathFunc]
{引數既可為標量,也可為矢量的內建數學函式表}
{%% acos-related
% acos
\startbuffer[funcproto:acos]
gentype acos(gentype)
\stopbuffer
\startbuffer[funcdesc:acos]
反餘弦函數。
\stopbuffer

% acosh
\startbuffer[funcproto:acosh]
gentype acosh(gentype)
\stopbuffer
\startbuffer[funcdesc:acosh]
反雙曲餘弦函數。
\stopbuffer

% acospi
\startbuffer[funcproto:acospi]
gentype acospi(gentype)
\stopbuffer
\startbuffer[funcdesc:acospi]
計算 \math{acos(x) / \pi}。
\stopbuffer

%% asin-related
% asin
\startbuffer[funcproto:asin]
gentype asin(gentype)
\stopbuffer
\startbuffer[funcdesc:asin]
反正弦函數。
\stopbuffer

% asinh
\startbuffer[funcproto:asinh]
gentype asinh(gentype)
\stopbuffer
\startbuffer[funcdesc:asinh]
反雙曲正弦函數。
\stopbuffer

% asinpi
\startbuffer[funcproto:asinpi]
gentype asinpi(gentype)
\stopbuffer
\startbuffer[funcdesc:asinpi]
計算 \math{asin(x) / \pi}。
\stopbuffer

%% atan-related
% atan
\startbuffer[funcproto:atan]
gentype atan (gentype y_over_x)
\stopbuffer
\startbuffer[funcdesc:atan]
反正切函數。
\stopbuffer

% atan2
\startbuffer[funcproto:atan2]
gentype atan2 (gentype y,
		gentype x)
\stopbuffer
\startbuffer[funcdesc:atan2]
\math{y / x} 的反正切。
\stopbuffer

% atanh
\startbuffer[funcproto:atanh]
gentype atanh (gentype)
\stopbuffer
\startbuffer[funcdesc:atanh]
反雙曲正切函數。
\stopbuffer

% atanpi
\startbuffer[funcproto:atanpi]
gentype atanpi (gentype x)
\stopbuffer
\startbuffer[funcdesc:atanpi]
計算 \math{atan(x) / \pi}。
\stopbuffer

% atan2pi
\startbuffer[funcproto:atan2pi]
gentype atan2pi (gentype y,
		gentype x)
\stopbuffer
\startbuffer[funcdesc:atan2pi]
計算 \math{atan2(y,x) / \pi}。
\stopbuffer

% cbrt
\startbuffer[funcproto:cbrt]
gentype cbrt(gentype)
\stopbuffer
\startbuffer[funcdesc:cbrt]
計算立方根。
\stopbuffer

% ceil
\startbuffer[funcproto:ceil]
gentype ceil(gentype)
\stopbuffer
\startbuffer[funcdesc:ceil]
向正無窮捨入成整數值。
\stopbuffer

% copysign
\startbuffer[funcproto:copysign]
gentype copysign(gentype x,
		 gentype y)
\stopbuffer
\startbuffer[funcdesc:copysign]
將 \carg{x} 的符號改成 \carg{y} 的,並將其返回。
\stopbuffer

% cos
\startbuffer[funcproto:cos]
gentype cos(gentype)
\stopbuffer
\startbuffer[funcdesc:cos]
計算餘弦。
\stopbuffer

% cosh
\startbuffer[funcproto:cosh]
gentype cosh(gentype)
\stopbuffer
\startbuffer[funcdesc:cosh]
計算雙曲餘弦。
\stopbuffer

% cospi
\startbuffer[funcproto:cospi]
gentype cospi(gentype x)
\stopbuffer
\startbuffer[funcdesc:cospi]
計算 \math{cos(\pi x)}。
\stopbuffer

% erfc
\startbuffer[funcproto:erfc]
gentype erfc(gentype)
\stopbuffer
\startbuffer[funcdesc:erfc]
餘補誤差函數
 \math{1 - erf(x) = \frac{2}{\sqrt{\pi}} \intop^{\infty}_{x}e^{-\theta^2}d\theta}。
\stopbuffer

% erf
\startbuffer[funcproto:erf]
gentype erf(gentype)
\stopbuffer
\startbuffer[funcdesc:erf]
誤差函數,表示正態分布的積分
 \math{\frac{2}{\sqrt{\pi}} \intop^{\infty}_{x}e^{-\theta^2}d\theta}。
\stopbuffer

% exp
\startbuffer[funcproto:exp]
gentype exp(gentype x)
\stopbuffer
\startbuffer[funcdesc:exp]
計算 \math{e} 的 \carg{x} 次冪 \math{e^x}。
\stopbuffer

% exp
\startbuffer[funcproto:exp]
gentype exp(gentype x)
\stopbuffer
\startbuffer[funcdesc:exp]
計算 \math{e} 的 \carg{x} 次冪 \math{e^x}。
\stopbuffer

% exp2
\startbuffer[funcproto:exp2]
gentype exp2(gentype)
\stopbuffer
\startbuffer[funcdesc:exp2]
底數為 \math{2} 的冪。
\stopbuffer

% exp10
\startbuffer[funcproto:exp10]
gentype exp10(gentype)
\stopbuffer
\startbuffer[funcdesc:exp10]
底數為 \math{10} 的冪。
\stopbuffer

% expm1
\startbuffer[funcproto:expm1]
gentype expm1(gentype x)
\stopbuffer
\startbuffer[funcdesc:expm1]
計算 \math{e^x-1.0}。
\stopbuffer

% fabs
\startbuffer[funcproto:fabs]
gentype fabs(gentype)
\stopbuffer
\startbuffer[funcdesc:fabs]
計算浮點數的絕對值。
\stopbuffer

% fdim
\startbuffer[funcproto:fdim]
gentype fdim(gentype x,
	     gentype y)
\stopbuffer
\startbuffer[funcdesc:fdim]
如果 \math{x < y},返回 \math{x - y};否則返回 \math{+0}。
\stopbuffer

% floor
\startbuffer[funcproto:floor]
gentype floor(gentype)
\stopbuffer
\startbuffer[funcdesc:floor]
向負無窮捨入成整數值。
\stopbuffer

% fma
\startbuffer[funcproto:fma]
gentype fma(gentype a,
	    gentype b,
	    gentype c)
\stopbuffer
\startbuffer[funcdesc:fma]
返回 \math{a \times b + c},其中乘法具有無限精度即不會進行捨入,但會對加法進行正確地捨入。
邊界條件下的行為遵循 IEEE 754-2008 標準。
\stopbuffer

% fmax
\startbuffer[funcproto:fmax]
gentype fmax (gentype x, gentype y)
gentypef fmax (gentypef x, float y)
gentyped fmax (gentyped x, double y)
\stopbuffer
\startbuffer[funcdesc:fmax]
如果 \math{x < y},則返回 \math{y},否則返回 \math{x}。
如果一個引數是 NaN,則返回另一個引數;
如果兩個引數都是 NaN,則返回 NaN。
\stopbuffer
\stopbuffer

% fmin
\startbuffer[funcproto:fmin]
gentype fmin (gentype x, gentype y)
gentypef fmin (gentypef x, float y)
gentyped fmin (gentyped x, double y)
\stopbuffer
\startbuffer[funcdesc:fmin]
如果 \math{y < x},則返回 \math{y},否則返回 \math{x}。
如果一個引數是 NaN,則返回另一個引數;
如果兩個引數都是 NaN,則返回 NaN\footnote{
在處理 signaling NaN 時, \capi{fmin} 和 \capi{fmax} 的行為遵守 C99 中的定義,
但可能與 IEEE 754-2008 中定義的 \capi{minNum} 和 \capi{maxNum} 處理方式不同。
特別是,可能將 signaling NaN 當成 quiet NaN。}。
\stopbuffer

% fmod
\startbuffer[funcproto:fmod]
gentype fmod (gentyped x, gentype y)
\stopbuffer
\startbuffer[funcdesc:fmod]
模。返回 \math{x - y * trunc(x/y)}。
\stopbuffer

% fract
\startbuffer[funcproto:fract]
gentype fract (gentype x,
	__global gentype *iptr)
gentype fract (gentype x,
	__local gentype *iptr)
gentype fract (gentype x,
	__private gentype *iptr)
\stopbuffer
\startbuffer[funcdesc:fract]
返回 \math{\text{\capi{fmin}} (x - \text{\capi{floor}} ( x ), 0x1.fffffep - 1f}。
而 \math{\text{\capi{floor}}(x)} 將在 \carg{iptr} 中返回\footnote{
此處的 min() 是為了避免 \capi{fract}(-small) 返回 1.0。
有了 min(),這種情況就會返回小於 1.0 的最大正浮點數。}。
\stopbuffer

% frexp float
\startbuffer[funcproto:frexpf]
floatn frexp (floatn x,
	__global intn *exp)
floatn frexp (floatn x,
	__local intn *exp)
floatn frexp (floatn x,
	__private intn *exp)
float frexp (float x,
	__global int *exp)
float frexp (float x,
	__local int *exp)
float frexp (float x,
	__private int *exp)
\stopbuffer
\startbuffer[funcdesc:frexpf]
從 \carg{x} 中分離出尾數和指數。
返回的尾數(記為 m)型別為 \ctype{float},值為 0 或者屬於 \math{[1/2, 1)}。
\carg{x} 的每個組件都等於 \math{m \times 2^{exp}}。
\stopbuffer

% frexp double
\startbuffer[funcproto:frexpd]
doublen frexp (doublen x,
	__global intn *exp)
doublen frexp (doublen x,
	__local intn *exp)
doublen frexp (doublen x,
	__private intn *exp)
double frexp (double x,
	__global int *exp)
double frexp (double x,
	__local int *exp)
double frexp (double x,
	__private int *exp)
\stopbuffer
\startbuffer[funcdesc:frexpd]
從 \carg{x} 中分離出尾數和指數。
返回的尾數(記為 m)型別為 \ctype{double},值為 0 或者屬於 \math{[1/2, 1)}。
\carg{x} 的每個組件都等於 \math{m \times 2^{exp}}。
\stopbuffer

% hypot
\startbuffer[funcproto:hypot]
gentype hypot (gentype x, gentype y)
\stopbuffer
\startbuffer[funcdesc:hypot]
計算 \math{\sqrt{x^2+y^2}},不會有過分的上溢或下溢。
\stopbuffer

% ilogb
\startbuffer[funcproto:ilogb]
intn ilogb (floatn x)
int ilogb (float x)
intn ilogb (doublen x)
int ilogb (double x)
\stopbuffer
\startbuffer[funcdesc:ilogb]
返回整形對數。
\stopbuffer

% ldexp
\startbuffer[funcproto:ldexp]
floatn ldexp (floatn x, intn k)
floatn ldexp (floatn x, int k)
float ldexp (float x, int k)
doublen ldexp (doublen x, intn k)
doublen ldexp (doublen x, int k)
double ldexp (double x, int k)
\stopbuffer
\startbuffer[funcdesc:ldexp]
返回 \math{x \times 2^k}。
\stopbuffer

% lgamma
\startbuffer[funcproto:lgamma]
gentype lgamma (gentype x)
floatn lgamma_r (floatn x,
	__global intn *signp)
floatn lgamma_r (floatn x,
	__local intn *signp)
floatn lgamma_r (floatn x,
	__private intn *signp)
float lgamma_r (float x,
	__global int *signp)
float lgamma_r (float x,
	__local int *signp)
float lgamma_r (float x,
	__private int *signp)
doublen lgamma_r (doublen x,
	__global intn *signp)
doublen lgamma_r (doublen x,
	__local intn *signp)
doublen lgamma_r (doublen x,
	__private intn *signp)
double lgamma_r (double x,
	__global int *signp)
double lgamma_r (double x,
	__local int *signp)
double lgamma_r (double x,
	__private int *signp)
\stopbuffer
\startbuffer[funcdesc:lgamma]
返回伽馬函數絕對值的自然對數。
\capi{lgamma_r} 的引數 \carg{signp} 中會返回伽馬函數的符號。
 \math{ln|\Gamma(x)|}
\stopbuffer

% log
\startbuffer[funcproto:log]
gentype log (gentype)
\stopbuffer
\startbuffer[funcdesc:log]
計算自然對數。
\stopbuffer

% log2
\startbuffer[funcproto:log2]
gentype log2 (gentype)
\stopbuffer
\startbuffer[funcdesc:log2]
計算以 2 為底的對數。
\stopbuffer

% log10
\startbuffer[funcproto:log10]
gentype log10 (gentype)
\stopbuffer
\startbuffer[funcdesc:log10]
計算以 10 為底的對數。
\stopbuffer

% log1p
\startbuffer[funcproto:log1p]
gentype log1p (gentype x)
\stopbuffer
\startbuffer[funcdesc:log1p]
計算 \math{log_e(1.0+x)}。
\stopbuffer

% logb
\startbuffer[funcproto:logb]
gentype logb (gentype x)
\stopbuffer
\startbuffer[funcdesc:logb]
\math{log_r|x|}的整數部分。
\stopbuffer

% mad
\startbuffer[funcproto:mad]
gentype mad (gentype a,
	gentype b,
	gentype c)
\stopbuffer
\startbuffer[funcdesc:mad]
\capi{mad} 逼近 \math{a \times b + c}。
至於 \math{a \times b} 是否要捨入、怎樣捨入,
以及如何處理超常(supernormal)或次常(subnormal)的乘法都沒有定義。
在那些速度比準確更重要的地方,就可以使用 \capi{mad}\footnote{
需要提醒用戶的是,對於一些情況,
如 \math{\text{\capi{mad}}(a, b, -a\times b)}, \capi{mad} 定義的非常寬鬆,
以至於當 \carg{a} 和 \carg{b} 為某些特定值時,返回任何值都有可能}。
\stopbuffer

% maxmag
\startbuffer[funcproto:maxmag]
gentype maxmag (gentype x, gentype y)
\stopbuffer
\startbuffer[funcdesc:maxmag]
如果 \math{|x| > |y|},則返回 \carg{x};
如果 \math{|y| > |x|},則返回 \carg{y};
否則返回 \math{\text{\capi{fmax}}(x,y)}。
\stopbuffer

% minmag
\startbuffer[funcproto:minmag]
gentype minmag (gentype x, gentype y)
\stopbuffer
\startbuffer[funcdesc:minmag]
如果 \math{|x| < |y|},則返回 \carg{x};
如果 \math{|y| < |x|},則返回 \carg{y};
否則返回 \math{\text{\capi{fmin}}(x,y)}。
\stopbuffer

% modf
\startbuffer[funcproto:modf]
gentype modf (gentype x,
	__global gentype *iptr)
gentype modf (gentype x,
	__local gentype *iptr)
gentype modf (gentype x,
	__private gentype *iptr)
\stopbuffer
\startbuffer[funcdesc:modf]
分解浮點數,將引數 \carg{x} 分解為整數部分和小數部分,
兩部分的符號都與 \carg{x} 相同。
整數部分存儲在 \carg{iptr} 所指對象中。
\stopbuffer

% nan
\startbuffer[funcproto:nan]
floatn nan (uintn nancode)
float nan (uint nancode)
doublen nan (ulongn nancode)
double nan (ulong nancode)
\stopbuffer
\startbuffer[funcdesc:nan]
返回 quiet NaN。
其中將 \carg{nancode} 放在尾數的位置。
\stopbuffer

% nextafter
\startbuffer[funcproto:nextafter]
gentype nextafter (gentype x,
		gentype y)
\stopbuffer
\startbuffer[funcdesc:nextafter]
返回 \carg{x} 在往 \carg{y} 的方向上下一個可表示的單精度浮點值。
因此,如果 \carg{y} 小於 \carg{x},
則返回小於 \carg{x} 的最大的可表示的浮點數。
\stopbuffer

% pow
\startbuffer[funcproto:pow]
gentype pow (gentype x, gentype y)
\stopbuffer
\startbuffer[funcdesc:pow]
計算 \math{x^y}。
\stopbuffer

% pown
\startbuffer[funcproto:pown]
floatn pown (floatn x, intn y)
float pown (float x, int y)
doublen pown (doublen x, intn y)
double pown (double x, int y)
\stopbuffer
\startbuffer[funcdesc:pown]
計算 \math{x^y},其中 \carg{y} 是整數。
\stopbuffer

% powr
\startbuffer[funcproto:powr]
gentype powr (gentype x,
		gentype y)
\stopbuffer
\startbuffer[funcdesc:powr]
計算 \math{x^y},其中 \math{x\geq 0}。
\stopbuffer

% remainder
\startbuffer[funcproto:remainder]
gentype remainder (gentype x,
		gentype y)
\stopbuffer
\startbuffer[funcdesc:remainder]
返回值記為 \math{r},則 \math{r = x - n \times y}。
其中 \math{n} 是最接近精確值 \math{x/y} 的整數。
如果有兩個這樣的整數,則選擇偶數作為 \math{n}。
如果 \math{r} 是零,則符號與 \carg{x} 一樣。
\stopbuffer

% remquof
\startbuffer[funcproto:remquof]
floatn remquo (floatn x,
	floatn y,
	__global intn *quo)
floatn remquo (floatn x,
	floatn y,
	__local intn *quo)
floatn remquo (floatn x,
	floatn y,
	__private intn *quo)
float remquo (float x,
	float y,
	__global int *quo)
float remquo (float x,
	float y,
	__local int *quo)
float remquo (float x,
	float y,
	__private int *quo)
\stopbuffer
\startbuffer[funcdesc:remquof]
返回值記為 \math{r},則 \math{r = x - k \times y}。
其中 \math{k} 是最接近精確值 \math{x/y} 的整數。
如果有兩個這樣的整數,則選擇偶數作為 \math{k}。
如果 \math{r} 是零,則符號與 \carg{x} 一樣。
\math{r} 跟 \capi{remainder} 返回的值一樣。
區別就是 \capi{remquo} 還會計算整數商 \math{x/y} 的最低七位,
連帶 \math{x/y} 的符號一同存儲在 \carg{quo} 所指對象中。
\stopbuffer

% remquod
\startbuffer[funcproto:remquod]
doublen remquo (doublen x,
	doublen y,
	__global intn *quo)
doublen remquo (doublen x,
	doublen y,
	__local intn *quo)
doublen remquo (doublen x,
	doublen y,
	__private intn *quo)
double remquo (double x,
	double y,
	__global int *quo)
double remquo (double x,
	double y,
	__local int *quo)
double remquo (double x,
	double y,
	__private int *quo)
\stopbuffer
\startbuffer[funcdesc:remquod]
返回值記為 \math{r},則 \math{r = x - k \times y}。
其中 \math{k} 是最接近精確值 \math{x/y} 的整數。
如果有兩個這樣的整數,則選擇偶數作為 \math{k}。
如果 \math{r} 是零,則符號與 \carg{x} 一樣。
\math{r} 跟 \capi{remainder} 返回的值一樣。
區別就是 \capi{remquo} 還會計算整數商 \math{x/y} 的最低七位,
連帶 \math{x/y} 的符號一同存儲在 \carg{quo} 所指對象中。
\stopbuffer

% rint
\startbuffer[funcproto:rint]
gentype rint (gentype)
\stopbuffer
\startbuffer[funcdesc:rint]
捨入為最近偶數,雖然值為整數,但用的是浮點格式。
關於捨入模式請參考\refsec{roundingMode}。
\stopbuffer

% rootn
\startbuffer[funcproto:rootn]
floatn rootn (floatn x, intn y)
float rootn (float x, int y)

doublen rootn (doublen x, intn y)
doublen rootn (double x, int y)
\stopbuffer
\startbuffer[funcdesc:rootn]
計算 \math{x^{1/y}}。
\stopbuffer

% round
\startbuffer[funcproto:round]
gentype round (gentype x)
\stopbuffer
\startbuffer[funcdesc:round]
返回整數值,從零往外捨入,無視當前捨入方向。
\stopbuffer

% rsqrt
\startbuffer[funcproto:rsqrt]
gentype rsqrt (gentype)
\stopbuffer
\startbuffer[funcdesc:rsqrt]
計算 \math{ 1 / \sqrt{x}}。
\stopbuffer

% sin
\startbuffer[funcproto:sin]
gentype sin (gentype)
\stopbuffer
\startbuffer[funcdesc:sin]
計算正弦。
\stopbuffer

% sincos
\startbuffer[funcproto:sincos]
gentype sincos (gentype x,
	__global gentype *cosval)
gentype sincos (gentype x,
	__local gentype *cosval)
gentype sincos (gentype x,
	__private gentype *cosval)
\stopbuffer
\startbuffer[funcdesc:sincos]
計算 \carg{x} 的正弦和餘弦。
返回的是正弦,餘弦放到 \carg{cosval} 中。
\stopbuffer

% sinh
\startbuffer[funcproto:sinh]
gentype sinh (gentype)
\stopbuffer
\startbuffer[funcdesc:sinh]
計算雙曲正弦。
\stopbuffer

% sinpi
\startbuffer[funcproto:sinpi]
gentype sinpi (gentype x)
\stopbuffer
\startbuffer[funcdesc:sinpi]
計算 \math{\text{\capi{sin}}(\pi x)}。
\stopbuffer

% sqrt
\startbuffer[funcproto:sqrt]
gentype sqrt (gentype)
\stopbuffer
\startbuffer[funcdesc:sqrt]
計算平方根。
\stopbuffer

% tan
\startbuffer[funcproto:tan]
gentype tan (gentype)
\stopbuffer
\startbuffer[funcdesc:tan]
計算正切。
\stopbuffer

% tanh
\startbuffer[funcproto:tanh]
gentype tanh (gentype)
\stopbuffer
\startbuffer[funcdesc:tanh]
計算雙曲正切。
\stopbuffer

% tanpi
\startbuffer[funcproto:tanpi]
gentype tanpi (gentype)
\stopbuffer
\startbuffer[funcdesc:tanpi]
計算 \math{\text{\capi{tan}}(\pi x)}。
\stopbuffer

% tgamma
\startbuffer[funcproto:tgamma]
gentype tgamma (gentype)
\stopbuffer
\startbuffer[funcdesc:tgamma]
計算 \math{\Gamma(x)}。
\stopbuffer

% trunc
\startbuffer[funcproto:trunc]
gentype trunc (gentype)
\stopbuffer
\startbuffer[funcdesc:trunc]
向零捨入成整數值。
\stopbuffer


% begin TABLE
\startCLFD

\clFD{acos}
\clFD{acosh}
\clFD{acospi}
\clFD{asin}
\clFD{asinh}
\clFD{asinpi}
\clFD{atan}
\clFD{atan2}
\clFD{atanh}
\clFD{atanpi}
\clFD{atan2pi}
\clFD{cbrt}
\clFD{ceil}
\clFD{copysign}
\clFD{cos}
\clFD{cosh}
\clFD{cospi}
\clFD{erfc}
\clFD{erf}
\clFD{exp}
\clFD{exp2}
\clFD{exp10}
\clFD{expm1}
\clFD{fabs}
\clFD{fdim}
\clFD{floor}
\clFD{fma}
\clFD{fmax}
\clFD{fmin}
\clFD{fmod}
\clFD{fract}
\clFD{frexpf}
\clFD{frexpd}
\clFD{hypot}
\clFD{ilogb}
\clFD{ldexp}
\clFD{lgamma}
\clFD{log}
\clFD{log2}
\clFD{log10}
\clFD{log1p}
\clFD{logb}
\clFD{mad}
\clFD{maxmag}
\clFD{minmag}
\clFD{modf}
\clFD{nan}
\clFD{nextafter}
\clFD{pow}
\clFD{pown}
\clFD{powr}
\clFD{remainder}
\clFD{remquof}
\clFD{remquod}
\clFD{rint}
\clFD{rootn}
\clFD{rsqrt}
\clFD{sin}
\clFD{sincos}
\clFD{sinh}
\clFD{sinpi}
\clFD{sqrt}
\clFD{tan}
\clFD{tanh}
\clFD{tanpi}
\clFD{tgamma}
\clFD{trunc}

\stopCLFD

}

% half_ & native_ math function
\reftab{hnMathFunc}中列出了下列函式:
\startigBase
\item \reftab{svMathFunc}中的部分函式,但定義時帶有前綴 \ccmm{half_}。
實現這些函式時,精度至少要有 10 位,即所有 ULP 值都要小於等於 8192 ulp
(ULP:units in the last place,最後一位的進退位)。

\item \reftab{svMathFunc}中的部分函式,但定義時帶有前綴 \ccmm{native_}。
這些函式可能會映射到一條或多條原生的\cnglo{device}指令上,
性能通常比對應的不帶前綴 \ccmm{native_} 的函式更好。
這些函式的精度(以及某些情況下的輸入範圍)\cnglo{impdef}。

\item 用於除法和倒數運算的 \ccmm{half_} 和 \ccmm{native_} 函式。
\stopigBase

在\reftab{hnMathFunc}中,泛型 \ctype{gentype} 表示函式引數的型別可以是
 \ctype{float}、 \ctype{float2}、 \ctype{float3}、 \ctype{float4}、
 \ctype{float8} 或 \ctype{float16}。

\placetable[here,force,split][tab:hnMathFunc]
{內建的 \ccmm{half_} 和 \ccmm{native_} 數學函式}
{% half_cos
\startbuffer[funcproto:half_cos]
gentype half_cos (gentype x) 
\stopbuffer
\startbuffer[funcdesc:half_cos]
計算餘弦。 \carg{x} 的取值範圍為 \math{-2^{16} \cdots +2^{16}}。
\stopbuffer

% half_divide
\startbuffer[funcproto:half_divide]
gentype half_divide (gentype x, 
		gentype y) 
\stopbuffer
\startbuffer[funcdesc:half_divide]
計算 \math{x/y}。
\stopbuffer

% half_exp
\startbuffer[funcproto:half_exp]
gentype half_exp (gentype x) 
\stopbuffer
\startbuffer[funcdesc:half_exp]
計算 \math{e^x}。
\stopbuffer

% half_exp2
\startbuffer[funcproto:half_exp2]
gentype half_exp2 (gentype x) 
\stopbuffer
\startbuffer[funcdesc:half_exp2]
計算 \math{2^x}。
\stopbuffer

% half_exp10
\startbuffer[funcproto:half_exp10]
gentype half_exp10 (gentype x) 
\stopbuffer
\startbuffer[funcdesc:half_exp10]
計算 \math{10^x}。
\stopbuffer

% half_log
\startbuffer[funcproto:half_log]
gentype half_log (gentype x) 
\stopbuffer
\startbuffer[funcdesc:half_log]
計算自然對數。
\stopbuffer

% half_log2
\startbuffer[funcproto:half_log2]
gentype half_log2 (gentype x) 
\stopbuffer
\startbuffer[funcdesc:half_log2]
計算底為 2 的對數。
\stopbuffer

% half_log10
\startbuffer[funcproto:half_log10]
gentype half_log10 (gentype x) 
\stopbuffer
\startbuffer[funcdesc:half_log10]
計算底為 10 的對數。
\stopbuffer

% half_powr
\startbuffer[funcproto:half_powr]
gentype half_powr (gentype x,
		gentype y)
\stopbuffer
\startbuffer[funcdesc:half_powr]
計算 \math{x^y},其中 \math{x\geq 0}。
\stopbuffer

% half_recip
\startbuffer[funcproto:half_recip]
gentype half_recip (gentype x)
\stopbuffer
\startbuffer[funcdesc:half_recip]
計算倒數。
\stopbuffer

% half_rsqrt
\startbuffer[funcproto:half_rsqrt]
gentype half_rsqrt (gentype x)
\stopbuffer
\startbuffer[funcdesc:half_rsqrt]
計算 \math{ 1 / \sqrt{x}}。
\stopbuffer

% half_sin
\startbuffer[funcproto:half_sin]
gentype half_sin (gentype x)
\stopbuffer
\startbuffer[funcdesc:half_sin]
計算正弦。 \carg{x} 的取值範圍為 \math{-2^{16} \cdots +2^{16}}。
\stopbuffer

% half_sqrt
\startbuffer[funcproto:half_sqrt]
gentype half_sqrt (gentype x)
\stopbuffer
\startbuffer[funcdesc:half_sqrt]
計算 \math{\sqrt{x}}。
\stopbuffer

% half_tan
\startbuffer[funcproto:half_tan]
gentype half_tan (gentype x)
\stopbuffer
\startbuffer[funcdesc:half_tan]
計算正切。 \carg{x} 的取值範圍為 \math{-2^{16} \cdots +2^{16}}。
\stopbuffer

% native_cos
\startbuffer[funcproto:native_cos]
gentype native_cos (gentype x) 
\stopbuffer
\startbuffer[funcdesc:native_cos]
計算餘弦。
參數的取值範圍以及取最大值時會產生什麼錯誤都\cnglo{impdef}。
\stopbuffer

% native_divide
\startbuffer[funcproto:native_divide]
gentype native_divide (gentype x, 
		gentype y) 
\stopbuffer
\startbuffer[funcdesc:native_divide]
計算 \math{x/y}。
參數的取值範圍以及取最大值時會產生什麼錯誤都\cnglo{impdef}。
\stopbuffer

% native_exp
\startbuffer[funcproto:native_exp]
gentype native_exp (gentype x) 
\stopbuffer
\startbuffer[funcdesc:native_exp]
計算 \math{e^x}。
參數的取值範圍以及取最大值時會產生什麼錯誤都\cnglo{impdef}。
\stopbuffer

% native_exp2
\startbuffer[funcproto:native_exp2]
gentype native_exp2 (gentype x) 
\stopbuffer
\startbuffer[funcdesc:native_exp2]
計算 \math{2^x}。
參數的取值範圍以及取最大值時會產生什麼錯誤都\cnglo{impdef}。
\stopbuffer

% native_exp10
\startbuffer[funcproto:native_exp10]
gentype native_exp10 (gentype x) 
\stopbuffer
\startbuffer[funcdesc:native_exp10]
計算 \math{10^x}。
參數的取值範圍以及取最大值時會產生什麼錯誤都\cnglo{impdef}。
\stopbuffer

% native_log
\startbuffer[funcproto:native_log]
gentype native_log (gentype x) 
\stopbuffer
\startbuffer[funcdesc:native_log]
計算自然對數。
參數的取值範圍以及取最大值時會產生什麼錯誤都\cnglo{impdef}。
\stopbuffer

% native_log2
\startbuffer[funcproto:native_log2]
gentype native_log2 (gentype x) 
\stopbuffer
\startbuffer[funcdesc:native_log2]
計算底為 2 的對數。
參數的取值範圍以及取最大值時會產生什麼錯誤都\cnglo{impdef}。
\stopbuffer

% native_log10
\startbuffer[funcproto:native_log10]
gentype native_log10 (gentype x) 
\stopbuffer
\startbuffer[funcdesc:native_log10]
計算底為 10 的對數。
參數的取值範圍以及取最大值時會產生什麼錯誤都\cnglo{impdef}。
\stopbuffer

% native_powr
\startbuffer[funcproto:native_powr]
gentype native_powr (gentype x,
		gentype y)
\stopbuffer
\startbuffer[funcdesc:native_powr]
計算 \math{x^y},其中 \math{x\geq 0}。
參數的取值範圍以及取最大值時會產生什麼錯誤都\cnglo{impdef}。
\stopbuffer

% native_recip
\startbuffer[funcproto:native_recip]
gentype native_recip (gentype x)
\stopbuffer
\startbuffer[funcdesc:native_recip]
計算倒數。
參數的取值範圍以及取最大值時會產生什麼錯誤都\cnglo{impdef}。
\stopbuffer

% native_rsqrt
\startbuffer[funcproto:native_rsqrt]
gentype native_rsqrt (gentype x)
\stopbuffer
\startbuffer[funcdesc:native_rsqrt]
計算 \math{ 1 / \sqrt{x}}。
參數的取值範圍以及取最大值時會產生什麼錯誤都\cnglo{impdef}。
\stopbuffer

% native_sin
\startbuffer[funcproto:native_sin]
gentype native_sin (gentype x)
\stopbuffer
\startbuffer[funcdesc:native_sin]
計算正弦。
參數的取值範圍以及取最大值時會產生什麼錯誤都\cnglo{impdef}。
\stopbuffer

% native_sqrt
\startbuffer[funcproto:native_sqrt]
gentype native_sqrt (gentype x)
\stopbuffer
\startbuffer[funcdesc:native_sqrt]
計算 \math{\sqrt{x}}。
參數的取值範圍以及取最大值時會產生什麼錯誤都\cnglo{impdef}。
\stopbuffer

% native_tan
\startbuffer[funcproto:native_tan]
gentype native_tan (gentype x)
\stopbuffer
\startbuffer[funcdesc:native_tan]
計算正切。
參數的取值範圍以及取最大值時會產生什麼錯誤都\cnglo{impdef}。
\stopbuffer


% begin table
\startCLFD
\clFD{half_cos}
\clFD{half_divide}
\clFD{half_exp}
\clFD{half_exp2}
\clFD{half_exp10}
\clFD{half_log}
\clFD{half_log2}
\clFD{half_log10}
\clFD{half_powr}
\clFD{half_recip}
\clFD{half_rsqrt}
\clFD{half_sin}
\clFD{half_sqrt}
\clFD{half_tan}

\clFD{native_cos}
\clFD{native_divide}
\clFD{native_exp}
\clFD{native_exp2}
\clFD{native_exp10}
\clFD{native_log}
\clFD{native_log2}
\clFD{native_log10}
\clFD{native_powr}
\clFD{native_recip}
\clFD{native_rsqrt}
\clFD{native_sin}
\clFD{native_sqrt}
\clFD{native_tan}
\stopCLFD
}

實作可以自行決定 \capi{half_} 函式是否支持去規格化值。
如果引數是去規格化數, \capi{half_} 函式可以返回任何值,只要\todo{節 7.5.3}允許就行,
無論 \ccmm{-cl-denorms-are-zero} (參見\refsec{MathIntrinsicsOption})是否有效。

有下列符號常量可用。這些值的型別都是 \ctype{float},在單精度浮點數的精度內是精確的。
\startCLOD[常量名][描述]

\clOD{\cmacro{MAXFLOAT}}{
最大的有限單精度浮點數。
}

\clOD{\cmacro{HUGE_VALF}}{
正浮點常量算式。其求值結果為 \math{+\infty},由內建數學函式用作返回值表明錯誤。
}

\clOD{\cmacro{INFINITY}}{
常量算式,型別為 \ctype{float},表示正的或無符號的無窮。
}

\clOD{\cmacro{NAN}}{
常量算式,型別為 \ctype{float},表示 quiet NaN。
}
\stopCLOD


如果\cnglo{device}支持雙精度浮點數,還有下列符號常量可用:
\startCLOD[常量名][描述]

\clOD{\cmacro{HUGE_VAL}}{
正浮點常量算式。型別為 \ctype{double}。
其求值結果為 \math{+\infty},由內建數學函式用作返回值表明錯誤。
}

\stopCLOD


% Floating-point macros and pragmas
\subsubsection{浮點巨集和雜注}

雜注(pragma) \cpragmaemp{FP_CONTRACT} 可用來允許(如果狀態是 \ccmm{on})
或禁止(如果狀態是 \ccmm{off})實作化簡算式。
他可位於外部聲明的外面,也可位於複合語句中的顯式聲明或語句的前面。
當在外部聲明外面時,在遇到下一個 \cpragmaemp{FP_CONTRACT} 或者翻譯單元結束時就無效了。
當在複合語句中時,在遇到下一個 \cpragmaemp{FP_CONTRACT}
(包括嵌套的複合語句中的 \cpragmaemp{FP_CONTRACT})或者複合語句結束時就無效了;
在複合語句末尾處,會恢復成此語句之前的狀態。
在其他任何上下文中使用此雜注,其行為都是未定義的。

這樣設置 \cpragmaemp{FP_CONTRACT}:
\startclc
#pragma OPENCL FP_CONTRACT on-off-switch

on-off-switch is one of:
	/BTEX\ftEmp{ON}/ETEX, /BTEX\ftEmp{OFF}/ETEX or /BTEX\ftEmp{DEFAULT}/ETEX.
	The /BTEX\ftEmp{DEFAULT}/ETEX value is /BTEX\ftEmp{ON}/ETEX.
\stopclc

% float - single precision
巨集 \cmacroemp{FP_FAST_FMAF} 用來指明對於單精度浮點數,
函式 \capi{fma} 是否比直接編碼更快。
如果定義了此巨集,則表明對 \ctype{float} 算元的乘、加運算,
函式 \capi{fma} 一般跟直接編碼一樣快,或者更快。

OpenCL C 編程語言定義了如下巨集,他們必須使用指定的值。
可以在預處理指示 \ccmm{#if} 中使用這些常量算式。
\startclc
#define FLT_DIG		6
#define FLT_MANT_DIG	24
#define FLT_MAX_10_EXP	+38
#define FLT_MAX_EXP	+128
#define FLT_MIN_10_EXP	-37
#define FLT_MIN_EXP	-125
#define FLT_RADIX	2
#define FLT_MAX		0x1.fffffep127f
#define FLT_MIN		0x1.0p-126f
#define FLT_EPSILON	0x1.0p-23f
\stopclc

下表給出了上面所列巨集與\cnglo{app}所用的巨集名字之間的對應關係。

\startCLOO[OpenCL 語言中的巨集][\cnglo{app}所用的巨集]

\clMMF{DIG}
\clMMF{MANT_DIG}
\clMMF{MAX_10_EXP}
\clMMF{MAX_EXP}
\clMMF{MIN_10_EXP}
\clMMF{MIN_EXP}
\clMMF{RADIX}
\clMMF{MAX}
\clMMF{MIN}
\clMMF{EPSILSON}

\stopCLOO


下列兩個巨集將會展開成整數常量算式。
如果 \carg{x} 是 0 或 NaN,則 \math{\text{\capi{ilogb}}(x)} 會分別返回這兩個值。
\startigBase
\item \cmacroemp{FP_ILOGB0} 為 \ccmm{{INT_MIN}} 或 \ccmm{-{INT_MAX}}。
\item \cmacroemp{FP_ILOGBNAN} 為 \ccmm{{INT_MAX}} 或 \ccmm{{INT_MIN}}。
\stopigBase

除此之外,還有下列常量可用。
他們的型別都是 \ctype{float},在 \ctype{float} 型別的精度內是精確的。

\startCLOO[常量][描述]

\clCM{M_E_F}{e}
\clCM{M_LOG2E_F}{log_{2}e}
\clCM{M_LOG10E_F}{log_{10}e}
\clCM{M_LN2_F}{log_{e}2}
\clCM{M_LN10_F}{log_{e}10}
\clCM{M_PI_F}{\pi}
\clCM{M_PI_2_F}{\pi/2}
\clCM{M_PI_4_F}{\pi/4}
\clCM{M_1_PI_F}{1/\pi}
\clCM{M_2_PI_F}{2/\pi}
\clCM{M_2_SQRTPI_F}{2/\sqrt{\pi}}
\clCM{M_SQRT2_F}{\sqrt{2}}
\clCM{M_SQRT1_2_F}{1/\sqrt{2}}

\stopCLOO


% double - double precision
如果\cnglo{device}支持雙精度浮點數,還有下列巨集和常量可用:
\startigBase
\item 巨集 \cmacroemp{FP_FAST_FMA} 指明處理雙精度浮點數時,
 \capi{fma} 系列函式是否比直接編碼更快。
如果定義了此巨集,則表明對 \ctype{double} 算元的乘、加運算,
函式 \capi{fma} 一般跟直接編碼一樣快,或者更快。
\stopigBase

OpenCL C 編程語言定義了如下巨集,他們必須使用指定的值。
可以在預處理指示 \ccmm{#if} 中使用這些常量算式。
\startclc
#define DBL_DIG		15
#define DBL_MANT_DIG	53
#define DBL_MAX_10_EXP	+308
#define DBL_MAX_EXP	+1024
#define DBL_MIN_10_EXP	-307
#define DBL_MIN_EXP	-1021
#define DBL_MAX		0x1.fffffffffffffp1023
#define DBL_MIN		0x1.0p-1022
#define DBL_EPSILON	0x1.0p-52
\stopclc

下表給出了上面所列巨集與\cnglo{app}所用的巨集名字之間的對應關係。

\startCLOO[OpenCL 語言中的巨集][應用所用的巨集]

\clMMD{DIG}
\clMMD{MANT_DIG}
\clMMD{MAX_10_EXP}
\clMMD{MAX_EXP}
\clMMD{MIN_10_EXP}
\clMMD{MIN_EXP}
\clMMD{MAX}
\clMMD{MIN}
\clMMD{EPSILSON}

\stopCLOO


除此之外,還有下列常量可用。
他們的型別都是 \ctype{double},在 \ctype{double} 型別的精度內是精確的。

\startCLOO[常量][描述]

\clCM{M_E}{e}
\clCM{M_LOG2E}{log_{2}e}
\clCM{M_LOG10E}{log_{10}e}
\clCM{M_LN2}{log_{e}2}
\clCM{M_LN10}{log_{e}10}
\clCM{M_PI}{\pi}
\clCM{M_PI_2}{\pi/2}
\clCM{M_PI_4}{\pi/4}
\clCM{M_1_PI}{1/\pi}
\clCM{M_2_PI}{2/\pi}
\clCM{M_2_SQRTPI}{2/\sqrt{\pi}}
\clCM{M_SQRT2}{\sqrt{2}}
\clCM{M_SQRT1_2}{1/\sqrt{2}}

\stopCLOO


