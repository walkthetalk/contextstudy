%Integer Functions
\subsection{整數函式}

\reftab{}中列出了內建的整數函式,其引數即可為標量,亦可為矢量。
矢量版本的整數函式按組件逐一運算。其中的描述針對單個組件的。

泛型 \ctype{gentype} 表示函式引數的型別可以是
 \ctype{char}、 \ctype{char{2|3|4|8|16}}、
 \ctype{uchar}、 \ctype{uchar{2|3|4|8|16}}、
 \ctype{short}、 \ctype{short{2|3|4|8|16}}、
 \ctype{ushort}、 \ctype{ushort{2|3|4|8|16}}、
 \ctype{int}、 \ctype{int{2|3|4|8|16}}、
 \ctype{uint}、 \ctype{uint{2|3|4|8|16}}、
 \ctype{long}、 \ctype{long{2|3|4|8|16}}、
 \ctype{ulong} 或 \ctype{ulong{2|3|4|8|16}}。
泛型 \ctype{ugentype} 指代無符號版本的 \ctype{gentype}。
例如,如果 \ctype{gentype} 為 \ctype{char4},則 \ctype{ugentype} 為 \ctype{uchar4}。
同時,泛型 \ctype{sgentype} 指明函式的引數可以是標量(即
 \ctype{char}、 \ctype{uchar}、 \ctype{short}、 \ctype{ushort}、
 \ctype{int}、 \ctype{uint}、 \ctype{long} 或 \ctype{ulong})。
對於既有 \ctype{gentype} 引數,又有 \ctype{sgentype} 引數的內建整數函式,
 \ctype{gentype} 必須是標量或矢量版本的 \ctype{sgentype}。
例如,如果 \ctype{sgentype} 是 \ctype{uchar},
則 \ctype{gentype} 必須是 \ctype{uchar} 或 \ctype{uchar{2|3|4|8|16}}。
對於矢量版本, \ctype{sgentype} 只是簡單的拓寬成 \ctype{gentype},
參見\refsec{operator}中的\refitem{arithoperator}。

