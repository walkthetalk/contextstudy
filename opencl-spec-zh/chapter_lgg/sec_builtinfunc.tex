\section{內建函式}

OpenCL C 編程語言提供了一套豐富的內建函式用於標量和矢量運算。
這些函式中有很多跟通用 C 庫中提供的函式名字類似,
不同的是他們所支持的參數型別即可是標量,也可是矢量。
\cnglo{app}應儘可能使用這些內建函式,而不是去實現自己的版本。

用戶自定義的 OpenCL C 函式,按照函式的 C 標準規則運行(C99-TC2-節 6.9.1)。
在函式入口處,對於所有可動態修改的參數(variably modified parameter),都會求出其大小,
並且按照\refsec{usualArithConv}中描述的常見算術轉換規則
將所有引數算式的值都轉換成對應參數的型別。
本節所描述的內建函式行為類似,
不過由於同一個內建函式可能有多種形式,為避免歧義,不會有隱式的標量拓寬。
然而需要注意的是,可能某些內建函式的某些形式是針對標量以及矢量型別的混合運算。

% work-item functions
\subsection{\cnglo{workitem}函式}

\reftab{workItemFunction}列出了內建的\cnglo{workitem}函式,
這些函式可用來查詢指定給 \capi{clEnqueueNDRangeKernel} 的維數、全局和局部的索引空間大小、
以及在\cnglo{device}上執行此\cnglo{kernel}時
每個\cnglo{workitem}的\cnglo{glbid}和\cnglo{locid}。
如果使用的是 \capi{clEnqueueTask},則維數、全局和局部索引空間的大小都是一。

\placetable[here,split][tab:workItemFunction]
{\cnglo{workitem}函式表}
{% get_work_dim
\startbuffer[funcproto:get_work_dim]
uint get_work_dim()
\stopbuffer

\startbuffer[funcdesc:get_work_dim]
返回所用的維度數目。即 \capi{clEnqueueNDRangeKernel} 的引數 \carg{work_dim} 的值。

如果用的是 \capi{clEnqueueTask},則返回 1。
\stopbuffer

% get_global_size
\startbuffer[funcproto:get_global_size]
size_t get_global_size(uint dimindx)
\stopbuffer

\startbuffer[funcdesc:get_global_size]
返回在 \carg{dimindx} 所標識的維度上指定的全局\cnglo{workitem}的數目。
即 \capi{clEnqueueNDRangeKernel} 的引數 \carg{global_work_size}。
\carg{dimindx} 的有效範圍為 0 到 \math{\text{\capi{get_work_dim}}()-1}。
如果 \carg{dimindx} 是其他值,則 \capi{get_global_size} 返回 1。

如果用的是 \capi{clEnqueueTask},則返回 1。
\stopbuffer

% get_global_id
\startbuffer[funcproto:get_global_id]
size_t get_global_id(uint dimindx)
\stopbuffer

\startbuffer[funcdesc:get_global_id]
返回\cnglo{workitem}在 \carg{dimindx} 所標識的維度上的唯一\cnglo{glbid}。
此 ID 基於用來執行此\cnglo{kernel}的全局\cnglo{workitem}的數目。
\carg{dimindx} 的有效範圍為 0 到 \math{\text{\capi{get_work_dim}}() - 1}。
如果 \carg{dimindx} 是其他值,則 \capi{get_global_id} 返回 0。

如果用的是 \capi{clEnqueueTask},則返回 0。
\stopbuffer

% get_local_size
\startbuffer[funcproto:get_local_size]
size_t get_local_size(uint dimindx)
\stopbuffer

\startbuffer[funcdesc:get_local_size]
返回在 \carg{dimindx} 所標識的維度上指定的局部\cnglo{workitem}的數目。
如果 \capi{clEnqueueNDRangeKernel} 的引數 \carg{local_work_size} 不是 \cenum{NULL},
則返回的就是此引數的值;
否則 OpenCL 實作會選擇一個恰當的 \carg{local_work_size} 並將其返回。
\carg{dimindx} 的有效範圍為 0 到 \math{\text{\capi{get_work_dim}}()-1}。
如果 \carg{dimindx} 是其他值,則 \capi{get_local_size} 返回 1。

如果用的是 \capi{clEnqueueTask},則返回 1。
\stopbuffer

% get_local_id
\startbuffer[funcproto:get_local_id]
size_t get_local_id(uint dimindx)
\stopbuffer

\startbuffer[funcdesc:get_local_id]
返回\cnglo{workitem}在\cnglo{workgrp}內 \carg{dimindx} 所標識的維度上的唯一\cnglo{locid}。
\carg{dimindx} 的有效範圍為 0 到 \math{\text{\capi{get_work_dim}}() - 1}。
如果 \carg{dimindx} 是其他值,則 \capi{get_local_id} 返回 0。

如果用的是 \capi{clEnqueueTask},則返回 0。
\stopbuffer

% get_num_groups
\startbuffer[funcproto:get_num_groups]
size_t get_num_groups(uint dimindx)
\stopbuffer

\startbuffer[funcdesc:get_num_groups]
返回執行此\cnglo{kernel}的\cnglo{workgrp}在 \carg{dimindx} 所標識的維度上的數目。
\carg{dimindx} 的有效範圍為 0 到 \math{\text{\capi{get_work_dim}}() - 1}。
如果 \carg{dimindx} 是其他值,則 \capi{get_num_groups} 返回 1。

如果用的是 \capi{clEnqueueTask},則返回 1。
\stopbuffer

% get_group_id
\startbuffer[funcproto:get_group_id]
size_t get_group_id(uint dimindx)
\stopbuffer

\startbuffer[funcdesc:get_group_id]
所返回的\cnglo{workgrp} ID 取值範圍為
 \math{0 ... \text{\capi{get_work_dim}(\carg{dimindx})} - 1}。
\carg{dimindx} 的有效範圍為 0 到 \math{\text{\capi{get_work_dim}}() - 1}。
如果 \carg{dimindx} 是其他值,則 \capi{get_group_id} 返回 0。

如果用的是 \capi{clEnqueueTask},則返回 0。
\stopbuffer

% get_global_offset
\startbuffer[funcproto:get_global_offset]
size_t get_global_offset (uint dimindx)
\stopbuffer

\startbuffer[funcdesc:get_global_offset]
返回的是
為 \capi{clEnqueueNDRangeKernel} 的引數 \carg{global_work_offset} 所指定的偏移值。
\carg{dimindx} 的有效範圍為 0 到 \math{\text{\capi{get_work_dim}}() - 1}。
如果 \carg{dimindx} 是其他值,則 \capi{get_group_id} 返回 0。

如果用的是 \capi{clEnqueueTask},則返回 0。
\stopbuffer


% begin TABLE
\startCLFD

\clFD{get_work_dim}
\clFD{get_global_size}
\clFD{get_global_id}
\clFD{get_local_size}
\clFD{get_local_id}
\clFD{get_num_groups}
\clFD{get_group_id}
\clFD{get_global_offset}

\stopCLFD

}

% Math Functions
\subsection[sec:mathFunc]{數學函式}

\reftab{svMathFunc}中列出了內建的數學函式。
內建的數學函式分為兩種:
\startigBase
\item 第一種函式有兩個版本,一個版本的引數是標量,一個版本的引數是矢量;

\item 第二種函式只有一個版本,引數為標量浮點數。
\stopigBase

矢量版本的數學函式按組件逐一進行運算。
描述也是針對單個組件的。

無論調用環境中使用哪種捨入模式,
內建數學函式始終捨入為最近偶數,返回的結果也始終如一。

\reftab{svMathFunc}中所列函式即可接受標量引數,也可接受矢量引數。
泛型 \ctype{gentype} 表示函式引數的型別可以是
 \ctype{float}、 \ctype{float2}、 \ctype{float3}、 \ctype{float4}、
 \ctype{float8}、 \ctype{float16}、 \ctype{double}、 \ctype{double2}、
 \ctype{double3}、 \ctype{double4}、 \ctype{double8} 或 \ctype{double16}。
泛型 \ctype{gentypef} 表示函式引數的型別可以是
 \ctype{float}、 \ctype{float2}、 \ctype{float3}、 \ctype{float4}、
 \ctype{float8} 或 \ctype{float16}。
泛型 \ctype{gentyped} 表示函式引數的型別可以是
 \ctype{double}、 \ctype{double2}、 \ctype{double3}、 \ctype{double4}、
 \ctype{double8} 或 \ctype{double16}。
如果沒有特殊說明,函式的返回值與引數的型別都相同。

\placetable[here,force,split][tab:svMathFunc]
{引數既可為標量,也可為矢量的內建數學函式表}
{%% acos-related
% acos
\startbuffer[funcproto:acos]
gentype acos(gentype)
\stopbuffer
\startbuffer[funcdesc:acos]
反餘弦函數。
\stopbuffer

% acosh
\startbuffer[funcproto:acosh]
gentype acosh(gentype)
\stopbuffer
\startbuffer[funcdesc:acosh]
反雙曲餘弦函數。
\stopbuffer

% acospi
\startbuffer[funcproto:acospi]
gentype acospi(gentype)
\stopbuffer
\startbuffer[funcdesc:acospi]
計算 \math{acos(x) / \pi}。
\stopbuffer

%% asin-related
% asin
\startbuffer[funcproto:asin]
gentype asin(gentype)
\stopbuffer
\startbuffer[funcdesc:asin]
反正弦函數。
\stopbuffer

% asinh
\startbuffer[funcproto:asinh]
gentype asinh(gentype)
\stopbuffer
\startbuffer[funcdesc:asinh]
反雙曲正弦函數。
\stopbuffer

% asinpi
\startbuffer[funcproto:asinpi]
gentype asinpi(gentype)
\stopbuffer
\startbuffer[funcdesc:asinpi]
計算 \math{asin(x) / \pi}。
\stopbuffer

%% atan-related
% atan
\startbuffer[funcproto:atan]
gentype atan (gentype y_over_x)
\stopbuffer
\startbuffer[funcdesc:atan]
反正切函數。
\stopbuffer

% atan2
\startbuffer[funcproto:atan2]
gentype atan2 (gentype y,
		gentype x)
\stopbuffer
\startbuffer[funcdesc:atan2]
\math{y / x} 的反正切。
\stopbuffer

% atanh
\startbuffer[funcproto:atanh]
gentype atanh (gentype)
\stopbuffer
\startbuffer[funcdesc:atanh]
反雙曲正切函數。
\stopbuffer

% atanpi
\startbuffer[funcproto:atanpi]
gentype atanpi (gentype x)
\stopbuffer
\startbuffer[funcdesc:atanpi]
計算 \math{atan(x) / \pi}。
\stopbuffer

% atan2pi
\startbuffer[funcproto:atan2pi]
gentype atan2pi (gentype y,
		gentype x)
\stopbuffer
\startbuffer[funcdesc:atan2pi]
計算 \math{atan2(y,x) / \pi}。
\stopbuffer

% cbrt
\startbuffer[funcproto:cbrt]
gentype cbrt(gentype)
\stopbuffer
\startbuffer[funcdesc:cbrt]
計算立方根。
\stopbuffer

% ceil
\startbuffer[funcproto:ceil]
gentype ceil(gentype)
\stopbuffer
\startbuffer[funcdesc:ceil]
向正無窮捨入成整數值。
\stopbuffer

% copysign
\startbuffer[funcproto:copysign]
gentype copysign(gentype x,
		 gentype y)
\stopbuffer
\startbuffer[funcdesc:copysign]
將 \carg{x} 的符號改成 \carg{y} 的,並將其返回。
\stopbuffer

% cos
\startbuffer[funcproto:cos]
gentype cos(gentype)
\stopbuffer
\startbuffer[funcdesc:cos]
計算餘弦。
\stopbuffer

% cosh
\startbuffer[funcproto:cosh]
gentype cosh(gentype)
\stopbuffer
\startbuffer[funcdesc:cosh]
計算雙曲餘弦。
\stopbuffer

% cospi
\startbuffer[funcproto:cospi]
gentype cospi(gentype x)
\stopbuffer
\startbuffer[funcdesc:cospi]
計算 \math{cos(\pi x)}。
\stopbuffer

% erfc
\startbuffer[funcproto:erfc]
gentype erfc(gentype)
\stopbuffer
\startbuffer[funcdesc:erfc]
餘補誤差函數
 \math{1 - erf(x) = \frac{2}{\sqrt{\pi}} \intop^{\infty}_{x}e^{-\theta^2}d\theta}。
\stopbuffer

% erf
\startbuffer[funcproto:erf]
gentype erf(gentype)
\stopbuffer
\startbuffer[funcdesc:erf]
誤差函數,表示正態分布的積分
 \math{\frac{2}{\sqrt{\pi}} \intop^{\infty}_{x}e^{-\theta^2}d\theta}。
\stopbuffer

% exp
\startbuffer[funcproto:exp]
gentype exp(gentype x)
\stopbuffer
\startbuffer[funcdesc:exp]
計算 \math{e} 的 \carg{x} 次冪 \math{e^x}。
\stopbuffer

% exp
\startbuffer[funcproto:exp]
gentype exp(gentype x)
\stopbuffer
\startbuffer[funcdesc:exp]
計算 \math{e} 的 \carg{x} 次冪 \math{e^x}。
\stopbuffer

% exp2
\startbuffer[funcproto:exp2]
gentype exp2(gentype)
\stopbuffer
\startbuffer[funcdesc:exp2]
底數為 \math{2} 的冪。
\stopbuffer

% exp10
\startbuffer[funcproto:exp10]
gentype exp10(gentype)
\stopbuffer
\startbuffer[funcdesc:exp10]
底數為 \math{10} 的冪。
\stopbuffer

% expm1
\startbuffer[funcproto:expm1]
gentype expm1(gentype x)
\stopbuffer
\startbuffer[funcdesc:expm1]
計算 \math{e^x-1.0}。
\stopbuffer

% fabs
\startbuffer[funcproto:fabs]
gentype fabs(gentype)
\stopbuffer
\startbuffer[funcdesc:fabs]
計算浮點數的絕對值。
\stopbuffer

% fdim
\startbuffer[funcproto:fdim]
gentype fdim(gentype x,
	     gentype y)
\stopbuffer
\startbuffer[funcdesc:fdim]
如果 \math{x < y},返回 \math{x - y};否則返回 \math{+0}。
\stopbuffer

% floor
\startbuffer[funcproto:floor]
gentype floor(gentype)
\stopbuffer
\startbuffer[funcdesc:floor]
向負無窮捨入成整數值。
\stopbuffer

% fma
\startbuffer[funcproto:fma]
gentype fma(gentype a,
	    gentype b,
	    gentype c)
\stopbuffer
\startbuffer[funcdesc:fma]
返回 \math{a \times b + c},其中乘法具有無限精度即不會進行捨入,但會對加法進行正確地捨入。
邊界條件下的行為遵循 IEEE 754-2008 標準。
\stopbuffer

% fmax
\startbuffer[funcproto:fmax]
gentype fmax (gentype x, gentype y)
gentypef fmax (gentypef x, float y)
gentyped fmax (gentyped x, double y)
\stopbuffer
\startbuffer[funcdesc:fmax]
如果 \math{x < y},則返回 \math{y},否則返回 \math{x}。
如果一個引數是 NaN,則返回另一個引數;
如果兩個引數都是 NaN,則返回 NaN。
\stopbuffer
\stopbuffer

% fmin
\startbuffer[funcproto:fmin]
gentype fmin (gentype x, gentype y)
gentypef fmin (gentypef x, float y)
gentyped fmin (gentyped x, double y)
\stopbuffer
\startbuffer[funcdesc:fmin]
如果 \math{y < x},則返回 \math{y},否則返回 \math{x}。
如果一個引數是 NaN,則返回另一個引數;
如果兩個引數都是 NaN,則返回 NaN\footnote{
在處理 signaling NaN 時, \capi{fmin} 和 \capi{fmax} 的行為遵守 C99 中的定義,
但可能與 IEEE 754-2008 中定義的 \capi{minNum} 和 \capi{maxNum} 處理方式不同。
特別是,可能將 signaling NaN 當成 quiet NaN。}。
\stopbuffer

% fmod
\startbuffer[funcproto:fmod]
gentype fmod (gentyped x, gentype y)
\stopbuffer
\startbuffer[funcdesc:fmod]
模。返回 \math{x - y * trunc(x/y)}。
\stopbuffer

% fract
\startbuffer[funcproto:fract]
gentype fract (gentype x,
	__global gentype *iptr)
gentype fract (gentype x,
	__local gentype *iptr)
gentype fract (gentype x,
	__private gentype *iptr)
\stopbuffer
\startbuffer[funcdesc:fract]
返回 \math{\text{\capi{fmin}} (x - \text{\capi{floor}} ( x ), 0x1.fffffep - 1f}。
而 \math{\text{\capi{floor}}(x)} 將在 \carg{iptr} 中返回\footnote{
此處的 min() 是為了避免 \capi{fract}(-small) 返回 1.0。
有了 min(),這種情況就會返回小於 1.0 的最大正浮點數。}。
\stopbuffer

% frexp float
\startbuffer[funcproto:frexpf]
floatn frexp (floatn x,
	__global intn *exp)
floatn frexp (floatn x,
	__local intn *exp)
floatn frexp (floatn x,
	__private intn *exp)
float frexp (float x,
	__global int *exp)
float frexp (float x,
	__local int *exp)
float frexp (float x,
	__private int *exp)
\stopbuffer
\startbuffer[funcdesc:frexpf]
從 \carg{x} 中分離出尾數和指數。
返回的尾數(記為 m)型別為 \ctype{float},值為 0 或者屬於 \math{[1/2, 1)}。
\carg{x} 的每個組件都等於 \math{m \times 2^{exp}}。
\stopbuffer

% frexp double
\startbuffer[funcproto:frexpd]
doublen frexp (doublen x,
	__global intn *exp)
doublen frexp (doublen x,
	__local intn *exp)
doublen frexp (doublen x,
	__private intn *exp)
double frexp (double x,
	__global int *exp)
double frexp (double x,
	__local int *exp)
double frexp (double x,
	__private int *exp)
\stopbuffer
\startbuffer[funcdesc:frexpd]
從 \carg{x} 中分離出尾數和指數。
返回的尾數(記為 m)型別為 \ctype{double},值為 0 或者屬於 \math{[1/2, 1)}。
\carg{x} 的每個組件都等於 \math{m \times 2^{exp}}。
\stopbuffer

% hypot
\startbuffer[funcproto:hypot]
gentype hypot (gentype x, gentype y)
\stopbuffer
\startbuffer[funcdesc:hypot]
計算 \math{\sqrt{x^2+y^2}},不會有過分的上溢或下溢。
\stopbuffer

% ilogb
\startbuffer[funcproto:ilogb]
intn ilogb (floatn x)
int ilogb (float x)
intn ilogb (doublen x)
int ilogb (double x)
\stopbuffer
\startbuffer[funcdesc:ilogb]
返回整形對數。
\stopbuffer

% ldexp
\startbuffer[funcproto:ldexp]
floatn ldexp (floatn x, intn k)
floatn ldexp (floatn x, int k)
float ldexp (float x, int k)
doublen ldexp (doublen x, intn k)
doublen ldexp (doublen x, int k)
double ldexp (double x, int k)
\stopbuffer
\startbuffer[funcdesc:ldexp]
返回 \math{x \times 2^k}。
\stopbuffer

% lgamma
\startbuffer[funcproto:lgamma]
gentype lgamma (gentype x)
floatn lgamma_r (floatn x,
	__global intn *signp)
floatn lgamma_r (floatn x,
	__local intn *signp)
floatn lgamma_r (floatn x,
	__private intn *signp)
float lgamma_r (float x,
	__global int *signp)
float lgamma_r (float x,
	__local int *signp)
float lgamma_r (float x,
	__private int *signp)
doublen lgamma_r (doublen x,
	__global intn *signp)
doublen lgamma_r (doublen x,
	__local intn *signp)
doublen lgamma_r (doublen x,
	__private intn *signp)
double lgamma_r (double x,
	__global int *signp)
double lgamma_r (double x,
	__local int *signp)
double lgamma_r (double x,
	__private int *signp)
\stopbuffer
\startbuffer[funcdesc:lgamma]
返回伽馬函數絕對值的自然對數。
\capi{lgamma_r} 的引數 \carg{signp} 中會返回伽馬函數的符號。
 \math{ln|\Gamma(x)|}
\stopbuffer

% log
\startbuffer[funcproto:log]
gentype log (gentype)
\stopbuffer
\startbuffer[funcdesc:log]
計算自然對數。
\stopbuffer

% log2
\startbuffer[funcproto:log2]
gentype log2 (gentype)
\stopbuffer
\startbuffer[funcdesc:log2]
計算以 2 為底的對數。
\stopbuffer

% log10
\startbuffer[funcproto:log10]
gentype log10 (gentype)
\stopbuffer
\startbuffer[funcdesc:log10]
計算以 10 為底的對數。
\stopbuffer

% log1p
\startbuffer[funcproto:log1p]
gentype log1p (gentype x)
\stopbuffer
\startbuffer[funcdesc:log1p]
計算 \math{log_e(1.0+x)}。
\stopbuffer

% logb
\startbuffer[funcproto:logb]
gentype logb (gentype x)
\stopbuffer
\startbuffer[funcdesc:logb]
\math{log_r|x|}的整數部分。
\stopbuffer

% mad
\startbuffer[funcproto:mad]
gentype mad (gentype a,
	gentype b,
	gentype c)
\stopbuffer
\startbuffer[funcdesc:mad]
\capi{mad} 逼近 \math{a \times b + c}。
至於 \math{a \times b} 是否要捨入、怎樣捨入,
以及如何處理超常(supernormal)或次常(subnormal)的乘法都沒有定義。
在那些速度比準確更重要的地方,就可以使用 \capi{mad}\footnote{
需要提醒用戶的是,對於一些情況,
如 \math{\text{\capi{mad}}(a, b, -a\times b)}, \capi{mad} 定義的非常寬鬆,
以至於當 \carg{a} 和 \carg{b} 為某些特定值時,返回任何值都有可能}。
\stopbuffer

% maxmag
\startbuffer[funcproto:maxmag]
gentype maxmag (gentype x, gentype y)
\stopbuffer
\startbuffer[funcdesc:maxmag]
如果 \math{|x| > |y|},則返回 \carg{x};
如果 \math{|y| > |x|},則返回 \carg{y};
否則返回 \math{\text{\capi{fmax}}(x,y)}。
\stopbuffer

% minmag
\startbuffer[funcproto:minmag]
gentype minmag (gentype x, gentype y)
\stopbuffer
\startbuffer[funcdesc:minmag]
如果 \math{|x| < |y|},則返回 \carg{x};
如果 \math{|y| < |x|},則返回 \carg{y};
否則返回 \math{\text{\capi{fmin}}(x,y)}。
\stopbuffer

% modf
\startbuffer[funcproto:modf]
gentype modf (gentype x,
	__global gentype *iptr)
gentype modf (gentype x,
	__local gentype *iptr)
gentype modf (gentype x,
	__private gentype *iptr)
\stopbuffer
\startbuffer[funcdesc:modf]
分解浮點數,將引數 \carg{x} 分解為整數部分和小數部分,
兩部分的符號都與 \carg{x} 相同。
整數部分存儲在 \carg{iptr} 所指對象中。
\stopbuffer

% nan
\startbuffer[funcproto:nan]
floatn nan (uintn nancode)
float nan (uint nancode)
doublen nan (ulongn nancode)
double nan (ulong nancode)
\stopbuffer
\startbuffer[funcdesc:nan]
返回 quiet NaN。
其中將 \carg{nancode} 放在尾數的位置。
\stopbuffer

% nextafter
\startbuffer[funcproto:nextafter]
gentype nextafter (gentype x,
		gentype y)
\stopbuffer
\startbuffer[funcdesc:nextafter]
返回 \carg{x} 在往 \carg{y} 的方向上下一個可表示的單精度浮點值。
因此,如果 \carg{y} 小於 \carg{x},
則返回小於 \carg{x} 的最大的可表示的浮點數。
\stopbuffer

% pow
\startbuffer[funcproto:pow]
gentype pow (gentype x, gentype y)
\stopbuffer
\startbuffer[funcdesc:pow]
計算 \math{x^y}。
\stopbuffer

% pown
\startbuffer[funcproto:pown]
floatn pown (floatn x, intn y)
float pown (float x, int y)
doublen pown (doublen x, intn y)
double pown (double x, int y)
\stopbuffer
\startbuffer[funcdesc:pown]
計算 \math{x^y},其中 \carg{y} 是整數。
\stopbuffer

% powr
\startbuffer[funcproto:powr]
gentype powr (gentype x,
		gentype y)
\stopbuffer
\startbuffer[funcdesc:powr]
計算 \math{x^y},其中 \math{x\geq 0}。
\stopbuffer

% remainder
\startbuffer[funcproto:remainder]
gentype remainder (gentype x,
		gentype y)
\stopbuffer
\startbuffer[funcdesc:remainder]
返回值記為 \math{r},則 \math{r = x - n \times y}。
其中 \math{n} 是最接近精確值 \math{x/y} 的整數。
如果有兩個這樣的整數,則選擇偶數作為 \math{n}。
如果 \math{r} 是零,則符號與 \carg{x} 一樣。
\stopbuffer

% remquof
\startbuffer[funcproto:remquof]
floatn remquo (floatn x,
	floatn y,
	__global intn *quo)
floatn remquo (floatn x,
	floatn y,
	__local intn *quo)
floatn remquo (floatn x,
	floatn y,
	__private intn *quo)
float remquo (float x,
	float y,
	__global int *quo)
float remquo (float x,
	float y,
	__local int *quo)
float remquo (float x,
	float y,
	__private int *quo)
\stopbuffer
\startbuffer[funcdesc:remquof]
返回值記為 \math{r},則 \math{r = x - k \times y}。
其中 \math{k} 是最接近精確值 \math{x/y} 的整數。
如果有兩個這樣的整數,則選擇偶數作為 \math{k}。
如果 \math{r} 是零,則符號與 \carg{x} 一樣。
\math{r} 跟 \capi{remainder} 返回的值一樣。
區別就是 \capi{remquo} 還會計算整數商 \math{x/y} 的最低七位,
連帶 \math{x/y} 的符號一同存儲在 \carg{quo} 所指對象中。
\stopbuffer

% remquod
\startbuffer[funcproto:remquod]
doublen remquo (doublen x,
	doublen y,
	__global intn *quo)
doublen remquo (doublen x,
	doublen y,
	__local intn *quo)
doublen remquo (doublen x,
	doublen y,
	__private intn *quo)
double remquo (double x,
	double y,
	__global int *quo)
double remquo (double x,
	double y,
	__local int *quo)
double remquo (double x,
	double y,
	__private int *quo)
\stopbuffer
\startbuffer[funcdesc:remquod]
返回值記為 \math{r},則 \math{r = x - k \times y}。
其中 \math{k} 是最接近精確值 \math{x/y} 的整數。
如果有兩個這樣的整數,則選擇偶數作為 \math{k}。
如果 \math{r} 是零,則符號與 \carg{x} 一樣。
\math{r} 跟 \capi{remainder} 返回的值一樣。
區別就是 \capi{remquo} 還會計算整數商 \math{x/y} 的最低七位,
連帶 \math{x/y} 的符號一同存儲在 \carg{quo} 所指對象中。
\stopbuffer

% rint
\startbuffer[funcproto:rint]
gentype rint (gentype)
\stopbuffer
\startbuffer[funcdesc:rint]
捨入為最近偶數,雖然值為整數,但用的是浮點格式。
關於捨入模式請參考\refsec{roundingMode}。
\stopbuffer

% rootn
\startbuffer[funcproto:rootn]
floatn rootn (floatn x, intn y)
float rootn (float x, int y)

doublen rootn (doublen x, intn y)
doublen rootn (double x, int y)
\stopbuffer
\startbuffer[funcdesc:rootn]
計算 \math{x^{1/y}}。
\stopbuffer

% round
\startbuffer[funcproto:round]
gentype round (gentype x)
\stopbuffer
\startbuffer[funcdesc:round]
返回整數值,從零往外捨入,無視當前捨入方向。
\stopbuffer

% rsqrt
\startbuffer[funcproto:rsqrt]
gentype rsqrt (gentype)
\stopbuffer
\startbuffer[funcdesc:rsqrt]
計算 \math{ 1 / \sqrt{x}}。
\stopbuffer

% sin
\startbuffer[funcproto:sin]
gentype sin (gentype)
\stopbuffer
\startbuffer[funcdesc:sin]
計算正弦。
\stopbuffer

% sincos
\startbuffer[funcproto:sincos]
gentype sincos (gentype x,
	__global gentype *cosval)
gentype sincos (gentype x,
	__local gentype *cosval)
gentype sincos (gentype x,
	__private gentype *cosval)
\stopbuffer
\startbuffer[funcdesc:sincos]
計算 \carg{x} 的正弦和餘弦。
返回的是正弦,餘弦放到 \carg{cosval} 中。
\stopbuffer

% sinh
\startbuffer[funcproto:sinh]
gentype sinh (gentype)
\stopbuffer
\startbuffer[funcdesc:sinh]
計算雙曲正弦。
\stopbuffer

% sinpi
\startbuffer[funcproto:sinpi]
gentype sinpi (gentype x)
\stopbuffer
\startbuffer[funcdesc:sinpi]
計算 \math{\text{\capi{sin}}(\pi x)}。
\stopbuffer

% sqrt
\startbuffer[funcproto:sqrt]
gentype sqrt (gentype)
\stopbuffer
\startbuffer[funcdesc:sqrt]
計算平方根。
\stopbuffer

% tan
\startbuffer[funcproto:tan]
gentype tan (gentype)
\stopbuffer
\startbuffer[funcdesc:tan]
計算正切。
\stopbuffer

% tanh
\startbuffer[funcproto:tanh]
gentype tanh (gentype)
\stopbuffer
\startbuffer[funcdesc:tanh]
計算雙曲正切。
\stopbuffer

% tanpi
\startbuffer[funcproto:tanpi]
gentype tanpi (gentype)
\stopbuffer
\startbuffer[funcdesc:tanpi]
計算 \math{\text{\capi{tan}}(\pi x)}。
\stopbuffer

% tgamma
\startbuffer[funcproto:tgamma]
gentype tgamma (gentype)
\stopbuffer
\startbuffer[funcdesc:tgamma]
計算 \math{\Gamma(x)}。
\stopbuffer

% trunc
\startbuffer[funcproto:trunc]
gentype trunc (gentype)
\stopbuffer
\startbuffer[funcdesc:trunc]
向零捨入成整數值。
\stopbuffer


% begin TABLE
\startCLFD

\clFD{acos}
\clFD{acosh}
\clFD{acospi}
\clFD{asin}
\clFD{asinh}
\clFD{asinpi}
\clFD{atan}
\clFD{atan2}
\clFD{atanh}
\clFD{atanpi}
\clFD{atan2pi}
\clFD{cbrt}
\clFD{ceil}
\clFD{copysign}
\clFD{cos}
\clFD{cosh}
\clFD{cospi}
\clFD{erfc}
\clFD{erf}
\clFD{exp}
\clFD{exp2}
\clFD{exp10}
\clFD{expm1}
\clFD{fabs}
\clFD{fdim}
\clFD{floor}
\clFD{fma}
\clFD{fmax}
\clFD{fmin}
\clFD{fmod}
\clFD{fract}
\clFD{frexpf}
\clFD{frexpd}
\clFD{hypot}
\clFD{ilogb}
\clFD{ldexp}
\clFD{lgamma}
\clFD{log}
\clFD{log2}
\clFD{log10}
\clFD{log1p}
\clFD{logb}
\clFD{mad}
\clFD{maxmag}
\clFD{minmag}
\clFD{modf}
\clFD{nan}
\clFD{nextafter}
\clFD{pow}
\clFD{pown}
\clFD{powr}
\clFD{remainder}
\clFD{remquof}
\clFD{remquod}
\clFD{rint}
\clFD{rootn}
\clFD{rsqrt}
\clFD{sin}
\clFD{sincos}
\clFD{sinh}
\clFD{sinpi}
\clFD{sqrt}
\clFD{tan}
\clFD{tanh}
\clFD{tanpi}
\clFD{tgamma}
\clFD{trunc}

\stopCLFD

}

% half_ & native_ math function
\reftab{hnMathFunc}中列出了下列函式:
\startigBase
\item \reftab{svMathFunc}中的部分函式,但定義時帶有前綴 \ccmm{half_}。
實現這些函式時,精度至少要有 10 位,即所有 ULP 值都要小於等於 8192 ulp
(ULP:units in the last place,最後一位的進退位)。

\item \reftab{svMathFunc}中的部分函式,但定義時帶有前綴 \ccmm{native_}。
這些函式可能會映射到一條或多條原生的\cnglo{device}指令上,
性能通常比對應的不帶前綴 \ccmm{native_} 的函式更好。
這些函式的精度(以及某些情況下的輸入範圍)\cnglo{impdef}。

\item 用於除法和倒數運算的 \ccmm{half_} 和 \ccmm{native_} 函式。
\stopigBase

在\reftab{hnMathFunc}中,泛型 \ctype{gentype} 表示函式引數的型別可以是
 \ctype{float}、 \ctype{float2}、 \ctype{float3}、 \ctype{float4}、
 \ctype{float8} 或 \ctype{float16}。

\placetable[here,force,split][tab:hnMathFunc]
{內建的 \ccmm{half_} 和 \ccmm{native_} 數學函式}
{% half_cos
\startbuffer[funcproto:half_cos]
gentype half_cos (gentype x) 
\stopbuffer
\startbuffer[funcdesc:half_cos]
計算餘弦。 \carg{x} 的取值範圍為 \math{-2^{16} \cdots +2^{16}}。
\stopbuffer

% half_divide
\startbuffer[funcproto:half_divide]
gentype half_divide (gentype x, 
		gentype y) 
\stopbuffer
\startbuffer[funcdesc:half_divide]
計算 \math{x/y}。
\stopbuffer

% half_exp
\startbuffer[funcproto:half_exp]
gentype half_exp (gentype x) 
\stopbuffer
\startbuffer[funcdesc:half_exp]
計算 \math{e^x}。
\stopbuffer

% half_exp2
\startbuffer[funcproto:half_exp2]
gentype half_exp2 (gentype x) 
\stopbuffer
\startbuffer[funcdesc:half_exp2]
計算 \math{2^x}。
\stopbuffer

% half_exp10
\startbuffer[funcproto:half_exp10]
gentype half_exp10 (gentype x) 
\stopbuffer
\startbuffer[funcdesc:half_exp10]
計算 \math{10^x}。
\stopbuffer

% half_log
\startbuffer[funcproto:half_log]
gentype half_log (gentype x) 
\stopbuffer
\startbuffer[funcdesc:half_log]
計算自然對數。
\stopbuffer

% half_log2
\startbuffer[funcproto:half_log2]
gentype half_log2 (gentype x) 
\stopbuffer
\startbuffer[funcdesc:half_log2]
計算底為 2 的對數。
\stopbuffer

% half_log10
\startbuffer[funcproto:half_log10]
gentype half_log10 (gentype x) 
\stopbuffer
\startbuffer[funcdesc:half_log10]
計算底為 10 的對數。
\stopbuffer

% half_powr
\startbuffer[funcproto:half_powr]
gentype half_powr (gentype x,
		gentype y)
\stopbuffer
\startbuffer[funcdesc:half_powr]
計算 \math{x^y},其中 \math{x\geq 0}。
\stopbuffer

% half_recip
\startbuffer[funcproto:half_recip]
gentype half_recip (gentype x)
\stopbuffer
\startbuffer[funcdesc:half_recip]
計算倒數。
\stopbuffer

% half_rsqrt
\startbuffer[funcproto:half_rsqrt]
gentype half_rsqrt (gentype x)
\stopbuffer
\startbuffer[funcdesc:half_rsqrt]
計算 \math{ 1 / \sqrt{x}}。
\stopbuffer

% half_sin
\startbuffer[funcproto:half_sin]
gentype half_sin (gentype x)
\stopbuffer
\startbuffer[funcdesc:half_sin]
計算正弦。 \carg{x} 的取值範圍為 \math{-2^{16} \cdots +2^{16}}。
\stopbuffer

% half_sqrt
\startbuffer[funcproto:half_sqrt]
gentype half_sqrt (gentype x)
\stopbuffer
\startbuffer[funcdesc:half_sqrt]
計算 \math{\sqrt{x}}。
\stopbuffer

% half_tan
\startbuffer[funcproto:half_tan]
gentype half_tan (gentype x)
\stopbuffer
\startbuffer[funcdesc:half_tan]
計算正切。 \carg{x} 的取值範圍為 \math{-2^{16} \cdots +2^{16}}。
\stopbuffer

% native_cos
\startbuffer[funcproto:native_cos]
gentype native_cos (gentype x) 
\stopbuffer
\startbuffer[funcdesc:native_cos]
計算餘弦。
參數的取值範圍以及取最大值時會產生什麼錯誤都\cnglo{impdef}。
\stopbuffer

% native_divide
\startbuffer[funcproto:native_divide]
gentype native_divide (gentype x, 
		gentype y) 
\stopbuffer
\startbuffer[funcdesc:native_divide]
計算 \math{x/y}。
參數的取值範圍以及取最大值時會產生什麼錯誤都\cnglo{impdef}。
\stopbuffer

% native_exp
\startbuffer[funcproto:native_exp]
gentype native_exp (gentype x) 
\stopbuffer
\startbuffer[funcdesc:native_exp]
計算 \math{e^x}。
參數的取值範圍以及取最大值時會產生什麼錯誤都\cnglo{impdef}。
\stopbuffer

% native_exp2
\startbuffer[funcproto:native_exp2]
gentype native_exp2 (gentype x) 
\stopbuffer
\startbuffer[funcdesc:native_exp2]
計算 \math{2^x}。
參數的取值範圍以及取最大值時會產生什麼錯誤都\cnglo{impdef}。
\stopbuffer

% native_exp10
\startbuffer[funcproto:native_exp10]
gentype native_exp10 (gentype x) 
\stopbuffer
\startbuffer[funcdesc:native_exp10]
計算 \math{10^x}。
參數的取值範圍以及取最大值時會產生什麼錯誤都\cnglo{impdef}。
\stopbuffer

% native_log
\startbuffer[funcproto:native_log]
gentype native_log (gentype x) 
\stopbuffer
\startbuffer[funcdesc:native_log]
計算自然對數。
參數的取值範圍以及取最大值時會產生什麼錯誤都\cnglo{impdef}。
\stopbuffer

% native_log2
\startbuffer[funcproto:native_log2]
gentype native_log2 (gentype x) 
\stopbuffer
\startbuffer[funcdesc:native_log2]
計算底為 2 的對數。
參數的取值範圍以及取最大值時會產生什麼錯誤都\cnglo{impdef}。
\stopbuffer

% native_log10
\startbuffer[funcproto:native_log10]
gentype native_log10 (gentype x) 
\stopbuffer
\startbuffer[funcdesc:native_log10]
計算底為 10 的對數。
參數的取值範圍以及取最大值時會產生什麼錯誤都\cnglo{impdef}。
\stopbuffer

% native_powr
\startbuffer[funcproto:native_powr]
gentype native_powr (gentype x,
		gentype y)
\stopbuffer
\startbuffer[funcdesc:native_powr]
計算 \math{x^y},其中 \math{x\geq 0}。
參數的取值範圍以及取最大值時會產生什麼錯誤都\cnglo{impdef}。
\stopbuffer

% native_recip
\startbuffer[funcproto:native_recip]
gentype native_recip (gentype x)
\stopbuffer
\startbuffer[funcdesc:native_recip]
計算倒數。
參數的取值範圍以及取最大值時會產生什麼錯誤都\cnglo{impdef}。
\stopbuffer

% native_rsqrt
\startbuffer[funcproto:native_rsqrt]
gentype native_rsqrt (gentype x)
\stopbuffer
\startbuffer[funcdesc:native_rsqrt]
計算 \math{ 1 / \sqrt{x}}。
參數的取值範圍以及取最大值時會產生什麼錯誤都\cnglo{impdef}。
\stopbuffer

% native_sin
\startbuffer[funcproto:native_sin]
gentype native_sin (gentype x)
\stopbuffer
\startbuffer[funcdesc:native_sin]
計算正弦。
參數的取值範圍以及取最大值時會產生什麼錯誤都\cnglo{impdef}。
\stopbuffer

% native_sqrt
\startbuffer[funcproto:native_sqrt]
gentype native_sqrt (gentype x)
\stopbuffer
\startbuffer[funcdesc:native_sqrt]
計算 \math{\sqrt{x}}。
參數的取值範圍以及取最大值時會產生什麼錯誤都\cnglo{impdef}。
\stopbuffer

% native_tan
\startbuffer[funcproto:native_tan]
gentype native_tan (gentype x)
\stopbuffer
\startbuffer[funcdesc:native_tan]
計算正切。
參數的取值範圍以及取最大值時會產生什麼錯誤都\cnglo{impdef}。
\stopbuffer


% begin table
\startCLFD
\clFD{half_cos}
\clFD{half_divide}
\clFD{half_exp}
\clFD{half_exp2}
\clFD{half_exp10}
\clFD{half_log}
\clFD{half_log2}
\clFD{half_log10}
\clFD{half_powr}
\clFD{half_recip}
\clFD{half_rsqrt}
\clFD{half_sin}
\clFD{half_sqrt}
\clFD{half_tan}

\clFD{native_cos}
\clFD{native_divide}
\clFD{native_exp}
\clFD{native_exp2}
\clFD{native_exp10}
\clFD{native_log}
\clFD{native_log2}
\clFD{native_log10}
\clFD{native_powr}
\clFD{native_recip}
\clFD{native_rsqrt}
\clFD{native_sin}
\clFD{native_sqrt}
\clFD{native_tan}
\stopCLFD
}

實作可以自行決定 \capi{half_} 函式是否支持去規格化值。
如果引數是去規格化數, \capi{half_} 函式可以返回任何值,只要\todo{節 7.5.3}允許就行,
無論 \ccmm{-cl-denorms-are-zero} (參見\refsec{MathIntrinsicsOption})是否有效。

有下列符號常量可用。這些值的型別都是 \ctype{float},在單精度浮點數的精度內是精確的。
\startCLOD[常量名][描述]

\clOD{\cmacro{MAXFLOAT}}{
最大的有限單精度浮點數。
}

\clOD{\cmacro{HUGE_VALF}}{
正浮點常量算式。其求值結果為 \math{+\infty},由內建數學函式用作返回值表明錯誤。
}

\clOD{\cmacro{INFINITY}}{
常量算式,型別為 \ctype{float},表示正的或無符號的無窮。
}

\clOD{\cmacro{NAN}}{
常量算式,型別為 \ctype{float},表示 quiet NaN。
}
\stopCLOD


如果\cnglo{device}支持雙精度浮點數,還有下列符號常量可用:
\startCLOD[常量名][描述]

\clOD{\cmacro{HUGE_VAL}}{
正浮點常量算式。型別為 \ctype{double}。
其求值結果為 \math{+\infty},由內建數學函式用作返回值表明錯誤。
}

\stopCLOD


% Floating-point macros and pragmas
\subsubsection{浮點巨集和雜注}

雜注(pragma) \cpragmaemp{FP_CONTRACT} 可用來允許(如果狀態是 \ccmm{on})
或禁止(如果狀態是 \ccmm{off})實作化簡算式。
他可位於外部聲明的外面,也可位於複合語句中的顯式聲明或語句的前面。
當在外部聲明外面時,在遇到下一個 \cpragmaemp{FP_CONTRACT} 或者翻譯單元結束時就無效了。
當在複合語句中時,在遇到下一個 \cpragmaemp{FP_CONTRACT}
(包括嵌套的複合語句中的 \cpragmaemp{FP_CONTRACT})或者複合語句結束時就無效了;
在複合語句末尾處,會恢復成此語句之前的狀態。
在其他任何上下文中使用此雜注,其行為都是未定義的。

這樣設置 \cpragmaemp{FP_CONTRACT}:
\startclc
#pragma OPENCL FP_CONTRACT on-off-switch

on-off-switch is one of:
	/BTEX\ftEmp{ON}/ETEX, /BTEX\ftEmp{OFF}/ETEX or /BTEX\ftEmp{DEFAULT}/ETEX.
	The /BTEX\ftEmp{DEFAULT}/ETEX value is /BTEX\ftEmp{ON}/ETEX.
\stopclc

% float - single precision
巨集 \cmacroemp{FP_FAST_FMAF} 用來指明對於單精度浮點數,
函式 \capi{fma} 是否比直接編碼更快。
如果定義了此巨集,則表明對 \ctype{float} 算元的乘、加運算,
函式 \capi{fma} 一般跟直接編碼一樣快,或者更快。

OpenCL C 編程語言定義了如下巨集,他們必須使用指定的值。
可以在預處理指示 \ccmm{#if} 中使用這些常量算式。
\startclc
#define FLT_DIG		6
#define FLT_MANT_DIG	24
#define FLT_MAX_10_EXP	+38
#define FLT_MAX_EXP	+128
#define FLT_MIN_10_EXP	-37
#define FLT_MIN_EXP	-125
#define FLT_RADIX	2
#define FLT_MAX		0x1.fffffep127f
#define FLT_MIN		0x1.0p-126f
#define FLT_EPSILON	0x1.0p-23f
\stopclc

\reftab{tblFltMacroAndApp}中給出了上面所列巨集與\cnglo{app}所用的巨集名字之間的對應關係。

\placetable[here][tab:tblFltMacroAndApp]
{單精度浮點巨集與應用所用巨集的對應關係}
{\startCLOO[OpenCL 語言中的巨集][\cnglo{app}所用的巨集]

\clMMF{DIG}
\clMMF{MANT_DIG}
\clMMF{MAX_10_EXP}
\clMMF{MAX_EXP}
\clMMF{MIN_10_EXP}
\clMMF{MIN_EXP}
\clMMF{RADIX}
\clMMF{MAX}
\clMMF{MIN}
\clMMF{EPSILSON}

\stopCLOO
}

下列兩個巨集將會展開成整數常量算式。
如果 \carg{x} 是 0 或 NaN,則 \math{\mapiemp{ilogb}(x)} 會分別返回這兩個值。
\startigBase
\item \cmacroemp{FP_ILOGB0} 為 \ccmm{{INT_MIN}} 或 \ccmm{-{INT_MAX}}。
\item \cmacroemp{FP_ILOGBNAN} 為 \ccmm{{INT_MAX}} 或 \ccmm{{INT_MIN}}。
\stopigBase

除此之外,還有下列常量可用。
他們的型別都是 \ctype{float},在 \ctype{float} 型別的精度內是精確的。

\startCLOO[常量][描述]

\clCM{M_E_F}{e}
\clCM{M_LOG2E_F}{log_{2}e}
\clCM{M_LOG10E_F}{log_{10}e}
\clCM{M_LN2_F}{log_{e}2}
\clCM{M_LN10_F}{log_{e}10}
\clCM{M_PI_F}{\pi}
\clCM{M_PI_2_F}{\pi/2}
\clCM{M_PI_4_F}{\pi/4}
\clCM{M_1_PI_F}{1/\pi}
\clCM{M_2_PI_F}{2/\pi}
\clCM{M_2_SQRTPI_F}{2/\sqrt{\pi}}
\clCM{M_SQRT2_F}{\sqrt{2}}
\clCM{M_SQRT1_2_F}{1/\sqrt{2}}

\stopCLOO


% double - double precision
如果\cnglo{device}支持雙精度浮點數,還有下列巨集和常量可用:
\startigBase
\item 巨集 \cmacroemp{FP_FAST_FMA} 指明處理雙精度浮點數時,
 \capi{fma} 系列函式是否比直接編碼更快。
如果定義了此巨集,則表明對 \ctype{double} 算元的乘、加運算,
函式 \capi{fma} 一般跟直接編碼一樣快,或者更快。
\stopigBase

OpenCL C 編程語言定義了如下巨集,他們必須使用指定的值。
可以在預處理指示 \ccmm{#if} 中使用這些常量算式。
\startclc
#define DBL_DIG		15
#define DBL_MANT_DIG	53
#define DBL_MAX_10_EXP	+308
#define DBL_MAX_EXP	+1024
#define DBL_MIN_10_EXP	-307
#define DBL_MIN_EXP	-1021
#define DBL_MAX		0x1.fffffffffffffp1023
#define DBL_MIN		0x1.0p-1022
#define DBL_EPSILON	0x1.0p-52
\stopclc

\reftab{tblDblMacroAndApp}中給出了上面所列巨集與\cnglo{app}所用的巨集名字之間的對應關係。

\placetable[here][tab:tblDblMacroAndApp]
{雙精度浮點巨集與應用所用巨集的對應關係}
{\startCLOO[OpenCL 語言中的巨集][應用所用的巨集]

\clMMD{DIG}
\clMMD{MANT_DIG}
\clMMD{MAX_10_EXP}
\clMMD{MAX_EXP}
\clMMD{MIN_10_EXP}
\clMMD{MIN_EXP}
\clMMD{MAX}
\clMMD{MIN}
\clMMD{EPSILSON}

\stopCLOO
}

除此之外,還有下列常量可用。
他們的型別都是 \ctype{double},在 \ctype{double} 型別的精度內是精確的。

\startCLOO[常量][描述]

\clCM{M_E}{e}
\clCM{M_LOG2E}{log_{2}e}
\clCM{M_LOG10E}{log_{10}e}
\clCM{M_LN2}{log_{e}2}
\clCM{M_LN10}{log_{e}10}
\clCM{M_PI}{\pi}
\clCM{M_PI_2}{\pi/2}
\clCM{M_PI_4}{\pi/4}
\clCM{M_1_PI}{1/\pi}
\clCM{M_2_PI}{2/\pi}
\clCM{M_2_SQRTPI}{2/\sqrt{\pi}}
\clCM{M_SQRT2}{\sqrt{2}}
\clCM{M_SQRT1_2}{1/\sqrt{2}}

\stopCLOO



%Integer Functions
\subsection{整數函式}

\reftab{}中列出了內建的整數函式,其引數即可為標量,亦可為矢量。
矢量版本的整數函式按組件逐一運算。其中的描述針對單個組件的。

泛型 \ctype{gentype} 表示函式引數的型別可以是
 \ctype{char}、 \ctype{char{2|3|4|8|16}}、
 \ctype{uchar}、 \ctype{uchar{2|3|4|8|16}}、
 \ctype{short}、 \ctype{short{2|3|4|8|16}}、
 \ctype{ushort}、 \ctype{ushort{2|3|4|8|16}}、
 \ctype{int}、 \ctype{int{2|3|4|8|16}}、
 \ctype{uint}、 \ctype{uint{2|3|4|8|16}}、
 \ctype{long}、 \ctype{long{2|3|4|8|16}}、
 \ctype{ulong} 或 \ctype{ulong{2|3|4|8|16}}。
泛型 \ctype{ugentype} 指代無符號版本的 \ctype{gentype}。
例如,如果 \ctype{gentype} 為 \ctype{char4},則 \ctype{ugentype} 為 \ctype{uchar4}。
同時,泛型 \ctype{sgentype} 指明函式的引數可以是標量(即
 \ctype{char}、 \ctype{uchar}、 \ctype{short}、 \ctype{ushort}、
 \ctype{int}、 \ctype{uint}、 \ctype{long} 或 \ctype{ulong})。
對於既有 \ctype{gentype} 引數,又有 \ctype{sgentype} 引數的內建整數函式,
 \ctype{gentype} 必須是標量或矢量版本的 \ctype{sgentype}。
例如,如果 \ctype{sgentype} 是 \ctype{uchar},
則 \ctype{gentype} 必須是 \ctype{uchar} 或 \ctype{uchar{2|3|4|8|16}}。
對於矢量版本, \ctype{sgentype} 只是簡單的拓寬成 \ctype{gentype},
參見\refsec{operator}中的\refitem{arithoperator}。


\subsection[sec:commonFunc]{公共函式\footnote{
可以使用化簡(如 \capi{mad} 或 \capi{fma})來實現 \capi{mix} 和 \capi{smoothstep}。}}

\reftab{svCommonFunc}中列出了內建的公共函式。
這些函式都是按組件逐一運算的,其中的描述也是針對單個組件的。
泛型 \ctype{gentype} 表明函式的引數可以是:
\startigBase[indentnext=no]
\item \ctype{float}、
\item \ctype{float{2|3|4|8|16}}、
\item \ctype{double}、
\item \ctype{double{2|3|4|8|16}}。
\stopigBase
泛型 \ctype{gentypef} 表明函式的引數可以是:
\startigBase[indentnext=no]
\item \ctype{float}、
\item \ctype{float{2|3|4|8|16}}。
\stopigBase
泛型 \ctype{gentyped} 表明函式的引數可以是:
\startigBase[indentnext=no]
\item \ctype{double}、
\item \ctype{double{2|3|4|8|16}}。
\stopigBase

內建的公共函式實現時用的捨入模式是捨入為最近偶數。

\placetable[here][tab:svCommonFunc]
{引數既可為標量整數,也可為矢量整數的內建公共函式}
{% clamp
\startbuffer[funcproto:clamp]
gentype clamp (gentype x,
		gentype minval,
		gentype maxval)
gentypef clamp (gentypef x,
		float minval,
		float maxval)
gentyped clamp (gentyped x,
		double minval,
		double maxval)
\stopbuffer
\startbuffer[funcdesc:clamp]
返回 \math{\text{\capi{fmin}}(\text{\capi{fmax}}(x,minval),maxval)}。
如果 \math{minval > maxval},則結果未定義。
\stopbuffer

% degrees
\startbuffer[funcproto:degrees]
gentype degrees (gentype radians)
\stopbuffer
\startbuffer[funcdesc:degrees]
將弧度轉換成角度,即 \math{(180/\pi)\times radians}。
\stopbuffer

% max
\startbuffer[funcproto:max]
gentype max (gentype x, gentype y)
gentypef max (gentypef x, float y)
gentyped max (gentyped x, double y)
\stopbuffer
\startbuffer[funcdesc:max]
如果 \math{x<y},則返回 \math{y},否則返回 \math{x}。
如果 \math{x} 或 \math{y} 是無窮或 NaN,則返回的值未定義。
\stopbuffer

% min
\startbuffer[funcproto:min]
gentype min (gentype x, gentype y)
gentypef min (gentypef x, float y)
gentyped min (gentyped x, double y)
\stopbuffer
\startbuffer[funcdesc:min]
如果 \math{y<x},則返回 \math{y},否則返回 \math{x}。
如果 \math{x} 或 \math{y} 是無窮或 NaN,則返回的值未定義。
\stopbuffer

% mix
\startbuffer[funcproto:mix]
gentype mix (gentype x,
		gentype y,
		gentype a)
gentypef mix (gentypef x,
		gentypef y,
		float a)
gentyped mix (gentyped x,
		gentyped y,
		double a)
\stopbuffer
\startbuffer[funcdesc:mix]
返回 \math{x + (y-x)\times a}。
其中 \math{a} 必須在區間 \math{0.0 \cdots 1.0} 內,否則返回值未定義。
\stopbuffer

% radians
\startbuffer[funcproto:radians]
gentype radians (gentype degrees)
\stopbuffer
\startbuffer[funcdesc:radians]
將角度轉換成弧度,即 \math{(\pi/180)\times degrees}。
\stopbuffer

% step
\startbuffer[funcproto:step]
gentype step (gentype edge,
		gentype x)
gentypef step (float edge,
		gentypef x)
gentyped step (double edge,
		gentyped x)
\stopbuffer
\startbuffer[funcdesc:step]
如果 \math{x < edge},則返回 \math{0.0},否則返回 \math{1.0}。
\stopbuffer

% smoothstep
\startbuffer[funcproto:smoothstep]
gentype smoothstep (gentype edge0,
		gentype edge1,
		gentype x)
gentypef smoothstep (float edge0,
		float edge1,
		gentypef x)
gentyped smoothstep (double edge0,
		double edge1,
		gentyped x)
\stopbuffer
\startbuffer[funcdesc:smoothstep]
如果 \math{x \leq edge0},則返回 \math{0.0};

如果 \math{x \geq edge1},則返回 \math{1.0};

如果 \math{edge0 < x < edge1},
則實施平滑埃爾米特插值(smooth Hermite interpolation)。
在某些地方需要能進行平滑過渡的臨界函式,這時就可使用此函式。

此函式等同於:
{\setuptyping[option=none]\starttyping
gentype t;
t = clamp((x - edge0) / (edge1 - edge1), 0, 1);
return t * t * (3 - 2 * t);
\stoptyping}

如果 \math{edge0 \geq edge1},
或者 \cmath{x}、 \cmath{edge0}、 \cmath{edge1} 中的任意一個是 NaN,則結果未定義。
\stopbuffer

% sign
\startbuffer[funcproto:sign]
gentype sign (gentype x)
\stopbuffer
\startbuffer[funcdesc:sign]
如果 \math{x > 0},則返回 \math{1.0};
如果 \math{x = -0.0},則返回 \math{-0.0};
如果 \math{x = +0.0},則返回 \math{+0.0};
如果 \math{x < 0},則返回 \math{-1.0};
如果 \cmath{x} 是 NaN,則返回 \math{0.0}。
\stopbuffer

% begin TABLE
\startCLFD

\clFD{clamp}
\clFD{degrees}
\clFD{max}
\clFD{min}
\clFD{mix}
\clFD{radians}
\clFD{step}
\clFD{smoothstep}
\clFD{sign}

\stopCLFD

}

\subsection{幾何函式\footnote{
可以使用化簡(如 \capi{mad} 或 \capi{fma})來實現幾何函式。}}

\reftab{svGeometricFunc}中列出了內建的幾何函式。
這些函式都是按組件逐一運算的,其中的描述也是針對單個組件的。
\cldtv{float} 表示
 \ctype{float}、 \ctype{float2}、 \ctype{float3} 或 \ctype{float4};
而 \cldtv{double} 則表示
 \ctype{double}、 \ctype{double2}、 \ctype{double3} 或 \ctype{double4}。

內建的幾何函式實現時用的捨入模式是向最近偶數捨入。

\placetable[here,force,split][tab:svGeometricFunc]
{引數既可為標量,也可為矢量的內建幾何函式}
{% cross
\startbuffer[funcproto:cross]
float4 cross (float4 p0, float4 p1)
float3 cross (float3 p0, float3 p1)
double4 cross (double4 p0, double4 p1)
double3 cross (double3 p0, double3 p1)
\stopbuffer
\startbuffer[funcdesc:cross]
返回 \math{p0.xyz} 和 \math{p1.xyz} 的叉乘。
所返回的 \ctype{float4} 中的組件 \math{w} 為 \math{0.0}。
\stopbuffer

% dot
\startbuffer[funcproto:dot]
float dot (floatn p0, floatn p1)
double dot (doublen p0, doublen p1)
\stopbuffer
\startbuffer[funcdesc:dot]
計算點乘。
\stopbuffer

% distance
\startbuffer[funcproto:distance]
float distance (floatn p0,
		floatn p1)
double distance (doublen p0,
		doublen p1)
\stopbuffer
\startbuffer[funcdesc:distance]
返回 \carg{p0} 和 \carg{p1} 的距離。
即 \math{\mapiemp{length}(\marg{p0} - \marg{p1})}。
\stopbuffer

% length
\startbuffer[funcproto:length]
float length (floatn p)
double length (gentype p)
\stopbuffer
\startbuffer[funcdesc:length]
返回矢量 \carg{p} 的長度,即 \math{\sqrt{p.x^2+p.y^2+\cdots}}。
\stopbuffer

% normalize
\startbuffer[funcproto:normalize]
floatn normalize (floatn p)
doublen normalize (doublen p)
\stopbuffer
\startbuffer[funcdesc:normalize]
返回的矢量方向與 \carg{p} 相同,長度為 \math{1}。
\stopbuffer

% fast_distance
\startbuffer[funcproto:fast_distance]
float fast_distance (floatn p0,
		floatn p1)
\stopbuffer
\startbuffer[funcdesc:fast_distance]
返回 \math{\mapiemp{fast_length}(\marg{p0} - \marg{p1})}。
\stopbuffer

% fast_length
\startbuffer[funcproto:fast_length]
float fast_length (floatn p)
\stopbuffer
\startbuffer[funcdesc:fast_length]
返回 \math{\mapiemp{half_sqrt}(p.x^2 + p.y^2 + \cdots)}。
\stopbuffer

% fast_normalize
\startbuffer[funcproto:fast_normalize]
floatn fast_normalize (floatn p)
\stopbuffer
\startbuffer[funcdesc:fast_normalize]
返回的矢量方向與 \math{p} 相同,長度為 \math{1}。
\capi{fast_normalize} 是這樣計算的:

\math{p * \mapiemp{half_rsqrt}(p.x^2 + p.y^2 + \cdots)}

返回值與下面語句結果(具有無窮精度)的誤差在 8192 ulp 內:
\startcintbl[escape=yes,indentnext=no]
if (/BTEX\capi{all}/ETEX(/BTEX\math{p == 0.0f}/ETEX))
	/BTEX\math{result = p}/ETEX;
else
	/BTEX\math{result = p / \mapiemp{sqrt}(p.x^2 + p.y^2 + \cdots )}/ETEX;
\stopcintbl
但是下列情況例外:
\startigNum
\item 如果平方和大於 \cmacro{FLT_MAX},則結果中對應的浮點值未定義。

\item 如果平方和小於 \cmacro{FLT_MIN},則實作可能直接返回 \math{p}。

\item 如果\cnglo{device}處於“將去規格化數刷成零”的模式,
則在計算前會將算元中
量級小於 \math{\mapiemp{sqrt}(\mmacro{FLT_MIN})} 的元素刷成零。
\stopigNum
\stopbuffer

% begin TABLE
\startCLFD

\clFD{cross}
\clFD{dot}
\clFD{distance}
\clFD{length}
\clFD{normalize}
\clFD{fast_distance}
\clFD{fast_length}
\clFD{fast_normalize}

\stopCLFD

}

\subsection{關係函式}

可以使用關係算子和相等算子(<、 <=、 >、 >=、 !=、 ==)對內建標量和矢量型別進行關係運算,
所產生的結果分別為標量或矢量帶符號整形,參見\refsec{operator}。

\reftab{svRelationalFunc}中所列函式\footnote{
如果實作對規範進行了擴充,從而支持 IEEE-754 标志和異常,
則當有一個或多個算數是 NaN 時,
\reftab{svRelationalFunc}中所定義的內建函式不會引發{\ftRef{無效(invalid)}}浮點異常。}
可以內建標量或矢量型別為引數,返回的結果為標量或矢量整形。
泛型 \ctype{gentype} 指代下列內建型別:
 \cldts{char}、 \cldtv{char}、 \cldts{uchar}、 \cldtv{uchar}、
 \cldts{short}、 \cldtv{short}、 \cldts{ushort}、 \cldtv{ushort}、
 \cldts{int}、 \cldtv{int}、 \cldts{uint}、 \cldtv{uint}、
 \cldts{long}、 \cldtv{long}、 \cldts{ulong}、 \cldtv{ulong}、
 \cldts{float}、 \cldtv{float}、 \cldts{double} 和 \cldtv{double}。
泛型 \ctype{igentype} 指代內建帶符號整形,即:
 \cldts{char}、 \cldtv{char}、 \cldts{short}、 \cldtv{short}、
 \cldts{int}、 \cldtv{int}、 \cldts{long} 和 \cldtv{long}。
泛型 \ctype{ugentype} 指代內建無符號整形,即:
 \cldts{uchar}、 \cldtv{uchar}、 \cldts{ushort}、 \cldtv{ushort}、
 \cldts{uint}、 \cldtv{uint}、 \cldts{ulong} 和 \cldtv{ulong}。
其中 \cldtvfix{n} 為 2、 3、 4、 8 或 16。

對於標量型別的引數,如果所指定的關係為 {\ftRef{false}},則\reftab{svRelationalFunc}中的函式
 \capi{isequal}、 \capi{isnotequal}、 \capi{isgreater}、 \capi{isgreaterequal}、
 \capi{isless}、 \capi{islessequal}、 \capi{islessgreater}、 \capi{isfinite}、
 \capi{isinf}、 \capi{isnan}、 \capi{isnormal}、 \capi{isordered}、
 \capi{isunordered} 和 \capi{signbit} 會返回 0,否則返回 1。
而對於矢量型別的引數,如果所指定的關係為 {\ftRef{false}},則返回 0,
否則返回 -1 (即所有位都是 1)。

如果任一引數為 NaN,則關係函式
 \capi{isequal}、 \capi{isgreater}、 \capi{isgreaterequal}、
 \capi{isless}、 \capi{islessequal} 和 \capi{islessgreater}
返回 0。
如果引數為標量,則當任一引數為 NaN 時, \capi{isnotequal} 返回 1;
而如果引數為矢量,則當任一引數為 NaN 時, \capi{isnotequal} 返回 -1。

\placetable[here,force,split][tab:svRelationalFunc]
{標量和矢量關係函式}
{% isequal
\startbuffer[funcproto:isequal]
int isequal (float x, float y)
intn isequal (floatn x, floatn y) 
int isequal (double x, double y)
longn isequal (doublen x, doublen y)
\stopbuffer
\startbuffer[funcdesc:isequal]
按組件逐一比較 \math{\marg{x} == \marg{y}}。
\stopbuffer

% isnotequal
\startbuffer[funcproto:isnotequal]
int isnotequal (float x, float y)
intn isnotequal (floatn x, floatn y)
int isnotequal (double x, double y)
longn isnotequal (doublen x, doublen y)
\stopbuffer
\startbuffer[funcdesc:isnotequal]
按組件逐一比較 \math{\marg{x} != \marg{y}}。
\stopbuffer

% isgreater
\startbuffer[funcproto:isgreater]
int isgreater (float x, float y)
intn isgreater (floatn x, floatn y)
int isgreater (double x, double y)
longn isgreater (doublen x, doublen y)
\stopbuffer
\startbuffer[funcdesc:isgreater]
按組件逐一比較 \math{\marg{x} > \marg{y}}。
\stopbuffer

% isgreaterequal
\startbuffer[funcproto:isgreaterequal]
int isgreaterequal (float x, float y)
intn isgreaterequal (floatn x, floatn y)
int isgreaterequal (double x,
		double y)
longn isgreaterequal (doublen x,
		doublen y)
\stopbuffer
\startbuffer[funcdesc:isgreaterequal]
按組件逐一比較 \math{\marg{x} \geq \marg{y}}。
\stopbuffer

% isless
\startbuffer[funcproto:isless]
int isless (float x, float y)
intn isless (floatn x, floatn y)
int isless (double x, double y)
longn isless (doublen x, doublen y)
\stopbuffer
\startbuffer[funcdesc:isless]
按組件逐一比較 \math{\marg{x} < \marg{y}}。
\stopbuffer

% islessequal
\startbuffer[funcproto:islessequal]
int islessequal (float x, float y)
intn islessequal (floatn x, floatn y)
int islessequal (double x, double y)
longn islessequal (doublen x, doublen y)
\stopbuffer
\startbuffer[funcdesc:islessequal]
按組件逐一比較 \math{\marg{x} \leq \marg{y}}。
\stopbuffer

% islessgreater
\startbuffer[funcproto:islessgreater]
int islessgreater (float x, float y)
intn islessgreater (floatn x, floatn y)
int islessgreater (double x, double y)
longn islessgreater (doublen x, doublen y)
\stopbuffer
\startbuffer[funcdesc:islessgreater]
按組件逐一比較 \math{(\marg{x} < \marg{y}) \mcmm{||} (\marg{x} > \marg{y})}。
\stopbuffer

% isfinite
\startbuffer[funcproto:isfinite]
int isfinite (float)
intn isfinite (floatn)
int isfinite (double)
longn isfinite (doublen)
\stopbuffer
\startbuffer[funcdesc:isfinite]
測試引數是否為有限值。
\stopbuffer

% isinf
\startbuffer[funcproto:isinf]
int isinf (float)
intn isinf (floatn)
int isinf (double)
longn isinf (doublen)
\stopbuffer
\startbuffer[funcdesc:isinf]
測試引數是否為無窮值(正數或負數)。
\stopbuffer

% isnan
\startbuffer[funcproto:isnan]
int isnan (float)
intn isnan (floatn)
int isnan (double)
longn isnan (doublen)
\stopbuffer
\startbuffer[funcdesc:isnan]
測試引數是否為 NaN。
\stopbuffer

% isnormal
\startbuffer[funcproto:isnormal]
int isnormal (float)
intn isnormal (floatn)
int isnormal (double)
longn isnormal (doublen)
\stopbuffer
\startbuffer[funcdesc:isnormal]
測試引數是否為規格化值。
\stopbuffer

% isordered
\startbuffer[funcproto:isordered]
int isordered (float x, float y)
intn isordered (floatn x, floatn y)
int isordered (double x, double y)
longn isordered (doublen x, doublen y)
\stopbuffer
\startbuffer[funcdesc:isordered]
測試引數是否規則。
相當於 \math{\mapiemp{isequal}(\marg{x}, \marg{x}) \mcmm{&&} \mapiemp{isequal}(\marg{y}, \marg{y})}。
\stopbuffer

% isunordered
\startbuffer[funcproto:isunordered]
int isunordered (float x, float y)
intn isunordered (floatn x, floatn y)
int isunordered (double x, double y)
longn isunordered (doublen x, doublen y)
\stopbuffer
\startbuffer[funcdesc:isunordered]
測試引數是否不規則。
如果引數 \carg{x} 或 \carg{y} 是 NaN,則返回非零值,否則返回零。
\stopbuffer

% signbit
\startbuffer[funcproto:signbit]
int signbit (float)
intn signbit (floatn)
int signbit (double)
longn signbit (doublen)
\stopbuffer
\startbuffer[funcdesc:signbit]
測試符號位。
對於此函式的標量版本,如果設置了符號位,則返回 1,否則返回 0。
而在此函式的標量版本中,對於矢量的每個組件,
如果設置了符號位則返回 -1 (即所有位都是 1),否則返回 0。
\stopbuffer

% any
\startbuffer[funcproto:any]
int any (igentype x)
\stopbuffer
\startbuffer[funcdesc:any]
如果 \carg{x} 中任一組件的最高位是 1,則返回 1;否則返回 0。
\stopbuffer

% all
\startbuffer[funcproto:all]
int all (igentype x)
\stopbuffer
\startbuffer[funcdesc:all]
如果 \carg{x} 中所有組件的最高位都是 1,則返回 1;否則返回 0。
\stopbuffer

% bitselect
\startbuffer[funcproto:bitselect]
gentype bitselect (gentype a,
		gentype b,
		gentype c)
\stopbuffer
\startbuffer[funcdesc:bitselect]
如果 \carg{c} 中的某一位為 0,則選取 \carg{a} 中的對應位作為結果中對應位的值;
否則選取 \carg{b} 中的對應位作為結果中對應位的值。
\stopbuffer

% select
\startbuffer[funcproto:select]
gentype select (gentype a,
		gentype b,
		igentype c)
gentype select (gentype a,
		gentype b,
		ugentype c)
\stopbuffer
\startbuffer[funcdesc:select]
對於矢量型別中的每個組件,
如果 \math{\marg{c}[i]} 的最高位為 1,則結果為 \math{\marg{b}[i]},否則為 \math{\marg{a}[i]}。

對於標量型別,\math{result = \marg{c} ? \marg{b} : \marg{a}}。

\ctype{igentype} 和 \ctype{ugentype} 的元素數目以及元素的位數
都必須與 \ctype{gentype} 相同。
\stopbuffer

% begin table
\startCLFD

\clFD{isequal}
\clFD{isnotequal}
\clFD{isgreater}
\clFD{isgreaterequal}
\clFD{isless}
\clFD{islessequal}
\clFD{islessgreater}
\clFD{isfinite}
\clFD{isinf}
\clFD{isnan}
\clFD{isnormal}
\clFD{isordered}
\clFD{isunordered}
\clFD{signbit}
\clFD{any}
\clFD{all}
\clFD{bitselect}
\clFD{select}

\stopCLFD
}


\subsection{矢量數據裝載和存儲函式}

\startbuffer
僅在擴展 \clext{cl_khr_fp16} 中才定義有 \cldt[n]{half}%
(參見 OpenCL 1.2 擴展規範的節 9.5)。
\stopbuffer
\reftab{vectorLsFunc}中列出了用來讀寫矢量型別數據的內建函式。
泛型 \cldt{gentype} 表示內建數據型別
 \cldt{char}、 \cldt{uchar}、 \cldt{short}、 \cldt{ushort}、
 \cldt{int}、 \cldt{uint}、 \cldt{long}、 \cldt{ulong}、
 \cldt{float} 或 \cldt{double}。
泛型 \cldt[n]{gentype} 表示具有 n 個 \cldt{gentype} 元素的矢量。
我們用 \cldt[n]{half} 表示具有 n 個 \cldt{half} 元素的矢量\footnote{\getbuffer}。
函式名中也有後綴 \ctypesuffix{n} (即 \clapiv{vload}、 \clapiv{vstore} 等),
其中 \ctypesuffix{n} 為 2、 3、 4、 8 或 16。

\placetable[here,force,split][tab:vectorLsFunc]
{矢量數據裝載、存儲函式表}
{

% vloadn
\startbuffer[funcproto:vloadn]
\stopbuffer
\startbuffer[funcdesc:vloadn]
\stopbuffer

% vloadn
\startbuffer[funcproto:vloadn]
gentypen vloadn (size_t offset, 
	const __global gentype *p) 
gentypen vloadn (size_t offset,
	const __local gentype *p)
gentypen vloadn (size_t offset,
	const __constant gentype *p)
gentypen vloadn (size_t offset,
	const __private gentype *p)
\stopbuffer
\startbuffer[funcdesc:vloadn]
由位址 \math{(p + (offset \times n))} 讀取
 \math{\text{\capi{sizeof}}(\text{\cldt[n]{gentype}})} 字節的數據並將其返回。
對於位址 \math{(p + (offset \times n))} 而言,
如果 \cldt{gentype} 為 \cldt{char}、 \cldt{uchar},則他必須按 8 位對齊;
如果 \cldt{gentype} 為 \cldt{short}、 \cldt{ushort},則他必須按 16 位對齊;
如果 \cldt{gentype} 為 \cldt{int}、 \cldt{uint}、 \cldt{float},
則他必須按 32 位對齊;
如果 \cldt{gentype} 為 \cldt{long}、 \cldt{ulong},則他必須按 64 位對齊。
\stopbuffer

% vstoren
\startbuffer[funcproto:vstoren]
void vstoren (gentypen data,
	size_t offset,
	__global gentype *p)
void vstoren (gentypen data,
	size_t offset,
	__local gentype *p)
void vstoren (gentypen data,
	size_t offset,
	__private gentype *p)
\stopbuffer
\startbuffer[funcdesc:vstoren]
將 \carg{data} 中
 \math{\text{\capi{sizeof}}(\text{\cldt[n]{gentype}})} 字節的數據寫入
位址 \math{(p + (offset \times n))} 中。
對於位址 \math{(p + (offset \times n))} 而言,
如果 \cldt{gentype} 為 \cldt{char}、 \cldt{uchar},則他必須按 8 位對齊;
如果 \cldt{gentype} 為 \cldt{short}、 \cldt{ushort},則他必須按 16 位對齊;
如果 \cldt{gentype} 為 \cldt{int}、 \cldt{uint}、 \cldt{float},
則他必須按 32 位對齊;
如果 \cldt{gentype} 為 \cldt{long}、 \cldt{ulong},則他必須按 64 位對齊。
\stopbuffer

% vload_half
\startbuffer[funcproto:vload_half]
float vload_half (size_t offset,
	const __global half *p)
float vload_half (size_t offset,
	const __local half *p)
float vload_half (size_t offset,
	const __constant half *p)
float vload_half (size_t offset,
	const __private half *p)
\stopbuffer
\startbuffer[funcdesc:vload_half]
由位址 \math{(p + offset)} 讀取
 \math{\text{\capi{sizeof}}(\text{\cldt{half}})} 字節的數據。
將讀到的數據按 \cldt{half} 值解釋,將其轉換為 \cldt{float} 後返回。
位址 \math{(p + offset)} 必須按 16 位對齊。
\stopbuffer

% vload_halfn
\startbuffer[funcproto:vload_halfn]
floatn vload_halfn (size_t offset,
	const __global half *p)
floatn vload_halfn (size_t offset,
	const __local half *p)
floatn vload_halfn (size_t offset,
	const __constant half *p)
floatn vload_halfn (size_t offset,
	const __private half *p)
\stopbuffer
\startbuffer[funcdesc:vload_halfn]
由位址 \math{(p + (offset \times n))} 讀取
 \math{\text{\capi{sizeof}}(\text{\cldt[n]{half}})} 字節的數據。
將讀到的數據按 \cldt[n]{half} 值解釋,將其轉換為 \cldt[n]{float} 後返回。
位址 \math{(p + (offset \times n))} 必須按 16 位對齊。
\stopbuffer

% vstore_half_float
\startbuffer[funcproto:vstore_half_float]
void vstore_half (float data,
	size_t offset,
	__global half *p)
void vstore_half_rte (float data,
	size_t offset,
	__global half *p)
void vstore_half_rtz (float data,
	size_t offset,
	__global half *p)
void vstore_half_rtp (float data,
	size_t offset,
	__global half *p)
void vstore_half_rtn (float data,
	size_t offset,
	__global half *p)

void vstore_half (float data,
	size_t offset,
	__local half *p)
void vstore_half_rte (float data,
	size_t offset,
	__local half *p)
void vstore_half_rtz (float data,
	size_t offset,
	__local half *p)
void vstore_half_rtp (float data,
	size_t offset,
	__local half *p)
void vstore_half_rtn (float data,
	size_t offset,
	__local half *p)

void vstore_half (float data,
	size_t offset,
	__private half *p)
void vstore_half_rte (float data,
	size_t offset,
	__private half *p)
void vstore_half_rtz (float data,
	size_t offset,
	__private half *p)
void vstore_half_rtp (float data,
	size_t offset,
	__private half *p)
void vstore_half_rtn (float data,
	size_t offset,
	__private half *p)
\stopbuffer
\startbuffer[funcdesc:vstore_half_float]
先按某種捨入模式將 \carg{data} 中的 \cldt{float} 值轉換為 \cldt{half} 值。
然後將其寫入位址 \math{p + offset} 中。
位址 \math{p + offset} 必須按 16 位對齊。

\clapis{vstore_half} 使用缺省的捨入模式。
缺省的捨入模式為捨入為最近偶數。
\stopbuffer

% vstore_halfn_float
\startbuffer[funcproto:vstore_halfn_float]
void vstore_halfn (floatn data,
	size_t offset,
	__global half *p)
void vstore_halfn_rte (floatn data,
	size_t offset,
	__global half *p)
void vstore_halfn_rtz (floatn data,
	size_t offset,
	__global half *p)
void vstore_halfn_rtp (floatn data,
	size_t offset,
	__global half *p)
void vstore_halfn_rtn (floatn data,
	size_t offset,
	__global half *p)

void vstore_halfn (floatn data,
	size_t offset,
	__local half *p)
void vstore_halfn_rte (floatn data,
	size_t offset,
	__local half *p)
void vstore_halfn_rtz (floatn data,
	size_t offset,
	__local half *p)
void vstore_halfn_rtp (floatn data,
	size_t offset,
	__local half *p)
void vstore_halfn_rtn (floatn data,
	size_t offset,
	__local half *p)

void vstore_halfn (floatn data,
	size_t offset,
	__private half *p)
void vstore_halfn_rte (floatn data,
	size_t offset,
	__private half *p)
void vstore_halfn_rtz (floatn data,
	size_t offset,
	__private half *p)
void vstore_halfn_rtp (floatn data,
	size_t offset,
	__private half *p)
void vstore_halfn_rtn (floatn data,
	size_t offset,
	__private half *p)
\stopbuffer
\startbuffer[funcdesc:vstore_halfn_float]
先按某種捨入模式將 \carg{data} 中的 \cldt[n]{float} 值轉換為 \cldt[n]{half} 值。
然後將其寫入位址 \math{(p + (offset \times n))} 中。
位址 \math{(p + (offset \times n))} 必須按 16 位對齊。

\clapiv{vstore_half} 使用缺省的捨入模式。
缺省的捨入模式為捨入為最近偶數。
\stopbuffer

% vstore_half_double
\startbuffer[funcproto:vstore_half_double]
void vstore_half (double data,
	size_t offset,
	__global half *p)
void vstore_half_rte (double data,
	size_t offset,
	__global half *p)
void vstore_half_rtz (double data,
	size_t offset,
	__global half *p)
void vstore_half_rtp (double data,
	size_t offset,
	__global half *p)
void vstore_half_rtn (double data,
	size_t offset,
	__global half *p)

void vstore_half (double data,
	size_t offset,
	__local half *p)
void vstore_half_rte (double data,
	size_t offset,
	__local half *p)
void vstore_half_rtz (double data,
	size_t offset,
	__local half *p)
void vstore_half_rtp (double data,
	size_t offset,
	__local half *p)
void vstore_half_rtn (double data,
	size_t offset,
	__local half *p)

void vstore_half (double data,
	size_t offset,
	__private half *p)
void vstore_half_rte (double data,
	size_t offset,
	__private half *p)
void vstore_half_rtz (double data,
	size_t offset,
	__private half *p)
void vstore_half_rtp (double data,
	size_t offset,
	__private half *p)
void vstore_half_rtn (double data,
	size_t offset,
	__private half *p)
\stopbuffer
\startbuffer[funcdesc:vstore_half_double]
先按某種捨入模式將 \carg{data} 中的 \cldt{double} 值轉換為 \cldt{half} 值。
然後將其寫入位址 \math{p + offset} 中。
位址 \math{p + offset} 必須按 16 位對齊。

\clapis{vstore_half} 使用缺省的捨入模式。
缺省的捨入模式為捨入為最近偶數。
\stopbuffer

% vstore_halfn_double
\startbuffer[funcproto:vstore_halfn_double]
void vstore_halfn (doublen data,
	size_t offset,
	__global half *p)
void vstore_halfn_rte (doublen data,
	size_t offset,
	__global half *p)
void vstore_halfn_rtz (doublen data,
	size_t offset,
	__global half *p)
void vstore_halfn_rtp (doublen data,
	size_t offset,
	__global half *p)
void vstore_halfn_rtn (doublen data,
	size_t offset,
	__global half *p)

void vstore_halfn (doublen data,
	size_t offset,
	__local half *p)
void vstore_halfn_rte (doublen data,
	size_t offset,
	__local half *p)
void vstore_halfn_rtz (doublen data,
	size_t offset,
	__local half *p)
void vstore_halfn_rtp (doublen data,
	size_t offset,
	__local half *p)
void vstore_halfn_rtn (doublen data,
	size_t offset,
	__local half *p)

void vstore_halfn (doublen data,
	size_t offset,
	__private half *p)
void vstore_halfn_rte (doublen data,
	size_t offset,
	__private half *p)
void vstore_halfn_rtz (doublen data,
	size_t offset,
	__private half *p)
void vstore_halfn_rtp (doublen data,
	size_t offset,
	__private half *p)
void vstore_halfn_rtn (doublen data,
	size_t offset,
	__private half *p)
\stopbuffer
\startbuffer[funcdesc:vstore_halfn_double]
先按某種捨入模式將 \carg{data} 中的 \cldt[n]{double} 值轉換為 \cldt[n]{half} 值。
然後將其寫入位址 \math{(p + (offset \times n))} 中。
位址 \math{(p + (offset \times n))} 必須按 16 位對齊。

\clapiv{vstore_half} 使用缺省的捨入模式。
缺省的捨入模式為捨入為最近偶數。
\stopbuffer

% vloada_halfn
\startbuffer[funcproto:vloada_halfn]
floatn vloada_halfn (size_t offset,
	const __global half *p)
floatn vloada_halfn (size_t offset,
	const __local half *p)
floatn vloada_halfn (size_t offset,
	const __constant half *p)
floatn vloada_halfn (size_t offset,
	const __private half *p)
\stopbuffer
\startbuffer[funcdesc:vloada_halfn]
對於 n 為 1、 2、 4、 8 和 16,
由位址 \math{(p + (offset \times n))} 讀取
 \math{\text{\capi{sizeof}}(\text{\cldt[n]{half}})} 字節的數據。
讀到的數據解釋為 \cldt[n]{half} 值,將其轉換為 \cldt[n]{float} 值後返回。

位址 \math{(p + (offset \times n))} 必須按
 \math{\text{\capi{sizeof}}(\text{\cldt[n]{half}})} 字節對齊。

如果 n = 3,則由位址 \math{(p + (offset \times 4))} 讀取
 \ctype{half3} 並返回 \ctype{float3}。
位址 \math{(p + (offset \times 4))} 按
 \math{\text{\capi{sizeof}}(\text{\cldt{half}}) \times 4} 字節對齊。
\stopbuffer

% vstorea_halfn_float
\startbuffer[funcproto:vstorea_halfn_float]
void vstorea_halfn (floatn data,
	size_t offset,
	__global half *p)
void vstorea_halfn_rte (floatn data,
	size_t offset,
	__global half *p)
void vstorea_halfn_rtz (floatn data,
	size_t offset,
	__global half *p)
void vstorea_halfn_rtp (floatn data,
	size_t offset,
	__global half *p)
void vstorea_halfn_rtn (floatn data,
	size_t offset,
	__global half *p)

void vstorea_halfn (floatn data,
	size_t offset,
	__local half *p)
void vstorea_halfn_rte (floatn data,
	size_t offset,
	__local half *p)
void vstorea_halfn_rtz (floatn data,
	size_t offset,
	__local half *p)
void vstorea_halfn_rtp (floatn data,
	size_t offset,
	__local half *p)
void vstorea_halfn_rtn (floatn data,
	size_t offset,
	__local half *p)

void vstorea_halfn (floatn data,
	size_t offset,
	__private half *p)
void vstorea_halfn_rte (floatn data,
	size_t offset,
	__private half *p)
void vstorea_halfn_rtz (floatn data,
	size_t offset,
	__private half *p)
void vstorea_halfn_rtp (floatn data,
	size_t offset,
	__private half *p)
void vstorea_halfn_rtn (floatn data,
	size_t offset,
	__private half *p)
\stopbuffer
\startbuffer[funcdesc:vstorea_halfn_float]
按某種捨入模式將 \carg{data} 中的 \cldt[n]{float} 轉換為 \cldt[n]{half}。

如果 n 為 1、 2、 4、 8 和 16,
則將 \cldt[n]{half} 值寫入位址 \math{(p + (offset \times n))}
位址 \math{(p + (offset \times n))} 必須按
 \math{\text{\capi{sizeof}}(\text{\cldt[n]{half}})} 字節對齊。

如果 n = 3,
則將 \ctype{half3} 值寫入位址 \math{(p + (offset \times 4))}
位址 \math{(p + (offset \times 4))} 按
 \math{\text{\capi{sizeof}}(\text{\cldt{half}}) \times 4} 字節對齊。

\capi{vstorea_halfn} 使用缺省的捨入模式。
缺省的捨入模式為捨入為最近偶數。
\stopbuffer

% vstorea_halfn_double
\startbuffer[funcproto:vstorea_halfn_double]
void vstorea_halfn (doublen data,
	size_t offset,
	__global half *p)
void vstorea_halfn_rte (doublen data,
	size_t offset,
	__global half *p)
void vstorea_halfn_rtz (doublen data,
	size_t offset,
	__global half *p)
void vstorea_halfn_rtp (doublen data,
	size_t offset,
	__global half *p)
void vstorea_halfn_rtn (doublen data,
	size_t offset,
	__global half *p)

void vstorea_halfn (doublen data,
	size_t offset,
	__local half *p)
void vstorea_halfn_rte (doublen data,
	size_t offset,
	__local half *p)
void vstorea_halfn_rtz (doublen data,
	size_t offset,
	__local half *p)
void vstorea_halfn_rtp (doublen data,
	size_t offset,
	__local half *p)
void vstorea_halfn_rtn (doublen data,
	size_t offset,
	__local half *p)

void vstorea_halfn (doublen data,
	size_t offset,
	__private half *p)
void vstorea_halfn_rte (doublen data,
	size_t offset,
	__private half *p)
void vstorea_halfn_rtz (doublen data,
	size_t offset,
	__private half *p)
void vstorea_halfn_rtp (doublen data,
	size_t offset,
	__private half *p)
void vstorea_halfn_rtn (doublen data,
	size_t offset,
	__private half *p)
\stopbuffer
\startbuffer[funcdesc:vstorea_halfn_double]
按某種捨入模式將 \carg{data} 中的 \cldt[n]{double} 轉換為 \cldt[n]{half}。

如果 n 為 1、 2、 4、 8 和 16,
則將 \cldt[n]{half} 值寫入位址 \math{(p + (offset \times n))}
位址 \math{(p + (offset \times n))} 必須按
 \math{\text{\capi{sizeof}}(\text{\cldt[n]{half}})} 字節對齊。

如果 n = 3,
則將 \ctype{half3} 值寫入位址 \math{(p + (offset \times 4))}
位址 \math{(p + (offset \times 4))} 按
 \math{\text{\capi{sizeof}}(\text{\cldt{half}}) \times 4} 字節對齊。

\capi{vstorea_halfn} 使用缺省的捨入模式。
缺省的捨入模式為捨入為最近偶數。
\stopbuffer


% begin table
\startCLFD
\clFD{vloadn}
\clFD{vstoren}
\clFD{vload_half}
\clFD{vload_halfn}
\clFD{vstore_half_float}
\clFD{vstore_halfn_float}
\clFD{vstore_half_double}
\clFD{vstore_halfn_double}
\clFD{vloada_halfn}
\clFD{vstorea_halfn_float}
\clFD{vstorea_halfn_double}
\stopCLFD

}

注意:

\capi{vload3}、 \capi{vload_half3}、 \capi{vstore3} 和 \capi{vstore_half3}
 所用位址為 \math{(p+(offset\times 3))};
而 \capi{vloada_half3} 和 \capi{vstorea_half3}
 所用位址為 \math{(p+(offset\times 4))}。

使用這些函式裝載、存儲矢量數據時,
如果所讀寫的位址沒有按\reftab{vectorLsFunc}中所描述的方式對齊,
則結果未定義。
\reftab{vectorLsFunc}中存儲函式的指針引數 \carg{p} 可以指向
 \cqlf{__global}、 \cqlf{__local} 或 \cqlf{__private} 內存。
\reftab{vectorLsFunc}中裝載函式的指針引數 \carg{p} 可以指向
 \cqlf{__global}、 \cqlf{__local}、 \cqlf{__constant} 或 \cqlf{__private} 內存。


\subsection{同步函式}

OpenCL C 編程語言實現了如下同步函式。

\placetable[here,force,split][tab:syncFunc]
{內建同步函式}
{% barrier
\startbuffer[funcproto:barrier]
void barrier (cl_mem_fence_flags flags) 
\stopbuffer
\startbuffer[funcdesc:barrier]
同一\cnglo{workgrp}中的\cnglo{workitem}在處理器上執行此\cnglo{kernel}時,
其中任一\cnglo{workitem}要想越過 \capi{barrier} 繼續執行,
所有\cnglo{workitem}都得先執行此函式。
所有\cnglo{workitem}必須都能執行到此函式。

如果 \capi{barrier} 在條件語句內,
只要有一個\cnglo{workitem}會進入此條件語句具有並執行 \capi{barrier},
那麼所有\cnglo{workitem}都必須進入此條件語句。

如果 \capi{barrier} 在迴圈語句內,
那麼在每一次迭代過程中,
任一\cnglo{workitem}要想越過 \capi{barrier} 繼續執行,
所有\cnglo{workitem}都得先執行此函式。

\capi{barrier} 函式還會用內存隔柵(讀、寫都包括)來確保局部、全局內存操作的正確順序。

引數 \carg{flags} 指定內存位址空間,可以是下列常值的組合:
\startigBase
\item \cenum{CLK_LOCAL_MEM_FENCE},
函式 \capi{barrier} 會通過刷新存儲在局部內存中的所有變量,
或者用內存隔柵確保局部內存操作的正確順序。

\item \cenum{CLK_GLOBAL_MEM_FENCE},
函式 \capi{barrier} 會用內存隔柵確保全局內存操作的正確順序。
例如,\cnglo{workitem}寫入\cnglo{bufobj}或\cnglo{imgobj}後又想讀取更新過的數據,
這時此功能就派上用場了。
\stopigBase
\stopbuffer

% begin table
\startCLFD

\clFD{barrier}

\stopCLFD

}


\subsection{顯式內存隔柵函式}

OpenCL C 編程語言實現了如下顯式內存隔柵函式,
可對\cnglo{workitem}中的內存操作進行定序。

\placetable[here,force,split][tab:memfenceFunc]
{內建顯式內存隔柵函式}
{% mem_fence
\startbuffer[funcproto:mem_fence]
void mem_fence (cl_mem_fence_flags flags) 
\stopbuffer
\startbuffer[funcdesc:mem_fence]
\cnglo{workitem}執行\cnglo{kernel}時,為其中的裝載和存儲進行定序。
這意味着在執行 \capi{mem_fence} 之後的裝載和存儲之前,
會先將 \capi{mem_fence} 之前的裝載和存儲提交給內存。

引數 \carg{flags} 指定內存位址空間,可以是下列常值的組合:
\startigBase
\item \cenum{CLK_LOCAL_MEM_FENCE}
\item \cenum{CLK_GLOBAL_MEM_FENCE}
\stopigBase
\stopbuffer

% read_mem_fence
\startbuffer[funcproto:read_mem_fence]
void read_mem_fence (cl_mem_fence_flags flags)
\stopbuffer
\startbuffer[funcdesc:read_mem_fence]
讀內存屏障,僅對裝載定序。

引數 \carg{flags} 指定內存位址空間,可以是下列常值的組合:
\startigBase
\item \cenum{CLK_LOCAL_MEM_FENCE}
\item \cenum{CLK_GLOBAL_MEM_FENCE}
\stopigBase
\stopbuffer

% write_mem_fence
\startbuffer[funcproto:write_mem_fence]
void write_mem_fence (cl_mem_fence_flags flags)
\stopbuffer
\startbuffer[funcdesc:write_mem_fence]
寫內存屏障,僅對存儲定序。

引數 \carg{flags} 指定內存位址空間,可以是下列常值的組合:
\startigBase
\item \cenum{CLK_LOCAL_MEM_FENCE}
\item \cenum{CLK_GLOBAL_MEM_FENCE}
\stopigBase
\stopbuffer


% begin table
\startCLFD
\clFD{mem_fence}
\clFD{read_mem_fence}
\clFD{write_mem_fence}
\stopCLFD
}


\subsection{在全局內存和局部內存間的異步拷貝以及預取}

OpenCL C 編程語言實現了下列函式,
可在\cnglo{glbmem}和\cnglo{locmem}間進行異步拷貝,
以及從\cnglo{glbmem}中預取(prefetch)。

如無特殊說明,泛型 \ctype{gentype} 表示函式引數可以是內建數據型別
 \cldt{char}、 \cldt[n]{char}、 \cldt{uchar}、 \cldt[n]{uchar}、
 \cldt{short}、 \cldt[n]{short}、 \cldt{ushort}、 \cldt[n]{ushort}、
 \cldt{int}、 \cldt[n]{int}、 \cldt{uint}、 \cldt[n]{uint}、
 \cldt{long}、 \cldt[n]{long}、 \cldt{ulong}、 \cldt[n]{ulong}、
 \cldt{float}、 \cldt[n]{float} 或 \cldt{double}、 \cldt[n]{double},
其中 \ctypesuffix{n} 可以是 2、 3\footnote{
對 \capi{async_work_group_copy} 和 \capi{async_work_group_strided_copy} 而言,
矢量型別的組件數目是 3 還是 4 沒有什麼區別。}、 4、 8、 16。

\placetable[here,force,split][tab:asyncCopyPrefetch]
{內建異步拷貝和預取函式}
{% async_work_group_copy
\startbuffer[funcproto:async_work_group_copy]
event_t async_work_group_copy ( 
	__local gentype *dst, 
	const __global gentype *src, 
	size_t num_gentypes, 
	event_t event) 
event_t async_work_group_copy (
	__global gentype *dst,
	const __local gentype *src,
	size_t num_gentypes,
	event_t event)
\stopbuffer
\startbuffer[funcdesc:async_work_group_copy]
從 \carg{src} 異步拷貝 \carg{num_gentypes} 個 \ctype{gentype} 元素
到 \carg{dst} 中。
\cnglo{workgrp}中的所有\cnglo{workitem}都會實施此異步拷貝,
因此在\cnglo{workgrp}中,
使用相同引數值執行\cnglo{kernel}的所有\cnglo{workitem}必須都能執行到此函式,
否則結果未定義。

返回的\cnglo{evtobj}可由 \capi{wait_group_events} 用來等待異步拷貝完畢。
可以使用引數 \carg{event} 將 \capi{async_work_group_copy} 與之前的異步拷貝關聯在一起,
從而使得多個異步拷貝間可共享同一事件;否則 \carg{event} 必須是零。

如果引數 \carg{event} 非零,則會將其中的\cnglo{evtobj}返回。

此函式不會對源數據實施隱式同步,如在拷貝前執行 \capi{barrier}。
\stopbuffer

% async_work_group_strided_copy
\startbuffer[funcproto:async_work_group_strided_copy]
event_t async_work_group_strided_copy (
	__local gentype *dst,
	const __global gentype *src,
	size_t num_gentypes,
	size_t src_stride,
	event_t event)
event_t async_work_group_strided_copy (
	__global gentype *dst,
	const __local gentype *src,
	size_t num_gentypes,
	size_t dst_stride,
	event_t event)
\stopbuffer
\startbuffer[funcdesc:async_work_group_strided_copy]
從 \carg{src} 異步採集 \carg{num_gentypes} 個 \ctype{gentype} 元素
到 \carg{dst} 中。
參數 \carg{src_stride} 為從 \carg{src} 中讀取元素時所用跨距。
參數 \carg{dst_stride} 為將元素寫入 \carg{dst} 中時所用跨距。
\cnglo{workgrp}中的所有\cnglo{workitem}都會實施此異步採集,
因此在\cnglo{workgrp}中,
使用相同引數值執行\cnglo{kernel}的所有\cnglo{workitem}必須都能執行到此函式,
否則結果未定義。

返回的\cnglo{evtobj}可由 \capi{wait_group_events} 用來等待異步拷貝完畢。
可以使用引數 \carg{event} 將 \capi{async_work_group_strided_copy} 與之前的異步拷貝
關聯在一起,
從而使得多個異步拷貝間可共享同一事件;
否則 \carg{event} 必須是零。

如果引數 \carg{event} 非零,則會將其中的\cnglo{evtobj}返回。

此函式不會對源數據實施隱式同步,如在拷貝前執行 \capi{barrier}。

如果 \carg{src_stride} 或 \carg{dst_stride} 是 0,
或者拷貝時, \carg{src_stride} 或 \carg{dst_stride} 使得
 \carg{src} 或 \carg{dst} 指針超過了位址空間的上界,
則 \capi{async_work_group_strided_copy} 的行為未定義。
\stopbuffer

% wait_group_events
\startbuffer[funcproto:wait_group_events]
void wait_group_events (
	int num_events,
	event_t *event_list)
\stopbuffer
\startbuffer[funcdesc:wait_group_events]
等待用來表示 \capi{async_work_group_copy} 操作完成的事件。
實施等待後會釋放 \carg{event_list} 中的\cnglo{evtobj}。
對於某個\cnglo{workgrp}中的\cnglo{workitem}而言,
如果執行\cnglo{kernel}時
用的 \carg{num_events} 以及 \carg{event_list} 中的\cnglo{evtobj}一樣,
則它們必須都能執行到此函式;
否則結果未定義。
\stopbuffer

% prefetch
\startbuffer[funcproto:prefetch]
void prefetch (
	const __global gentype *p,
	size_t num_gentypes)
\stopbuffer
\startbuffer[funcdesc:prefetch]
預取 \math{\text{\carg{num_gentypes}}
 \times \text{\capi{sizeof}}(\text{\ctype{gentype}})} 字節到全局緩存中。
預取指令會作用到\cnglo{workgrp}中的\cnglo{workitem}上,
不會影響\cnglo{kernel}的功能性行為。
\stopbuffer


% begin table
\startCLFD
\clFD{async_work_group_copy}
\clFD{async_work_group_strided_copy}
\clFD{wait_group_events}
\clFD{prefetch}
\stopCLFD
}

注意:

\cnglo{kernel}必須用內建函式 \capi{wait_group_events} 等待所有異步拷貝全部完成後再退出,
否則其行為未定義。


\subsection{原子函式}

OpenCL C 編程語言實現了下列函式,
可用來對位於 \cqlf{__global} 或 \cqlf{__local} 內存中的 32 位帶符號、
無符號整數以及單精度浮點數\footnote{
只有 \capi{atomic_xchg} 才支持單精度浮點數據型別。}進行原子操作。

\placetable[here,force,split][tab:atomicFunc]
{內建異步拷貝和預取函式}
{% atomic_add
\startbuffer[funcproto:atomic_add]
int atomic_add (
	volatile __global int *p,
	int val)
unsigned int atomic_add (
	volatile __global unsigned int *p,
	unsigned int val)

int atomic_add (
	volatile __local int *p,
	int val)
unsigned int atomic_add (
	volatile __local unsigned int *p,
	unsigned int val)
\stopbuffer
\startbuffer[funcdesc:atomic_add]
讀取 \carg{p} 所指向的 32 位值(記為 \math{old})。
計算 \math{(old + \marg{val})} 並將結果存儲到 \carg{p} 所指位置中。
此函式返回 \math{old}。
\stopbuffer

% atomic_sub
\startbuffer[funcproto:atomic_sub]
int atomic_sub (
	volatile __global int *p,
	int val)
unsigned int atomic_sub (
	volatile __global unsigned int *p,
	unsigned int val)

int atomic_sub (
	volatile __local int *p,
	int val)
unsigned int atomic_sub (
	volatile __local unsigned int *p,
	unsigned int val)
\stopbuffer
\startbuffer[funcdesc:atomic_sub]
讀取 \carg{p} 所指向的 32 位值(記為 \math{old})。
計算 \math{(old - \marg{val})} 並將結果存儲到 \carg{p} 所指位置中。
此函式返回 \math{old}。
\stopbuffer

% atomic_xchg
\startbuffer[funcproto:atomic_xchg]
int atomic_xchg (
	volatile __global int *p,
	int val)
unsigned int atomic_xchg (
	volatile __global unsigned int *p,
	unsigned int val)
float atomic_xchg (
	volatile __global float *p,
	float val)

int atomic_xchg (
	volatile __local int *p,
	int val)
unsigned int atomic_xchg (
	volatile __local unsigned int *p,
	unsigned int val)
float atomic_xchg (
	volatile __local float *p,
	float val)
\stopbuffer
\startbuffer[funcdesc:atomic_xchg]
將位置 \carg{p} 中所存儲的值 \math{old} 和 \carg{val} 中的新值相互交換。
返回 \math{old}。
\stopbuffer

% atomic_inc
\startbuffer[funcproto:atomic_inc]
int atomic_inc (volatile __global int *p)
unsigned int atomic_inc (
	volatile __global unsigned int *p)

int atomic_inc (volatile __local int *p)
unsigned int atomic_inc (
	volatile __local unsigned int *p)
\stopbuffer
\startbuffer[funcdesc:atomic_inc]
讀取 \carg{p} 所指向的 32 位值(記為 \math{old})。
計算 \math{(old+1)} 並將結果存儲到 \carg{p} 所指位置中。
此函式返回 \math{old}。
\stopbuffer

% atomic_dec
\startbuffer[funcproto:atomic_dec]
int atomic_dec (volatile __global int *p)
unsigned int atomic_dec (
	volatile __global unsigned int *p)

int atomic_dec (volatile __local int *p)
unsigned int atomic_dec (
	volatile __local unsigned int *p)
\stopbuffer
\startbuffer[funcdesc:atomic_dec]
讀取 \carg{p} 所指向的 32 位值(記為 \math{old})。
計算 \math{(old-1)} 並將結果存儲到 \carg{p} 所指位置中。
此函式返回 \math{old}。
\stopbuffer

% atomic_cmpchg
\startbuffer[funcproto:atomic_cmpchg]
int atomic_cmpxchg (
	volatile __global int *p,
	int cmp, int val)
unsigned int atomic_cmpxchg (
	volatile __global unsigned int *p,
	unsigned int cmp,
	unsigned int val)

int atomic_cmpxchg (
	volatile __local int *p,
	int cmp,
	int val)
unsigned int atomic_cmpxchg (
	volatile __local unsigned int *p,
	unsigned int cmp,
	unsigned int val)
\stopbuffer
\startbuffer[funcdesc:atomic_cmpchg]
讀取 \carg{p} 所指向的 32 位值(記為 \math{old})。
計算 \math{(old == cmp) ? val : old} 並將結果存儲到 \carg{p} 所指位置中。
此函式返回 \math{old}。
\stopbuffer

% atomic_min
\startbuffer[funcproto:atomic_min]
int atomic_min (
	volatile __global int *p,
	int val)
unsigned int atomic_min (
	volatile __global unsigned int *p,
	unsigned int val)

int atomic_min (
	volatile __local int *p,
	int val)
unsigned int atomic_min (
	volatile __local unsigned int *p,
	unsigned int val)
\stopbuffer
\startbuffer[funcdesc:atomic_min]
讀取 \carg{p} 所指向的 32 位值(記為 \math{old})。
計算 \math{\mapiemp{min}(old, \marg{val})} 並將結果存儲到 \carg{p} 所指位置中。
此函式返回 \math{old}。
\stopbuffer

% atomic_max
\startbuffer[funcproto:atomic_max]
int atomic_max (
	volatile __global int *p,
	int val)
unsigned int atomic_max (
	volatile __global unsigned int *p,
	unsigned int val)

int atomic_max (
	volatile __local int *p,
	int val)
unsigned int atomic_max (
	volatile __local unsigned int *p,
	unsigned int val)
\stopbuffer
\startbuffer[funcdesc:atomic_max]
讀取 \carg{p} 所指向的 32 位值(記為 \math{old})。
計算 \math{\mapiemp{max}(old, \marg{val})} 並將結果存儲到 \carg{p} 所指位置中。
此函式返回 \math{old}。
\stopbuffer

% atomic_and
\startbuffer[funcproto:atomic_and]
int atomic_and (
	volatile __global int *p,
	int val)
unsigned int atomic_and (
	volatile __global unsigned int *p,
	unsigned int val)

int atomic_and (
	volatile __local int *p,
	int val)
unsigned int atomic_and (
	volatile __local unsigned int *p,
	unsigned int val)
\stopbuffer
\startbuffer[funcdesc:atomic_and]
讀取 \carg{p} 所指向的 32 位值(記為 \math{old})。
計算 \math{(old \mcmm{&} \marg{val})} 並將結果存儲到 \carg{p} 所指位置中。
此函式返回 \math{old}。
\stopbuffer

% atomic_or
\startbuffer[funcproto:atomic_or]
int atomic_or (
	volatile __global int *p,
	int val)
unsigned int atomic_or (
	volatile __global unsigned int *p,
	unsigned int val)

int atomic_or (
	volatile __local int *p,
	int val)
unsigned int atomic_or (
	volatile __local unsigned int *p,
	unsigned int val)
\stopbuffer
\startbuffer[funcdesc:atomic_or]
讀取 \carg{p} 所指向的 32 位值(記為 \math{old})。
計算 \math{(old \mcmm{|} \marg{val})} 並將結果存儲到 \carg{p} 所指位置中。
此函式返回 \math{old}。
\stopbuffer

% atomic_xor
\startbuffer[funcproto:atomic_xor]
int atomic_xor (
	volatile __global int *p,
	int val)
unsigned int atomic_xor (
	volatile __global unsigned int *p,
	unsigned int val)

int atomic_xor (
	volatile __local int *p,
	int val)
unsigned int atomic_xor (
	volatile __local unsigned int *p,
	unsigned int val)
\stopbuffer
\startbuffer[funcdesc:atomic_xor]
讀取 \carg{p} 所指向的 32 位值(記為 \math{old})。
計算 \math{(old \mcmm{^} \marg{val})} 並將結果存儲到 \carg{p} 所指位置中。
此函式返回 \math{old}。
\stopbuffer


% begin table
\startCLFD
\clFD{atomic_add}
\clFD{atomic_sub}
\clFD{atomic_xchg}
\clFD{atomic_inc}
\clFD{atomic_dec}
\clFD{atomic_cmpxchg}
\clFD{atomic_min}
\clFD{atomic_max}
\clFD{atomic_and}
\clFD{atomic_or}
\clFD{atomic_xor}
\stopCLFD

}

注意:

OpenCL 1.0 規範的節 9.5 和節 9.6 中列有如下擴展:
\startigBase
\item \clext{cl_khr_global_int32_base_atomics}
\item \clext{cl_khr_global_int32_extended_atomics}
\item \clext{cl_khr_local_int32_base_atomics}
\item \clext{cl_khr_local_int32_extended_atomics}
\stopigBase
其中所定義的帶有前綴 \capi{atom_} 的內建原子函式也在支持之列。



