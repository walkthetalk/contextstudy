\section{內建函式}

OpenCL C 編程語言提供了一套豐富的內建函式用於標量和矢量運算。
這些函式中有很多跟通用 C 庫中提供的函式名字類似,
不同的是他們所支持的參數型別即可是標量,也可是矢量。
\cnglo{app}應儘可能使用這些內建函式,而不是去實現自己的版本。

用戶自定義的 OpenCL C 函式,按照函式的 C 標準規則運行(C99-TC2-節 6.9.1)。
在函式入口處,對於所有可動態修改的參數(variably modified parameter),都會求出其大小,
並且按照\refsec{usualArithConv}中描述的常見算術轉換規則
將所有引數算式的值都轉換成對應參數的型別。
本節所描述的內建函式行為類似,
不過由於同一個內建函式可能有多種形式,為避免歧義,不會有隱式的標量擴展。
然而需要注意的是,可能某些內建函式的某些形式是針對標量以及矢量型別的混合運算。

% work-item functions
\subsection{\cnglo{workitem}函式}

\reftab{workItemFunction}列出了內建的\cnglo{workitem}函式,
這些函式可用來查詢指定給 \capi{clEnqueueNDRangeKernel} 的維數、全局和局部的索引空間大小、
以及在\cnglo{device}上執行此\cnglo{kernel}時
每個\cnglo{workitem}的\cnglo{glbid}和\cnglo{locid}。
如果使用的是 \capi{clEnqueueTask},則維數、全局和局部索引空間的大小都是一。

\placetable[here,force,split][tab:workItemFunction]
{\cnglo{workitem}函式表}
{\startCLOD[函式][描述]

\stopCLOD

}
