\section{函式限定符}

% __kernel (or kernel)
\subsection{__kernel (或 kernel)}

限定符 \cqlfemp{__kernel} (或 \cqlfemp{kernel})可以將函式聲明為\cnglo{kernel},
使其可由\cnglo{app}在 OpenCL \cnglo{device}上執行。
對於用此限定符聲明的函式有如下規定:
\startigBase
\item 僅可在\cnglo{device}上執行;
\item 可由\cnglo{host}調用;
\item 當被另一個\cnglo{kernel}函式調用時, \cqlf{__kernel} 函式僅僅是一個常規函式。
\stopigBase

注意:

如果\cnglo{kernel}函式中聲明的變量帶有限定符 \cqlf{__local} 或 \cqlf{local},
則\cnglo{host}可以使用恰當的 API
 像 \capi{clEnqueueNDRangeKernel} 和 \capi{clEnqueueTask} 來調用他。

如果\cnglo{kernel}函式中聲明的變量帶有限定符 \cqlf{__local} 或 \cqlf{local},
那麼在其他\cnglo{kernel}函式中調用他會發生什麼\cnglo{impdef}。

保留關鍵字 \cqlfemp{__kernel} 和 \cqlfemp{kernel} 只能用作函式的限定符,
不能挪作他用。

% optional attribute qualifiers
\subsection{可選特性限定符}

限定符 \cqlf{__kernel} 可以與關鍵字 \cqlf{__attribute__} 一起使用,
從而為\cnglo{kernel}函式聲明額外的資訊,如下所述。

% safe 一般指防止“無意的破壞”,
% secure 一般指防止“有意的破壞”。
% 單獨出現均可為“安全”,同時出現時可翻譯為“安全”與“保安”/“守密”
可選特性 \ccmm{__attribute__((vec_type_hint(<type>)))}\footnote{
自動矢量化(autovectorization)時會做隱式的假設:
 \cqlf{__kernel} 中調用的所有庫都必須可在運行時重新編譯,
只有這樣,編譯器才可以合併或分離作業項。
這可能意味着,這樣的庫不能是編死的二元碼,
如果是編死的二元碼,則必須帶有源碼或可重定向的中間表示。
在某些情況下,這可能導致代碼安全問題。}
 可以給編譯器暗示 \cqlf{__kernel} 可計算的{\ftRef{寬度}},
編譯器在對代碼進行自動矢量化時,可以將其作為基準來計算處理器的可利用帶寬。
其中 \ccmm{<type>} 是\reftab{builtInVectorDataTypes}中所列的內建矢量型別,
或者是構成其元素的標量型別。
如果沒有指定 \ccmm{vec_type_hint (<type>)},
則假定\cnglo{kernel}具有限定符 \ccmm{__attribute__((vec_type_hint(int)))}。

例如,如果開發人員指定的寬度是 \ctype{float4},
則編譯器應當假定通常使用四路浮點矢量進行計算,
並可以決定合併\cnglo{workitem}或者甚至可能將一個\cnglo{workitem}分成多個線程,
以更好的匹配硬件的能力。
對於合格的實作,不要求可以對代碼自動矢量化,但應當支持這個暗示。
即使沒有這個暗示,編譯器也可能自動矢量化。
如果實作將 N 個\cnglo{workitem}合併到一個線程中,
則他要負責正確地處理這種情況:
全局或局部\cnglo{workitem}的數目在任一維度上模 N 不為零。

例:
\startclc
// autovectorize assuming float4 as the
// basic computation width
__kernel __attribute__((vec_type_hint(float4)))
void foo( __global float4 *p ) { .... }

// autovectorize assuming double as the
// basic computation width
__kernel __attribute__((vec_type_hint(double)))
void foo( __global float4 *p ) { .... }

// autovectorize assuming int (default)
// as the basic computation width
__kernel
void foo( __global float4 *p ) { .... }
\stopclc

舉個例子,如果聲明 \cqlf{__kernel} 函式時帶有
 \ccmm{__attribute__((vec_type_hint (float4)))}
 (即 \cqlf{__kernel} 函式中大部分運算都已使用 \ctype{float4} 顯式矢量化),
並且此\cnglo{kernel}在使用
 Intel® AVX(Intel® Advanced Vector Instructions,
實現了 8 個 \ctype{float} 寬度的矢量單元),
則自動矢量化時可能選擇將兩個\cnglo{workitem}合併成一個線程,
第二個\cnglo{workitem}將在 256 位 AVX 寄存器的高 128 位中運行。

% select重点在于强调选择的动作;
% choose强调选择的结果
另一個例子, Power4 機器具有兩個雙精度浮點單元,均為六級流水線結構(6-cycle deep pipe)。
針對這種機器,自動矢量化時可能選擇將
六個聲明時帶有限定符 \ccmm{__attribute__(( vec_type_hint (double2)))} 的\cnglo{kernel}
間插到同一個硬件線程中,
從而可以讓 FPU 可以一直 12 路全速並行。
考慮到資源利用或者一些偏好,如 2 的指數,
即僅將 4 個或 8 個(或者其他數目的)\cnglo{workitem}進行合併,
只要自動矢量化時認定這種抉擇更好就行。

可選特性 \ccmm{__attribute__((work_group_size_hint(X, Y, Z)))}
 可以給編譯器暗示\cnglo{workgrp}的大小,
即最有可能傳給 \capi{clEnqueueNDRangeKernel} 的引數 \carg{local_work_size} 的值。
例如, \ccmm{__attribute__((work_group_size_hint(1, 1, 1)))}
 就是暗示編譯器\cnglo{kernel}最有可能在大小為 1 的\cnglo{workgrp}中執行。

可選特性 \ccmm{__attribute__((reqd_work_group_size(X, Y, Z)))}
 則是 \capi{clEnqueueNDRangeKernel} 的引數 \carg{local_work_size} 必須使用的值。
這樣編譯器就可以針對此\cnglo{kernel}對生成的代碼進行適當的優化。
如果\cnglo{kernel}是通過 \capi{clEnqueueTask} 執行的,
則指定此特性時必須是 \ccmm{__attribute__((reqd_work_group_size(1, 1, 1)))}。

如果 \ccmm{Z} 是一,
則 \capi{clEnqueueNDRangeKernel} 的引數 \carg{work_dim} 可以是 2 或 3。
如果 \ccmm{Y} 和 \ccmm{Z} 都是一,
則 \capi{clEnqueueNDRangeKernel} 的引數 \carg{work_dim} 可以是 1、2 或 3。

