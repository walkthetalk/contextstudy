%% acos-related
% acos
\startbuffer[funcproto:acos]
gentype acos(gentype)
\stopbuffer
\startbuffer[funcdesc:acos]
反餘弦函數。
\stopbuffer

% acosh
\startbuffer[funcproto:acosh]
gentype acosh(gentype)
\stopbuffer
\startbuffer[funcdesc:acosh]
反雙曲餘弦函數。
\stopbuffer

% acospi
\startbuffer[funcproto:acospi]
gentype acospi(gentype)
\stopbuffer
\startbuffer[funcdesc:acospi]
計算 \math{acos(x) / \pi}。
\stopbuffer

%% asin-related
% asin
\startbuffer[funcproto:asin]
gentype asin(gentype)
\stopbuffer
\startbuffer[funcdesc:asin]
反正弦函數。
\stopbuffer

% asinh
\startbuffer[funcproto:asinh]
gentype asinh(gentype)
\stopbuffer
\startbuffer[funcdesc:asinh]
反雙曲正弦函數。
\stopbuffer

% asinpi
\startbuffer[funcproto:asinpi]
gentype asinpi(gentype)
\stopbuffer
\startbuffer[funcdesc:asinpi]
計算 \math{asin(x) / \pi}。
\stopbuffer

%% atan-related
% atan
\startbuffer[funcproto:atan]
gentype atan (gentype y_over_x)
\stopbuffer
\startbuffer[funcdesc:atan]
反正切函數。
\stopbuffer

% atan2
\startbuffer[funcproto:atan2]
gentype atan2 (gentype y, gentype x)
\stopbuffer
\startbuffer[funcdesc:atan2]
\math{y / x} 的反正切。
\stopbuffer

% atanh
\startbuffer[funcproto:atanh]
gentype atanh (gentype)
\stopbuffer
\startbuffer[funcdesc:atanh]
反雙曲正切函數。
\stopbuffer

% atanpi
\startbuffer[funcproto:atanpi]
gentype atanpi (gentype x)
\stopbuffer
\startbuffer[funcdesc:atanpi]
計算 \math{atan(x) / \pi}。
\stopbuffer

% atan2pi
\startbuffer[funcproto:atan2pi]
gentype atan2pi (gentype y, gentype x)
\stopbuffer
\startbuffer[funcdesc:atan2pi]
計算 \math{atan2(y,x) / \pi}。
\stopbuffer

% cbrt
\startbuffer[funcproto:cbrt]
gentype cbrt(gentype)
\stopbuffer
\startbuffer[funcdesc:cbrt]
計算立方根。
\stopbuffer

% ceil
\startbuffer[funcproto:ceil]
gentype ceil(gentype)
\stopbuffer
\startbuffer[funcdesc:ceil]
向正無窮舍入成整數值。
\stopbuffer

% copysign
\startbuffer[funcproto:copysign]
gentype copysign(gentype x,
		 gentype y)
\stopbuffer
\startbuffer[funcdesc:copysign]
將 \carg{x} 的符號改成 \carg{y} 的,並將其返回。
\stopbuffer

% cos
\startbuffer[funcproto:cos]
gentype cos(gentype)
\stopbuffer
\startbuffer[funcdesc:cos]
計算餘弦。
\stopbuffer

% cosh
\startbuffer[funcproto:cosh]
gentype cosh(gentype)
\stopbuffer
\startbuffer[funcdesc:cosh]
計算雙曲餘弦。
\stopbuffer

% cospi
\startbuffer[funcproto:cospi]
gentype cospi(gentype x)
\stopbuffer
\startbuffer[funcdesc:cospi]
計算 \math{cos(\pi x)}。
\stopbuffer

% erfc
\startbuffer[funcproto:erfc]
gentype erfc(gentype)
\stopbuffer
\startbuffer[funcdesc:erfc]
餘補誤差函數
 \math{1 - erf(x) = \frac{2}{\sqrt{\pi}} \intop^{\infty}_{x}e^{-\theta^2}d\theta}。
\stopbuffer

% erf
\startbuffer[funcproto:erf]
gentype erf(gentype)
\stopbuffer
\startbuffer[funcdesc:erf]
誤差函數,表示正態分布的積分
 \math{\frac{2}{\sqrt{\pi}} \intop^{\infty}_{x}e^{-\theta^2}d\theta}。
\stopbuffer

% exp
\startbuffer[funcproto:exp]
gentype exp(gentype x)
\stopbuffer
\startbuffer[funcdesc:exp]
計算 \math{e} 的 \carg{x} 次冪 \math{e^x}。
\stopbuffer

% exp
\startbuffer[funcproto:exp]
gentype exp(gentype x)
\stopbuffer
\startbuffer[funcdesc:exp]
計算 \math{e} 的 \carg{x} 次冪 \math{e^x}。
\stopbuffer

% exp2
\startbuffer[funcproto:exp2]
gentype exp2(gentype)
\stopbuffer
\startbuffer[funcdesc:exp2]
底數為 \math{2} 的冪。
\stopbuffer

% exp10
\startbuffer[funcproto:exp10]
gentype exp10(gentype)
\stopbuffer
\startbuffer[funcdesc:exp10]
底數為 \math{10} 的冪。
\stopbuffer

% expm1
\startbuffer[funcproto:expm1]
gentype expm1(gentype x)
\stopbuffer
\startbuffer[funcdesc:expm1]
計算 \math{e^x-1.0}。
\stopbuffer

% fabs
\startbuffer[funcproto:fabs]
gentype fabs(gentype)
\stopbuffer
\startbuffer[funcdesc:fabs]
計算浮點數的絕對值。
\stopbuffer

% fdim
\startbuffer[funcproto:fdim]
gentype fdim(gentype x,
	     gentype y)
\stopbuffer
\startbuffer[funcdesc:fdim]
如果 \math{x < y},返回 \math{x - y};否則返回 \math{+0}。
\stopbuffer

% floor
\startbuffer[funcproto:floor]
gentype floor(gentype)
\stopbuffer
\startbuffer[funcdesc:floor]
向負無窮舍入成整數值。
\stopbuffer

% fma
\startbuffer[funcproto:fma]
gentype fma(gentype a,
	    gentype b,
	    gentype c)
\stopbuffer
\startbuffer[funcdesc:fma]
返回 \math{a \times b + c},其中乘法具有無限精度即不會進行舍入,但會對加法進行正確地舍入。
邊界案例下的行為遵循 IEEE 754-2008 標準。
\stopbuffer

% fmax
\startbuffer[funcproto:fmax]
gentype fmax (gentype x, gentype y)
gentypef fmax (gentypef x, float y)
gentyped fmax (gentyped x, double y)
\stopbuffer
\startbuffer[funcdesc:fmax]
如果 \math{x < y},則返回 \math{y},否則返回 \math{x}。
如果一個引數是 NaN,則返回另一個引數;
如果兩個引數都是 NaN,則返回 NaN。
\stopbuffer
\stopbuffer

% fmin
\startbuffer[funcproto:fmin]
gentype fmin (gentype x, gentype y)
gentypef fmin (gentypef x, float y)
gentyped fmin (gentyped x, double y)
\stopbuffer
\startbuffer[funcdesc:fmin]
如果 \math{y < x},則返回 \math{y},否則返回 \math{x}。
如果一個引數是 NaN,則返回另一個引數;
如果兩個引數都是 NaN,則返回 NaN。
\stopbuffer

% fmod
\startbuffer[funcproto:fmod]
gentype fmod (gentyped x, gentype y)
\stopbuffer
\startbuffer[funcdesc:fmod]
模。返回 \math{x - y * trunc(x/y)}。
\stopbuffer

% fract


% begin TABLE
\startCLFD

\clFD{acos}
\clFD{acosh}
\clFD{acospi}
\clFD{asin}
\clFD{asinh}
\clFD{asinpi}
\clFD{atan}
\clFD{atan2}
\clFD{atanh}
\clFD{atanpi}
\clFD{atan2pi}
\clFD{cbrt}
\clFD{ceil}
\clFD{copysign}
\clFD{cos}
\clFD{cosh}
\clFD{cospi}
\clFD{erfc}
\clFD{erf}
\clFD{exp}
\clFD{exp2}
\clFD{exp10}
\clFD{expm1}
\clFD{fabs}
\clFD{fdim}
\clFD{floor}
\clFD{fma}
\clFD{fmax}
\clFD{fmin}
\clFD{fmod}

\stopCLFD

