% barrier
\startbuffer[funcproto:barrier]
void barrier (cl_mem_fence_flags flags) 
\stopbuffer
\startbuffer[funcdesc:barrier]
同一\cnglo{workgrp}中的\cnglo{workitem}在處理器上執行此\cnglo{kernel}時,
其中任一\cnglo{workitem}要想越過 \capi{barrier} 繼續執行,
所有\cnglo{workitem}都得先執行此函式。
所有\cnglo{workitem}必須都能執行到此函式。

如果 \capi{barrier} 在條件語句內,
只要有一個\cnglo{workitem}會進入此條件語句具有並執行 \capi{barrier},
那麼所有\cnglo{workitem}都必須進入此條件語句。

如果 \capi{barrier} 在循環語句內,
那麼在此循環的每一次迭代中,
任一\cnglo{workitem}要想越過 \capi{barrier} 繼續執行,
所有\cnglo{workitem}都得先執行此函式。

\capi{barrier} 函式還會用內存隔柵(讀、寫都包括)來確保局部、全局內存操作的正確順序。

引數 \carg{flags} 指定內存位址空間,可以是下列常值的組合:
\startigBase
\item \cenum{CLK_LOCAL_MEM_FENCE},
函式 \capi{barrier} 會通過刷新存儲在局部內存中的所有變量,
或者用內存隔柵確保局部內存操作的正確順序。

\item \cenum{CLK_GLOBAL_MEM_FENCE},
函式 \capi{barrier} 會用內存隔柵確保全局內存操作的正確順序。
例如,\cnglo{workitem}寫入\cnglo{bufobj}或\cnglo{imgobj}後又想讀取更新過的數據,
這時此功能就派上用場了。
\stopigBase
\stopbuffer

% begin table
\startCLFD

\clFD{barrier}

\stopCLFD

