% cross
\startbuffer[funcproto:cross]
float4 cross (float4 p0, float4 p1)
float3 cross (float3 p0, float3 p1)
double4 cross (double4 p0, double4 p1)
double3 cross (double3 p0, double3 p1)
\stopbuffer
\startbuffer[funcdesc:cross]
返回 \math{p0.xyz} 和 \math{p1.xyz} 的叉乘。
所返回的 \ctype{float4} 中的組件 \math{w} 為 \math{0.0}。
\stopbuffer

% dot
\startbuffer[funcproto:dot]
float dot (floatn p0, floatn p1)
double dot (doublen p0, doublen p1)
\stopbuffer
\startbuffer[funcdesc:dot]
計算點乘。
\stopbuffer

% distance
\startbuffer[funcproto:distance]
float distance (floatn p0,
		floatn p1)
double distance (doublen p0,
		doublen p1)
\stopbuffer
\startbuffer[funcdesc:distance]
返回 \carg{p0} 和 \carg{p1} 的距離。
即 \math{\mapiemp{length}(\marg{p0} - \marg{p1})}。
\stopbuffer

% length
\startbuffer[funcproto:length]
float length (floatn p)
double length (gentype p)
\stopbuffer
\startbuffer[funcdesc:length]
返回矢量 \carg{p} 的長度,即 \math{\sqrt{p.x^2+p.y^2+\cdots}}。
\stopbuffer

% normalize
\startbuffer[funcproto:normalize]
floatn normalize (floatn p)
doublen normalize (doublen p)
\stopbuffer
\startbuffer[funcdesc:normalize]
返回的矢量方向與 \carg{p} 相同,長度為 \math{1}。
\stopbuffer

% fast_distance
\startbuffer[funcproto:fast_distance]
float fast_distance (floatn p0,
		floatn p1)
\stopbuffer
\startbuffer[funcdesc:fast_distance]
返回 \math{\mapiemp{fast_length}(\marg{p0} - \marg{p1})}。
\stopbuffer

% fast_length
\startbuffer[funcproto:fast_length]
float fast_length (floatn p)
\stopbuffer
\startbuffer[funcdesc:fast_length]
返回 \math{\mapiemp{half_sqrt}(p.x^2 + p.y^2 + \cdots)}。
\stopbuffer

% fast_normalize
\startbuffer[funcproto:fast_normalize]
floatn fast_normalize (floatn p)
\stopbuffer
\startbuffer[funcdesc:fast_normalize]
返回的矢量方向與 \math{p} 相同,長度為 \math{1}。
\capi{fast_normalize} 是這樣計算的:

\math{p * \mapiemp{half_rsqrt}(p.x^2 + p.y^2 + \cdots)}

返回值與下面語句結果(具有無窮精度)的誤差在 8192 ulp 內:
\startcintbl[escape=yes,indentnext=no]
if (/BTEX\capi{all}/ETEX(/BTEX\math{p == 0.0f}/ETEX))
	/BTEX\math{result = p}/ETEX;
else
	/BTEX\math{result = p / \mapiemp{sqrt}(p.x^2 + p.y^2 + \cdots )}/ETEX;
\stopcintbl
但是下列情況例外:
\startigNum
\item 如果平方和大於 \cmacro{FLT_MAX},則結果中對應的浮點值未定義。

\item 如果平方和小於 \cmacro{FLT_MIN},則實作可能直接返回 \math{p}。

\item 如果\cnglo{device}處於“將去規格化數刷成零”的模式,
則在計算前會將算元中
量級小於 \math{\mapiemp{sqrt}(\mmacro{FLT_MIN})} 的元素刷成零。
\stopigNum
\stopbuffer

% begin TABLE
\startCLFD

\clFD{cross}
\clFD{dot}
\clFD{distance}
\clFD{length}
\clFD{normalize}
\clFD{fast_distance}
\clFD{fast_length}
\clFD{fast_normalize}

\stopCLFD

