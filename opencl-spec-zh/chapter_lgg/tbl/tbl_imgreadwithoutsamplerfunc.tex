
% read_imagef_2d_s
\startbuffer[funcproto:read_imagef_2d_s]
float4 read_imagef (
	image2d_t image,
	int2 coord)
\stopbuffer
\startbuffer[funcdesc:read_imagef_2d_s]
用坐標 \math{(coord.x, coord.y)} 查找 2D \cnglo{imgobj} \carg{image} 中的元素。

如果創建\cnglo{imgobj}時所用的 \carg{image_channel_data_type} 是
預定義的壓縮過的格式或 \cenum{CL_UNORM_INT8} 或 \cenum{CL_UNORM_INT16},
則返回的浮點值在區間 \math{[0.0 \cdots 1.0]} 內。

如果創建\cnglo{imgobj}時所用的 \carg{image_channel_data_type} 是
 \cenum{CL_SNORM_INT8} 或 \cenum{CL_SNORM_INT16},
則返回的浮點值在區間 \math{[-1.0 \cdots 1.0]} 內。

如果創建\cnglo{imgobj}時所用的 \carg{image_channel_data_type} 是
 \cenum{CL_HALF_FLOAT} 或 \cenum{CL_FLOAT},
則返回原始浮點值。

如果創建\cnglo{imgobj}時所用的 \carg{image_channel_data_type} 不再上面所列範圍內,
則結果未定義。
\stopbuffer

% read_imagei_2d_s
\startbuffer[funcproto:read_imagei_2d_s]
int4 read_imagei (
	image2d_t image,
	int2 coord)
uint4 read_imageui (
	image2d_t image,
	int2 coord)
\stopbuffer
\startbuffer[funcdesc:read_imagei_2d_s]
用坐標 \math{(coord.x, coord.y)} 查找 2D \cnglo{imgobj} \carg{image} 中的元素。

\capi{read_imagei} 和 \capi{read_imageui} 所返回的值分別為
非歸一化帶符號整數和非歸一化無符號整數。

對於 \capi{read_imagei} 而言,
創建\cnglo{imgobj}時所用的 \carg{image_channel_data_type} 必須是下列值之一:
\startigBase[indentnext=no]
\item \cenum{CL_SIGNED_INT8}
\item \cenum{CL_SIGNED_INT16}
\item \cenum{CL_SIGNED_INT32}
\stopigBase
如果 \carg{image_channel_data_type} 不在上述值之列,則結果未定義。

對於 \capi{read_imageui} 而言,
創建\cnglo{imgobj}時所用的 \carg{image_channel_data_type} 必須是下列值之一:
\startigBase[indentnext=no]
\item \cenum{CL_UNSIGNED_INT8}
\item \cenum{CL_UNSIGNED_INT16}
\item \cenum{CL_UNSIGNED_INT32}
\stopigBase
如果 \carg{image_channel_data_type} 不在上述值之列,則結果未定義。
\stopbuffer

% read_imagef_3d_s
\startbuffer[funcproto:read_imagef_3d_s]
float4 read_imagef (
	image3d_t image,
	int4 coord )
\stopbuffer
\startbuffer[funcdesc:read_imagef_3d_s]
用坐標 \math{(coord.x, coord.y, coord.z)}
 查找 3D \cnglo{imgobj} \carg{image} 中的元素。
其中 \math{coord.w} 被忽略。

如果創建\cnglo{imgobj}時所用的 \carg{image_channel_data_type} 是
預定義的壓縮過的格式或 \cenum{CL_UNORM_INT8} 或 \cenum{CL_UNORM_INT16},
則返回的浮點值在區間 \math{[0.0 \cdots 1.0]} 內。

如果創建\cnglo{imgobj}時所用的 \carg{image_channel_data_type} 是
 \cenum{CL_SNORM_INT8} 或 \cenum{CL_SNORM_INT16},
則返回的浮點值在區間 \math{[-1.0 \cdots 1.0]} 內。

如果創建\cnglo{imgobj}時所用的 \carg{image_channel_data_type} 是
 \cenum{CL_HALF_FLOAT} 或 \cenum{CL_FLOAT},
則返回原始浮點值。

如果創建\cnglo{imgobj}時所用的 \carg{image_channel_data_type} 不再上面所列範圍內,
則結果未定義。
\stopbuffer

% read_imagei_3d_s
\startbuffer[funcproto:read_imagei_3d_s]
int4 read_imagei (
	image3d_t image,
	int4 coord)
uint4 read_imageui (
	image3d_t image,
	int4 coord)
\stopbuffer
\startbuffer[funcdesc:read_imagei_3d_s]
用坐標 \math{(coord.x, coord.y, coord.z)} 查找
 3D \cnglo{imgobj} \carg{image} 中的元素。
其中 \math{coord.w} 被忽略。

\capi{read_imagei} 和 \capi{read_imageui} 所返回的值分別為
非歸一化帶符號整數和非歸一化無符號整數。
每個通道的值都是 32 位整數。

對於 \capi{read_imagei} 而言,
創建\cnglo{imgobj}時所用的 \carg{image_channel_data_type} 必須是下列值之一:
\startigBase[indentnext=no]
\item \cenum{CL_SIGNED_INT8}
\item \cenum{CL_SIGNED_INT16}
\item \cenum{CL_SIGNED_INT32}
\stopigBase
如果 \carg{image_channel_data_type} 不在上述值之列,則結果未定義。

對於 \capi{read_imageui} 而言,
創建\cnglo{imgobj}時所用的 \carg{image_channel_data_type} 必須是下列值之一:
\startigBase[indentnext=no]
\item \cenum{CL_UNSIGNED_INT8}
\item \cenum{CL_UNSIGNED_INT16}
\item \cenum{CL_UNSIGNED_INT32}
\stopigBase
如果 \carg{image_channel_data_type} 不在上述值之列,則結果未定義。
\stopbuffer

% read_imagef_2da_s
\startbuffer[funcproto:read_imagef_2da_s]
float4 read_imagef (
	image2d_array_t image,
	int4 coord)
\stopbuffer
\startbuffer[funcdesc:read_imagef_2da_s]
用坐標 \math{coord.z} 確定 2D 圖像陣列 \carg{image} 中的某一個 2D 圖像;
用坐標 \math{(corrd.x, coord.y)} 來查找此 2D 圖像中的元素。
其中 \math{coord.w} 被忽略。

如果創建\cnglo{imgobj}時所用的 \carg{image_channel_data_type} 是
預定義的壓縮過的格式或 \cenum{CL_UNORM_INT8} 或 \cenum{CL_UNORM_INT16},
則返回的浮點值在區間 \math{[0.0 \cdots 1.0]} 內。

如果創建\cnglo{imgobj}時所用的 \carg{image_channel_data_type} 是
 \cenum{CL_SNORM_INT8} 或 \cenum{CL_SNORM_INT16},
則返回的浮點值在區間 \math{[-1.0 \cdots 1.0]} 內。

如果創建\cnglo{imgobj}時所用的 \carg{image_channel_data_type} 是
 \cenum{CL_HALF_FLOAT} 或 \cenum{CL_FLOAT},
則返回原始浮點值。

如果創建\cnglo{imgobj}時所用的 \carg{image_channel_data_type} 不再上面所列範圍內,
則結果未定義。
\stopbuffer

% read_imagei_2da_s
\startbuffer[funcproto:read_imagei_2da_s]
int4 read_imagei (
	image2d_array_t image,
	int4 coord)
uint4 read_imageui (
	image2d_array_t image,
	int4 coord)
\stopbuffer
\startbuffer[funcdesc:read_imagei_2da_s]
用坐標 \math{coord.z} 確定 2D 圖像陣列 \carg{image} 中的某一個 2D 圖像;
用坐標 \math{(corrd.x, coord.y)} 來查找此 2D 圖像中的元素。
其中 \math{coord.w} 被忽略。

\capi{read_imagei} 和 \capi{read_imageui} 所返回的值分別為
非歸一化帶符號整數和非歸一化無符號整數。
每個通道的值都是 32 位整數。

對於 \capi{read_imagei} 而言,
創建\cnglo{imgobj}時所用的 \carg{image_channel_data_type} 必須是下列值之一:
\startigBase[indentnext=no]
\item \cenum{CL_SIGNED_INT8}
\item \cenum{CL_SIGNED_INT16}
\item \cenum{CL_SIGNED_INT32}
\stopigBase
如果 \carg{image_channel_data_type} 不在上述值之列,則結果未定義。

對於 \capi{read_imageui} 而言,
創建\cnglo{imgobj}時所用的 \carg{image_channel_data_type} 必須是下列值之一:
\startigBase[indentnext=no]
\item \cenum{CL_UNSIGNED_INT8}
\item \cenum{CL_UNSIGNED_INT16}
\item \cenum{CL_UNSIGNED_INT32}
\stopigBase
如果 \carg{image_channel_data_type} 不在上述值之列,則結果未定義。
\stopbuffer

% read_imagef_1d_s
\startbuffer[funcproto:read_imagef_1d_s]
float4 read_imagef (
	image1d_t image,
	int coord)
float4 read_imagef (
	image1d_buffer_t image,
	int coord)
\stopbuffer
\startbuffer[funcdesc:read_imagef_1d_s]
用坐標 \math{coord} 查找 1D \cnglo{imgobj}或 1D 圖像緩衝對象 \carg{image} 中的元素。

如果創建\cnglo{imgobj}時所用的 \carg{image_channel_data_type} 是
預定義的壓縮過的格式或 \cenum{CL_UNORM_INT8} 或 \cenum{CL_UNORM_INT16},
則返回的浮點值在區間 \math{[0.0 \cdots 1.0]} 內。

如果創建\cnglo{imgobj}時所用的 \carg{image_channel_data_type} 是
 \cenum{CL_SNORM_INT8} 或 \cenum{CL_SNORM_INT16},
則返回的浮點值在區間 \math{[-1.0 \cdots 1.0]} 內。

如果創建\cnglo{imgobj}時所用的 \carg{image_channel_data_type} 是
 \cenum{CL_HALF_FLOAT} 或 \cenum{CL_FLOAT},
則返回原始浮點值。

如果創建\cnglo{imgobj}時所用的 \carg{image_channel_data_type} 不再上面所列範圍內,
則結果未定義。
\stopbuffer

% read_imagei_1d_s
\startbuffer[funcproto:read_imagei_1d_s]
int4 read_imagei (
	image1d_t image,
	int coord)
uint4 read_imageui (
	image1d_t image,
	int coord)
int4 read_imagei (
	image1d_buffer_t image,
	int coord)
uint4 read_imageui (
	image1d_buffer_t image,
	int coord)
\stopbuffer
\startbuffer[funcdesc:read_imagei_1d_s]
用坐標 \math{coord} 查找 1D \cnglo{imgobj}或 1D 圖像緩衝對象 \carg{image} 中的元素。

\capi{read_imagei} 和 \capi{read_imageui} 所返回的值分別為
非歸一化帶符號整數和非歸一化無符號整數。
每個通道的值都是 32 位整數。

對於 \capi{read_imagei} 而言,
創建\cnglo{imgobj}時所用的 \carg{image_channel_data_type} 必須是下列值之一:
\startigBase[indentnext=no]
\item \cenum{CL_SIGNED_INT8}
\item \cenum{CL_SIGNED_INT16}
\item \cenum{CL_SIGNED_INT32}
\stopigBase
如果 \carg{image_channel_data_type} 不在上述值之列,則結果未定義。

對於 \capi{read_imageui} 而言,
創建\cnglo{imgobj}時所用的 \carg{image_channel_data_type} 必須是下列值之一:
\startigBase[indentnext=no]
\item \cenum{CL_UNSIGNED_INT8}
\item \cenum{CL_UNSIGNED_INT16}
\item \cenum{CL_UNSIGNED_INT32}
\stopigBase
如果 \carg{image_channel_data_type} 不在上述值之列,則結果未定義。
\stopbuffer

% read_imagef_1da_s
\startbuffer[funcproto:read_imagef_1da_s]
float4 read_imagef (
	image1d_array_t image,
	int2 coord)
\stopbuffer
\startbuffer[funcdesc:read_imagef_1da_s]
用坐標 \math{coord.y} 確定 1D 圖像陣列 \carg{image} 中的某一個 1D 圖像;
用坐標 \math{corrd.x} 來查找此 1D 圖像中的元素。

如果創建\cnglo{imgobj}時所用的 \carg{image_channel_data_type} 是
預定義的壓縮過的格式或 \cenum{CL_UNORM_INT8} 或 \cenum{CL_UNORM_INT16},
則返回的浮點值在區間 \math{[0.0 \cdots 1.0]} 內。

如果創建\cnglo{imgobj}時所用的 \carg{image_channel_data_type} 是
 \cenum{CL_SNORM_INT8} 或 \cenum{CL_SNORM_INT16},
則返回的浮點值在區間 \math{[-1.0 \cdots 1.0]} 內。

如果創建\cnglo{imgobj}時所用的 \carg{image_channel_data_type} 是
 \cenum{CL_HALF_FLOAT} 或 \cenum{CL_FLOAT},
則返回原始浮點值。

如果創建\cnglo{imgobj}時所用的 \carg{image_channel_data_type} 不再上面所列範圍內,
則結果未定義。
\stopbuffer

% read_imagei_1da_s
\startbuffer[funcproto:read_imagei_1da_s]
int4 read_imagei (
	image1d_array_t image,
	int2 coord)
uint4 read_imageui (
	image1d_array_t image,
	int2 coord)
\stopbuffer
\startbuffer[funcdesc:read_imagei_1da_s]
用坐標 \math{coord.y} 確定 1D 圖像陣列 \carg{image} 中的某一個 1D 圖像;
用坐標 \math{corrd.x} 來查找此 1D 圖像中的元素。

\capi{read_imagei} 和 \capi{read_imageui} 所返回的值分別為
非歸一化帶符號整數和非歸一化無符號整數。
每個通道的值都是 32 位整數。

對於 \capi{read_imagei} 而言,
創建\cnglo{imgobj}時所用的 \carg{image_channel_data_type} 必須是下列值之一:
\startigBase[indentnext=no]
\item \cenum{CL_SIGNED_INT8}
\item \cenum{CL_SIGNED_INT16}
\item \cenum{CL_SIGNED_INT32}
\stopigBase
如果 \carg{image_channel_data_type} 不在上述值之列,則結果未定義。

對於 \capi{read_imageui} 而言,
創建\cnglo{imgobj}時所用的 \carg{image_channel_data_type} 必須是下列值之一:
\startigBase[indentnext=no]
\item \cenum{CL_UNSIGNED_INT8}
\item \cenum{CL_UNSIGNED_INT16}
\item \cenum{CL_UNSIGNED_INT32}
\stopigBase
如果 \carg{image_channel_data_type} 不在上述值之列,則結果未定義。
\stopbuffer

% begin table
\startCLFD
\clFD{read_imagef_2d_s}
\clFD{read_imagei_2d_s}
\clFD{read_imagef_3d_s}
\clFD{read_imagei_3d_s}
\clFD{read_imagef_2da_s}
\clFD{read_imagei_2da_s}
\clFD{read_imagef_1d_s}
\clFD{read_imagei_1d_s}
\clFD{read_imagef_1da_s}
\clFD{read_imagei_1da_s}
\stopCLFD
