\startCLOD[型別][描述]

\clOD{\cldt[n]{char}}{
帶符號 8 位整數矢量,為二的補碼。
}

\clOD{\cldt[n]{uchar}}{
無符號 8 位整數矢量。
}

\clOD{\cldt[n]{short}}{
帶符號 16 位整數矢量,為二的補碼。
}

\clOD{\cldt[n]{ushort}}{
無符號 16 位整數矢量。
}

\clOD{\cldt[n]{int}}{
帶符號 32 位整數矢量,為二的補碼。
}

\clOD{\cldt[n]{uint}}{
無符號 32 位整數矢量。
}

\clOD{\cldt[n]{long}}{
帶符號 64 位整數矢量,為二的補碼。
}

\clOD{\cldt[n]{ulong}}{
無符號 64 位整數矢量。
}

\clOD{\cldt[n]{float}}{
32 位浮點數矢量。
}

\clOD{\cldt[n]{double}
\footnote{\cldt[n]{double} 是可選的,
只有設備的 \cenum{CL_DEVICE_DOUBLE_FP_CONFIG}
(參見\reftab{cldevquery})不是零時才需要支持。}
}{
64 位浮點數矢量。
}
\stopCLOD
