\startCLTD

\clTD{\cldtv{char}}{
帶符號 8 位整數矢量,為二的補碼。
}

\clTD{\cldtv{uchar}}{
無符號 8 位整數矢量。
}

\clTD{\cldtv{short}}{
帶符號 16 位整數矢量,為二的補碼。
}

\clTD{\cldtv{ushort}}{
無符號 16 位整數矢量。
}

\clTD{\cldtv{int}}{
帶符號 32 位整數矢量,為二的補碼。
}

\clTD{\cldtv{uint}}{
無符號 32 位整數矢量。
}

\clTD{\cldtv{long}}{
帶符號 64 位整數矢量,為二的補碼。
}

\clTD{\cldtv{ulong}}{
無符號 64 位整數矢量。
}

\clTD{\cldtv{float}}{
32 位浮點數矢量。
}

\clTD{\cldtv{double}
\footnote{\cldtv{double} 是可選類型,只有設備的 \cenum{CL_DEVICE_DOUBLE_FP_CONFIG}
(參見\reftab{cldevquery})不是零時才需要支持。}
}{
64 位浮點數矢量。
}
\stopCLTD
