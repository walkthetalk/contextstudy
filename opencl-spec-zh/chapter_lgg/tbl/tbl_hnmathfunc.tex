% half_cos
\startbuffer[funcproto:half_cos]
gentype half_cos (gentype x) 
\stopbuffer
\startbuffer[funcdesc:half_cos]
計算餘弦。 \carg{x} 的取值範圍為 \math{-2^{16} \cdots +2^{16}}。
\stopbuffer

% half_divide
\startbuffer[funcproto:half_divide]
gentype half_divide (gentype x, 
		gentype y) 
\stopbuffer
\startbuffer[funcdesc:half_divide]
計算 \math{x/y}。
\stopbuffer

% half_exp
\startbuffer[funcproto:half_exp]
gentype half_exp (gentype x) 
\stopbuffer
\startbuffer[funcdesc:half_exp]
計算 \math{e^x}。
\stopbuffer

% half_exp2
\startbuffer[funcproto:half_exp2]
gentype half_exp2 (gentype x) 
\stopbuffer
\startbuffer[funcdesc:half_exp2]
計算 \math{2^x}。
\stopbuffer

% half_exp10
\startbuffer[funcproto:half_exp10]
gentype half_exp10 (gentype x) 
\stopbuffer
\startbuffer[funcdesc:half_exp10]
計算 \math{10^x}。
\stopbuffer

% half_log
\startbuffer[funcproto:half_log]
gentype half_log (gentype x) 
\stopbuffer
\startbuffer[funcdesc:half_log]
計算自然對數。
\stopbuffer

% half_log2
\startbuffer[funcproto:half_log2]
gentype half_log2 (gentype x) 
\stopbuffer
\startbuffer[funcdesc:half_log2]
計算底為 2 的對數。
\stopbuffer

% half_log10
\startbuffer[funcproto:half_log10]
gentype half_log10 (gentype x) 
\stopbuffer
\startbuffer[funcdesc:half_log10]
計算底為 10 的對數。
\stopbuffer

% half_powr
\startbuffer[funcproto:half_powr]
gentype half_powr (gentype x,
		gentype y)
\stopbuffer
\startbuffer[funcdesc:half_powr]
計算 \math{x^y},其中 \math{x\geq 0}。
\stopbuffer

% half_recip
\startbuffer[funcproto:half_recip]
gentype half_recip (gentype x)
\stopbuffer
\startbuffer[funcdesc:half_recip]
計算倒數。
\stopbuffer

% half_rsqrt
\startbuffer[funcproto:half_rsqrt]
gentype half_rsqrt (gentype x)
\stopbuffer
\startbuffer[funcdesc:half_rsqrt]
計算 \math{ 1 / \sqrt{x}}。
\stopbuffer

% half_sin
\startbuffer[funcproto:half_sin]
gentype half_sin (gentype x)
\stopbuffer
\startbuffer[funcdesc:half_sin]
計算正弦。 \carg{x} 的取值範圍為 \math{-2^{16} \cdots +2^{16}}。
\stopbuffer

% half_sqrt
\startbuffer[funcproto:half_sqrt]
gentype half_sqrt (gentype x)
\stopbuffer
\startbuffer[funcdesc:half_sqrt]
計算 \math{\sqrt{x}}。
\stopbuffer

% half_tan
\startbuffer[funcproto:half_tan]
gentype half_tan (gentype x)
\stopbuffer
\startbuffer[funcdesc:half_tan]
計算正切。 \carg{x} 的取值範圍為 \math{-2^{16} \cdots +2^{16}}。
\stopbuffer

% native_cos
\startbuffer[funcproto:native_cos]
gentype native_cos (gentype x) 
\stopbuffer
\startbuffer[funcdesc:native_cos]
計算餘弦。
參數的取值範圍以及取最大值時會產生什麼錯誤都\cnglo{impdef}。
\stopbuffer

% native_divide
\startbuffer[funcproto:native_divide]
gentype native_divide (gentype x, 
		gentype y) 
\stopbuffer
\startbuffer[funcdesc:native_divide]
計算 \math{x/y}。
參數的取值範圍以及取最大值時會產生什麼錯誤都\cnglo{impdef}。
\stopbuffer

% native_exp
\startbuffer[funcproto:native_exp]
gentype native_exp (gentype x) 
\stopbuffer
\startbuffer[funcdesc:native_exp]
計算 \math{e^x}。
參數的取值範圍以及取最大值時會產生什麼錯誤都\cnglo{impdef}。
\stopbuffer

% native_exp2
\startbuffer[funcproto:native_exp2]
gentype native_exp2 (gentype x) 
\stopbuffer
\startbuffer[funcdesc:native_exp2]
計算 \math{2^x}。
參數的取值範圍以及取最大值時會產生什麼錯誤都\cnglo{impdef}。
\stopbuffer

% native_exp10
\startbuffer[funcproto:native_exp10]
gentype native_exp10 (gentype x) 
\stopbuffer
\startbuffer[funcdesc:native_exp10]
計算 \math{10^x}。
參數的取值範圍以及取最大值時會產生什麼錯誤都\cnglo{impdef}。
\stopbuffer

% native_log
\startbuffer[funcproto:native_log]
gentype native_log (gentype x) 
\stopbuffer
\startbuffer[funcdesc:native_log]
計算自然對數。
參數的取值範圍以及取最大值時會產生什麼錯誤都\cnglo{impdef}。
\stopbuffer

% native_log2
\startbuffer[funcproto:native_log2]
gentype native_log2 (gentype x) 
\stopbuffer
\startbuffer[funcdesc:native_log2]
計算底為 2 的對數。
參數的取值範圍以及取最大值時會產生什麼錯誤都\cnglo{impdef}。
\stopbuffer

% native_log10
\startbuffer[funcproto:native_log10]
gentype native_log10 (gentype x) 
\stopbuffer
\startbuffer[funcdesc:native_log10]
計算底為 10 的對數。
參數的取值範圍以及取最大值時會產生什麼錯誤都\cnglo{impdef}。
\stopbuffer

% native_powr
\startbuffer[funcproto:native_powr]
gentype native_powr (gentype x,
		gentype y)
\stopbuffer
\startbuffer[funcdesc:native_powr]
計算 \math{x^y},其中 \math{x\geq 0}。
參數的取值範圍以及取最大值時會產生什麼錯誤都\cnglo{impdef}。
\stopbuffer

% native_recip
\startbuffer[funcproto:native_recip]
gentype native_recip (gentype x)
\stopbuffer
\startbuffer[funcdesc:native_recip]
計算倒數。
參數的取值範圍以及取最大值時會產生什麼錯誤都\cnglo{impdef}。
\stopbuffer

% native_rsqrt
\startbuffer[funcproto:native_rsqrt]
gentype native_rsqrt (gentype x)
\stopbuffer
\startbuffer[funcdesc:native_rsqrt]
計算 \math{ 1 / \sqrt{x}}。
參數的取值範圍以及取最大值時會產生什麼錯誤都\cnglo{impdef}。
\stopbuffer

% native_sin
\startbuffer[funcproto:native_sin]
gentype native_sin (gentype x)
\stopbuffer
\startbuffer[funcdesc:native_sin]
計算正弦。
參數的取值範圍以及取最大值時會產生什麼錯誤都\cnglo{impdef}。
\stopbuffer

% native_sqrt
\startbuffer[funcproto:native_sqrt]
gentype native_sqrt (gentype x)
\stopbuffer
\startbuffer[funcdesc:native_sqrt]
計算 \math{\sqrt{x}}。
參數的取值範圍以及取最大值時會產生什麼錯誤都\cnglo{impdef}。
\stopbuffer

% native_tan
\startbuffer[funcproto:native_tan]
gentype native_tan (gentype x)
\stopbuffer
\startbuffer[funcdesc:native_tan]
計算正切。
參數的取值範圍以及取最大值時會產生什麼錯誤都\cnglo{impdef}。
\stopbuffer


% begin table
\startCLFD
\clFD{half_cos}
\clFD{half_divide}
\clFD{half_exp}
\clFD{half_exp2}
\clFD{half_exp10}
\clFD{half_log}
\clFD{half_log2}
\clFD{half_log10}
\clFD{half_powr}
\clFD{half_recip}
\clFD{half_rsqrt}
\clFD{half_sin}
\clFD{half_sqrt}
\clFD{half_tan}

\clFD{native_cos}
\clFD{native_divide}
\clFD{native_exp}
\clFD{native_exp2}
\clFD{native_exp10}
\clFD{native_log}
\clFD{native_log2}
\clFD{native_log10}
\clFD{native_powr}
\clFD{native_recip}
\clFD{native_rsqrt}
\clFD{native_sin}
\clFD{native_sqrt}
\clFD{native_tan}
\stopCLFD
