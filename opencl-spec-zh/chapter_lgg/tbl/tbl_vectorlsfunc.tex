

% vloadn
\startbuffer[funcproto:vloadn]
\stopbuffer
\startbuffer[funcdesc:vloadn]
\stopbuffer

% vloadn
\startbuffer[funcproto:vloadn]
gentypen vloadn (size_t offset, 
	const __global gentype *p) 
gentypen vloadn (size_t offset,
	const __local gentype *p)
gentypen vloadn (size_t offset,
	const __constant gentype *p)
gentypen vloadn (size_t offset,
	const __private gentype *p)
\stopbuffer
\startbuffer[funcdesc:vloadn]
由位址 \math{(\marg{p} + (\marg{offset} \times n))} 讀取
 \math{\mapiemp{sizeof}(\text{\cldt[n]{gentype}})} 字節的數據並將其返回。
對於位址 \math{(\marg{p} + (\marg{offset} \times n))} 而言,
如果 \cldt{gentype} 為 \cldt{char}、 \cldt{uchar},則他必須按 8 位對齊;
如果 \cldt{gentype} 為 \cldt{short}、 \cldt{ushort},則他必須按 16 位對齊;
如果 \cldt{gentype} 為 \cldt{int}、 \cldt{uint}、 \cldt{float},
則他必須按 32 位對齊;
如果 \cldt{gentype} 為 \cldt{long}、 \cldt{ulong},則他必須按 64 位對齊。
\stopbuffer

% vstoren
\startbuffer[funcproto:vstoren]
void vstoren (gentypen data,
	size_t offset,
	__global gentype *p)
void vstoren (gentypen data,
	size_t offset,
	__local gentype *p)
void vstoren (gentypen data,
	size_t offset,
	__private gentype *p)
\stopbuffer
\startbuffer[funcdesc:vstoren]
將 \carg{data} 中
 \math{\mapiemp{sizeof}(\text{\cldt[n]{gentype}})} 字節的數據寫入
位址 \math{(\marg{p} + (\marg{offset} \times n))} 中。
對於位址 \math{(\marg{p} + (\marg{offset} \times n))} 而言,
如果 \cldt{gentype} 為 \cldt{char}、 \cldt{uchar},則他必須按 8 位對齊;
如果 \cldt{gentype} 為 \cldt{short}、 \cldt{ushort},則他必須按 16 位對齊;
如果 \cldt{gentype} 為 \cldt{int}、 \cldt{uint}、 \cldt{float},
則他必須按 32 位對齊;
如果 \cldt{gentype} 為 \cldt{long}、 \cldt{ulong},則他必須按 64 位對齊。
\stopbuffer

% vload_half
\startbuffer[funcproto:vload_half]
float vload_half (size_t offset,
	const __global half *p)
float vload_half (size_t offset,
	const __local half *p)
float vload_half (size_t offset,
	const __constant half *p)
float vload_half (size_t offset,
	const __private half *p)
\stopbuffer
\startbuffer[funcdesc:vload_half]
由位址 \math{(\marg{p} + \marg{offset})} 讀取
 \math{\mapiemp{sizeof}(\mtype{half})} 字節的數據。
將讀到的數據按 \cldt{half} 值解釋,將其轉換為 \cldt{float} 後返回。
位址 \math{(\marg{p} + \marg{offset})} 必須按 16 位對齊。
\stopbuffer

% vload_halfn
\startbuffer[funcproto:vload_halfn]
floatn vload_halfn (size_t offset,
	const __global half *p)
floatn vload_halfn (size_t offset,
	const __local half *p)
floatn vload_halfn (size_t offset,
	const __constant half *p)
floatn vload_halfn (size_t offset,
	const __private half *p)
\stopbuffer
\startbuffer[funcdesc:vload_halfn]
由位址 \math{(\marg{p} + (\marg{offset} \times n))} 讀取
 \math{\mapiemp{sizeof}(\text{\cldt[n]{half}})} 字節的數據。
將讀到的數據按 \cldt[n]{half} 值解釋,將其轉換為 \cldt[n]{float} 後返回。
位址 \math{(\marg{p} + (\marg{offset} \times n))} 必須按 16 位對齊。
\stopbuffer

% vstore_half_float
\startbuffer[funcproto:vstore_half_float]
void vstore_half (float data,
	size_t offset,
	__global half *p)
void vstore_half_rte (float data,
	size_t offset,
	__global half *p)
void vstore_half_rtz (float data,
	size_t offset,
	__global half *p)
void vstore_half_rtp (float data,
	size_t offset,
	__global half *p)
void vstore_half_rtn (float data,
	size_t offset,
	__global half *p)

void vstore_half (float data,
	size_t offset,
	__local half *p)
void vstore_half_rte (float data,
	size_t offset,
	__local half *p)
void vstore_half_rtz (float data,
	size_t offset,
	__local half *p)
void vstore_half_rtp (float data,
	size_t offset,
	__local half *p)
void vstore_half_rtn (float data,
	size_t offset,
	__local half *p)

void vstore_half (float data,
	size_t offset,
	__private half *p)
void vstore_half_rte (float data,
	size_t offset,
	__private half *p)
void vstore_half_rtz (float data,
	size_t offset,
	__private half *p)
void vstore_half_rtp (float data,
	size_t offset,
	__private half *p)
void vstore_half_rtn (float data,
	size_t offset,
	__private half *p)
\stopbuffer
\startbuffer[funcdesc:vstore_half_float]
先按某種捨入模式將 \carg{data} 中的 \cldt{float} 值轉換為 \cldt{half} 值。
然後將其寫入位址 \math{\marg{p} + \marg{offset}} 中。
位址 \math{\marg{p} + \marg{offset}} 必須按 16 位對齊。

\clapi{vstore_half} 使用缺省的捨入模式。
缺省的捨入模式為捨入為最近偶數。
\stopbuffer

% vstore_halfn_float
\startbuffer[funcproto:vstore_halfn_float]
void vstore_halfn (floatn data,
	size_t offset,
	__global half *p)
void vstore_halfn_rte (floatn data,
	size_t offset,
	__global half *p)
void vstore_halfn_rtz (floatn data,
	size_t offset,
	__global half *p)
void vstore_halfn_rtp (floatn data,
	size_t offset,
	__global half *p)
void vstore_halfn_rtn (floatn data,
	size_t offset,
	__global half *p)

void vstore_halfn (floatn data,
	size_t offset,
	__local half *p)
void vstore_halfn_rte (floatn data,
	size_t offset,
	__local half *p)
void vstore_halfn_rtz (floatn data,
	size_t offset,
	__local half *p)
void vstore_halfn_rtp (floatn data,
	size_t offset,
	__local half *p)
void vstore_halfn_rtn (floatn data,
	size_t offset,
	__local half *p)

void vstore_halfn (floatn data,
	size_t offset,
	__private half *p)
void vstore_halfn_rte (floatn data,
	size_t offset,
	__private half *p)
void vstore_halfn_rtz (floatn data,
	size_t offset,
	__private half *p)
void vstore_halfn_rtp (floatn data,
	size_t offset,
	__private half *p)
void vstore_halfn_rtn (floatn data,
	size_t offset,
	__private half *p)
\stopbuffer
\startbuffer[funcdesc:vstore_halfn_float]
先按某種捨入模式將 \carg{data} 中的 \cldt[n]{float} 值轉換為 \cldt[n]{half} 值。
然後將其寫入位址 \math{(\marg{p} + (\marg{offset} \times n))} 中。
位址 \math{(\marg{p} + (\marg{offset} \times n))} 必須按 16 位對齊。

\clapi[n]{vstore_half} 使用缺省的捨入模式。
缺省的捨入模式為捨入為最近偶數。
\stopbuffer

% vstore_half_double
\startbuffer[funcproto:vstore_half_double]
void vstore_half (double data,
	size_t offset,
	__global half *p)
void vstore_half_rte (double data,
	size_t offset,
	__global half *p)
void vstore_half_rtz (double data,
	size_t offset,
	__global half *p)
void vstore_half_rtp (double data,
	size_t offset,
	__global half *p)
void vstore_half_rtn (double data,
	size_t offset,
	__global half *p)

void vstore_half (double data,
	size_t offset,
	__local half *p)
void vstore_half_rte (double data,
	size_t offset,
	__local half *p)
void vstore_half_rtz (double data,
	size_t offset,
	__local half *p)
void vstore_half_rtp (double data,
	size_t offset,
	__local half *p)
void vstore_half_rtn (double data,
	size_t offset,
	__local half *p)

void vstore_half (double data,
	size_t offset,
	__private half *p)
void vstore_half_rte (double data,
	size_t offset,
	__private half *p)
void vstore_half_rtz (double data,
	size_t offset,
	__private half *p)
void vstore_half_rtp (double data,
	size_t offset,
	__private half *p)
void vstore_half_rtn (double data,
	size_t offset,
	__private half *p)
\stopbuffer
\startbuffer[funcdesc:vstore_half_double]
先按某種捨入模式將 \carg{data} 中的 \cldt{double} 值轉換為 \cldt{half} 值。
然後將其寫入位址 \math{\marg{p} + \marg{offset}} 中。
位址 \math{\marg{p} + \marg{offset}} 必須按 16 位對齊。

\clapi{vstore_half} 使用缺省的捨入模式。
缺省的捨入模式為捨入為最近偶數。
\stopbuffer

% vstore_halfn_double
\startbuffer[funcproto:vstore_halfn_double]
void vstore_halfn (doublen data,
	size_t offset,
	__global half *p)
void vstore_halfn_rte (doublen data,
	size_t offset,
	__global half *p)
void vstore_halfn_rtz (doublen data,
	size_t offset,
	__global half *p)
void vstore_halfn_rtp (doublen data,
	size_t offset,
	__global half *p)
void vstore_halfn_rtn (doublen data,
	size_t offset,
	__global half *p)

void vstore_halfn (doublen data,
	size_t offset,
	__local half *p)
void vstore_halfn_rte (doublen data,
	size_t offset,
	__local half *p)
void vstore_halfn_rtz (doublen data,
	size_t offset,
	__local half *p)
void vstore_halfn_rtp (doublen data,
	size_t offset,
	__local half *p)
void vstore_halfn_rtn (doublen data,
	size_t offset,
	__local half *p)

void vstore_halfn (doublen data,
	size_t offset,
	__private half *p)
void vstore_halfn_rte (doublen data,
	size_t offset,
	__private half *p)
void vstore_halfn_rtz (doublen data,
	size_t offset,
	__private half *p)
void vstore_halfn_rtp (doublen data,
	size_t offset,
	__private half *p)
void vstore_halfn_rtn (doublen data,
	size_t offset,
	__private half *p)
\stopbuffer
\startbuffer[funcdesc:vstore_halfn_double]
先按某種捨入模式將 \carg{data} 中的 \cldt[n]{double} 值轉換為 \cldt[n]{half} 值。
然後將其寫入位址 \math{(\marg{p} + (\marg{offset} \times n))} 中。
位址 \math{(\marg{p} + (\marg{offset} \times n))} 必須按 16 位對齊。

\clapi[n]{vstore_half} 使用缺省的捨入模式。
缺省的捨入模式為捨入為最近偶數。
\stopbuffer

% vloada_halfn
\startbuffer[funcproto:vloada_halfn]
floatn vloada_halfn (size_t offset,
	const __global half *p)
floatn vloada_halfn (size_t offset,
	const __local half *p)
floatn vloada_halfn (size_t offset,
	const __constant half *p)
floatn vloada_halfn (size_t offset,
	const __private half *p)
\stopbuffer
\startbuffer[funcdesc:vloada_halfn]
對於 n 為 1、 2、 4、 8 和 16,
由位址 \math{(\marg{p} + (\marg{offset} \times n))} 讀取
 \math{\mapiemp{sizeof}(\text{\cldt[n]{half}})} 字節的數據。
讀到的數據解釋為 \cldt[n]{half} 值,將其轉換為 \cldt[n]{float} 值後返回。

位址 \math{(\marg{p} + (\marg{offset} \times n))} 必須按
 \math{\mapiemp{sizeof}(\text{\cldt[n]{half}})} 字節對齊。

如果 n = 3,則由位址 \math{(\marg{p} + (\marg{offset} \times 4))} 讀取
 \ctype{half3} 並返回 \ctype{float3}。
位址 \math{(\marg{p} + (\marg{offset} \times 4))} 按
 \math{\mapiemp{sizeof}(\text{\cldt{half}}) \times 4} 字節對齊。
\stopbuffer

% vstorea_halfn_float
\startbuffer[funcproto:vstorea_halfn_float]
void vstorea_halfn (floatn data,
	size_t offset,
	__global half *p)
void vstorea_halfn_rte (floatn data,
	size_t offset,
	__global half *p)
void vstorea_halfn_rtz (floatn data,
	size_t offset,
	__global half *p)
void vstorea_halfn_rtp (floatn data,
	size_t offset,
	__global half *p)
void vstorea_halfn_rtn (floatn data,
	size_t offset,
	__global half *p)

void vstorea_halfn (floatn data,
	size_t offset,
	__local half *p)
void vstorea_halfn_rte (floatn data,
	size_t offset,
	__local half *p)
void vstorea_halfn_rtz (floatn data,
	size_t offset,
	__local half *p)
void vstorea_halfn_rtp (floatn data,
	size_t offset,
	__local half *p)
void vstorea_halfn_rtn (floatn data,
	size_t offset,
	__local half *p)

void vstorea_halfn (floatn data,
	size_t offset,
	__private half *p)
void vstorea_halfn_rte (floatn data,
	size_t offset,
	__private half *p)
void vstorea_halfn_rtz (floatn data,
	size_t offset,
	__private half *p)
void vstorea_halfn_rtp (floatn data,
	size_t offset,
	__private half *p)
void vstorea_halfn_rtn (floatn data,
	size_t offset,
	__private half *p)
\stopbuffer
\startbuffer[funcdesc:vstorea_halfn_float]
按某種捨入模式將 \carg{data} 中的 \cldt[n]{float} 轉換為 \cldt[n]{half}。

如果 n 為 1、 2、 4、 8 和 16,
則將 \cldt[n]{half} 值寫入位址 \math{(\marg{p} + (\marg{offset} \times n))}
位址 \math{(\marg{p} + (\marg{offset} \times n))} 必須按
 \math{\mapiemp{sizeof}(\text{\cldt[n]{half}})} 字節對齊。

如果 n = 3,
則將 \ctype{half3} 值寫入位址 \math{(\marg{p} + (\marg{offset} \times 4))}
位址 \math{(\marg{p} + (\marg{offset} \times 4))} 按
 \math{\mapiemp{sizeof}(\text{\cldt{half}}) \times 4} 字節對齊。

\capi{vstorea_halfn} 使用缺省的捨入模式。
缺省的捨入模式為捨入為最近偶數。
\stopbuffer

% vstorea_halfn_double
\startbuffer[funcproto:vstorea_halfn_double]
void vstorea_halfn (doublen data,
	size_t offset,
	__global half *p)
void vstorea_halfn_rte (doublen data,
	size_t offset,
	__global half *p)
void vstorea_halfn_rtz (doublen data,
	size_t offset,
	__global half *p)
void vstorea_halfn_rtp (doublen data,
	size_t offset,
	__global half *p)
void vstorea_halfn_rtn (doublen data,
	size_t offset,
	__global half *p)

void vstorea_halfn (doublen data,
	size_t offset,
	__local half *p)
void vstorea_halfn_rte (doublen data,
	size_t offset,
	__local half *p)
void vstorea_halfn_rtz (doublen data,
	size_t offset,
	__local half *p)
void vstorea_halfn_rtp (doublen data,
	size_t offset,
	__local half *p)
void vstorea_halfn_rtn (doublen data,
	size_t offset,
	__local half *p)

void vstorea_halfn (doublen data,
	size_t offset,
	__private half *p)
void vstorea_halfn_rte (doublen data,
	size_t offset,
	__private half *p)
void vstorea_halfn_rtz (doublen data,
	size_t offset,
	__private half *p)
void vstorea_halfn_rtp (doublen data,
	size_t offset,
	__private half *p)
void vstorea_halfn_rtn (doublen data,
	size_t offset,
	__private half *p)
\stopbuffer
\startbuffer[funcdesc:vstorea_halfn_double]
按某種捨入模式將 \carg{data} 中的 \cldt[n]{double} 轉換為 \cldt[n]{half}。

如果 n 為 1、 2、 4、 8 和 16,
則將 \cldt[n]{half} 值寫入位址 \math{(\marg{p} + (\marg{offset} \times n))}
位址 \math{(\marg{p} + (\marg{offset} \times n))} 必須按
 \math{\mapiemp{sizeof}(\text{\cldt[n]{half}})} 字節對齊。

如果 n = 3,
則將 \ctype{half3} 值寫入位址 \math{(\marg{p} + (\marg{offset} \times 4))}
位址 \math{(\marg{p} + (\marg{offset} \times 4))} 按
 \math{\mapiemp{sizeof}(\text{\cldt{half}}) \times 4} 字節對齊。

\capi{vstorea_halfn} 使用缺省的捨入模式。
缺省的捨入模式為捨入為最近偶數。
\stopbuffer


% begin table
\startCLFD
\clFD{vloadn}
\clFD{vstoren}
\clFD{vload_half}
\clFD{vload_halfn}
\clFD{vstore_half_float}
\clFD{vstore_halfn_float}
\clFD{vstore_half_double}
\clFD{vstore_halfn_double}
\clFD{vloada_halfn}
\clFD{vstorea_halfn_float}
\clFD{vstorea_halfn_double}
\stopCLFD

