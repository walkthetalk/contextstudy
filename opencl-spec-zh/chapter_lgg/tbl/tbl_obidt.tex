\startCLTD

\clTD{\cldts{image2d_t}}{
2D 圖像。\todo{6.12.14}對於使用此類型的內建函數有詳細描述。
}

\clTD{\cldts{image3d_t}}{
3D 圖像。\todo{6.12.14}對於使用此類型的內建函數有詳細描述。
}

\clTD{\cldts{image2d_array_t}}{
2D 圖像陣列。\todo{6.12.14}對於使用此類型的內建函數有詳細描述。
}

\clTD{\cldts{image1d_t}}{
1D 圖像。\todo{6.12.14}對於使用此類型的內建函數有詳細描述。
}

\clTD{\cldts{image1d_buffer_t}}{
由\cnglo{bufobj}所創建的 1D 圖像。\todo{6.12.14}對於使用此類型的內建函數有詳細描述。
}

\clTD{\cldts{image1d_array_t}}{
1D 圖像陣列。\todo{6.12.14}對於使用此類型的內建函數有詳細描述。
}

\clTD{\cldts{sampler_t}}{
\cnglo{sampler}類型。\todo{6.12.14}對於使用此類型的內建函數有詳細描述。
}

\clTD{\cldts{event_t}}{
事件。可用來標識\cnglo{glbmem}和\cnglo{locmem}之間的異步拷貝,參見\todo{6.12.10}。
}

\stopCLTD
