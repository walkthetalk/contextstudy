\startCLOD[型別][描述]

\clOD{\cldt{image2d_t}}{
2D 圖像。\refsec{imgRwFunc}對於使用此型別的內建函式有詳細描述。
}

\clOD{\cldt{image3d_t}}{
3D 圖像。\refsec{imgRwFunc}對於使用此型別的內建函式有詳細描述。
}

\clOD{\cldt{image2d_array_t}}{
2D 圖像陣列。\refsec{imgRwFunc}對於使用此型別的內建函式有詳細描述。
}

\clOD{\cldt{image1d_t}}{
1D 圖像。\refsec{imgRwFunc}對於使用此型別的內建函式有詳細描述。
}

\clOD{\cldt{image1d_buffer_t}}{
由\cnglo{bufobj}所創建的 1D 圖像。\refsec{imgRwFunc}對於使用此型別的內建函式有詳細描述。
}

\clOD{\cldt{image1d_array_t}}{
1D 圖像陣列。\refsec{imgRwFunc}對於使用此型別的內建函式有詳細描述。
}

\clOD{\cldt{sampler_t}}{
\cnglo{sampler}型別。\refsec{imgRwFunc}對於使用此型別的內建函式有詳細描述。
}

\clOD{\cldt{event_t}}{
事件。可用來標識\cnglo{glbmem}和\cnglo{locmem}之間的異步拷貝,參見\refsec{asyncCopyPrefetch}。
}

\stopCLOD
