% clamp
\startbuffer[funcproto:clamp]
gentype clamp (gentype x,
		gentype minval,
		gentype maxval)
gentypef clamp (gentypef x,
		float minval,
		float maxval)
gentyped clamp (gentyped x,
		double minval,
		double maxval)
\stopbuffer
\startbuffer[funcdesc:clamp]
返回 \math{\mapiemp{fmin}(\mapiemp{fmax}(\marg{x}, \marg{minval}), \marg{maxval})}。
如果 \math{\marg{minval} > \marg{maxval}},則結果\cnglo{undef}。
\stopbuffer

% degrees
\startbuffer[funcproto:degrees]
gentype degrees (gentype radians)
\stopbuffer
\startbuffer[funcdesc:degrees]
將弧度轉換成角度,即 \math{(180/\pi)\times \marg{radians}}。
\stopbuffer

% max
\startbuffer[funcproto:max]
gentype max (gentype x, gentype y)
gentypef max (gentypef x, float y)
gentyped max (gentyped x, double y)
\stopbuffer
\startbuffer[funcdesc:max]
如果 \math{\marg{x} < \marg{y}},則返回 \carg{y},否則返回 \carg{x}。
如果 \carg{x} 或 \carg{y} 是無窮或 NaN,則返回的值\cnglo{undef}。
\stopbuffer

% min
\startbuffer[funcproto:min]
gentype min (gentype x, gentype y)
gentypef min (gentypef x, float y)
gentyped min (gentyped x, double y)
\stopbuffer
\startbuffer[funcdesc:min]
如果 \math{\marg{y} < \marg{x}},則返回 \carg{y},否則返回 \carg{x}。
如果 \carg{x} 或 \carg{y} 是無窮或 NaN,則返回的值\cnglo{undef}。
\stopbuffer

% mix
\startbuffer[funcproto:mix]
gentype mix (gentype x,
		gentype y,
		gentype a)
gentypef mix (gentypef x,
		gentypef y,
		float a)
gentyped mix (gentyped x,
		gentyped y,
		double a)
\stopbuffer
\startbuffer[funcdesc:mix]
返回 \math{\marg{x} + (\marg{y} - \marg{x})\times \marg{a}}。
其中 \carg{a} 必須在區間 \math{0.0 \cdots 1.0} 內,否則返回值\cnglo{undef}。
\stopbuffer

% radians
\startbuffer[funcproto:radians]
gentype radians (gentype degrees)
\stopbuffer
\startbuffer[funcdesc:radians]
將角度轉換成弧度,即 \math{(\pi/180)\times \marg{degrees}}。
\stopbuffer

% step
\startbuffer[funcproto:step]
gentype step (gentype edge,
		gentype x)
gentypef step (float edge,
		gentypef x)
gentyped step (double edge,
		gentyped x)
\stopbuffer
\startbuffer[funcdesc:step]
如果 \math{\marg{x} < \marg{edge}},則返回 \math{0.0},否則返回 \math{1.0}。
\stopbuffer

% smoothstep
\startbuffer[funcproto:smoothstep]
gentype smoothstep (gentype edge0,
		gentype edge1,
		gentype x)
gentypef smoothstep (float edge0,
		float edge1,
		gentypef x)
gentyped smoothstep (double edge0,
		double edge1,
		gentyped x)
\stopbuffer
\startbuffer[funcdesc:smoothstep]
如果 \math{\marg{x} \leq \marg{edge0}},則返回 \math{0.0};

如果 \math{\marg{x} \geq \marg{edge1}},則返回 \math{1.0};

如果 \math{\marg{edge0} < \marg{x} < \marg{edge1}},
則實施平滑埃爾米特插值(smooth Hermite interpolation)。
在某些地方需要能進行平滑過渡的臨界函式,這時就可使用此函式。

此函式等同於:
\startcintbl
gentype t;
t = clamp((x - edge0) / (edge1 - edge1), 0, 1);
return t * t * (3 - 2 * t);
\stopcintbl

如果 \math{\marg{edge0} \geq \marg{edge1}},
或者 \carg{x}、 \carg{edge0}、 \carg{edge1} 中的任意一個是 NaN,則結果\cnglo{undef}。
\stopbuffer

% sign
\startbuffer[funcproto:sign]
gentype sign (gentype x)
\stopbuffer
\startbuffer[funcdesc:sign]
如果 \math{\marg{x} > 0},則返回 \math{1.0};
如果 \math{\marg{x} = -0.0},則返回 \math{-0.0};
如果 \math{\marg{x} = +0.0},則返回 \math{+0.0};
如果 \math{\marg{x} < 0},則返回 \math{-1.0};
如果 \carg{x} 是 NaN,則返回 \math{0.0}。
\stopbuffer

% begin TABLE
\startCLFD

\clFD{clamp}
\clFD{degrees}
\clFD{max}
\clFD{min}
\clFD{mix}
\clFD{radians}
\clFD{step}
\clFD{smoothstep}
\clFD{sign}

\stopCLFD

