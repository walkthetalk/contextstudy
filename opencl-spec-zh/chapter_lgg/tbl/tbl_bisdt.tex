\startCLOD[型別][描述]

\clOD{\cldt{bool}
\footnote{任意標量值轉換為 \cldt{bool} 時,如果原始值等於 0,則結果為 0;
否則,結果為 1。}
}{
一種條件數據型別,值為 \ccmm{true} 或 \ccmm{false}。
其中 \ccmm{true} 展開後為整形常量 1;而 \ccmm{false} 展開後為整形常量 0。
}

\clOD{\cldt{char}}{
帶符號 8 位整數,為二的補碼。
}

\clOD{\cldt{unsigned char}\par\cldt{uchar}}{
無符號 8 位整數。
}

\clOD{\cldt{short}}{
帶符號 16 位整數,為二的補碼。
}

\clOD{\cldt{unsigned short}\par\cldt{ushort}}{
無符號 16 位整數。
}

\clOD{\cldt{int}}{
帶符號 32 位整數,為二的補碼。
}

\clOD{\cldt{unsigned int}\par\cldt{uint}}{
無符號 32 位整數。
}

\clOD{\cldt{long}}{
帶符號 64 位整數,為二的補碼。
}

\clOD{\cldt{unsigned long}\par\cldt{ulong}}{
無符號 64 位整數。
}

\clOD{\cldt{float}}{
32 位浮點數。必須符合 IEEE 754 中的單精度存儲格式。
}

\clOD{\cldt{double}
\footnote{\cldt{double} 是可選型別,只有設備的 \cenum{CL_DEVICE_DOUBLE_FP_CONFIG}
(參見\reftab{cldevquery})不是零時才需要支持。}
}{
64 位浮點數。必須符合 IEEE 754 中的雙精度存儲格式。
}

\clOD{\cldt{half}}{
16 位浮點數。必須符合 IEEE 754-2008 中的半精度存儲格式。
}

\clOD{\cldt{size_t}}{
無符號整數,算子 \ccmm{sizeof} 的結果。
如果 \cenum{CL_DEVICE_ADDRESS_BITS}(參見\reftab{cldevquery})
是 32 位,則此型別為 32 位無符號整數;
如果 \cenum{CL_DEVICE_ADDRESS_BITS} 是 64 位,則此型別為 64 位無符號整數。
}

\clOD{\cldt{ptrdiff_t}}{
帶符號整形,兩個指針相減的結果。
如果 \cenum{CL_DEVICE_ADDRESS_BITS}(參見\reftab{cldevquery})
是 32 位,則此型別為 32 位帶符號整數;
如果 \cenum{CL_DEVICE_ADDRESS_BITS} 是 64 位,則此型別為 64 位帶符號整數。
}

\clOD{\cldt{intptr_t}}{
帶符號整形,任意指向 \ccmm{void} 的有效指針都能轉換為此型別,
然後還可以轉換回指向 \ccmm{void} 的指針,其結果與原始指針相同。
如果 \cenum{CL_DEVICE_ADDRESS_BITS}(參見\reftab{cldevquery})
是 32 位,則此型別為 32 位帶符號整數;
如果 \cenum{CL_DEVICE_ADDRESS_BITS} 是 64 位,則此型別為 64 位帶符號整數。
}

\clOD{\cldt{uintptr_t}}{
無符號整形,任意指向 \ccmm{void} 的有效指針都能轉換為此型別,
然後還可以轉換回指向 \ccmm{void} 的指針,其結果與原始指針相同。
如果 \cenum{CL_DEVICE_ADDRESS_BITS}(參見\reftab{cldevquery})
是 32 位,則此型別為 32 位無符號整數;
如果 \cenum{CL_DEVICE_ADDRESS_BITS} 是 64 位,則此型別為 64 位無符號整數。
}

\clOD{\cldt{void}}{
此型別不包含任何值;他是一種不完全型別,不能被補全。
}
\stopCLOD
