% write_image_2d
\startbuffer[funcproto:write_image_2d]
void write_imagef (
	image2d_t image,
	int2 coord,
	float4 color)
void write_imagei (
	image2d_t image,
	int2 coord,
	int4 color)
void write_imageui (
	image2d_t image,
	int2 coord,
	uint4 color)
\stopbuffer
\startbuffer[funcdesc:write_image_2d]
將 \carg{color} 寫入 2D \cnglo{imgobj} \carg{image} 中
由坐標 \math{(coord.x, coord.y)} 所指定的位置上。
寫之前會進行適當的格式轉換。
\math{coord.x} 和 \math{coord.y} 會被當成非歸一化坐標,
其值必須分別在區間 \math{0 \cdots image width-1}
和 \math{0 \cdots image height-1} 內。

對於 \capi{write_imagef},
創建\cnglo{imgobj}時所用的 \carg{image_channel_data_type} 必須是預定義壓縮過的格式
或者 \cenum{CL_SNORM_INT8}、 \cenum{CL_UNORM_INT8}、 \cenum{CL_SNORM_INT16}、
 \cenum{CL_UNORM_INT16}、 \cenum{CL_HALF_FLOAT} 或 \cenum{CL_FLOAT}。

對於 \capi{write_imagei} 而言,
創建\cnglo{imgobj}時所用的 \carg{image_channel_data_type} 必須是下列值之一:
\startigBase
\item \cenum{CL_SIGNED_INT8}
\item \cenum{CL_SIGNED_INT16}
\item \cenum{CL_SIGNED_INT32}
\stopigBase

對於 \capi{write_imageui} 而言,
創建\cnglo{imgobj}時所用的 \carg{image_channel_data_type} 必須是下列值之一:
\startigBase
\item \cenum{CL_UNSIGNED_INT8}
\item \cenum{CL_UNSIGNED_INT16}
\item \cenum{CL_UNSIGNED_INT32}
\stopigBase

如果創建\cnglo{imgobj}時所用的 \carg{image_channel_data_type} 不再上述所列範圍內,
或者坐標 \math{(x, y)} 不在 \math{(0 \cdots image width-1, 0 \cdots image height-1)}
範圍內,
則 \capi{write_imagef}、 \capi{write_imagei} 和 \capi{write_imageui} 的行為未定義。
\stopbuffer

% write_image_2da
\startbuffer[funcproto:write_image_2da]
void write_imagef (
	image2d_array_t image,
	int4 coord,
	float4 color)
void write_imagei (
	image2d_array_t image,
	int4 coord,
	int4 color)
void write_imageui (
	image2d_array_t image,
	int4 coord,
	uint4 color)
\stopbuffer
\startbuffer[funcdesc:write_image_2da]
用 \math{coord.z} 確定 2D 圖像陣列 \carg{image} 中的某一 2D 圖像,
將 \carg{color} 寫入 此 2D 圖像中
由坐標 \math{(coord.x, coord.y)} 所指定的位置上。
寫之前會進行適當的格式轉換。
\math{coord.x}、 \math{coord.y} 以及 \math{coord.z} 會被當成非歸一化坐標,
其值必須分別在區間 \math{0 \cdots image width-1}、
\math{0 \cdots image height-1} 和 \math{0 \cdots image number-1} 內。

對於 \capi{write_imagef},
創建\cnglo{imgobj}時所用的 \carg{image_channel_data_type} 必須是預定義壓縮過的格式
或者 \cenum{CL_SNORM_INT8}、 \cenum{CL_UNORM_INT8}、 \cenum{CL_SNORM_INT16}、
 \cenum{CL_UNORM_INT16}、 \cenum{CL_HALF_FLOAT} 或 \cenum{CL_FLOAT}。

對於 \capi{write_imagei} 而言,
創建\cnglo{imgobj}時所用的 \carg{image_channel_data_type} 必須是下列值之一:
\startigBase
\item \cenum{CL_SIGNED_INT8}
\item \cenum{CL_SIGNED_INT16}
\item \cenum{CL_SIGNED_INT32}
\stopigBase

對於 \capi{write_imageui} 而言,
創建\cnglo{imgobj}時所用的 \carg{image_channel_data_type} 必須是下列值之一:
\startigBase
\item \cenum{CL_UNSIGNED_INT8}
\item \cenum{CL_UNSIGNED_INT16}
\item \cenum{CL_UNSIGNED_INT32}
\stopigBase

如果創建\cnglo{imgobj}時所用的 \carg{image_channel_data_type} 不再上述所列範圍內,
或者坐標 \math{(x, y, z)} 不在 \math{(0 \cdots image width-1, 0 \cdots image height-1, 0 \cdots image number-1)}
範圍內,
則 \capi{write_imagef}、 \capi{write_imagei} 和 \capi{write_imageui} 的行為未定義。
\stopbuffer

% write_image_1d
\startbuffer[funcproto:write_image_1d]
void write_imagef (
	image1d_t image,
	int coord,
	float4 color)
void write_imagei (
	image1d_t image,
	int coord,
	int4 color)
void write_imageui (
	image1d_t image,
	int coord,
	uint4 color)
void write_imagef (
	image1d_buffer_t image,
	int coord,
	float4 color)
void write_imagei (
	image1d_buffer_t image,
	int coord,
	int4 color)
void write_imageui (
	image1d_buffer_t image,
	int coord,
	uint4 color)
\stopbuffer
\startbuffer[funcdesc:write_image_1d]
將 \carg{color} 寫入 1D \cnglo{imgobj} \carg{image} 中
由坐標 \math{coord} 所指定的位置上。
寫之前會進行適當的格式轉換。
\math{coord} 會被當成非歸一化坐標,
其值必須在區間 \math{0 \cdots image width-1} 內。

對於 \capi{write_imagef},
創建\cnglo{imgobj}時所用的 \carg{image_channel_data_type} 必須是預定義壓縮過的格式
或者 \cenum{CL_SNORM_INT8}、 \cenum{CL_UNORM_INT8}、 \cenum{CL_SNORM_INT16}、
 \cenum{CL_UNORM_INT16}、 \cenum{CL_HALF_FLOAT} 或 \cenum{CL_FLOAT}。

對於 \capi{write_imagei} 而言,
創建\cnglo{imgobj}時所用的 \carg{image_channel_data_type} 必須是下列值之一:
\startigBase
\item \cenum{CL_SIGNED_INT8}
\item \cenum{CL_SIGNED_INT16}
\item \cenum{CL_SIGNED_INT32}
\stopigBase

對於 \capi{write_imageui} 而言,
創建\cnglo{imgobj}時所用的 \carg{image_channel_data_type} 必須是下列值之一:
\startigBase
\item \cenum{CL_UNSIGNED_INT8}
\item \cenum{CL_UNSIGNED_INT16}
\item \cenum{CL_UNSIGNED_INT32}
\stopigBase

如果創建\cnglo{imgobj}時所用的 \carg{image_channel_data_type} 不再上述所列範圍內,
或者坐標不在 \math{0 \cdots image width-1} 範圍內,
則 \capi{write_imagef}、 \capi{write_imagei} 和 \capi{write_imageui} 的行為未定義。
\stopbuffer

% write_image_1da
\startbuffer[funcproto:write_image_1da]
void write_imagef (
	image1d_array_t image,
	int2 coord,
	float4 color)
void write_imagei (
	image1d_array_t image,
	int2 coord,
	int4 color)
void write_imageui (
	image1d_array_t image,
	int2 coord,
	uint4 color)
\stopbuffer
\startbuffer[funcdesc:write_image_1da]
用 \math{coord.y} 確定 1D 圖像陣列 \carg{image} 中的某一 1D 圖像,
將 \carg{color} 寫入 此 1D 圖像中
由坐標 \math{coord.x} 所指定的位置上。
寫之前會進行適當的格式轉換。
\math{coord.x} 和 \math{coord.y} 會被當成非歸一化坐標,
其值必須分別在區間 \math{0 \cdots image width-1} 和 \math{0 \cdots image number-1} 之間。

對於 \capi{write_imagef},
創建\cnglo{imgobj}時所用的 \carg{image_channel_data_type} 必須是預定義壓縮過的格式
或者 \cenum{CL_SNORM_INT8}、 \cenum{CL_UNORM_INT8}、 \cenum{CL_SNORM_INT16}、
 \cenum{CL_UNORM_INT16}、 \cenum{CL_HALF_FLOAT} 或 \cenum{CL_FLOAT}。

對於 \capi{write_imagei} 而言,
創建\cnglo{imgobj}時所用的 \carg{image_channel_data_type} 必須是下列值之一:
\startigBase
\item \cenum{CL_SIGNED_INT8}
\item \cenum{CL_SIGNED_INT16}
\item \cenum{CL_SIGNED_INT32}
\stopigBase

對於 \capi{write_imageui} 而言,
創建\cnglo{imgobj}時所用的 \carg{image_channel_data_type} 必須是下列值之一:
\startigBase
\item \cenum{CL_UNSIGNED_INT8}
\item \cenum{CL_UNSIGNED_INT16}
\item \cenum{CL_UNSIGNED_INT32}
\stopigBase

如果創建\cnglo{imgobj}時所用的 \carg{image_channel_data_type} 不再上述所列範圍內,
或者坐標 \math{(x, y)} 不在 \math{(0 \cdots image width-1, 0 \cdots image number-1)}
範圍內,
則 \capi{write_imagef}、 \capi{write_imagei} 和 \capi{write_imageui} 的行為未定義。
\stopbuffer

% begin table
\startCLFD
\clFD{write_image_2d}
\clFD{write_image_2da}
\clFD{write_image_1d}
\clFD{write_image_1da}
\stopCLFD