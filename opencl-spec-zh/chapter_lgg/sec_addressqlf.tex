\section{位址空間限定符}

OpenCL 實現了下列相互分離的位址空間:
 \cqlfemp{__global}、 \cqlfemp{__local}、
 \cqlfemp{__constant}、 \cqlfemp{__private}。
在聲明變量時,可以使用位址空間限定符來指定用哪塊區域的內存來分配對象。
 OpenCL 對 C 中型別限定符的語法做了擴展, OpenCL 中的型別限定符可以包含一個位址空間名。
如果某個對象的型別中帶有位址空間名,則在指定的位址空間中分配此對象;
否則,在缺省的位址空間中分配此對象。

也可以使用不帶前綴 \cqlf{__} 的位址空間限定符即
 \cqlfemp{global}、 \cqlfemp{local}、 \cqlfemp{constant}、 \cqlfemp{private}
 來代替相應帶前綴 \cqlf{__} 的位址空間限定符。

對於\cnglo{program}中的函式引數以及函式的局部變量而言,缺省的位址空間為 \cqlf{__private}。
所有函式引數必須位於 \cqlf{__private} 位址空間中。

如果 \cqlf{__kernel} 函式的引數聲明為指針或某種型別的陣列,
則他指向的位址空間必須是 \cqlf{global}、 \cqlf{local}、 \cqlf{constant}。
兩個指針要想相互賦值,他們必須指向同一個位址空間。
將指向位址空間 A 的指針轉型成為指向位址空間 B 的指針時\cnglo{illegle}的。

如果函式引數的型別為
 \cqlf{image2d_t}、 \cqlf{image3d_t}、 \cqlf{image2d_array_t}、
 \cqlf{image1d_t}、 \cqlf{image1d_buffer_t} 或 \cqlf{image1d_array_t},
則必須在 \cqlf{__global} 位址空間中分配。

如果變量的作用域是整個\cnglo{program},則沒有對應的缺省位址空間。
這樣的變量在聲明時必須加上 \cqlf{__constant}。

例:
\startclc
// declares a pointer p in the __private address space that
// points to an int object in address space __global
__global int *p;

// declares an array of 4 floats in the __private address space.
float x[4];
\stopclc

函式所返回的值沒有對應的位址空間。
聲明函式時,如果在返回的型別上加了位址空間限定符,則會產生編譯錯誤;
除非函式返回的是指針,並且限定符是用在指針指向的對象上。

例:
\startclc
__private int f() { ... } // should generate an error
__local int *f() { ... } // allowed
__local int * __private f() { ... }; // should generate an error.
\stopclc

% __global or global
\subsection{__global (或 global)}

位址空間名 \cqlfemp{__global} 或 \cqlfemp{global} 用來
指代分配自\cnglo{glbmem}的\cnglo{memobj}(\cnglo{bufobj}或\cnglo{imgobj})。

\cnglo{bufobj} 可聲明為指針,指向標量、矢量或用戶自定義的結構體。
這樣\cnglo{kernel}就可以讀寫緩衝區中任意位置的內容。

在\cnglo{host}代碼中通過調用相應 API 分配\cnglo{memobj}陣列時,他的實際大小就確定了。

一些例子:
\startclc
__global float4	*color;		// An array of float4 elements
typedef struct {
	float	a[3];
	int	b[2];
} foo_t;
__global foo_t	*my_info;	// An array of foo_t elements.
\stopclc

如果聲明時帶有此限定符的引數上附着的是\cnglo{imgobj},
則根據所附着對象的類型不同,聲明此引數時其型別必須是
 \ctype{image2d_t} (2D \cnglo{imgobj})、 \ctype{image3d_t} (3D \cnglo{imgobj})、
 \ctype{image2d_array_t} (2D 圖像陣列對象)、 \ctype{image1d_t} (1D \cnglo{imgobj})、
 \ctype{image1d_buffer_t} (1D 圖像緩衝對象)或 \ctype{image1d_array_t} (1D 圖像陣列對象)。
不能直接存取\cnglo{imgobj}的元素。 OpenCL 提供有內建函式可用來讀寫\cnglo{imgobj}。

限定符 \cqlfemp{const} 可以與 \cqlfemp{__global} 一起使用來聲明一個只讀的\cnglo{bufobj}。

% __local or local
\subsection{__local (或 local)}

如果要在\cnglo{locmem}中分配變量,
並在某個\cnglo{workgrp}中的所有\cnglo{workitem}間共享,
則可以使用位址空間名 \cqlfemp{__local} 或 \cqlfemp{local}。
指向 \cqlfemp{__local} 位址空間的指針可以作為函式(包括\cnglo{kernel}函式)的引數。
對於在\cnglo{kernel}函式內聲明的 \cqlfemp{__local} 變量,
其作用域僅為此\cnglo{kernel}函式。

關於在\cnglo{kernel}函式內聲明的 \cqlfemp{__local} 變量,下面是一些例子:
\startclc
__kernel void my_func(...)
{
	__local float	a;	// A single float allocated
				// in local address space
	__local float	b[10];	// An array of 10 floats
				// allocated in local address space.

	if (...)
	{
		// example of variable in __local address space but not
		// declared at __kernel function scope.
		__local float	c;	<- not allowed.
	}
}
\stopclc

不能對在\cnglo{kernel}函式內聲明的 \cqlfemp{__local} 變量進行初始化:
\startclc
__kernel void my_func(...)
{
	__local float	a = 1;	<- not allowed

	__local float	b;
	b = 1;			<- allowed
}
\stopclc

注意:

如果\cnglo{kernel}函式內聲明了 \cqlfemp{__local} 變量,
則會為執行此\cnglo{kernel}的每個\cnglo{workgrp}都分配一個,
並且只在\cnglo{workgrp}的生命周期內才存在。

% __constant or constant
\subsection{__constant (或 constant)}

位址空間名 \cqlfemp{__constant} 或 \cqlfemp{constant} 所描述的變量分配自\cnglo{glbmem},
並且在\cnglo{kernel}中作為只讀變量來存取。
所有\cnglo{workitem}(包括全局\cnglo{workitem})在執行\cnglo{kernel}時都可以讀取這些變量。

所有常值字串都存儲在 \cqlf{__constant} 位址空間中。

注意:

在給\cnglo{kernel}的常數引數進行計數時,
每個指向 \cqlf{__constant} 位址空間的指針引數都會使計數增一。
參見\reftab{cldevquery}中的 \cenum{CL_DEVICE_MAX_CONSTANT_ARGS}。

如果變量的作用域是整個\cnglo{program}或\cnglo{kernel}函式的最外層,
則可以在 \cqlf{__constant} 位址空間中聲明他。
必須對這種變量進行初始化,並且所用的值必須是編譯時常數。
對這種變量的寫操作會導致編譯時錯誤。

不要求實作將這種聲明聚合進幾個常數引數中,以使引數個數最少。這種行為\cnglo{impdef}。

想要代碼是可移植的,就必須做最保守的假定,
即每個在函式或\cnglo{program}的作用域中聲明的 \cqlfemp{__constant} 變量
在計數時都算是一個單獨的引數。

% __private or private
\subsection{__private (或 private)}

\cnglo{kernel}函式中所聲明的不帶位址空間限定符的變量,非\cnglo{kernel}函式中的所有變量,
以及所有函式引數都位於 \cqlfemp{__private} 或 \cqlfemp{private} 位址空間中。
如果所聲明的變量是指針,並且沒有指定位址空間限定符,
則認為他指向的是 \cqlfemp{__private} 位址空間。

所保留的關鍵字
 \cqlf{__global}、 \cqlf{__constant}、 \cqlf{__local}、 \cqlf{__private}、
 \cqlf{global}、 \cqlf{constant}、 \cqlf{local} 和 \cqlf{private}
 只作為位址空間限定符使用,不作他用。
