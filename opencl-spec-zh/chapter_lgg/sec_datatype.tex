\section{所支持的數據類型}

所支持的數據類型如下所示。

% Built-in Scalar Data Types
\subsection{內建標量數據類型}

\reftab{builtInScalarDataTypes}中列出了內建的標量數據類型。

\splitfloat{
\placetable[here,force][tab:builtInScalarDataTypes]
{內建標量數據類型}
}{
{\startCLOD[型別][描述]

\clOD{\cldts{bool}
\footnote{任意標量值轉換為 \cldts{bool} 時,如果原始值等於 0,則結果為 0;
否則,結果為 1。}
}{
一種條件數據型別,值為 \ccmm{true} 或 \ccmm{false}。
其中 \ccmm{true} 展開後為整形常量 1;而 \ccmm{false} 展開後為整形常量 0。
}

\clOD{\cldts{char}}{
帶符號 8 位整數,為二的補碼。
}

\clOD{\cldts{unsigned char}\par\cldts{uchar}}{
無符號 8 位整數。
}

\clOD{\cldts{short}}{
帶符號 16 位整數,為二的補碼。
}

\clOD{\cldts{unsigned short}\par\cldts{ushort}}{
無符號 16 位整數。
}

\clOD{\cldts{int}}{
帶符號 32 位整數,為二的補碼。
}

\clOD{\cldts{unsigned int}\par\cldts{uint}}{
無符號 32 位整數。
}

\clOD{\cldts{long}}{
帶符號 64 位整數,為二的補碼。
}

\clOD{\cldts{unsigned long}\par\cldts{ulong}}{
無符號 64 位整數。
}

\clOD{\cldts{float}}{
32 位浮點數。必須符合 IEEE 754 中的單精度存儲格式。
}

\clOD{\cldts{double}
\footnote{\cldts{double} 是可選型別,只有設備的 \cenum{CL_DEVICE_DOUBLE_FP_CONFIG}
(參見\reftab{cldevquery})不是零時才需要支持。}
}{
64 位浮點數。必須符合 IEEE 754 中的雙精度存儲格式。
}

\clOD{\cldts{half}}{
16 位浮點數。必須符合 IEEE 754-2008 中的半精度存儲格式。
}

\clOD{\cldts{size_t}}{
無符號整數,算子 \ccmm{sizeof} 的結果。
如果 \cenum{CL_DEVICE_ADDRESS_BITS}(參見\reftab{cldevquery})
是 32 位,則此型別為 32 位無符號整數;
如果 \cenum{CL_DEVICE_ADDRESS_BITS} 是 64 位,則此型別為 64 位無符號整數。
}

\clOD{\cldts{ptrdiff_t}}{
帶符號整形,兩個指針相減的結果。
如果 \cenum{CL_DEVICE_ADDRESS_BITS}(參見\reftab{cldevquery})
是 32 位,則此型別為 32 位帶符號整數;
如果 \cenum{CL_DEVICE_ADDRESS_BITS} 是 64 位,則此型別為 64 位帶符號整數。
}

\clOD{\cldts{intptr_t}}{
帶符號整形,任意指向 \ccmm{void} 的有效指針都能轉換為此型別,
然後還可以轉換回指向 \ccmm{void} 的指針,其結果與原始指針相同。
如果 \cenum{CL_DEVICE_ADDRESS_BITS}(參見\reftab{cldevquery})
是 32 位,則此型別為 32 位帶符號整數;
如果 \cenum{CL_DEVICE_ADDRESS_BITS} 是 64 位,則此型別為 64 位帶符號整數。
}

\clOD{\cldts{uintptr_t}}{
無符號整形,任意指向 \ccmm{void} 的有效指針都能轉換為此型別,
然後還可以轉換回指向 \ccmm{void} 的指針,其結果與原始指針相同。
如果 \cenum{CL_DEVICE_ADDRESS_BITS}(參見\reftab{cldevquery})
是 32 位,則此型別為 32 位無符號整數;
如果 \cenum{CL_DEVICE_ADDRESS_BITS} 是 64 位,則此型別為 64 位無符號整數。
}

\clOD{\cldts{void}}{
此型別不包含任何值;他是一種不完全型別,不能被補全。
}
\stopCLOD
}
}

在 OpenCL API(以及頭文件)中,大多數內建標量類型都被聲明為其他類型,
以更好的為\cnglo{app}所用。
下表列出了 OpenCL C 編程語言中的內建標量類型與\cnglo{app}所用類型間的對應關係。

\startCLOO[OpenCL 語言中的型別][應用所用 API 中的型別]

\clOO{\cldts{bool}}{\cldts{n/a}}

\clOO{\cldts{char}}{\cldts{cl_char}}

\clOO{\cldts{unsigned char}\par \cldts{uchar}}{\cldts{cl_uchar}}

\clOO{\cldts{short}}{\cldts{cl_short}}

\clOO{\cldts{unsigned short}\par \cldts{ushort}}{\cldts{cl_ushort}}

\clOO{\cldts{int}}{\cldts{cl_int}}

\clOO{\cldts{unsigned int}\par \cldts{uint}}{\cldts{cl_uint}}

\clOO{\cldts{long}}{\cldts{cl_long}}

\clOO{\cldts{unsigned long}\par \cldts{ulong}}{\cldts{cl_ulong}}

\clOO{\cldts{float}}{\cldts{cl_float}}

\clOO{\cldts{double}}{\cldts{cl_double}}

\clOO{\cldts{half}}{\cldts{cl_half}}

\clOO{\cldts{size_t}}{\cldts{n/a}}

\clOO{\cldts{ptrdiff_t}}{\cldts{n/a}}

\clOO{\cldts{intptr_t}}{\cldts{n/a}}

\clOO{\cldts{uintptr_t}}{\cldts{n/a}}

\clOO{\cldts{void}}{\cldts{n/a}}
\stopCLOO



\subsubsection{數據類型 \cldts{half}}

數據類型 \cldts{half} 必須符合 IEEE 754-2008。
類型為 \cldts{half} 的數含有 1 個符號位, 5 個指數位以及 10 個尾數位。
符號、指數、尾數的含義與 IEEE 754 浮點數類似。
指數偏置值為 15。
數據類型 \cldts{half} 必須能夠表示有限規格化數,去規格化數,無限以及 NaN。
不能將 \cldts{half} 類型的去規格化數
(可能是用 vstore_half 將 \cldts{float} 變換成 \cldts{half} 時生成的,
也可能是用 vload_half 將 \cldts{half} 變換成 \cldts{float} 時生成的)刷成 0。
從 \cldts{float} 到 \cldts{half} 的變換會將尾數舍入成 11 位精度。
從  \cldts{half} 到 \cldts{float} 的變換是無損的;
所有 \cldts{half} 數都可精確地表示成 \cldts{float} 值。

數據類型 \cldts{half} 只能用來聲明含有 \cldts{half} 值的緩存指針。
下面是幾個例子。
\startclc
void bar (__global half *p)
{
	....
}
__kernel void foo (__global half *pg, __local half *pl)
{
	__global half *ptr;
	int offset;

	ptr = pg + offset;
	bar(ptr);
}
\stopclc

下面的例子是對類型 \cldts{half} 的不當應用:
\startclc
half a;
half b[100];

half *p;
a = *p;		<- not allowed. must use vload_half function
\stopclc

函數 \capi{vload_half}、 \capi{vload_halfn}、 \capi{vloada_halfn}
 和 \capi{vstore_half}、 \capi{vstore_halfn}、 \capi{vstorea_halfn}
 可分別加載和存儲 \cldts{half} 指針,參見\todo{6.12.7}。
加載函數可從內存中讀取標量或矢量 \cldts{half} 值 並將其變換成 \cldts{float} 值。
而存儲函數則將標量或矢量 \cldts{float} 值作為輸入,並(以恰當的舍入模式)
將其變換成 \cldts{half} 標量或矢量值後寫入內存中。


% Built-in Vector Data Types
\subsection{內建矢量數據類型}

% Other Built-in Data Types
\subsection{其他內建數據類型}

% Reserved Data Types
\subsection{保留的數據類型}

% Alignment of Types
\subsection{類型對齊}

% Vector Literals
\subsection{常值矢量}

% Vector Components
\subsection{矢量構件}

% Aliasing Rules
\subsection{別名規則}

% Keywords
\subsection{關鍵字}

