\section{訪問限定符}

\cnglo{kernel}參數中的\cnglo{imgobj}可以聲明為只讀的或只寫的。
\cnglo{kernel}不能對一個\cnglo{imgobj}即讀又寫。
對於\cnglo{kernel}參數中的\cnglo{imgobj},
如果\cnglo{kernel}要讀取他,則聲明時要加上限定符 \cqlfemp{__read_only} (或 \cqlfemp{read_only});
如果\cnglo{kernel}要寫入他,則聲明時要加上限定符 \cqlfemp{__write_only} (或 \cqlfemp{write_only})。
缺省的限定符為 \cqlfemp{__read_only}。

下面例子:
\startclc
__kernel void foo (read_only image2d_t imageA,
		   write_only image2d_t imageB)
{
	....
}
\stopclc
\cvar{imageA} 是一個只讀的 2D \cnglo{imgobj},
而 \cvar{imageB} 是一個只寫的 2D \cnglo{imgobj}。

下列保留關鍵字僅用作訪問限定符,不作他用:
\startigBase
\item \cqlf{__read_only}、 \cqlf{__write_only}、 \cqlf{__read_write}、
\item \cqlf{read_only}、 \cqlf{write_only} 和 \cqlf{read_write}。
\stopigBase
