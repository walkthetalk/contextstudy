\startcomponent commands
\product prd-bbum

\chapter{命令}
\placecontent
\section{bbread}
命令 bbread 用来显示 BitBake 的元数据。如果不带参数,
会解析“\code{conf/bitbake.conf}”(位于 \code{BBPATH} 中)并显示出来。
如果在命令行上提供了一个文件,如一个 .bb,则会使用签名所得到的配置元数据继续解析此文件。

{\ftemp 已经移除了独立的 bbread 命令,以 bitbake -e 来代替。}

\section{bitbake}
\subsubject{简介}
bitbake 是系统中最主要的命令。
它帮助执行单个 .bb 文件中的任务、执行多个 .bb 文件上的一个任务、或者分析它们之间的依赖关系。

\subsubject{用法和语法}
\startbb
$ bitbake --help
usage: bitbake [options] [package ...]

Executes the specified task (default is 'build') for a given set of BitBake files.
It expects that BBFILES is defined, which is a space seperated list of files to
be executed.  BBFILES does support wildcards.
Default BBFILES are the .bb files in the current directory.

options:
  --version             show program's version number and exit
  -h, --help            show this help message and exit
  -b BUILDFILE, --buildfile=BUILDFILE
                        execute the task against this .bb file, rather than a
                        package from BBFILES.
  -k, --continue        continue as much as possible after an error. While the
                        target that failed, and those that depend on it,
                        cannot be remade, the other dependencies of these
                        targets can be processed all the same.
  -f, --force           force run of specified cmd, regardless of stamp status
  -i, --interactive     drop into the interactive mode also called the BitBake
                        shell.
  -c CMD, --cmd=CMD     Specify task to execute. Note that this only executes
                        the specified task for the providee and the packages
                        it depends on, i.e. 'compile' does not implicitly call
                        stage for the dependencies (IOW: use only if you know
                        what you are doing). Depending on the base.bbclass a
                        listtaks tasks is defined and will show available
                        tasks
  -r FILE, --read=FILE  read the specified file before bitbake.conf
  -v, --verbose         output more chit-chat to the terminal
  -D, --debug           Increase the debug level. You can specify this more
                        than once.
  -n, --dry-run         don't execute, just go through the motions
  -p, --parse-only      quit after parsing the BB files (developers only)
  -d, --disable-psyco   disable using the psyco just-in-time compiler (not
                        recommended)
  -s, --show-versions   show current and preferred versions of all packages
  -e, --environment     show the global or per-package environment (this is
                        what used to be bbread)
  -g, --graphviz        emit the dependency trees of the specified packages in
                        the dot syntax
  -I IGNORED_DOT_DEPS, --ignore-deps=IGNORED_DOT_DEPS
                        Stop processing at the given list of dependencies when
                        generating dependency graphs. This can help to make
                        the graph more appealing

\stopbb

\example Executing a task against a single .bb

为单个文件执行任务相对简单一些。
你指定文件, bitbake 会解析此文件并执行所指定的任务(或者缺省执行“build”)。
在此过程中, bitbakek 会遵循任务间的依赖关系。

“clean”任务:
\startbb
$ bitbake -b blah_1.0.bb -c clean
\stopbb

“build”任务:
\startbb
$ bitbake -b blah_1.0.bb
\stopbb

\example Executing tasks against a set of .bb files

要管理多个 .bb 文件,会引入大量额外的复杂性。
显然,需要一个途径告诉 bitbake 哪些文件可用,我们想要执行哪些。
同样也需要有一种方法来描述 .bb 之间的依赖关系,无论是构建时还是运行时。
当多个 .bb 文件提供了相同的功能,
或者一个 .bb 有多个版本的时候,需要一种手段让用户进行选择。

下一节,元数据,勾勒除了要怎样做这些事情。

命令 \code{bitbake},当不使用 \code{--buildfile} 时,
接受一个 \code{PROVIDER},而不是一个文件或其它什么东西。
缺省情况下,一个 .bb 一般会提供它的 packagename、 packagename-version 以及 packagename-version-revision。
\startbb
$ bitbake blah
$ bitbake blah-1.0
$ bitbake blah-1.0-r0
$ bitbake -c clean blah
$ bitbake virtual/whatever
$ bitbake -c clean virtual/whatever
\stopbb

\example Generating dependency graphs

BitBake 可以使用 dot 语法生成依赖关系图。可以使用 graphviz 中的 dot 应用将这些图转换成图片。
会向当前工作目录中写入三个文件, {\ftref depends.dot} 包含 \code{DEPENDS} 变量,
{\ftref rdepends.dot} 和 {\ftref alldepends.dot} 包含 \code{DEPENDS} 和 \code{RDEPENDS}。
为了防止包含通用的依赖,可以使用 \code{-I depend} 将其从图中去掉。这样生成的图具有更好的可读性。
如:会在图中移除所继承的类中的 \code{DEPENDS} (base.bbclass)。
\startbb
$ bitbake -g blah
$ bitbake -g -I virtual/whatever -I bloom blah
\stopbb

\subsubject{元数据}
你可能在用法或 .bb 文件的相关信息中已经看到了,
变量 \code{BBFILES} 指明了 bitbake 工具怎样查找文件。
此变量是以空格分隔的文件清单,支持通配符。

\example Setting BBFILES

\startbb
BBFILES = "/path/to/bbfiles/*.bb"
\stopbb

考虑到依赖关系,需要 .bb 定义变量 \code{DEPENDS},
包含一个包名的清单,以空格分隔,这些包名就是变量 \code{PN}。
一般而言,缺省情况下变量 \code{PN} 是 .bb 文件名的一部分。

\example Depending on another .bb

a.bb:
\startbb
PN = "package-a"
DEPENDS += "package-b"
\stopbb

b.bb:
\startbb
PN = "package-b"
\stopbb

\example Using PROVIDES

本例会展示变量 \code{PROVIDES} 的用法,它使 .bb 可以指定自己提供什么功能。

package1.bb:
\startbb
PROVIDES += "virtual/package"
\stopbb

package2.bb:
\startbb
DEPENDS += "virtual/package"
\stopbb
package3.bb:
\startbb
PROVIDES += "virtual/package"
\stopbb

如您所见,此处有两个提供不同功能(virtual / package)的 .bb。
显然,需要一种方法让运行 bitbake 的用户选择。
实际上,确实有一种方法。

下面代码出现在 .conf 文件中,会选择 package1:
\startbb
PREFERRED_PROVIDER_virtual/package = "package1"
\stopbb

\example Specifying version preference

当一个包有多个版本时,缺省情况下 bitbake 会选择最高版本,除非指定了。
如果一个 .bb 中的变量 \code{DEFAULT_PREFERENCE} 小于其它 .bb 中的(默认是 0),则不会被选中。
这样,维护 .bb 文件库的人就可以指定他们的首选项,即缺省选择哪个版本。另外,用户也可以指定其首选版本。

第一个 .bb 名为 \code{a_1.1.bb},则其变量 \code{PN} 将为“\code{a}”,而变量 \code{PV} 将为 \code{1.1}。

如果还有一个文件 \code{a_1.2.bb},则 bitbake 缺省会选择 \code{1.2}。
不过,如果我们在 bitbake 所解析的 .conf 文件中定义了以下变量,可以改变这种情况。
\startbb
PREFERRED_VERSION_a = "1.1"
\stopbb

\example Using “bbfile collections”

可能有多个 bb 文件库都包含某一个包。例如,可能有上游库的一份本地拷贝,并做了一些修改,可以这样使用:
\startbb
BBFILES = "/stuff/openembedded/*/*.bb /stuff/openembedded.modified/*/*.bb"
BBFILE_COLLECTIONS = "upstream local"
BBFILE_PATTERN_upstream = "^/stuff/openembedded/"
BBFILE_PATTERN_local = "^/stuff/openembedded.modified/"
BBFILE_PRIORITY_upstream = "5"
BBFILE_PRIORITY_local = "10"
\stopbb

\stopcomponent

