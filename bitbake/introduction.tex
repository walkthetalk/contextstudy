\startcomponent inroduction
\product prd-bbum

\chapter{简介}
\placecontent
\section{概览}
简单来讲, BitBake 就是可以执行任务并管理元数据的一个工具。
就这点而言,显然它跟 GNU 的 make 和其它构建工具类似。
其理念来自于 Portage, Gentoo 所使用的包管理工具。
BitBake 是 OpenEmbedded 项目的基础,
用以构建以及维护大量嵌入式 Linux 发行版,包括 OpenZaurus 和 Familiar。

\section{背景和目标}
在 BitBake 之前,没有任何一个构建工具能够满足一个嵌入式 Linux 发行版的所有需求。
传统桌面 Linux 发行版所使用的构建系统都缺少一些重要的功能,
在嵌入式领域中所有点对点的 {\ftref buildroot} 系统中,
没有一个是可扩展或易维护的。

BitBake的重要目标有:
\startigBase
\item 应对交叉编译;
\item 应对包间的依赖关系(在目标架构上编译时、在本地架构上编译时,以及运行时);
\item 支持在一个包中运行任意数量的任务,这包括,但不限于,获取上游源码、解包、打补丁、配置等等;
\item 必须对 Linux 发行版没有要求(无论是构建用还是目标);
\item 必须对架构没有要求;
\item 必须支持多种构建和目标操作系统(包括 cygwin、 BSD 等等);
\item 必须可以自包含,而不是紧紧整合到构建机器的根文件系统中;
\item 必须可以应对带条件的元数据(是否在目标架构上、什么操作系统、什么发行版、什么机器);
\item 必须能让用户很容易的使用它操作自己的本地元数据和包;
\item 必须能很容易的在使用 BitBake 的多个项目间协作;
\item 应该提供继承机制,使得不同的包可以共享通用的元数据;
\item 等等。
\stopigBase

BitBake 满足以上所有目标,甚至更多。灵活性和能力一直依赖具有最高优先级。
它高度可扩展,支持嵌入式 Python 代码以及执行任意任务。

\stopcomponent

