\startcomponent inroduction
\product prd-bbum

\chapter{简介}
\placecontent
\section{概览}
简单来讲,BitBake就是可以执行任务并管理元数据的一个工具。就这点而言,显然它跟GNU的make和其它构建工具类似。
其理念来自于Portage,Gentoo所使用的包管理工具。BitBake是OpenEmbedded项目的基础,
用以构建以及维护大量嵌入式Linux发行版,包括OpenZaurus和Familiar。

\section{背景和目标}
在BitBake之前,没有任何一个构建工具能够满足一个嵌入式Linux发行版的所有需求。
传统桌面Linux发行版所使用的构建系统都缺少一些重要的功能,在嵌入式领域中所有点对点的{\ftref buildroot}系统中,
没有一个是可扩展或易维护的。

BitBake的重要目标有:
\startigBase
\item 应对交叉编译;
\item 应对包间的依赖关系(在目标架构上编译时、在本地架构上编译时,以及运行时);
\item 支持在一个包中运行任意数量的任务,这包括,但不限于,获取上游源码、解包、打补丁、配置等等;
\item 必须对Linux发行版没有要求(无论是构建用还是目标);
\item 必须对架构没有要求;
\item 必须支持多种构建和目标操作系统(包括cygwin、BSD等等);
\item 必须可以自包含,而不是紧紧整合到构建机器的根文件系统中;
\item 必须可以应对带条件的元数据(是否在目标架构上、什么操作系统、什么发行版、什么机器);
\item 必须能让用户很容易的使用它操作自己的本地元数据和包;
\item 必须能很容易的在使用BitBake的多个项目间协作;
\item 应该提供继承机制,使得不同的包可以共享通用的元数据;
\item 等等。
\stopigBase

BitBake满足以上所有目标,甚至更多。灵活性和能力一直依赖具有最高优先级。
它高度可扩展,支持嵌入式Python代码以及执行任意任务。

\stopcomponent

