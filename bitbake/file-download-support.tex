\startcomponent file-download-support
\product prd-bbum

\chapter{对文件下载的支持}
\placecontent
\section{概览}
BitBake 提供了对文件下载的支持,这个过程叫 fetching。
\code{SRC_URI} 一般用来告诉 BitBake 要 fetch 哪些文件。
下一节会描述所有可用的 fetcher 以及相应的选项。
每个 fetcher 都有一套变量,每个 URI 都有一系列参数
(以“\code{;}”分隔,包括一个 key 和一个 value)。
变量和参数的语义由 fetcher 所定义。
当然, BitBake 会尽量保证不同的 fetcher 有相同的语义。

\section{Local File Fetcher}
本地文件 fetcher 的 \code{URN} 是 {\ftref file}。
文件名既可以是绝对路径也可以是相对路径。
如果是相对路径,会使用 \code{FILESPATH} 和 \code{FILESDIR} 查找相应文件,
至于使用哪个随 \code{OVERRIDES} 而定。
即可以指定单个文件,也可以指定整个目录。
\startbb
SRC_URI= "file://relativefile.patch"
SRC_URI= "file://relativefile.patch;this=ignored"
SRC_URI= "file:///Users/ich/very_important_software"
\stopbb

\section{CVS File Fetcher}
CVS fetcher 的 \code{URN} 是 {\ftref cvs}。
它会使用变量 \code{DL_DIR}、 \code{SRCDATE}、 \code{FETCHCOMMAND_cvs}、 \code{UPDATECOMMAND_cvs}。
\code{DL_DIR} 指定一个临时目录用来保存检出的代码。
\code{SRCDATE} 则为 fetch 时使用什么时候的代码(如果是“\code{now}”则每次构建前都会检出最新代码)。
\code{FETCHCOMMAND} 和 \code{UPDATECOMMAND} 则指定在 checkout 和 update 时使用哪个可执行程序。

所支持的参数为 \code{module}、 \code{tag}、 \code{date}、 \code{method}、 \code{localdir}、 \code{rsh}。
\code{module} 指定要检出哪个模块, \code{tag} 即检出时使用那个 CVS TAG (默认为空)。
可以指定一个 \code{date} 来覆盖配置中的 \code{SRCDATE}。特殊值“\code{now}”会使得每次构建前都会更新代码。
\code{method} 默认为 {\ftref pserver},如果改为 {\ftref ext},则对参数 \code{rsh} 求值并设置 \code{CVS_RSH}。
最后使用 \code{localdir} 检出代码到一个特殊目录(相对于 \code{CVSDIR})。
\startbb
SRC_URI = "cvs://CVSROOT;module=mymodule;tag=some-version;method=ext"
SRC_URI = "cvs://CVSROOT;module=mymodule;date=20060126;localdir=usethat"
\stopbb

\section{HTTP/FTP Fetcher}
HTTP/FTP 的 \code{URN} 是 {\ftref http}、 {\ftref https} 和 {\ftref ftp}。
它会使用变量 \code{DL_DIR}、 \code{FETCHCOMMAND_wget}、 \code{PREMIRRORS}、 \code{MIRRORS}。
\code{DL_DIR} 定义了将 fetch 到的文件存储到哪里。
\code{FETCHCOMMAND} 包含 fetch 所使用的命令。
“\code{${URI}}”和“\code{${FILES}}”会被 uri 和文件的 basename 替换。
fetch 文件时会先尝试 \code{PREMIRRORS},如果失败会使用 \code{MIRRORS} 直到全部都试过一遍。

仅支持一个参数 \code{md5sum}。 fetch 后会计算文件的 \code{md5sum} 并将两者进行比较。
\startbb
SRC_URI = "http://oe.handhelds.org/not_there.aac;md5sum=12343"
SRC_URI = "ftp://oe.handhelds.org/not_there_as_well.aac;md5sum=1234"
SRC_URI = "ftp://you@oe.handheld.sorg/home/you/secret.plan;md5sum=1234"
\stopbb

\section{SVK Fetcher}
{\ftemp 暂时不支持}。

\section{SVN Fetcher}
SVN Fetcher 的 \code{URN} 是 {\ftref svn}。它会使用三个变量:
\code{FETCHCOMMAND_svn}、 \code{DL_DIR}、 \code{SRCDATE}。
\code{FETCHCOMMAND} 包含 subversion 命令。 \code{DL_DIR} 即源码包的存储目录。
\code{SRCDATE} 即 fetch 时所使用的时间(如果是“\code{now}”则每次构建前都会检出最新代码)。

所支持的参数有 \code{proto}、 \code{rev}。 \code{proto} 是 subversion 的协议类型,
\code{rev} 是 subversion 的版本。
\startbb
SRC_URI = "svn://svn.oe.handhelds.org/svn;module=vip;proto=http;rev=667"
SRC_URI = "svn://svn.oe.handhelds.org/svn/;module=opie;proto=svn+ssh;date=20060126"
\stopbb

\section{GIT Fetcher}
GIT Fetcher 的 \code{URN} 是 {\ftref git}。它会使用变量 \code{DL_DIR} 和 \code{GITDIR}。
\code{DL_DIR} 即所检出代码的存储目录。 \code{GITDIR} 即将 git 树克隆到哪。

所支持的参数有 \code{tag}、 \code{protocol}。 \code{tag} 是 git 标签,默认是“\code{master}”。
\code{protocol} 是所缺省的 git 协议,默认是“\code{rsync}”。
\startbb
SRC_URI = "git://git.oe.handhelds.org/git/vip.git;tag=version-1"
SRC_URI = "git://git.oe.handhelds.org/git/vip.git;protocol=http"
\stopbb

\stopcomponent

