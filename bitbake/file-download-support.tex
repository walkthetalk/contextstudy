\startcomponent file-download-support
\product prd-bbum

\chapter{对文件下载的支持}
\placecontent
\section{概览}
BitBake提供了对文件下载的支持,这个过程叫fetching。\code{SRC_URI}一般用来告诉BitBake要
fetch哪些文件。下一节会描述所有可用的fetcher以及相应的选项。每个fetcher都有一套变量,
每个URI都有一系列参数(以“\code{;}”分隔,包括一个key和一个value)。变量和参数的语义由fetcher所定义。
当然,BitBake会尽量保证不同的fetcher有相同的语义。

\section{Local File Fetcher}
本地文件fetcher的\code{URN}是{\ftref file}。文件名既可以是绝对路径也可以是相对路径。
如果是相对路径,会使用\code{FILESPATH}和\code{FILESDIR}查找相应文件,至于使用哪个随\code{OVERRIDES}而定。
即可以指定单个文件,也可以指定整个目录。
\startbb
SRC_URI= "file://relativefile.patch"
SRC_URI= "file://relativefile.patch;this=ignored"
SRC_URI= "file:///Users/ich/very_important_software"
\stopbb

\section{CVS File Fetcher}
CVS fetcher的\code{URN}是{\ftref cvs}。它会使用变量\code{DL_DIR}、\code{SRCDATE}、
\code{FETCHCOMMAND_cvs}、\code{UPDATECOMMAND_cvs}。
\code{DL_DIR}指定一个临时目录用来保存检出的代码。
\code{SRCDATE}则为fetch时使用什么时候的代码(如果是“\code{now}”则每次构建前都会检出最新代码)。
\code{FETCHCOMMAND}和\code{UPDATECOMMAND}则指定在checkout和update时使用哪个可执行程序。

所支持的参数为\code{module}、\code{tag}、\code{date}、\code{method}、\code{localdir}、\code{rsh}。
\code{module}指定要检出哪个模块,\code{tag}即检出时使用那个CVS TAG(默认为空)。
可以指定一个\code{date}来覆盖配置中的\code{SRCDATE}。特殊值“\code{now}”会使得每次构建前都会更新代码。
\code{method}默认为{\ftref pserver},如果改为{\ftref ext},则对参数\code{rsh}求值并设置\code{CVS_RSH}。
最后使用\code{localdir}检出代码到一个特殊目录(相对于\code{CVSDIR})。
\startbb
SRC_URI = "cvs://CVSROOT;module=mymodule;tag=some-version;method=ext"
SRC_URI = "cvs://CVSROOT;module=mymodule;date=20060126;localdir=usethat"
\stopbb

\section{HTTP/FTP Fetcher}
HTTP/FTP的\code{URN}是{\ftref http}、{\ftref https}和{\ftref ftp}。
它会使用变量\code{DL_DIR}、\code{FETCHCOMMAND_wget}、\code{PREMIRRORS}、\code{MIRRORS}。
\code{DL_DIR}定义了将fetch到的文件存储到哪里。
\code{FETCHCOMMAND}包含fetch所使用的命令。
“\code{${URI}}”和“\code{${FILES}}”会被uri和文件的basename替换。
fetch文件时会先尝试\code{PREMIRRORS},如果失败会使用\code{MIRRORS}直到全部都试过一遍。

仅支持一个参数\code{md5sum}。fetch后会计算文件的\code{md5sum}并将两者进行比较。
\startbb
SRC_URI = "http://oe.handhelds.org/not_there.aac;md5sum=12343"
SRC_URI = "ftp://oe.handhelds.org/not_there_as_well.aac;md5sum=1234"
SRC_URI = "ftp://you@oe.handheld.sorg/home/you/secret.plan;md5sum=1234"
\stopbb

\section{SVK Fetcher}
{\ftemp 暂时不支持}。

\section{SVN Fetcher}
SVN Fetcher的\code{URN}是{\ftref svn}。它会使用三个变量:
\code{FETCHCOMMAND_svn}、\code{DL_DIR}、\code{SRCDATE}。
\code{FETCHCOMMAND}包含subversion命令。\code{DL_DIR}即源码包的存储目录。
\code{SRCDATE}即fetch时所使用的时间(如果是“\code{now}”则每次构建前都会检出最新代码)。

所支持的参数有\code{proto}、\code{rev}。\code{proto}是subversion的协议类型,
\code{rev}是subversion的版本。
\startbb
SRC_URI = "svn://svn.oe.handhelds.org/svn;module=vip;proto=http;rev=667"
SRC_URI = "svn://svn.oe.handhelds.org/svn/;module=opie;proto=svn+ssh;date=20060126"
\stopbb

\section{GIT Fetcher}
GIT Fetcher的\code{URN}是{\ftref git}。它会使用变量\code{DL_DIR}和\code{GITDIR}。
\code{DL_DIR}即所检出代码的存储目录。\code{GITDIR}即将git树克隆到哪。

所支持的参数有\code{tag}、\code{protocol}。\code{tag}是git标签,默认是“\code{master}”。
\code{protocol}是所缺省的git协议,默认是“\code{rsync}”。
\startbb
SRC_URI = "git://git.oe.handhelds.org/git/vip.git;tag=version-1"
SRC_URI = "git://git.oe.handhelds.org/git/vip.git;protocol=http"
\stopbb

\stopcomponent

