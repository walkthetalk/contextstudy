\startcomponent metadata
\product prd-bbum

\chapter{元数据}
\placecontent
\section{描述}
BitBake 的元数据可以分成三种:
\startigBase
\item 配置文件
\item .bb文件
\item 类别文件
\stopigBase

下面是 BitBake 元数据的一些例子。
上面所说几种文件中,任何不支持的语法都会在文档中列出。
\subsubject{Basic variable setting}
%\starttyping[option=c]
\startbb
VARIABLE = "value"
\stopbb
%\stoptyping
本例中,\code{VARIABLE}是一个值。

\subsubject{Variable expansion}
BitBake 支持变量引用另一个变量的内容,语法跟 shell 脚本类似。
\startbb
A = "aval"
B = "pre${A}post"
\stopbb
则\code{A}为\code{aval},\code{B}为\code{preavalpost}。

\subsubject{Immediate variable expansion (:=)}
\code{:=} 会使变量内容立即展开,而不是用时再展开。
\startbb
T = "123"
A := "${B} ${A} test ${T}"
T = "456"
B = "${T} bval"

C = "cval"
C := "${C}append"
\stopbb
本例中,\code{A} 内容为 \code{test 123}, \code{B} 为 \code{456 bval},
而 \code{C} 则是 \code{cvalappend}。

\subsubject{Appending (+=) and prepending (=+)}
\startbb
B = "bval"
B += "additionaldata"
C = "cval"
C =+ "test"
\stopbb
本例中, \code{B} 为 \code{bval additinaldata},而 \code{C} 为 \code{test cval}。

\subsubject{Appending (.=) and prepending (=.) without spaces}
\startbb
B = "bval"
B .= "additionaldata"
C = "cval"
C =. "test"
\stopbb
本例中, \code{B} 为 \code{bvaladditionaldata} 而 \code{C} 为 \code{testcval}。
与上例相比,没有引入额外的空格。

\subsubject{Conditional metadata set}
\code{OVERRIDES} 是一个以“\code{:}”分隔的变量,所包含的每一项都是要满足的条件。
如果一个变量依赖于“\code{arm}”,而“\code{arm}”在 \code{OVERRIDES} 中,
则会使用“\code{arm}”版本的变量,而不是使用不带条件的版本。例如:
\startbb
OVERRIDES = "architecture:os:machine"
TEST = "defaultvalue"
TEST_os = "osspecificvalue"
TEST_condnotinoverrides = "othercondvalue"
\stopbb
本例中,\code{TEST} 将为 \code{osspecificvalue},
这是因为 \code{OVERRIDES} 中有条件“\code{os}”。

\subsubject{Conditional appending}
BitBake支持根据\code{OVERRIDES}中有什么来对变量进行追加和前置。例如:
\startbb
DEPENDS = "glibc ncurses"
OVERRIDES = "machine:local"
DEPENDS_append_machine = " libmad"
\stopbb
本例中, \code{DEPENDS} 被置为 \code{glibc ncurses libmad}。

\subsubject{Inclusion}
接下来,有一个 \code{include} 指令,它会使 BitBake 解析你所指定的文件,并在此处插入,
跟 {\ftemp make} 差不多。不过,如果 \code{include} 所指定路径是相对路径,
BitBake 会使用从 \code{BBPATH} 中找到的第一个。

\subsubject{Requiring Inclusion}
跟指令 \code{include} 相反,如果找不到所包含的文件会导致 \code{ParseError}。
否则跟 \code{include} 的行为一样。

\subsubject{Python variable expansion}
\startbb
DATE = "${@time.strftime('%Y%m%d',time.gmtime())}"
\stopbb
变量\code{DATE}中会包含当天日期。

\subsubject{Defining executable metadata}
{\ftemp NOTE}:只有.bb文件和.bbclass文件才支持。
\startbb
do_mytask () {
    echo "Hello, world!"
}
\stopbb
本质上跟设置一个变量没什么不同,只不过这个变量是可执行的shell代码。
\startbb
python do_printdate () {
    import time
    print time.strftime('%Y%m%d', time.gmtime())
}
\stopbb
跟前面的一样,只是将其标记为python,这样BitBake就知道这是python代码。

\subsubject{Defining python functions into the global python namespace}
{\ftemp NOTE}:只有.bb文件和.bbclass文件才支持。
\startbb
def get_depends(bb, d):
    if bb.data.getVar('SOMECONDITION', d, True):
        return "dependencywithcond"
    else:
        return "dependency"

SOMECONDITION = "1"
DEPENDS = "${@get_depends(bb, d)}"
\stopbb
最终\code{DEPENDS}会包含\code{dependencywithcond}。

\subsubject{Inheritance}
{\ftemp NOTE}:只有.bb文件和.bbclass文件才支持。

指令 \code{inherit} 提供了一种手段,用来指定你的.bb需要哪一类功能。它是继承的基本形式。例如,
你可以很容易的在使用 \code{autoconf} 和 \code{automake} 构建包的任务中抽出一些,
将其放到一个 bbclass 中,然后用它来构建你自己的包。
通过在 \code{BBPATH} 中搜索 \code{classes\filename.oeclass} 来查找一个 bbclass,
其中 \code{filename} 就是你要继承的。

\subsubject{Tasks}
{\ftemp NOTE}:只有 .bb 文件和 .bbclass 文件才支持。

在 BitBake 中,一个 .bb 中要运行的每一步骤都是一个 task。
有一个命令 \code{addtask} 可以用来添加新的任务
(必须定义为 python 执行体元数据,并且带有前缀“\code{do_}”)并定义任务间的依赖关系。
\startbb
python do_printdate () {
    import time
    print time.strftime('%Y%m%d', time.gmtime())
}

addtask printdate before do_build
\stopbb
此代码定义了一个 python 函数,并将其作为 \code{do_build} (缺省的任务)所依赖的一个任务添加。
无论是谁要执行任务 \code{do_build},都会导致先执行 \code{do_printdata}。

\subsubject{Events}
{\ftemp NOTE}:只有 .bb 文件和 .bbclass 文件才支持。

BitBake允许安装事件处理程序。在运行过程中的一些特定点上可以触发事件,
如,操作一个 .bb 之前、开始一个任务、任务失败、任务成功等等。
目的是像让诸如构建失败发送 email 通知等一些事情更容易一些。
\startbb
addhandler myclass_eventhandler
python myclass_eventhandler() {
    from bb.event import NotHandled, getName
    from bb import data

    print "The name of the Event is %s" % getName(e)
    print "The file we run for is %s" % data.getVar('FILE', e.data, True)

    return NotHandled
}
\stopbb
每当事件被触发后就会调用相应的处理程序。定义了一个全局变量 \code{e}。
\code{e.data} 包含 \code{bb.data} 的一个实例。
可以用方法 \code{getName(e)} 得到所触发事件的名字。

上面的事件处理程序会打印事件的名称以及变量\code{FILE}的内容。

\section{解析}
\subsubject{配置文件}
BitBake 中的第一类元数据就是配置元数据。此类元数据是全局的,
因此会影响所执行的{\ftemp 所有}包和任务。
目前, BitBake 只知道一个配置文件的存在(文件名是硬编码的)。
它期望在用户指定的 \code{BBPATH} 中可以找到文件“\code{conf/bitbake.conf}”。
此文件中一般会\code{include}其它元数据。

在配置文件中只允许定义变量和使用 \code{include} 指令。

\subsubject{类别文件}
BitBake 的类别提供了最基本的继承机制。
在介绍元数据时简单提了一下,当遇到 \code{inherit} 指令时
就会去解析这些文件,这些文件位于 \code{classes/} 中(相对于 \code{BBPATH})。

\subsubject{.bb文件}
.bb文件是所要执行任务的一个逻辑单元。一般是一个要构建的包。同时会遵循.bb间的依赖关系。
这些文件由变量 \code{BBFILES} 指定,此变量是以空格分隔的 .bb 文件清单,可以含有通配符。

\stopcomponent

